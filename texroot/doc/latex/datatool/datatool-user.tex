\documentclass[report,widecs]{nlctdoc}

\DeleteShortVerb{\|}

\usepackage[T1]{fontenc}
\usepackage[utf8]{inputenc}
\usepackage{ifthen}
\usepackage[verbose=false]{datatool}
\usepackage{mfirstuc}[2017/11/10]
\usepackage{datapie,databar,databib,person,datagidx}
\usepackage{textcomp}
\usepackage{graphicx}
\usepackage{colortbl}
\usepackage{booktabs}
\usepackage{metalogo}
\usepackage{cmap}
\usepackage{alltt}
\ifpdf
\usepackage{mathpazo}
\usepackage[scaled=.88]{helvet}
\usepackage{courier}
\fi
\usepackage{caption}
\usepackage[colorlinks,
            bookmarks,
            hyperindex=false,
            pdfauthor={Nicola L.C. Talbot},
            pdftitle={datatool: Databases and data manipulation},
            pdfkeywords={LaTeX,package,database,data,chart,plot}]{hyperref}
\usepackage{xr-hyper}
\usetikzlibrary{snakes}

\PageIndex
\IndexPrologue{\chapter*{Index}}
\setcounter{IndexColumns}{2}

\ifpdf
  \externaldocument{datatool-code}
\fi

\doxitem{Counter}{counter}{counters}
\doxitem{Option}{option}{package options}

\newcommand*{\desctr}[1]{\DescribeCounter{#1}\ctrfmt{#1}}

\newrobustcmd{\dbs}{\texttt{\string\\}}

\newcommand{\idbs}{%
 \dbs\index{\string\\\actualchar\dbs\encapchar usage}%
}

\setlength\marginparwidth{70pt}


 %bibliography database
\DTLnewdb{docbib}
\DTLnewrow{docbib}
\DTLnewdbentry{docbib}{CiteKey}{Goossens}
\DTLnewdbentry{docbib}{EntryType}{book}
\DTLnewdbentry{docbib}{Author}{{}{Goossens}{}{Michel},{}{Mittelbach}{}{Frank},{}{Samarin}{}{Alexander}}
\DTLnewdbentry{docbib}{Title}{The \LaTeX\ Companion}
\DTLnewdbentry{docbib}{Publisher}{Addison-Wesley}
\DTLnewdbentry{docbib}{Year}{1994}

 % datagidx index example database

\newgidx{gidx-index}{Index}% define a database for the index

\DTLgidxSetDefaultDB{gidx-index}% set this as the default 

\newterm{mac\'edoine}
\newterm{macram\'e}
\newterm[label=elite]{{\'e}lite}
\newterm{reptile}
\newterm[seealso={reptile}]{crocodylian}

\newterm
 [%
   parent=crocodylian
 ]
 {crocodile}

\newterm
 [%
   parent=crocodylian
 ]
 {alligator}

\newterm
 [%
   parent=crocodylian,
   description={(also cayman)}
 ]
 {caiman}

\newterm[see={caiman}]{cayman}

\begin{document}
% Some of the commands are too long to produce nicely formatted
% paragraphs so use ragged-right:
\raggedright
\setlength{\parindent}{1em}%

\MakeShortVerb{"}

 \title{User Manual for datatool bundle version~2.32}
 \author{Nicola L.C. Talbot\\
\url{http://www.dickimaw-books.com/}}

 \date{2019-09-27}
 \maketitle

\pagenumbering{roman}

\noindent
The \styfmt{datatool} bundle comes with the following documentation:
\begin{description}
  \item[datatool-user.pdf]
  This document is the main user guide for the \styfmt{datatool}
  bundle.

  \item[\url{datatool-code.pdf}]
  Advanced users wishing to know more about the inner workings of
  all the packages provided in the \styfmt{datatool} bundle should
  read \qt{Documented Code for datatool v2.32}

  \item[INSTALL] Installation instructions.

  \item[CHANGES] Change log.

  \item[README] Package summary.
\end{description}
Additional online resources:
\begin{itemize}
\item \styfmt{datatool} FAQ:
\href{https://www.dickimaw-books.com/faqs/datatoolfaq.html}{dickimaw-books.com/faqs/datatoolfaq.html}
\item Bug tracker: \href{https://www.dickimaw-books.com/bugtracker.php}{dickimaw-books.com/bugtracker.php}
\item \styfmt{datatool} performance: \href{https://www.dickimaw-books.com/gallery/datatool-performance.shtml}{dickimaw-books.com/gallery/datatool-performance.shtml}
\end{itemize}

\begin{important}
There's an old adage, ``use the right tool for the right job.''
A carpenter's fine chisel is the right tool for delicate carving,
but if you try to use it to hack off a tree branch it will take a
long time. That doesn't mean there's something wrong with the
chisel. It just means you're using the wrong tool for the job.

The \styfmt{datatool} bundle is provided to help perform repetitive
commands, such as mail merging, but since \TeX\ is designed as a
typesetting language, don't expect this bundle to perform as
efficiently as custom database systems or a dedicated mathematical
or scripting language. \textbf{If the provided packages take a frustratingly
long time to compile your document, use another language to perform
your calculations or data manipulation and save the results in a
file that can be input into your document.} For large amounts of
data that need to be sorted or filtered or joined, consider storing your data 
in an~SQL database and use
\app{datatooltk}\footnote{\url{http://www.dickimaw-books.com/software/datatooltk/}} to import the data, 
using SQL syntax to filter, sort and otherwise
manipulate the values.
\end{important}

This bundle consists of the following packages:
\begin{description}
\item[\sty{datatool}] Main package providing database support.
Automatically loads \sty{datatool-base}.

\item[\sty{datatool-base}] Provides the main library code for
numerical and string functions. Automatically
loads \sty{datatool-fp} or \sty{datatool-pgfmath} depending on
package options.

\item[\sty{datagidx}] Package for generating indexes and glossaries.
Automatically loads \sty{datatool}.

\item[\sty{databar}] Package for drawing bar charts.
Automatically loads \sty{datatool}.

\item[\sty{datapie}] Package for drawing pie charts.
Automatically loads \sty{datatool}.

\item[\sty{dataplot}] Package for drawing simple line graphs.
Automatically loads \sty{datatool}.

\item[\sty{databib}] Package for loading a bibliography into a
database. Automatically loads \sty{datatool}.

\item[\sty{person}] Package for referencing people by the
appropriate gender pronouns. Automatically loads \sty{datatool}.

\end{description}

In addition, there are two mutually exclusive packages
\sty{datatool-fp} and \sty{datatool-pgfmath} that provide
mathematical related commands that are just wrapper functions for
\sty{fp} or \sty{pgfmath} commands. These can be loaded individually
without loading \sty{datatool}. For example, the following documents
produce the same results, but the first uses the \sty{fp} package
and the second uses the \sty{pgfmath} package:

\begin{enumerate}
\item Using \sty{fp} macros:
\begin{verbatim}
\documentclass{article}
\usepackage{datatool-fp}
\begin{document}
1=2: \dtlifnumeq{1}{2}{true}{false}.
\end{document}
\end{verbatim}

\item Using \sty{pgfmath} macros:
\begin{verbatim}
\documentclass{article}
\usepackage{datatool-pgfmath}
\begin{document}
1=2: \dtlifnumeq{1}{2}{true}{false}.
\end{document}
\end{verbatim}

\end{enumerate}

\begin{important}
Both \sty{fp} and \sty{pgfmath} have some limitations. These limitations
will therefore also be present in the various packages provided
with \sty{datatool}, according to the underlying package used.
\end{important}

\cleardoublepage
\pdfbookmark[0]{Contents}{contents}
\tableofcontents
\cleardoublepage
\pdfbookmark[0]{List of Examples}{examples}
\listofexamples
\cleardoublepage
\pdfbookmark[0]{List of Figures}{figures}
\listoffigures
\cleardoublepage
\pdfbookmark[0]{List of Tables}{tables}
\listoftables

\pagenumbering{arabic}
\chapter{Introduction}

 \changes{1.0}{2007 July 22}{Initial version}
The \sty{datatool} bundle consists of the following
packages: \sty{datatool} (which loads \sty{datatool-base} and either
\sty{datatool-fp} or \sty{datatool-pgfmath}), \sty{datagidx}, \sty{datapie}, \sty{dataplot},
\sty{databar}, \sty{databib} and \sty{person}. 

\begin{itemize}
\item The \sty{datatool} package can be used to:
\begin{itemize}
\item Create or load databases.
\item Sort rows of a database (either numerically or alphabetically,
ascending or descending).
\item Perform repetitive operations on each row of a database
(e.g.\ mail merging). Conditions may be imposed to exclude rows.
\end{itemize}
Package Options:
  \begin{description}
   \item[\pkgopt{utf8}] Boolean key. May be used to switch off the UTF-8
   support described on \pageref{utf8support} in \sectionref{sec:sort}.
   If you both load \sty{inputenc} with UTF-8
   support and you use accent commands like \cs{'} or \cs{c} then make
   sure you have at least version 2.05 of \sty{mfirstuc}
   if you want to use commands like \ics{makefirstuc}.
   \item[\pkgopt{verbose}] Boolean key. If
    \pkgoptval{true}{verbose}, prints informational messages in
     transcript.
   \item[\pkgopt{math}] May take one of two values:
    \pkgoptval{fp}{math} (load \sty{datatool-fp}) or
    \pkgoptval{pgfmath}{math} (load \sty{datatool-pgfmath}). Default is:
    \pkgoptval{fp}{math}.
   \item[\pkgopt{delimiter}] Delimiter used in CSV files.
   Default is a double quote (\texttt{\string"}).
   \item[\pkgopt{separator}] Delimiter used in CSV files.
   Default is a comma (\texttt{,}).
  \end{description}

\item The \sty{datatool-base} package can be used to:
\begin{itemize}
\item Determine whether an argument is an integer, a real number,
currency or a string. (Scientific notation is currently not
supported.) Locale dependent number settings are supported (such
as a comma as a decimal character and a full stop as a number
group character).
\item Convert locale dependent numbers/currency to the decimal
format required by the \sty{fp} or \sty{pgfmath} packages, enabling fixed point
arithmetic to be performed on elements of the database.
\item Names can be converted to initials.
\item Determine if strings are all upper or lower case.
\item Perform string comparisons (both case sensitive and case insensitive).
\end{itemize}
Package Options:
  \begin{description}
   \item[\pkgopt{verbose}] Boolean key. If
    \pkgoptval{true}{verbose}, prints informational messages in
     transcript.
   \item[\pkgopt{math}] May take one of two values:
    \pkgoptval{fp}{math} (load \sty{datatool-fp}) or
    \pkgoptval{pgfmath}{math} (load \sty{datatool-pgfmath}). Default is:
    \pkgoptval{fp}{math}.
  \end{description}

\item The \sty{datagidx} package (see \autoref{sec:datagidx}) can be used
to generate indexes or glossaries as an alternative to packages
such as \sty{glossaries}.

\item The \sty{datapie} package (see \autoref{sec:datapie}) can be used to convert a database into
a pie chart:
\begin{itemize}
\item Segments can be separated from the rest of the chart to make
them stand out.
\item Colour/grey scale options.
\item Predefined segment colours can be changed.
\item Hooks provided to add extra information to the chart
\end{itemize}

\item The \sty{databar} package (see \autoref{sec:databar}) can be used to convert a database into
a bar chart:
\begin{itemize}
\item Colour/grey scale options.
\item Predefined bar colours can be changed.
\item Hooks provided to add extra information to the chart
\end{itemize}

(The \sty{datapie} and \sty{databar} packages do not support the
creation of 3D charts, and I have no plans to implement them at any
later date. The use of 3D charts should be discouraged. They may look
pretty, but the purpose of a chart is to be informative. Three
dimensional graphics cause distortion, which can result in misleading
impressions. The \sty{pgf} manual provides a more in-depth 
discussion on the matter.)

\item The \sty{dataplot} package (see \autoref{sec:dataplot}) can be used to convert a database into
a two dimensional plot using markers and/or lines. Three dimensional
plots are currently not supported.

\item The \sty{databib} package (see \autoref{sec:databib}) can be used to convert a \BibTeX\ database
into a \sty{datatool} database.

\item The \sty{person} package (see \autoref{sec:person}) can be
used for gender-specific mail-merging and similar uses to avoid the
cumbersome use of the impersonal \qt{he\slash she}.

\end{itemize}

\chapter{Data Types}
\label{sec:datatypes}

The \sty{datatool-base} package recognises four data types: integers,
real numbers, currency and strings.

\begin{description}
\item[Integers] An integer is a sequence of digits, optionally 
groups of three digits may be separated by the number group character.
The default number group character is a comma (,) but may be
changed using \cs{DTLsetnumberchars} (see below).

\item[Real Numbers] A real number is an integer followed by the
decimal character followed by one or more digits. The decimal
character is a full stop (.) by default. The number group 
and decimal characters may be changed using
\begin{definition}[\DescribeMacro{\DTLsetnumberchars}]%
\cs{DTLsetnumberchars}\marg{number group character}\marg{decimal character}
\end{definition}
\begin{important}
Note that scientific notation is not supported, and the number group
character may not be used after the decimal character.
\end{important}

\item[Currency] A currency symbol followed by an integer or
real number is considered to be the currency data type.
There are two predefined currency symbols, "\$" and \cs{pounds}.
In addition, if any of the following commands are defined at the
start of the document, they are also considered to be a currency 
symbol: \cs{texteuro}, \cs{textdollar}, \cs{textstirling},
\cs{textyen}, \cs{textwon}, \cs{textcurrency}, \cs{euro}
and \cs{yen}. Additional currency symbols can be defined using
\begin{definition}[\DescribeMacro{\DTLnewcurrencysymbol}]%
\cs{DTLnewcurrencysymbol}\marg{symbol}
\end{definition}

\item[Strings] Anything that doesn't belong to the above three
types is considered to be a string.

\end{description}

\section{Conditionals}
\label{sec:ifconditions}

The following conditionals are provided by the \sty{datatool-base}
package:
\begin{definition}[\DescribeMacro{\DTLifint}]%
\cs{DTLifint}\marg{text}\marg{true part}\marg{false part}
\end{definition}
If \meta{text} is an integer then do \meta{true part}, otherwise
do \meta{false part}. For example
\begin{verbatim}
\DTLifint{2536}{integer}{not an integer}
\end{verbatim}
produces:
\DTLifint{2536}{integer}{not an integer}.

The number group character may appear in the number, for example:
\begin{verbatim}
\DTLifint{2,536}{integer}{not an integer}
\end{verbatim}
produces:
\DTLifint{2,536}{integer}{not an integer}.
However, the number group character may only be followed by a group 
of three digits.  For example:
\begin{verbatim}
\DTLifint{2,5,3,6}{integer}{not an integer}
\end{verbatim}
produces:
\DTLifint{2,5,3,6}{integer}{not an integer}.
The number group character may be changed. For example:
\begin{verbatim}
\DTLsetnumberchars{.}{,}%
\DTLifint{2,536}{integer}{not an integer}
\end{verbatim}
this now produces: 
\DTLsetnumberchars{.}{,}\relax
\DTLifint{2,536}{integer}{not an integer}, since 2,536 is now
a real number.

Note that nothing else can be appended or prepended to the
number. For example:
\begin{verbatim}
\DTLsetnumberchars{,}{.}%
\DTLifint{2,536m}{integer}{not an integer}
\end{verbatim}
produces:
\DTLsetnumberchars{,}{.}\relax
\DTLifint{2,536m}{integer}{not an integer}.

\begin{definition}[\DescribeMacro{\DTLifreal}]%
\cs{DTLifreal}\marg{text}\marg{true part}\marg{false part}
\end{definition}
If \meta{text} is a real number then do \meta{true part}, otherwise
do \meta{false part}. For example
\begin{verbatim}
\DTLifreal{1000.0}{real}{not real}
\end{verbatim}
produces:
\DTLifreal{1000.0}{real}{not real}.

Note that an integer is not considered a real number:
\begin{verbatim}
\DTLifreal{1,000}{real}{not real}
\end{verbatim}
produces:
\DTLifreal{1,000}{real}{not real}.

Whereas
\begin{verbatim}
\DTLifreal{1,000.0}{real}{not real}
\end{verbatim}
produces:
\DTLifreal{1,000.0}{real}{not real}.

However
\begin{verbatim}
\DTLsetnumberchars{.}{,}%
\DTLifreal{1,000}{real}{not real}
\end{verbatim}
produces:
\DTLsetnumberchars{.}{,}\relax
\DTLifreal{1,000}{real}{not real} since the comma is now
the decimal character.

Currency is not considered to be real:
\begin{verbatim}
\DTLsetnumberchars{,}{.}%
\DTLifreal{\$1.00}{real}{not real}
\end{verbatim}
produces:
\DTLsetnumberchars{,}{.}\relax
\DTLifreal{\$1.00}{real}{not real}.

\begin{definition}[\DescribeMacro{\DTLifcurrency}]%
\cs{DTLifcurrency}\marg{text}\marg{true part}\marg{false part}
\end{definition}
If \meta{text} is currency, then do \meta{true part}, otherwise
do false part. For example:
\begin{verbatim}
\DTLifcurrency{\$5.99}{currency}{not currency}
\end{verbatim}
produces:
\DTLifcurrency{\$5.99}{currency}{not currency}. Similarly:
\begin{verbatim}
\DTLifcurrency{\pounds5.99}{currency}{not currency}
\end{verbatim}
produces:
\DTLifcurrency{\pounds5.99}{currency}{not currency}.
Note, however, that
\begin{verbatim}
\DTLifcurrency{US\$5.99}{currency}{not currency}
\end{verbatim}
produces:
\DTLifcurrency{US\$5.99}{currency}{not currency}. If you want
this to be considered currency, you will have to add the
sequence "US\$" to the set of currency symbols:
\begin{verbatim}
\DTLnewcurrencysymbol{US\$}%
\DTLifcurrency{US\$5.99}{currency}{not currency}
\end{verbatim}
this now produces:
\DTLnewcurrencysymbol{US\$}\relax
\DTLifcurrency{US\$5.99}{currency}{not currency}.

This document has used the \sty{textcomp} package which defines
\cs{texteuro}, so this is also considered to be currency. For
example:
\begin{verbatim}
\DTLifcurrency{\texteuro5.99}{currency}{not currency}
\end{verbatim}
produces:
\DTLifcurrency{\texteuro5.99}{currency}{not currency}.

The preferred method is to display the euro symbol in a sans-serif
font, but
\begin{verbatim}
\DTLifcurrency{\textsf{\texteuro}5.99}{currency}{not currency}
\end{verbatim}
will produce:
\DTLifcurrency{\textsf{\texteuro}5.99}{currency}{not currency}.

It is better to define a new command, for example:
\begin{verbatim}
\DeclareRobustCommand*{\euro}{\textsf{\texteuro}}
\end{verbatim}
and add that command to the list of currency symbols. In fact,
in this case, if you define the command \cs{euro} in the 
preamble, it will automatically be added to the list of known
currency symbols. If however you define \cs{euro} in the document,
you will have to add it using \ics{DTLnewcurrencysymbol}. For
example:
\begin{verbatim}
\newcommand*{\euro}{\textsf{\texteuro}}%
\DTLnewcurrencysymbol{\euro}%
\DTLifcurrency{\euro5.99}{currency}{not currency}
\end{verbatim}
produces:
\DeclareRobustCommand*{\euro}{\textsf{\texteuro}}\relax
\DTLnewcurrencysymbol{\euro}\relax
\DTLifcurrency{\euro5.99}{currency}{not currency}.

\begin{definition}[\DescribeMacro{\DTLifcurrencyunit}]%
\cs{DTLifcurrencyunit}\marg{text}\marg{symbol}\marg{true part}\marg{false part}
\end{definition}
If \meta{text} is currency, and uses \meta{symbol} as the unit of
currency, then do \meta{true part} otherwise do \meta{false part}.
For example:
\begin{verbatim}
\DTLifcurrencyunit{\$6.99}{\$}{dollars}{not dollars}
\end{verbatim}
produces:
\DTLifcurrencyunit{\$6.99}{\$}{dollars}{not dollars}.
Another example:
\begin{verbatim}
\def\cost{\euro10.50}%
\DTLifcurrencyunit{\cost}{\euro}{euros}{not euros}
\end{verbatim}
produces:
\def\cost{\euro10.50}\relax
\DTLifcurrencyunit{\cost}{\euro}{euros}{not euros}.

\begin{definition}[\DescribeMacro{\DTLifnumerical}]%
\cs{DTLifnumerical}\marg{text}\marg{true part}\marg{false part}
\end{definition}
If \meta{text} is numerical (either an integer, real number or
currency) then do \meta{true part} otherwise do 
\meta{false part}. 
For example:
\begin{verbatim}
\DTLifnumerical{1,000.0}{number}{string}.
\end{verbatim}
produces: \DTLifnumerical{1,000.0}{number}{string}.
Whereas
\begin{verbatim}
\DTLsetnumberchars{.}{,}%
\DTLifnumerical{1,000.0}{number}{string}.
\end{verbatim}
produces:
\DTLsetnumberchars{.}{,}\relax
\DTLifnumerical{1,000.0}{number}{string}.
Since the number group character is now a full stop, and the
decimal character is now a comma. (The number group character
may only appear before the decimal character, not after it.)

Currency is also considered to be numerical:
\begin{verbatim}
\DTLsetnumberchars{,}{.}%
\DTLifnumerical{\$1,000.0}{number}{string}.
\end{verbatim}
produces:
\DTLsetnumberchars{,}{.}\relax
\DTLifnumerical{\$1,000.0}{number}{string}.

\begin{definition}[\DescribeMacro{\DTLifstring}]%
\cs{DTLifstring}\marg{text}\marg{true part}\marg{false part}
\end{definition}
This is the opposite of \cs{DTLifnumerical}. If \meta{text} is
not numerical, do \meta{true part}, otherwise do \meta{false part}.

\begin{definition}[\DescribeMacro{\DTLifcasedatatype}]%
\cs{DTLifcasedatatype}\marg{text}\marg{string case}\marg{int case}\marg{real case}\marg{currency case}
\end{definition}
If \meta{text} is a string do \meta{string case}, if \meta{text}
is an integer do \meta{int case}, if \meta{text} is a real number
do \meta{real case}, if \meta{text} is currency do 
\meta{currency case}. For example:
\begin{verbatim}
\DTLifcasedatatype{1,000}{string}{integer}{real}{currency}
\end{verbatim}
produces:
\DTLifcasedatatype{1,000}{string}{integer}{real}{currency}.

\begin{definition}[\DescribeMacro{\dtlifnumeq}]%
\cs{dtlifnumeq}\marg{num1}\marg{num2}\marg{true part}\marg{false
paty}
\end{definition}
If \meta{num1} is equal to \meta{num2}, then do \meta{true part},
otherwise to \meta{false part} where \meta{num1} and \meta{num2}
are plain numbers using a full stop as the decimal point and no
number group separator. For currency or locale dependent numbers use
\cs{DTLifnumeq}.

\begin{definition}[\DescribeMacro{\DTLifnumeq}]%
\cs{DTLifnumeq}\marg{num1}\marg{num2}\marg{true part}\marg{false part}
\end{definition}
If \meta{num1} is equal to \meta{num2}, then do \meta{true part},
otherwise do \meta{false part}. Note that both \meta{num1} and
\meta{num2} must be numerical (either integers, real numbers or
currency). The currency symbol is ignored when determining 
equality. For example
\begin{verbatim}
\DTLifnumeq{\pounds10.50}{10.5}{true}{false}
\end{verbatim}
produces:
\DTLifnumeq{\pounds10.50}{10.5}{true}{false}, since they are
considered to be numerically equivalent. Likewise:
\begin{verbatim}
\DTLifnumeq{\pounds10.50}{\$10.50}{true}{false}
\end{verbatim}
produces:
\DTLifnumeq{\pounds10.50}{\$10.50}{true}{false}.

\pagebreak
\begin{definition}[\DescribeMacro{\DTLifstringeq}]%
\cs{DTLifstringeq}\marg{string1}\marg{string2}\marg{true part}\marg{false part}
\end{definition}
\begin{definition}[\DescribeMacro{\DTLifstringeq*}]%
\cs{DTLifstringeq*}\marg{string1}\marg{string2}\marg{true part}\marg{false part}
\end{definition}
If \meta{string1} and \meta{string2} are the same, then do
\meta{true part}, otherwise do \meta{false part}. The starred
version ignores the case, the unstarred version is case
sensitive. Both
\meta{string1} and \meta{string2} are considered to be strings,
so for example:
\begin{verbatim}
\DTLifstringeq{10.50}{10.5}{true}{false}
\end{verbatim}
produces:
\DTLifstringeq{10.50}{10.5}{true}{false}.

Note that 
\begin{verbatim}
\DTLifstringeq{Text}{text}{true}{false}
\end{verbatim}
produces:
\DTLifstringeq{Text}{text}{true}{false}, whereas
\begin{verbatim}
\DTLifstringeq*{Text}{text}{true}{false}
\end{verbatim}
produces:
\DTLifstringeq*{Text}{text}{true}{false}, however it should also be
noted that many commands will be ignored, so:
\begin{verbatim}
\DTLifstringeq{\uppercase{t}ext}{text}{true}{false}
\end{verbatim}
produces:
\DTLifstringeq{\uppercase{t}ext}{text}{true}{false}.

Spaces are considered to be equivalent to \cs{space} and "~". For
example:
\begin{verbatim}
\DTLifstringeq{an apple}{an~apple}{true}{false}
\end{verbatim}
produces:
\DTLifstringeq{an apple}{an~apple}{true}{false}. Consecutive spaces
are treated as the same, for example:
\begin{verbatim}
\DTLifstringeq{an  apple}{an apple}{true}{false}
\end{verbatim}
produces:
\DTLifstringeq{an  apple}{an apple}{true}{false}.

\begin{definition}[\DescribeMacro{\DTLifeq}]%
\cs{DTLifeq}\marg{arg1}\marg{arg2}\marg{true part}\marg{false part}
\end{definition}
\begin{definition}[\DescribeMacro{\DTLifeq*}]%
\cs{DTLifeq*}\marg{arg1}\marg{arg2}\marg{true part}\marg{false part}
\end{definition}
If both \meta{arg1} and \meta{arg2} are numerical, then this is
equivalent to \cs{DTLifnumeq}, otherwise it is equivalent to
\cs{DTLifstringeq} (when using \cs{DTLifeq}) or \cs{DTLifstringeq*}
(when using \cs{DTLifeq*}).

\begin{definition}[\DescribeMacro{\dtlifnumlt}]%
\cs{dtlifnumlt}\marg{num1}\marg{num2}\marg{true part}\marg{false
paty}
\end{definition}
If \meta{num1} is less than \meta{num2}, then do \meta{true part},
otherwise to \meta{false part} where \meta{num1} and \meta{num2}
are plain numbers using a full stop as the decimal point and no
number group separator. For currency or locale dependent numbers use
\cs{DTLifnumlt}.

\begin{definition}[\DescribeMacro{\DTLifnumlt}]%
\cs{DTLifnumlt}\marg{num1}\marg{num2}\marg{true part}\marg{false part}
\end{definition}
If \meta{num1} is less than \meta{num2}, then do \meta{true part},
otherwise do \meta{false part}. Note that both \meta{num1} and
\meta{num2} must be numerical (either integers, real numbers or
currency).

\begin{definition}[\DescribeMacro{\DTLifstringlt}]%
\cs{DTLifstringlt}\marg{string1}\marg{string2}\marg{true 
part}\marg{false part}
\end{definition}
\begin{definition}[\DescribeMacro{\DTLifstringlt*}]%
\cs{DTLifstringlt*}\marg{string1}\marg{string2}\marg{true 
part}\marg{false part}
\end{definition}
If \meta{string1} is alphabetically less than \meta{string2}, then do
\meta{true part}, otherwise do \meta{false part}. The starred
version ignores the case, the unstarred version is case
sensitive.
For example:
\begin{verbatim}
\DTLifstringlt{aardvark}{zebra}{less}{not less}
\end{verbatim}
produces:
\DTLifstringlt{aardvark}{zebra}{less}{not less}.

Note that both \meta{string1} and \meta{string2} are considered to be
strings, so for example:
\begin{verbatim}
\DTLifstringlt{2}{10}{less}{not less}
\end{verbatim}
produces:
\DTLifstringlt{2}{10}{less}{not less}, since the string "2" 
comes after the string "10" when arranged alphabetically.

The case sensitive (unstarred) version considers uppercase characters
to be less than lowercase characters, so
\begin{verbatim}
\DTLifstringlt{B}{a}{less}{not less}
\end{verbatim}
produces:
\DTLifstringlt{B}{a}{less}{not less}, whereas
\begin{verbatim}
\DTLifstringlt*{B}{a}{less}{not less}
\end{verbatim}
produces:
\DTLifstringlt*{B}{a}{less}{not less}.

\begin{definition}[\DescribeMacro{\DTLiflt}]%
\cs{DTLiflt}\marg{arg1}\marg{arg2}\marg{true part}\marg{false part}
\end{definition}
\begin{definition}[\DescribeMacro{\DTLiflt*}]%
\cs{DTLiflt*}\marg{arg1}\marg{arg2}\marg{true part}\marg{false part}
\end{definition}
If \meta{arg1} and \meta{arg2} are both numerical, then this
is equivalent to \cs{DTLifnumlt}, otherwise it is equivalent
to \cs{DTLstringlt} (when using \cs{DTLiflt}) or
\cs{DTLstringlt*} (when using \cs{DTLiflt*}).

\begin{definition}[\DescribeMacro{\DTLifnumgt}]%
\cs{DTLifnumgt}\marg{num1}\marg{num2}\marg{true part}\marg{false part}
\end{definition}
If \meta{num1} is greater than \meta{num2}, then do \meta{true part},
otherwise do \meta{false part}. Note that both \meta{num1} and
\meta{num2} must be numerical (either integers, real numbers or
currency).

\begin{definition}[\DescribeMacro{\DTLifstringgt}]%
\cs{DTLifstringgt}\marg{string1}\marg{string2}\marg{true 
part}\marg{false part}
\end{definition}
\begin{definition}[\DescribeMacro{\DTLifstringgt*}]%
\cs{DTLifstringgt*}\marg{string1}\marg{string2}\marg{true 
part}\marg{false part}
\end{definition}
If \meta{string1} is alphabetically greater than \meta{string2}, then
do \meta{true part}, otherwise do \meta{false part}. The
starred version ignores the case, the unstarred version is
case sensitive.  For example:
\begin{verbatim}
\DTLifstringgt{aardvark}{zebra}{greater}{not greater}
\end{verbatim}
produces:
\DTLifstringgt{aardvark}{zebra}{greater}{not greater}.

Note that both \meta{string1} and \meta{string2} are considered to be
strings, so for example:
\begin{verbatim}
\DTLifstringgt{2}{10}{greater}{not greater}
\end{verbatim}
produces:
\DTLifstringgt{2}{10}{greater}{not greater}, since the string "2" 
comes after the string "10" when arranged alphabetically.

As with \cs{DTLifstringlt}, uppercase characters are considered
to be less than lower case characters when performing a
case sensitive comparison so:
\begin{verbatim}
\DTLifstringgt{B}{a}{greater}{not greater}
\end{verbatim}
produces:
\DTLifstringgt{B}{a}{greater}{not greater}, whereas
\begin{verbatim}
\DTLifstringgt*{B}{a}{greater}{not greater}
\end{verbatim}
produces:
\DTLifstringgt*{B}{a}{greater}{not greater}.

\begin{definition}[\DescribeMacro{\DTLifgt}]%
\cs{DTLifgt}\marg{arg1}\marg{arg2}\marg{true part}\marg{false part}
\end{definition}
\begin{definition}[\DescribeMacro{\DTLifgt*}]%
\cs{DTLifgt*}\marg{arg1}\marg{arg2}\marg{true part}\marg{false part}
\end{definition}
If \meta{arg1} and \meta{arg2} are both numerical, then this
is equivalent to \cs{DTLifnumgt}, otherwise it is equivalent
to \cs{DTLstringgt} (when using \cs{DTLifgt}) or
\cs{DTLstringgt*} (when using \cs{DTLifgt*}).

\begin{definition}[\DescribeMacro{\DTLifnumclosedbetween}]%
\cs{DTLifnumclosedbetween}\marg{num}\marg{min}\marg{max}\marg{true part}\marg{false part}
\end{definition}
If \meta{min} $\leq$ \meta{num} $\leq$ \meta{max}  then do \meta{true part},
otherwise do \meta{false part}. Note that \meta{num}, \meta{min} and
\meta{max} must be numerical (either integers, real numbers or
currency). The currency symbol is ignored when determining 
equality. For example:
\begin{verbatim}
\DTLifnumclosedbetween{5.4}{5}{7}{inside}{outside}
\end{verbatim}
produces:
\DTLifnumclosedbetween{5.4}{5}{7}{inside}{outside}.
Note that the closed range includes end points:
\begin{verbatim}
\DTLifnumclosedbetween{5}{5}{7}{inside}{outside}
\end{verbatim}
produces:
\DTLifnumclosedbetween{5}{5}{7}{inside}{outside}.

\begin{definition}[\DescribeMacro{\DTLifstringclosedbetween}]%
\cs{DTLifstringclosedbetween}\marg{string}\marg{min}\marg{max}\marg{true part}\marg{false part}
\end{definition}
\begin{definition}[\DescribeMacro{\DTLifstringclosedbetween*}]%
\cs{DTLifstringclosedbetween*}\marg{string}\marg{min}\marg{max}\marg{true part}\marg{false part}
\end{definition}
This determines if \meta{string} is between \meta{min} and
\meta{max} in the alphabetical sense, or is equal to either
\meta{min} or \meta{max}. The starred version ignores the case,
the unstarred version is case sensitive.

\begin{definition}[\DescribeMacro{\DTLifclosedbetween}]%
\cs{DTLifclosedbetween}\marg{arg}\marg{min}\marg{max}\marg{true part}\marg{false part}
\end{definition}
\begin{definition}[\DescribeMacro{\DTLifclosedbetween*}]%
\cs{DTLifclosedbetween*}\marg{arg}\marg{min}\marg{max}\marg{true part}\marg{false part}
\end{definition}
If \meta{arg}, \meta{min} and \meta{max} are numerical, then this is
equivalent to\newline
\cs{DTLifnumclosedbetween}\newline
otherwise it is equivalent to\newline
\cs{DTLifstringclosedbetween}\newline
(when using \cs{DTLifclosedbetween}) or\newline
\cs{DTLifstringclosedbetween*}\newline
(when using \cs{DTLifclosedbetween*}).

\begin{definition}[\DescribeMacro{\DTLifnumopenbetween}]%
\cs{DTLifnumopenbetween}\marg{num}\marg{min}\marg{max}\marg{true part}\marg{false part}
\end{definition}
If \meta{min} $<$ \meta{num} $<$ \meta{max}  then do \meta{true part},
otherwise do \meta{false part}. Note that \meta{num}, \meta{min} and
\meta{max} must be numerical (either integers, real numbers or
currency). Again, the currency symbol is ignored when determining 
equality. For example:
\begin{verbatim}
\DTLifnumopenbetween{5.4}{5}{7}{inside}{outside}
\end{verbatim}
produces:
\DTLifnumopenbetween{5.4}{5}{7}{inside}{outside}.
Note that end points are not included. For example:
\begin{verbatim}
\DTLifnumopenbetween{5}{5}{7}{inside}{outside}
\end{verbatim}
produces:
\DTLifnumopenbetween{5}{5}{7}{inside}{outside}.

\begin{definition}[\DescribeMacro{\DTLifstringopenbetween}]%
\cs{DTLifstringopenbetween}\marg{string}\marg{min}\marg{max}\marg{true part}\marg{false part}
\end{definition}
\begin{definition}[\DescribeMacro{\DTLifstringopenbetween*}]%
\cs{DTLifstringopenbetween*}\marg{string}\marg{min}\marg{max}\marg{true part}\marg{false part}
\end{definition}
This determines if \meta{string} is between \meta{min} and
\meta{max} in the alphabetical sense. 
The starred version ignores the case,
the unstarred version is case sensitive.

\begin{definition}[\DescribeMacro{\DTLifopenbetween}]%
\cs{DTLifopenbetween}\marg{arg}\marg{min}\marg{max}\marg{true part}\marg{false part}
\end{definition}
\begin{definition}[\DescribeMacro{\DTLifopenbetween*}]%
\cs{DTLifopenbetween*}\marg{arg}\marg{min}\marg{max}\marg{true part}\marg{false part}
\end{definition}
If \meta{arg}, \meta{min} and \meta{max} are numerical, then this is
equivalent to
\cs{DTLifnumopenbetween} 
otherwise it is equivalent to
\cs{DTLifstringopenbetween} 
(when using \cs{DTLifopenbetween}) or
\cs{DTLifstringopenbetween*}
(when using \cs{DTLifopenbetween*}).

\begin{definition}[\DescribeMacro{\DTLifFPclosedbetween}]%
\cs{DTLifFPclosedbetween}\marg{num}\marg{min}\marg{max}\marg{true part}\marg{false part}
\end{definition}
If \meta{min} $\leq$ \meta{num} $\leq$ \meta{max}  then do \meta{true part},
otherwise do \meta{false part} where \meta{num}, \meta{min}
and \meta{max} are all in standard fixed point notation (i.e.\
no number group separator, no currency symbols and a full stop as 
a decimal point).

\begin{definition}[\DescribeMacro{\DTLifFPopenbetween}]%
\cs{DTLifFPopenbetween}\marg{num}\marg{min}\marg{max}\marg{true part}\marg{false part}
\end{definition}
If \meta{min} $<$ \meta{num} $<$ \meta{max}  then do \meta{true part},
otherwise do \meta{false part} where \meta{num}, \meta{min}
and \meta{max} are all in standard fixed point notation (i.e.\
no number group separator, no currency symbols and a full stop as 
a decimal point).

\begin{definition}[\DescribeMacro{\DTLifAllUpperCase}]%
\cs{DTLifAllUpperCase}\marg{string}\marg{true part}\marg{false part}
\end{definition}
Tests if \meta{string} is all upper case. For example:
\begin{verbatim}
\DTLifAllUpperCase{WORD}{all upper}{not all upper}
\end{verbatim}
produces:
\DTLifAllUpperCase{WORD}{all upper}{not all upper},
whereas
\begin{verbatim}
\DTLifAllUpperCase{Word}{all upper}{not all upper}
\end{verbatim}
produces:
\DTLifAllUpperCase{Word}{all upper}{not all upper}.
Note also that:
\begin{verbatim}
\DTLifAllUpperCase{\MakeUppercase{word}}{all upper}{not all upper}
\end{verbatim}
also produces:
\DTLifAllUpperCase{\MakeUppercase{word}}{all upper}{not all upper}.
\cs{MakeTextUppercase} (defined in David Carlisle's \sty{textcase}
package) and \cs{uppercase} are also detected, otherwise, if
a command is encountered, the case of the command is considered.
For example:
\begin{verbatim}
\DTLifAllUpperCase{MAN{\OE}UVRE}{all upper}{not all upper}
\end{verbatim}
produces:
\DTLifAllUpperCase{MAN{\OE}UVRE}{all upper}{not all upper}.

\begin{definition}[\DescribeMacro{\DTLifAllLowerCase}]%
\cs{DTLifAllLowerCase}\marg{string}\marg{true part}\marg{false part}
\end{definition}
Tests if \meta{string} is all lower case. For example:
\begin{verbatim}
\DTLifAllLowerCase{word}{all lower}{not all lower}
\end{verbatim}
produces:
\DTLifAllLowerCase{word}{all lower}{not all lower},
whereas
\begin{verbatim}
\DTLifAllLowerCase{Word}{all lower}{not all lower}
\end{verbatim}
produces:
\DTLifAllLowerCase{Word}{all lower}{not all lower}.
Note also that:
\begin{verbatim}
\DTLifAllLowerCase{\MakeLowercase{WORD}}{all lower}{not all lower}
\end{verbatim}
also produces:
\DTLifAllLowerCase{\MakeLowercase{WORD}}{all lower}{not all lower}.
\cs{MakeTextLowercase} (defined in David Carlisle's \sty{textcase}
package) and \cs{lowercase} are also detected, otherwise, if
a command is encountered, the case of the command is considered.
For example:
\begin{verbatim}
\DTLifAllLowerCase{man{\oe}uvre}{all lower}{not all lower}
\end{verbatim}
produces:
\DTLifAllLowerCase{man{\oe}uvre}{all lower}{not all lower}.

\begin{definition}[\DescribeMacro{\DTLifSubString}]%
\cs{DTLifSubString}\marg{string}\marg{substring}\marg{true 
part}\marg{false part}
\end{definition}
This tests if \meta{substring} is a sub-string of \meta{string}.
This command performs a case sensitive match.  For example:
\begin{verbatim}
\DTLifSubString{An apple}{app}{is substring}{isn't substring}
\end{verbatim}
produces:
\DTLifSubString{An apple}{app}{is substring}{isn't substring}.
Note that spaces are considered to be equivalent to \cs{space}
or "~", so
\begin{verbatim}
\DTLifSubString{An apple}{n~a}{is substring}{isn't substring}
\end{verbatim}
produces:
\DTLifSubString{An apple}{n~a}{is substring}{isn't substring},
but other commands are skipped, so
\begin{verbatim}
\DTLifSubString{An \uppercase{a}pple}{app}{is substring}{isn't
substring}
\end{verbatim}
produces:
\DTLifSubString{An \uppercase{a}pple}{app}{is substring}{isn't
substring}, since the \cs{uppercase} command is ignored. Note also
that grouping is ignored, so:
\begin{verbatim}
\DTLifSubString{An {ap}ple}{app}{is substring}{isn't substring}
\end{verbatim}
produces:
\DTLifSubString{An {ap}ple}{app}{is substring}{isn't substring}.

\cs{DTLifSubString} is case sensitive, so:
\begin{verbatim}
\DTLifSubString{An Apple}{app}{is substring}{isn't substring}
\end{verbatim}
produces:
\DTLifSubString{An Apple}{app}{is substring}{isn't substring}.

\begin{definition}[\DescribeMacro{\DTLifStartsWith}]%
\cs{DTLifStartsWith}\marg{string}\marg{substring}\marg{true 
part}\marg{false part}
\end{definition}
This is like \cs{DTLifSubString}, except that \meta{substring} must
occur at the start of \meta{string}. This command performs a case 
sensitive match.  For example,
\begin{verbatim}
\DTLifStartsWith{An apple}{app}{prefix}{not a prefix}
\end{verbatim}
produces:
\DTLifStartsWith{An apple}{app}{prefix}{not a prefix}. All the
above remarks for \cs{DTLifSubString} also applies to 
\cs{DTLifStartsWith}. For example:
\begin{verbatim}
\DTLifStartsWith{\uppercase{a}n apple}{an~}{prefix}{not a prefix}
\end{verbatim}
produces:
\DTLifStartsWith{\uppercase{a}n apple}{an~}{prefix}{not a prefix},
since \cs{uppercase} is ignored, and "~" is considered to be the
same as a space, whereas
\begin{verbatim}
\DTLifStartsWith{An apple}{an~}{prefix}{not a prefix}
\end{verbatim}
produces:
\DTLifStartsWith{An apple}{an~}{prefix}{not a prefix}.

\section{\texorpdfstring{\sty{ifthen}}{ifthen} conditionals}
\label{sec:ifthen}

The commands described in the previous section can not be
used as the conditional part of the \cs{ifthenelse} or
\cs{whiledo} commands provided by the \sty{ifthen} package.
This section describes analogous commands which may only be
used in the conditional argument of \cs{ifthenelse} and
\cs{whiledo}. These may be used with the boolean operations
\cs{not}, \cs{and} and \cs{or} provided by the \sty{ifthen} package.
See the \sty{ifthen} documentation for further details.

\begin{definition}[\DescribeMacro{\DTLisstring}]%
\cs{DTLisstring}\marg{text}
\end{definition}
Tests if \meta{text} is a string. For example:
\begin{verbatim}
\ifthenelse{\DTLisstring{some text}}{string}{not a string}
\end{verbatim}
produces:
\ifthenelse{\DTLisstring{some text}}{string}{not a string}.

\begin{definition}[\DescribeMacro{\DTLisnumerical}]%
\cs{DTLisnumerical}\marg{text}
\end{definition}
Tests if \meta{text} is numerical (i.e.\ not a string). For example:
\begin{verbatim}
\ifthenelse{\DTLisnumerical{\$10.95}}{numerical}{not numerical}
\end{verbatim}
produces:
\ifthenelse{\DTLisnumerical{\$10.95}}{numerical}{not numerical}.

Note however that \cs{DTLisnumerical} requires more care than
\cs{DTLifnumerical} when used with some of the other currency
symbols. Consider:
\begin{verbatim}
\DTLifnumerical{\pounds10.95}{numerical}{not numerical}
\end{verbatim}
This produces:
\DTLifnumerical{\pounds10.95}{numerical}{not numerical}.
However
\begin{verbatim}
\ifthenelse{\DTLisnumerical{\pounds10.95}}{numerical}{not numerical}
\end{verbatim}
produces:
\ifthenelse{\DTLisnumerical{\pounds10.95}}{numerical}{not numerical}.
This is due to the expansion that occurs within \cs{ifthenelse}.
This can be prevented using \cs{noexpand}, for example:
\begin{verbatim}
\ifthenelse{\DTLisnumerical{\noexpand\pounds10.95}}{numerical}{not numerical}
\end{verbatim}
produces:
\ifthenelse{\DTLisnumerical{\noexpand\pounds10.95}}{numerical}{not numerical}.

Likewise:
\begin{verbatim}
\def\cost{\pounds10.95}%
\ifthenelse{\DTLisnumerical{\noexpand\cost}}{numerical}{not numerical}
\end{verbatim}
produces:
\def\cost{\pounds10.95}\relax
\ifthenelse{\DTLisnumerical{\noexpand\cost}}{numerical}{not numerical}.

\begin{definition}[\DescribeMacro{\DTLiscurrency}]%
\cs{DTLiscurrency}\marg{text}
\end{definition}
Tests if \meta{text} is currency. For example:
\begin{verbatim}
\ifthenelse{\DTLiscurrency{\$10.95}}{currency}{not currency}
\end{verbatim}
produces:
\ifthenelse{\DTLiscurrency{\$10.95}}{currency}{not currency}.

The same warning given above for \cs{DTLisnumerical} also applies
here.

\begin{definition}[\DescribeMacro{\DTLiscurrencyunit}]%
\cs{DTLiscurrencyunit}\marg{text}\marg{symbol}
\end{definition}
Tests if \meta{text} is currency and that currency uses \meta{symbol} as the unit 
of currency.
For example:
\begin{verbatim}
\ifthenelse{\DTLiscurrencyunit{\$6.99}{\$}}{dollars}{not dollars}
\end{verbatim}
produces:
\ifthenelse{\DTLiscurrencyunit{\$6.99}{\$}}{dollars}{not dollars}.
Another example:
\begin{verbatim}
\def\cost{\euro10.50}%
\ifthenelse{\DTLiscurrencyunit{\noexpand\cost}{\noexpand\euro}}%
{euros}{not euros}
\end{verbatim}
produces:
\def\cost{\euro10.50}\relax
\ifthenelse{\DTLiscurrencyunit{\noexpand\cost}{\noexpand\euro}}%
{euros}{not euros}. Again note the use of \cs{noexpand}.

\begin{definition}[\DescribeMacro{\DTLisreal}]%
\cs{DTLisreal}\marg{text}
\end{definition}
Tests if \meta{text} is a fixed point number (again, an integer is 
not considered to be a fixed point number). For example:
\begin{verbatim}
\ifthenelse{\DTLisreal{1.5}}{real}{not real}
\end{verbatim}
produces:
\ifthenelse{\DTLisreal{1.5}}{real}{not real}.

\begin{definition}[\DescribeMacro{\DTLisint}]%
\cs{DTLisint}\marg{text}
\end{definition}
Tests if \meta{text} is an integer. For example:
\begin{verbatim}
\ifthenelse{\DTLisint{153}}{integer}{not an integer}
\end{verbatim}
produces:
\ifthenelse{\DTLisint{153}}{integer}{not an integer}.

\begin{definition}[\DescribeMacro{\DTLislt}]%
\cs{DTLislt}\marg{arg1}\marg{arg2}
\end{definition}
This checks if \meta{arg1} is less than \meta{arg2}. As with
\cs{DTLiflt}, if \meta{arg1} and \meta{arg2} are numerical,
a numerical comparison is used, otherwise a case sensitive 
alphabetical comparison is used. (Note that there is no starred 
version of this command, but you can instead use \cs{DTLisilt}
to ignore the case.)

\begin{definition}[\DescribeMacro{\DTLisilt}]%
\cs{DTLisilt}\marg{arg1}\marg{arg2}
\end{definition}
This checks if \meta{arg1} is less than \meta{arg2}. As with
\cs{DTLiflt*}, if \meta{arg1} and \meta{arg2} are numerical,
a numerical comparison is used, otherwise a case insensitive 
alphabetical comparison is used.

\begin{definition}[\DescribeMacro{\DTLisgt}]%
\cs{DTLisgt}\marg{arg1}\marg{arg2}
\end{definition}
This checks if \meta{arg1} is greater than \meta{arg2}. As with
\cs{DTLifgt}, if \meta{arg1} and \meta{arg2} are numerical,
a numerical comparison is used, otherwise a case sensitive 
alphabetical comparison is used. (Note that there is no starred
version of this command, instead use \cs{DTLisigt} to 
ignore the case.)

\begin{definition}[\DescribeMacro{\DTLisigt}]%
\cs{DTLisigt}\marg{arg1}\marg{arg2}
\end{definition}
This checks if \meta{arg1} is greater than \meta{arg2}. As with
\cs{DTLifgt*}, if \meta{arg1} and \meta{arg2} are numerical,
a numerical comparison is used, otherwise a case insensitive 
alphabetical comparison is used.

\begin{definition}[\DescribeMacro{\DTLiseq}]%
\cs{DTLiseq}\marg{arg1}\marg{arg2}
\end{definition}
This checks if \meta{arg1} is equal to \meta{arg2}. As with
\cs{DTLifeq}, if \meta{arg1} and \meta{arg2} are numerical,
a numerical comparison is used, otherwise a case sensitive 
alphabetical comparison is used. (Note that there is no starred
version of this command, instead use \cs{DTLisieq}.)

\begin{definition}[\DescribeMacro{\DTLisieq}]%
\cs{DTLisieq}\marg{arg1}\marg{arg2}
\end{definition}
This checks if \meta{arg1} is equal to \meta{arg2}. As with
\cs{DTLifeq*}, if \meta{arg1} and \meta{arg2} are numerical,
a numerical comparison is used, otherwise a case insensitive 
alphabetical comparison is used.

\begin{definition}[\DescribeMacro{\DTLisclosedbetween}]%
\cs{DTLisclosedbetween}\marg{arg}\marg{min}\marg{max}
\end{definition}
This checks if \meta{arg} lies between \meta{min} and 
\meta{max} (end points included). As with
\cs{DTLifclosedbetween}, if the arguments are numerical,
a numerical comparison is used, otherwise a case sensitive 
alphabetical comparison is used. (Note that there is no starred
version of this command, instead use \cs{DTLisiclosedbetween}.)

\begin{definition}[\DescribeMacro{\DTLisiclosedbetween}]%
\cs{DTLisiclosedbetween}\marg{arg}\marg{min}\marg{max}
\end{definition}
This checks if \meta{arg} lies between \meta{min} and 
\meta{max} (end points included). As with
\cs{DTLifclosedbetween*}, if the arguments are numerical,
a numerical comparison is used, otherwise a case insensitive 
alphabetical comparison is used.

\begin{definition}[\DescribeMacro{\DTLisopenbetween}]%
\cs{DTLisopenbetween}\marg{arg}\marg{min}\marg{max}
\end{definition}
This checks if \meta{arg} lies between \meta{min} and 
\meta{max} (end points excluded). As with
\cs{DTLifopenbetween}, if the arguments are numerical,
a numerical comparison is used, otherwise a case sensitive 
alphabetical comparison is used. (Note that there is no starred
version of this command, instead use \cs{DTLisiopenbetween}.)

\begin{definition}[\DescribeMacro{\DTLisiopenbetween}]%
\cs{DTLisiopenbetween}\marg{arg}\marg{min}\marg{max}
\end{definition}
This checks if \meta{arg} lies between \meta{min} and 
\meta{max} (end points excluded). As with
\cs{DTLifopenbetween*}, if the arguments are numerical,
a numerical comparison is used, otherwise a case insensitive 
alphabetical comparison is used.

\begin{definition}[\DescribeMacro{\DTLisFPlt}]%
\cs{DTLisFPlt}\marg{num1}\marg{num2}
\end{definition}
This checks if \meta{num1} is less than \meta{num2}, where both
numbers are in standard fixed point format (i.e.\ no number group
separators, no currency and a full stop as a decimal point).

\begin{definition}[\DescribeMacro{\DTLisFPlteq}]%
\cs{DTLisFPlteq}\marg{num1}\marg{num2}
\end{definition}
This checks if \meta{num1} is less than or equal to \meta{num2}, where both
numbers are in standard fixed point format (i.e.\ no number group
separators, no currency and a full stop as a decimal point).

\begin{definition}[\DescribeMacro{\DTLisFPgt}]%
\cs{DTLisFPgt}\marg{num1}\marg{num2}
\end{definition}
This checks if \meta{num1} is greater than \meta{num2}, where both
numbers are in standard fixed point format (i.e.\ no number group
separators, no currency and a full stop as a decimal point).

\begin{definition}[\DescribeMacro{\DTLisFPgteq}]%
\cs{DTLisFPgteq}\marg{num1}\marg{num2}
\end{definition}
This checks if \meta{num1} is greater than or equal to \meta{num2}, where both
numbers are in standard fixed point format (i.e.\ no number group
separators, no currency and a full stop as a decimal point).

\begin{definition}[\DescribeMacro{\DTLisFPeq}]%
\cs{DTLisFPeq}\marg{num1}\marg{num2}
\end{definition}
This checks if \meta{num1} is equal to \meta{num2}, where both
numbers are in standard fixed point format (i.e.\ no number group
separators, no currency and a full stop as a decimal point).

\begin{definition}[\DescribeMacro{\DTLisFPclosedbetween}]%
\cs{DTLisFPclosedbetween}\marg{num}\marg{min}\marg{max}
\end{definition}
This checks if \meta{num} lies between \meta{min} and 
\meta{max} (end points included). All arguments must be
numbers in standard fixed point format (i.e.\ no number group
separators, no currency and a full stop as a decimal point).

\begin{definition}[\DescribeMacro{\DTLisFPopenbetween}]%
\cs{DTLisFPopenbetween}\marg{num}\marg{min}\marg{max}
\end{definition}
This checks if \meta{num} lies between \meta{min} and 
\meta{max} (end points excluded). All arguments must be
numbers in standard fixed point format (i.e.\ no number group
separators, no currency and a full stop as a decimal point).

\begin{definition}[\DescribeMacro{\DTLisSubString}]%
\cs{DTLisSubString}\marg{string}\marg{substring}
\end{definition}
This checks if \meta{substring} is contained in \meta{string}.
The remarks about \cs{DTLifSubString} also apply to 
\cs{DTLisSubString}. This command performs a case sensitive
match.

\begin{definition}[\DescribeMacro{\DTLisPrefix}]%
\cs{DTLisPrefix}\marg{string}\marg{prefix}
\end{definition}
This checks if \meta{string} starts with \meta{prefix}.
The remarks about \cs{DTLifStartsWith} also apply to 
\cs{DTLisPrefix}. This command performs a case sensitive
match.

\begin{definition}[\DescribeMacro{\DTLisinlist}]%
\cs{DTLisinlist}\marg{element}\marg{list}
\end{definition}
This checks if \meta{element} is in \meta{list}. (Internally uses
\cs{DTLifinlist}.)

\chapter{Fixed Point Arithmetic}
\label{sec:fp}

\begin{important}
The \sty{datatool} bundle doesn't support scientific
notation.
\end{important}

The \sty{datatool-base} package uses either the \sty{fp} or the \sty{pgfmath} package to perform
fixed point arithmetic, however all numbers must be converted
from the locale dependent format into the format required by the
\sty{fp} or \sty{pgfmath} packages. A numerical value (i.e.\ an integer, a real
or currency) can be converted into a plain decimal number using
\begin{definition}[\DescribeMacro{\DTLconverttodecimal}]%
\cs{DTLconverttodecimal}\marg{num}\marg{cmd}
\end{definition}
The decimal number will be stored in \meta{cmd} which must be
a control sequence. For example:
\begin{verbatim}
\DTLconverttodecimal{1,563.54}{\mynum}
\end{verbatim}
\DTLconverttodecimal{1,563.54}{\mynum}\relax
will define \cs{mynum} to be \texttt{\mynum}. The command \cs{mynum}
can then be used in any of the arithmetic macros provided by the
\sty{fp} or \sty{pgfmath} packages.

\begin{important}
The arguments to \cs{DTLconverttodecimal} don't get fully expanded
so, for example,
\begin{verbatim}
\def\myval{1.23}
\DTLconverttodecimal{\myval}{\mynum}
\end{verbatim}
will work, but the following \emph{\bfseries won't} work:
\begin{verbatim}
\def\myval{1.23}
\def\myotherval{\myval}
\DTLconverttodecimal{\myotherval}{\mynum}
\end{verbatim}
Nor will the following work:
\begin{verbatim}
\def\myval{9}
\DTLconverttodecimal{\myval 9}{\mynum}
\end{verbatim}
\end{important}
There are two commands provided to perform
the reverse:
\begin{definition}[\DescribeMacro{\DTLdecimaltolocale}]%
\cs{DTLdecimaltolocale}\marg{number}\marg{cmd}
\end{definition}
This converts a plain decimal number \meta{number} (that uses a full
stop as the decimal character and has no number group characters)
into a locale dependent format. The resulting number is stored
in \meta{cmd}, which must be a control sequence. For example:
\begin{verbatim}
\DTLdecimaltolocale{6795.3}{\mynum}
\end{verbatim}
\DTLdecimaltolocale{6795.3}{\mynum}
will define \cs{mynum} to be \texttt{\mynum}.

\begin{definition}[\DescribeMacro{\DTLdecimaltocurrency}]%
\cs{DTLdecimaltocurrency}\marg{number}\marg{cmd}
\end{definition}
This will convert a plain decimal number \meta{number} into a
locale dependent currency format. For example:
\begin{verbatim}
\DTLdecimaltocurrency{267.5}{\price}\price
\end{verbatim}
will produce:
\DTLdecimaltocurrency{267.5}{\price}\price.

The currency symbol used by \cs{DTLdecimaltocurrency} is 
initially "\$", but it will use the currency last encountered.
So, for example
\begin{verbatim}
\DTLifcurrency{\texteuro45.00}{}{}%
\DTLdecimaltocurrency{267.5}{\price}\price
\end{verbatim}
will produce:
\DTLifcurrency{\texteuro45.00}{}{}\relax
\DTLdecimaltocurrency{267.5}{\price}\price. This is because
the last currency symbol to be encountered was \cs{texteuro}.
You can reset the currency symbol using the command:
\begin{definition}[\DescribeMacro{\DTLsetdefaultcurrency}]%
\cs{DTLsetdefaultcurrency}\marg{symbol}
\end{definition}
For example:
\begin{verbatim}
\DTLsetdefaultcurrency{\textyen}%
\DTLdecimaltocurrency{267.5}{\price}\price
\end{verbatim}
will produce:
\DTLsetdefaultcurrency{\textyen}\relax
\DTLdecimaltocurrency{267.5}{\price}\price

The \sty{datatool-base} package provides convenience commands which
use \cs{DTLconverttodecimal}, and then use the basic macros provided
by the \sty{fp}\slash\sty{pgfmath} package. The resulting value is then converted
back into the locale format using
\cs{DTLdecimaltolocale} or \cs{DTLdecimaltocurrency}.
Note that since these commands use \cs{DTLconverttodecimal} the
caveat above regarding expansion also applies to all the commands.

\begin{important}
If you don't require currency or locale conversion, you can reduce
the package overheads by using the commands defined in the
\sty{datatool-fp} or \sty{datatool-pgfmath} packages which provide
interface commands to \sty{fp} or \sty{pgfmath}, respectively.
(See sections~\ref*{sec:code:datatool-fp}
and~\ref*{sec:code:datatool-pgfmath} of the documented code,
\texttt{datatool-code.pdf}.)
Alternatively, you can just use the \sty{fp} or \sty{pgfmath}
commands explicitly. (See the \sty{fp} or \sty{pgf} manuals for
further details.)
\end{important}


\begin{definition}[\DescribeMacro{\DTLadd}]%
\cs{DTLadd}\marg{cmd}\marg{num1}\marg{num2}
\end{definition}
\begin{definition}[\DescribeMacro{\DTLgadd}]%
\cs{DTLgadd}\marg{cmd}\marg{num1}\marg{num2}
\end{definition}
This sets the control sequence \meta{cmd} to \meta{num1}+\meta{num2}.
\cs{DLTadd} sets \meta{cmd} locally, while \cs{DTLgadd} sets
\meta{cmd} globally.

For example:
\begin{verbatim}
\DTLadd{\result}{3,562.65}{412.2}\result
\end{verbatim}
will produce:
\DTLadd{\result}{3,562.65}{412.2}\result. Since 
\cs{DTLconverttodecimal} can convert currency to a real
number, you can also add prices. For example:
\begin{verbatim}
\DTLadd{\result}{\pounds3,562.65}{\pounds452.2}\result
\end{verbatim}
produces:
\DTLadd{\result}{\pounds3,562.65}{\pounds452.2}\result.

Note that \sty{datatool} isn't aware of exchange rates! If you
use different currency symbols, the last symbol will be used.
For example
\begin{verbatim}
\DTLadd{\result}{\pounds3,562.65}{\euro452.2}\result
\end{verbatim}
produces:
\DTLadd{\result}{\pounds3,562.65}{\euro452.2}\result.

Likewise, if one value is a number and the other is a currency,
the type of the last value, \meta{num2}, will be used for the
result. For example:
\begin{verbatim}
\DTLadd{\result}{3,562.65}{\$452.2}\result
\end{verbatim}
produces:
\DTLadd{\result}{3,562.65}{\$452.2}\result.

\begin{definition}[\DescribeMacro{\DTLaddall}]%
\cs{DTLaddall}\marg{cmd}\marg{number list}
\end{definition}
\begin{definition}[\DescribeMacro{\DTLgaddall}]%
\cs{DTLgaddall}\marg{cmd}\marg{number list}
\end{definition}
This sets the control sequence \meta{cmd} to the sum of all
the numbers in \meta{number list}.
\cs{DLTaddall} sets \meta{cmd} locally, while \cs{DTLgaddall} sets
\meta{cmd} globally. Example:
\begin{verbatim}
\DTLaddall{\total}{25.1,45.2,35.6}\total
\end{verbatim}
produces:
\DTLaddall{\total}{25.1,45.2,35.6}\total.
Note that if any of the numbers in \meta{number list} contain
a comma, you must group the number. Example:
\begin{verbatim}
\DTLaddall{\total}{{1,525},{2,340},500}\total
\end{verbatim}
produces:
\DTLaddall{\total}{{1,525},{2,340},500}\total.

\begin{definition}[\DescribeMacro{\DTLsub}]%
\cs{DTLsub}\marg{cmd}\marg{num1}\marg{num2}
\end{definition}
\begin{definition}[\DescribeMacro{\DTLgsub}]%
\cs{DTLgsub}\marg{cmd}\marg{num1}\marg{num2}
\end{definition}
This sets the control sequence \meta{cmd} to 
\meta{num1}$-$\meta{num2}.
\cs{DLTsub} sets \meta{cmd} locally, while \cs{DTLgsub} sets
\meta{cmd} globally.

For example:
\begin{verbatim}
\DTLsub{\result}{3,562.65}{412.2}\result
\end{verbatim}
will produce:
\DTLsub{\result}{3,562.65}{412.2}\result. As with \cs{DTLadd},
\meta{num1} and \meta{num2} may be currency.

\begin{definition}[\DescribeMacro{\DTLmul}]%
\cs{DTLmul}\marg{cmd}\marg{num1}\marg{num2}
\end{definition}
\begin{definition}[\DescribeMacro{\DTLgmul}]%
\cs{DTLgmul}\marg{cmd}\marg{num1}\marg{num2}
\end{definition}
This sets the control sequence \meta{cmd} to 
\meta{num1}$\times$\meta{num2}.
\cs{DLTmul} sets \meta{cmd} locally, while \cs{DTLgmul} sets
\meta{cmd} globally.

For example:
\begin{verbatim}
\DTLmul{\result}{568.95}{2}\result
\end{verbatim}
will produce:
\DTLmul{\result}{568.95}{2}\result. Again, \meta{num1} or 
\meta{num2} may be currency, but unlike \cs{DTLadd} and \cs{DTLsub},
currency overrides integer/real. For example:
\begin{verbatim}
\DTLmul{\result}{\pounds568.95}{2}\result
\end{verbatim}
will produce:
\DTLmul{\result}{\pounds568.95}{2}\result. Likewise,
\begin{verbatim}
\DTLmul{\result}{2}{\pounds568.95}\result
\end{verbatim}
will produce:
\DTLmul{\result}{2}{\pounds568.95}\result. Although it doesn't make
sense to multiply two currencies, \sty{datatool} will allow
\begin{verbatim}
\DTLmul{\result}{\$2}{\pounds568.95}\result
\end{verbatim}
which will produce:
\DTLmul{\result}{\$2}{\pounds568.95}\result.

\begin{definition}[\DescribeMacro{\DTLdiv}]%
\cs{DTLdiv}\marg{cmd}\marg{num1}\marg{num2}
\end{definition}
\begin{definition}[\DescribeMacro{\DTLgdiv}]%
\cs{DTLgdiv}\marg{cmd}\marg{num1}\marg{num2}
\end{definition}
This sets the control sequence \meta{cmd} to 
\meta{num1}$\div$\meta{num2}.
\cs{DLTdiv} sets \meta{cmd} locally, while \cs{DTLgdiv} sets
\meta{cmd} globally.

For example:
\begin{verbatim}
\DTLdiv{\result}{501}{2}\result
\end{verbatim}
will produce:
\DTLdiv{\result}{501}{2}\result. Again, \meta{num1} or \meta{num2}
may be currency, but the resulting type will be not be a currency
if both \meta{num1} and \meta{num2} use the same currency symbol.
For example:
\begin{verbatim}
\DTLdiv{\result}{\$501}{\$2}\result
\end{verbatim}
will produce:
\DTLdiv{\result}{\$501}{\$2}\result. Whereas
\begin{verbatim}
\DTLdiv{\result}{\$501}{2}\result
\end{verbatim}
will produce:
\DTLdiv{\result}{\$501}{2}\result.

\begin{definition}[\DescribeMacro{\DTLabs}]%
\cs{DTLabs}\marg{cmd}\marg{num}
\end{definition}
\begin{definition}[\DescribeMacro{\DTLgabs}]%
\cs{DTLgabs}\marg{cmd}\marg{num}
\end{definition}
This sets \meta{cmd} to the absolute value of \meta{num}.
\cs{DLTabs} sets \meta{cmd} locally, while \cs{DTLgabs} sets
\meta{cmd} globally. Example:
\begin{verbatim}
\DTLabs{\result}{-\pounds2.50}\result
\end{verbatim}
produces:
\DTLabs{\result}{-\pounds2.50}\result.

\begin{definition}[\DescribeMacro{\DTLneg}]%
\cs{DTLneg}\marg{cmd}\marg{num}
\end{definition}
\begin{definition}[\DescribeMacro{\DTLgneg}]%
\cs{DTLgneg}\marg{cmd}\marg{num}
\end{definition}
This sets \meta{cmd} to the negative of \meta{num}.
\cs{DLTneg} sets \meta{cmd} locally, while \cs{DTLgneg} sets
\meta{cmd} globally. Example:
\begin{verbatim}
\DTLneg{\result}{\pounds2.50}\result
\end{verbatim}
produces:
\DTLneg{\result}{\pounds2.50}\result.

\begin{definition}[\DescribeMacro{\DTLsqrt}]%
\cs{DTLsqrt}\marg{cmd}\marg{num}
\end{definition}
\begin{definition}[\DescribeMacro{\DTLgsqrt}]%
\cs{DTLgsqrt}\marg{cmd}\marg{num}
\end{definition}
This sets \meta{cmd} to the sqrt root of \meta{num}.
\cs{DLTsqrt} sets \meta{cmd} locally, while \cs{DTLgsqrt} sets
\meta{cmd} globally. Example:
\begin{verbatim}
\DTLsqrt{\result}{2}\result
\end{verbatim}
produces:
\DTLsqrt{\result}{2}\result.

\begin{definition}[\DescribeMacro{\DTLmin}]%
\cs{DTLmin}\marg{cmd}\marg{num1}\marg{num2}
\end{definition}
\begin{definition}[\DescribeMacro{\DTLgmin}]%
\cs{DTLgmin}\marg{cmd}\marg{num1}\marg{num2}
\end{definition}
This sets the control sequence \meta{cmd} to the minimum of
\meta{num1} and \meta{num2}.
\cs{DLTmin} sets \meta{cmd} locally, while \cs{DTLgmin} sets
\meta{cmd} globally. For example:
\begin{verbatim}
\DTLmin{\result}{256}{32}\result
\end{verbatim}
produces:
\DTLmin{\result}{256}{32}\result.
Again, \meta{num1} and \meta{num2} may
be currency. For example:
\begin{verbatim}
\DTLmin{\result}{256}{\pounds32}\result
\end{verbatim}
produces:
\DTLmin{\result}{256}{\pounds32}\result, whereas
\begin{verbatim}
\DTLmin{\result}{\pounds256}{32}\result
\end{verbatim}
produces:
\DTLmin{\result}{\pounds256}{32}\result. As mentioned above,
\sty{datatool} doesn't know about exchange rates, so be careful
about mixing currencies. For example:
\begin{verbatim}
\DTLmin{\result}{\pounds5}{\$6}\result
\end{verbatim}
produces:
\DTLmin{\result}{\pounds5}{\$6}\result, which may not necessarily
be true!

\begin{definition}[\DescribeMacro{\DTLminall}]%
\cs{DTLminall}\marg{cmd}\marg{number list}
\end{definition}
\begin{definition}[\DescribeMacro{\DTLgminall}]%
\cs{DTLgminall}\marg{cmd}\marg{number list}
\end{definition}
This sets the control sequence \meta{cmd} to the minimum of all
the numbers in \meta{number list}.
\cs{DLTminall} sets \meta{cmd} locally, while \cs{DTLgminall} sets
\meta{cmd} globally. Example:
\begin{verbatim}
\DTLminall{\theMin}{25.1,45.2,35.6}\theMin
\end{verbatim}
produces:
\DTLminall{\theMin}{25.1,45.2,35.6}\theMin.
Note that if any of the numbers in \meta{number list} contain
a comma, you must group the number. Example:
\begin{verbatim}
\DTLminall{\theMin}{{1,525},{2,340},500}\theMin
\end{verbatim}
produces:
\DTLminall{\theMin}{{1,525},{2,340},500}\theMin.

\begin{definition}[\DescribeMacro{\DTLmax}]%
\cs{DTLmax}\marg{cmd}\marg{num1}\marg{num2}
\end{definition}
\begin{definition}[\DescribeMacro{\DTLgmax}]%
\cs{DTLgmax}\marg{cmd}\marg{num1}\marg{num2}
\end{definition}
This sets the control sequence \meta{cmd} to the maximum of
\meta{num1} and \meta{num2}.
\cs{DLTmax} sets \meta{cmd} locally, while \cs{DTLgmax} sets
\meta{cmd} globally. For example:
\begin{verbatim}
\DTLmax{\result}{256}{32}\result
\end{verbatim}
produces:
\DTLmax{\result}{256}{32}\result.
Again, \meta{num1} and \meta{num2} may
be currency, but the same warnings for \cs{DTLmin} apply.

\begin{definition}[\DescribeMacro{\DTLmaxall}]%
\cs{DTLmaxall}\marg{cmd}\marg{number list}
\end{definition}
\begin{definition}[\DescribeMacro{\DTLgmaxall}]%
\cs{DTLgmaxall}\marg{cmd}\marg{number list}
\end{definition}
This sets the control sequence \meta{cmd} to the maximum of all
the numbers in \meta{number list}.
\cs{DLTmaxall} sets \meta{cmd} locally, while \cs{DTLgmaxall} sets
\meta{cmd} globally. Example:
\begin{verbatim}
\DTLmaxall{\theMax}{25.1,45.2,35.6}\theMax
\end{verbatim}
produces:
\DTLmaxall{\theMax}{25.1,45.2,35.6}\theMax.
Note that if any of the numbers in \meta{number list} contain
a comma, you must group the number. Example:
\begin{verbatim}
\DTLmaxall{\theMax}{{1,525},{2,340},500}\theMax
\end{verbatim}
produces:
\DTLmaxall{\theMax}{{1,525},{2,340},500}\theMax.

\begin{definition}[\DescribeMacro{\DTLmeanforall}]%
\cs{DTLmeanforall}\marg{cmd}\marg{number list}
\end{definition}
\begin{definition}[\DescribeMacro{\DTLgmeanall}]%
\cs{DTLgmeanforall}\marg{cmd}\marg{number list}
\end{definition}
This sets the control sequence \meta{cmd} to the arithmetic mean of all
the numbers in \meta{number list}.
\cs{DLTmeanforall} sets \meta{cmd} locally, while \cs{DTLgmeanforall} sets
\meta{cmd} globally. Example:
\begin{verbatim}
\DTLmeanforall{\theMean}{25.1,45.2,35.6}\theMean
\end{verbatim}
produces:
\DTLmeanforall{\theMean}{25.1,45.2,35.6}\theMean.
Note that if any of the numbers in \meta{number list} contain
a comma, you must group the number. Example:
\begin{verbatim}
\DTLmeanforall{\theMean}{{1,525},{2,340},500}\theMean
\end{verbatim}
produces:
\DTLmeanforall{\theMean}{{1,525},{2,340},500}\theMean.

\begin{definition}[\DescribeMacro{\DTLvarianceforall}]%
\cs{DTLvarianceforall}\marg{cmd}\marg{number list}
\end{definition}
\begin{definition}[\DescribeMacro{\DTLgvarianceforall}]%
\cs{DTLgvarianceforall}\marg{cmd}\marg{number list}
\end{definition}
This sets the control sequence \meta{cmd} to the variance of all
the numbers in \meta{number list}.
\cs{DLTvarianceforall} sets \meta{cmd} locally, while \cs{DTLgvarianceforall} sets
\meta{cmd} globally. Example:
\begin{verbatim}
\DTLvarianceforall{\theVar}{25.1,45.2,35.6}\theVar
\end{verbatim}
produces:
\DTLvarianceforall{\theVar}{25.1,45.2,35.6}\theVar.
Again note that if any of the numbers in \meta{number list} 
contain a comma, you must group the number.

\begin{definition}[\DescribeMacro{\DTLsdforall}]%
\cs{DTLsdforall}\marg{cmd}\marg{number list}
\end{definition}
\pagebreak
\begin{definition}[\DescribeMacro{\DTLgsdforall}]%
\cs{DTLgsdforall}\marg{cmd}\marg{number list}
\end{definition}
This sets the control sequence \meta{cmd} to the standard deviation of all
the numbers in \meta{number list}.
\cs{DLTsdforall} sets \meta{cmd} locally, while \cs{DTLgsdforall} sets
\meta{cmd} globally. Example:
\begin{verbatim}
\DTLsdforall{\theSD}{25.1,45.2,35.6}\theSD
\end{verbatim}
produces:
\DTLsdforall{\theSD}{25.1,45.2,35.6}\theSD.
Note that if any of the numbers in \meta{number list} contain
a comma, you must group the number. Example:
\begin{verbatim}
\DTLsdforall{\theSD}{{1,525},{2,340},500}\theSD
\end{verbatim}
produces:
\DTLsdforall{\theSD}{{1,525},{2,340},500}\theSD.

\begin{definition}[\DescribeMacro{\DTLround}]%
\cs{DTLround}\marg{cmd}\marg{num}\marg{num digits}
\end{definition}
\begin{definition}[\DescribeMacro{\DTLground}]%
\cs{DTLground}\marg{cmd}\marg{num}\marg{num digits}
\end{definition}
This sets \meta{cmd} to \meta{num} rounded to \meta{num digits}
after the decimal character.
\cs{DLTround} sets \meta{cmd} locally, while \cs{DTLground} sets
\meta{cmd} globally. Example:
\begin{verbatim}
\DTLround{\result}{3.135276}{2}\result
\end{verbatim}
produces: \DTLround{\result}{3.135276}{2}\result.

\begin{definition}[\DescribeMacro{\DTLtrunc}]%
\cs{DTLtrunc}\marg{cmd}\marg{num}\marg{num digits}
\end{definition}
\begin{definition}[\DescribeMacro{\DTLgtrunc}]%
\cs{DTLgtrunc}\marg{cmd}\marg{num}\marg{num digits}
\end{definition}
This sets \meta{cmd} to \meta{num} truncated to \meta{num digits}
after the decimal character.
\cs{DLTtrunc} sets \meta{cmd} locally, while \cs{DTLgtrunc} sets
\meta{cmd} globally. Example:
\begin{verbatim}
\DTLtrunc{\result}{3.135276}{2}\result
\end{verbatim}
produces: \DTLtrunc{\result}{3.135276}{2}\result.

\pagebreak
\begin{definition}[\DescribeMacro{\DTLclip}]%
\cs{DTLclip}\marg{cmd}\marg{num}
\end{definition}
\begin{definition}[\DescribeMacro{\DTLgclip}]%
\cs{DTLgclip}\marg{cmd}\marg{num}
\end{definition}
This sets \meta{cmd} to \meta{num} with all unnecessary 0's
removed.
\cs{DLTclip} sets \meta{cmd} locally, while \cs{DTLgclip} sets
\meta{cmd} globally.

\chapter{Strings}
\label{sec:strings}

Strings are considered to be anything non-numerical. The 
\sty{datatool} package loads the \sty{substr} package, so
you can use the commands defined in that package to determine
if one string is contained in another string. In addition,
the \sty{datatool} provides the following macros:

\begin{definition}[\DescribeMacro{\DTLsubstitute}]%
\cs{DTLsubstitute}\marg{cmd}\marg{original}\marg{replacement}
\end{definition}
This replaces the first occurrence of \meta{original} in
\meta{cmd} with \meta{replacement}. Note that \meta{cmd} must
be the name of a command. For example:
\begin{verbatim}
\def\mystr{abcdce}\DTLsubstitute{\mystr}{c}{z}\mystr
\end{verbatim}
produces:
\def\mystr{abcdce}\DTLsubstitute{\mystr}{c}{z}\mystr.

\begin{definition}[\DescribeMacro{\DTLsubstituteall}]%
\cs{DTLsubstituteall}\marg{cmd}\marg{original}\marg{replacement}
\end{definition}
This replaces all occurrences of \meta{original} in
\meta{cmd} with \meta{replacement}, where again, \meta{cmd} must
be the name of a command. For example:
\begin{verbatim}
\def\mystr{abcdce}\DTLsubstituteall{\mystr}{c}{z}\mystr
\end{verbatim}
produces:
\def\mystr{abcdce}\DTLsubstituteall{\mystr}{c}{z}\mystr.

\begin{definition}[\DescribeMacro{\DTLsplitstring}]%
\cs{DTLsplitstring}\marg{string}\marg{split text}\marg{before 
cmd}\marg{after cmd}
\end{definition}
This splits \meta{string} at the first occurrence of \meta{split text}
and stores the before part in the command \meta{before cmd} and
the after part in the command \meta{after cmd}. For example:
\begin{verbatim}
\DTLsplitstring{abcdce}{c}{\beforepart}{\afterpart}%
Before part: ``\beforepart''. After part: ``\afterpart''
\end{verbatim}
produces:
\DTLsplitstring{abcdce}{c}{\beforepart}{\afterpart}\relax
Before part: ``\beforepart''. After part: ``\afterpart''.
Note that for \cs{DTLsplitstring}, \meta{string} is not
expanded, so
\begin{verbatim}
\def\mystr{abcdce}%
\DTLsplitstring{\mystr}{c}{\beforepart}{\afterpart}%
Before part: ``\beforepart''. After part: ``\afterpart''
\end{verbatim}
produces:
\def\mystr{abcdce}\relax
\DTLsplitstring{\mystr}{c}{\beforepart}{\afterpart}%
Before part: ``\beforepart''. After part: ``\afterpart''. If you
want the string expanded, you will need to use \cs{expandafter}:
\begin{verbatim}
\def\mystr{abcdce}%
\expandafter\DTLsplitstring\expandafter
{\mystr}{c}{\beforepart}{\afterpart}%
Before part: ``\beforepart''. After part: ``\afterpart''
\end{verbatim}
which produces:
\def\mystr{abcdce}\relax
\expandafter\DTLsplitstring\expandafter
{\mystr}{c}{\beforepart}{\afterpart}\relax
Before part: ``\beforepart''. After part: ``\afterpart''.

\begin{definition}[\DescribeMacro{\DTLinitials}]%
\cs{DTLinitials}\marg{string}
\end{definition}
This converts \meta{string} (typically a name) into initials.
For example:
\begin{verbatim}
\DTLinitials{Mary Ann}
\end{verbatim}
produces:
\DTLinitials{Mary Ann} (including the final full stop). Note that
\begin{verbatim}
\DTLinitials{Mary-Ann}
\end{verbatim}
produces:
\DTLinitials{Mary-Ann} (including the final full stop). 
Be careful if the initial letter has an
accent. The accented letter needs to be placed in a group, if
you want the initial to also have an accent, otherwise the
accent command will be ignored. For example:
\begin{verbatim}
\DTLinitials{{\'E}lise Adams}
\end{verbatim}
produces:
\DTLinitials{{\'E}lise Adams}, whereas
\begin{verbatim}
\DTLinitials{\'Elise Adams}
\end{verbatim}
produces:
\DTLinitials{\'Elise Adams} In fact, any command which appears
at the start of the name that is not enclosed in a group will
be ignored. For example:
\begin{verbatim}
\DTLinitials{\MakeUppercase{m}ary ann}
\end{verbatim}
produces:
\DTLinitials{\MakeUppercase{m}ary ann}, whereas
\begin{verbatim}
\DTLinitials{{\MakeUppercase{m}}ary ann}
\end{verbatim}
produces:
\DTLinitials{{\MakeUppercase{m}}ary ann}, but note that
\begin{verbatim}
\DTLinitials{\MakeUppercase{mary ann}}
\end{verbatim}
produces:
\DTLinitials{\MakeUppercase{mary ann}}

\begin{definition}[\DescribeMacro{\DTLstoreinitials}]%
\cs{DTLstoreinitials}\marg{string}\marg{cmd}
\end{definition}
This converts \meta{string} into initials and stores the
result in \meta{cmd} which must be a command name. The remarks 
about \cs{DTLinitials} also relate to \cs{DTLstoreinitials}.
For example
\begin{verbatim}
\DTLstoreinitials{Marie-{\'E}lise del~Rosario}{\theInitials}\theInitials
\end{verbatim}
produces:
\DTLstoreinitials{Marie-{\'E}lise del~Rosario}{\theInitials}\theInitials

Both the above commands rely on the following to format the
initials:
\begin{definition}[\DescribeMacro{\DTLafterinitials}]%
\cs{DTLafterinitials}
\end{definition}
This indicates what to do at the end of the initials. This
simply does a full stop by default.

\begin{definition}[\DescribeMacro{\DTLbetweeninitials}]%
\cs{DTLbetweeninitials}
\end{definition}
This indicates what to do between initials. This does a 
full stop by default.

\begin{definition}[\DescribeMacro{\DTLinitialhyphen}]%
\cs{DTLinitialhyphen}
\end{definition}
This indicates what to do at a hyphen. This simply does a hyphen
by default, but can be redefined to do nothing to prevent the
hyphen appearing in the initials.

\begin{definition}[\DescribeMacro{\DTLafterinitialbeforehyphen}]%
\cs{DTLafterinitialbeforehyphen}
\end{definition}
This indicates what to do between an initial and a hyphen.
This simply does a full stop by default.

For example
\begin{verbatim}
\renewcommand*{\DTLafterinitialbeforehyphen}{}%
\DTLinitials{Marie-{\'E}lise del~Rosario}
\end{verbatim}
produces:
{\renewcommand*{\DTLafterinitialbeforehyphen}{}\relax
\DTLinitials{Marie-{\'E}lise del~Rosario}}
whereas
\begin{verbatim}
\renewcommand*{\DTLafterinitialbeforehyphen}{}%
\renewcommand*{\DTLafterinitials}{}%
\renewcommand*{\DTLbetweeninitials}{}%
\renewcommand*{\DTLinitialhyphen}{}%
\DTLinitials{Marie-{\'E}lise del~Rosario}
\end{verbatim}
produces:
{\renewcommand*{\DTLafterinitialbeforehyphen}{}\relax
\renewcommand*{\DTLafterinitials}{}\relax
\renewcommand*{\DTLbetweeninitials}{}\relax
\renewcommand*{\DTLinitialhyphen}{}\relax
\DTLinitials{Marie-{\'E}lise del~Rosario}}

\chapter{Comma-Separated Lists}
\label{sec:csvlists}

The \styfmt{datatool-base} package automatically loads the
\sty{etoolbox} package, so you can use any of the list
commands provided by that package, or you can use
the internal command \cs{@for} provided by the \LaTeX\ kernel
(and modified by the \sty{xfor} package, which is also loaded
by \styfmt{datatool-base}).

In addition to those commands, \styfmt{datatool-base}
provides some commands that deal with comma-separated lists.
Note that this just refers to a control sequence that stores
a list of elements separated by commas, for example:
\begin{verbatim}
\newcommand{\mylist}{elephant,ant,zebra,duck}
\end{verbatim}
This isn't the same as comma-separated files, which is dealt
with in \sectionref{sec:databases}.

\begin{definition}[\DescribeMacro{\DTLformatlist}]
\cs{DTLformatlist}\marg{list}
\end{definition}
Formats the comma-separated list. The unstarred version scopes
the internal operation. The starred form doesn't.
The \meta{list} may either be an explicit comma-separated list or 
a control sequence whose replacement
text is a comma-separated list. If the list contains empty
elements then \cs{DTLformatlist} will either include or
skip the empty element according to the conditional:
\begin{definition}[\DescribeMacro{\ifDTLlistskipempty}]
\cs{ifDTLlistskipempty}
\end{definition}
For example:
\begin{verbatim}
\newcommand{\mylist}{elephant,,ant,zebra,duck}%
\DTLlistskipemptytrue
\DTLformatlist{\mylist}\par
\DTLlistskipemptyfalse
\DTLformatlist{\mylist}
\end{verbatim}
which produces:
\begin{display}
\newcommand{\mylist}{elephant,,ant,zebra,duck}%
\DTLlistskipemptytrue
\DTLformatlist{\mylist}\par
\DTLlistskipemptyfalse
\DTLformatlist{\mylist}
\end{display}
The default setting is
\begin{verbatim}
\DTLlistskipemptytrue
\end{verbatim}
Note that this may cause a difference when upgrading to v2.31
as in previous versions \cs{DTLformatlist} didn't skip empty
elements. To restore the original behaviour use:
\begin{verbatim}
\DTLlistskipemptyfalse
\end{verbatim}
before \cs{DTLformatlist}.

The list formatting command inserts
\begin{definition}[\DescribeMacro\DTLlistformatsep]
\cs{DTLlistformatsep}
\end{definition}
between each item, except for the last pair which uses:
\begin{definition}[\DescribeMacro\DTLlistformatlastsep]
\cs{DTLlistformatlastsep}
\end{definition}
if there are only two items in the list or
\begin{definition}[\DescribeMacro\DTLlistformatoxford]
\cs{DTLlistformatoxford}\cs{DTLlistformatlastsep}
\end{definition}
if there are three or more items in the list.
\cs{DTLlistformatlastsep} uses
\begin{definition}[\DescribeMacro\DTLandname]
\cs{DTLandname}
\end{definition}
Each item in the list is formatted according to
\begin{definition}[\DescribeMacro\DTLlistformatitem]
\cs{DTLlistformatitem}\marg{item}
\end{definition}

The default definitions are:
\begin{itemize}
\item \cs{DTLlistformatitem}\marg{item}: just does \meta{item};
\item \cs{DTLlistformatsep}: \verb*|, | (comma followed by a space);
\item \cs{DTLandname}: if \ics{andname} is
defined this is defined as \cs{andname}
otherwise it's defined as \verb|\&|;
\item \cs{DTLlistformatlastsep}: does \verb*| \DTLandname\space|;
\item \cs{DTLlistformatoxford}: does nothing (so if you want an
Oxford comma you need to redefine this).
\end{itemize}
For example
\begin{verbatim}
\renewcommand{\DTLlistformatitem}[1]{\emph{#1}}%
\renewcommand{\DTLlistformatoxford}{,}%
\renewcommand{\DTLandname}{and}%
One: \DTLformatlist{elephant}.

Two: \DTLformatlist{elephant,ant}.

Three: \DTLformatlist{elephant,ant,zebra}.

Four: \DTLformatlist{elephant,ant,zebra,duck}.
\end{verbatim}
produces:
\begin{display}
\renewcommand{\DTLlistformatitem}[1]{\emph{#1}}%
\renewcommand{\DTLlistformatoxford}{,}%
\renewcommand{\DTLandname}{and}%
One: \DTLformatlist{elephant}.

Two: \DTLformatlist{elephant,ant}.

Three: \DTLformatlist{elephant,ant,zebra}.

Four: \DTLformatlist{elephant,ant,zebra,duck}.
\end{display}

You can test if an element is contained in a comma-separated list
using:
\begin{definition}[\DescribeMacro{\DTLifinlist}]
\cs{DTLifinlist}\marg{element}\marg{list}\marg{true part}\marg{false part}
\end{definition}
If \meta{element} is contained in the comma-separated list given
by \meta{list}, then this does \meta{true part} otherwise it does false
part. (Does a one level expansion on \meta{list}, but no
expansion on \meta{element}.)

\begin{definition}[\DescribeMacro{\DTLnumitemsinlist}]
\cs{DTLnumitemsinlist}\marg{list}\marg{cmd}
\end{definition}
Counts the number of elements in \meta{list} and stores the
result in \meta{cmd}, which must be a control sequence.
As from v2.31, this obeys the conditional \cs{ifDTLlistskipempty}
to determine if empty elements should be skipped. For example:
\begin{verbatim}
\newcommand{\mylist}{foo,,bar,baz,wibble}
Number of non-empty elements:
\DTLnumitemsinlist{\mylist}{\listnum}\listnum.\par
\DTLlistskipemptyfalse
Number of elements (including empty):
\DTLnumitemsinlist{\mylist}{\listnum}\listnum.
\end{verbatim}
This produces:
\begin{display}
\newcommand{\mylist}{foo,,bar,baz,wibble}
Number of non-empty elements: \DTLnumitemsinlist{\mylist}{\listnum}\listnum.\par
\DTLlistskipemptyfalse
Number of elements (including empty): \DTLnumitemsinlist{\mylist}{\listnum}\listnum.
\end{display}

\begin{definition}[\DescribeMacro{\DTLlistelement}]
\cs{DTLlistelement}\marg{list}\marg{index}
\end{definition}
Does the \meta{index}th element of the list (starting at 1 for
the first element) or generates a warning if the index is out 
of range. This obeys the conditional \cs{ifDTLlistskipempty}
to determine if empty elements should be skipped.
For example:
\begin{verbatim}
\newcommand{\mylist}{foo,,bar,baz,wibble}
3rd item in list (skip empty): \DTLlistelement{\mylist}{3}.\par
\ifDTLlistskipempty
3rd item in list (include empty): \DTLlistelement{\mylist}{3}.
\end{verbatim}
This produces:
\begin{display}
\newcommand{\mylist}{foo,,bar,baz,wibble}
3rd item in list (skip empty): \DTLlistelement{\mylist}{3}.\par
\ifDTLlistskipempty
3rd item in list (include empty): \DTLlistelement{\mylist}{3}.
\end{display}

\begin{definition}[\DescribeMacro{\DTLfetchlistelement}]
\cs{DTLfetchlistelement}\marg{list}\marg{index}\marg{cs}
\end{definition}
Like \cs{DTLlistelement} but it stores the \meta{index}th
element in the command given by \meta{cs}.

\begin{definition}[\DescribeMacro\dtlsortlist]
\cs{dtlsortlist}\marg{list cs}\marg{criteria cs}
\end{definition}
This sorts the comma-separated list stored in the command
\meta{list cs} according to the criteria command
\meta{criteria cs}. The criteria command must take three arguments:
a count register in which to store the result, element \meta{A} and
element \meta{B}. If \meta{A} is considered less that \meta{B}, the
count register should be set to $-1$, if \meta{A} and \meta{B} are
considered the same then the count register should be set to 0,
and if \meta{A} is considered greater than \meta{B}, then the count
register should be set to 1.
\begin{important}
It's inefficient to use \TeX\ to sort and comparisons are only
made according to the character codes and so are inappropriate
for extended Latin or non-Latin alphabets. It's better to use 
an external tool where possible.
\end{important}

The \styfmt{datatool-base} package provides some predefined
criteria commands:
\begin{definition}[\DescribeMacro\dtlcompare]
\cs{dtlcompare}\marg{register}\marg{A}\marg{B}
\end{definition}
A case-sensitive comparison.
\begin{definition}[\DescribeMacro\dtlicompare]
\cs{dtlicompare}\marg{register}\marg{A}\marg{B}
\end{definition}
A case-insensitive comparison. The two above commands work in much
the same way except that the first compares character codes and the
second compares the lowercase character codes. If control sequences
are found then the comparison is determined by the conditional:
\begin{definition}[\DescribeMacro\ifdtlcompareskipcs]
\cs{ifdtlcompareskipcs}
\end{definition}
This is switched on with:
\begin{definition}[\DescribeMacro\dtlcompareskipcstrue]
\cs{dtlcompareskipcstrue}
\end{definition}
and switched off with:
\begin{definition}[\DescribeMacro\dtlcompareskipcsfalse]
\cs{dtlcompareskipcsfalse}
\end{definition}
The default setting is false (off). If true control sequences will
be skipped. If false control sequences will considered as having the
code 0.

There are also two comparison commands designed for indexes:
\begin{definition}[\DescribeMacro\dtlwordindexcompare]
\cs{dtlwordindexcompare}\marg{register}\marg{A}\marg{B}
\end{definition}
English word-ordering comparison for indexes, as described by 
the Oxford Style Manual.
\begin{definition}[\DescribeMacro\dtlletterindexcompare]
\cs{dtlletterindexcompare}\marg{register}\marg{A}\marg{B}
\end{definition}
English letter-ordering comparison for indexes.
These last two commands are described in more detail in
\sectionref{sec:sort}.

For example:
\begin{verbatim}
\newcommand{\mylist}{elephant,ant,zebra,duck}
\dtlsortlist{\mylist}{\dtlcompare}

\mylist.
\end{verbatim}
produces:
\begin{display}
\newcommand{\mylist}{elephant,ant,zebra,duck}
\dtlsortlist{\mylist}{\dtlcompare}

\mylist.
\end{display}

If you are building up a list, you may prefer to use:
\begin{definition}[\DescribeMacro\dtlinsertinto]
\cs{dtlinsertinto}\marg{element}\marg{list cs}\marg{criteria cs}
\end{definition}
which inserts \meta{element} into the list stored in the
command \meta{list cs} according to the criteria command
\meta{criteria cs}. This is more efficient than first building the
list and then sorting it.

For example:
\begin{verbatim}
\newcommand{\mylist}{}
\dtlinsertinto{elephant}{\mylist}{\dtlcompare}
\dtlinsertinto{ant}{\mylist}{\dtlcompare}
\dtlinsertinto{zebra}{\mylist}{\dtlcompare}
\dtlinsertinto{duck}{\mylist}{\dtlcompare}

\mylist.
\end{verbatim}
produces:
\begin{display}
\newcommand{\mylist}{}
\dtlinsertinto{elephant}{\mylist}{\dtlcompare}
\dtlinsertinto{ant}{\mylist}{\dtlcompare}
\dtlinsertinto{zebra}{\mylist}{\dtlcompare}
\dtlinsertinto{duck}{\mylist}{\dtlcompare}

\mylist.
\end{display}

Note that \cs{dtlinsertinto} doesn't expand \meta{element}.
If the element is stored in a command, you need to
expand it first. For example:
\begin{verbatim}
\newcommand*{\element}{ant}
\expandafter\dtlinsertinto\expandafter{\element}{\mylist}{\dtlcompare}
\end{verbatim}

To ensure that the element is first fully expanded, you can
use:
\begin{definition}[\DescribeMacro\edtlinsertinto]
\cs{edtlinsertinto}\marg{element}\marg{list cs}\marg{criteria cs}
\end{definition}
This will fully expand \meta{element} using \cs{protected@edef}
and then use \cs{dtlinsertinto}.

\chapter{Databases}
\label{sec:databases}

The \sty{datatool} package provides a means of creating and
loading databases. Once a database has been created (or loaded),
it is possible to iterate through each row of data, to make it
easier to perform repetitive actions, such as mail merging.

\begin{important}
Whilst \TeX\ is an excellent typesetting language,
it is not designed as a database management system, and 
attempting to use it as such is like trying to fasten a screw
with a knife instead of a screwdriver: it can be done, but requires
great care and is more time consuming. Version 2.0 of the
\sty{datatool} package uses a completely different method of storing
the data to previous versions.\footnote{Many thanks to Morten H\o gholm
for providing the new code.} As a result, the code is much more efficient,
however, large databases and complex operations will still slow the
time taken to process your document. Therefore, if you can, it is
better to do the complex operations using whatever system created
the data in the first place.
\end{important}

Some advanced commands for accessing database information are
described in \autoref{sec:advanced}, but using \TeX\ is nowhere near
as efficient as, say, using a SQL database, so don't expect too much
from this package.

\label{datatooltk}%
I've written a Java helper application to accompany \sty{datatool}
called \app{datatooltk}. The installer is available
on CTAN at
\url{http://mirrors.ctan.org/support/datatooltk/datatooltk-installer.jar}. The application will allow you to edit
files saved using \ics{DTLsaverawdb} or \ics{DTLprotectedsaverawdb} in a~graphical interface or
import data from a~SQL database, a~CSV file or a~\sty{probsoln}
dataset.

Each database has the data stored internally in a token register and
the header information is stored in an internal control sequence.
In general you don't need to worry about the way it's stored, except
that row indexes start from 1 (the first row of data) and column
indexes also start from 1. Columns may be referenced by a label
but rows can only be referenced by the row index. Once the internal
commands and registers have been set, \styfmt{datatool} doesn't keep
a record of how or where the information came from (such as a CSV
file or through a file created by \app{datatooltk}). Changes to the
data only modify the internal commands and are lost at the end of
the \LaTeX\ run unless you save the data.

\section{Creating a New Database}
\label{sec:newdb}

\begin{definition}[\DescribeMacro{\DTLnewdb}]%
\cs{DTLnewdb}\marg{db name}
\end{definition}
\begin{definition}[\DescribeMacro{\DTLgnewdb}]%
\cs{DTLgnewdb}\marg{db name}
\end{definition}
This command creates a new empty database called \meta{db name}. The
second form is for global definitions. You
can test if a database is empty using:
\begin{definition}[\DescribeMacro{\DTLifdbempty}]%
\cs{DTLifdbempty}\marg{db name}\marg{true part}\marg{false part}
\end{definition}
If the database called \meta{db name} is empty, do \meta{true part},
otherwise do \meta{false part}.

\begin{definition}[\DescribeMacro{\DTLrowcount}]%
\cs{DTLrowcount}\marg{db name}
\end{definition}
This command displays the number of rows in the database called
\meta{db name}.

\begin{definition}[\DescribeMacro{\DTLcolumncount}]%
\cs{DTLcolumncount}\marg{db name}
\end{definition}
This command displays the number of columns (or keys) in the
database called \meta{db name}.

\begin{definition}[\DescribeMacro{\DTLnewrow}]%
\cs{DTLnewrow}\marg{db name}
\end{definition}
This command starts a new row in the database called \meta{db name}.
This new row becomes the current row when adding new entries.

For example, the following creates an empty database called
\texttt{mydata}:
\begin{verbatim}
\DTLnewdb{mydata}
\end{verbatim}
\DTLnewdb{mydata}\relax
The following tests if the database is empty:
\begin{verbatim}
\DTLifdbempty{mydata}{empty}{not empty}!
\end{verbatim}
This produces:
\DTLifdbempty{mydata}{empty}{not empty}!

The following adds an empty row to the database, this is the
first row of the database:
\begin{verbatim}
\DTLnewrow{mydata}
\end{verbatim}
\DTLnewrow{mydata}\relax
Note that even though the only row in the database is currently
empty, the database is no longer considered to be empty:
\begin{verbatim}
\DTLifdbempty{mydata}{empty}{not empty}!
\end{verbatim}
This now produces:
\DTLifdbempty{mydata}{empty}{not empty}! The row count is 
given by
\begin{verbatim}
\DTLrowcount{mydata}
\end{verbatim}
which produces:
\DTLrowcount{mydata}. The column count is given by
\begin{verbatim}
\DTLcolumncount{mydata}
\end{verbatim}
which produces: \DTLcolumncount{mydata}.

\begin{definition}[\DescribeMacro{\DTLnewdbentry}]%
\cs{DTLnewdbentry}\marg{db name}\marg{key}\marg{value}
\end{definition}
This creates a new entry with the identifier \meta{key} whose value
is \meta{value} and adds it to the last row of the database
called \meta{db name}. For example:
\begin{verbatim}
\DTLnewdbentry{mydata}{Surname}{Smith}
\DTLnewdbentry{mydata}{FirstName}{John}
\end{verbatim}
Adds an entry with identifier \texttt{Surname} and value
\texttt{Smith} to the last row of the database named
\texttt{mydata}, and then adds an entry with identifier
\texttt{FirstName} and value \texttt{John}. Note that the
key should not contain any fragile commands. It is generally
best to only use non-active characters in the key.

The value isn't expanded by default, but you can change this
using the declaration:
\begin{definition}[\DescribeMacro{\dtlexpandnewvalue}]
\cs{dtlexpandnewvalue}
\end{definition}
This can be localised by placing it in a group, or you can
switch back using:
\begin{definition}[\DescribeMacro{\dtlnoexpandnewvalue}]
\cs{dtlnoexpandnewvalue}
\end{definition}

\importantpar
Note that database entries can't contain paragraph 
breaks as many of the macros used by \sty{datatool} are short
commands.  If you do need a paragraph break in an entry, you can
instead use the command:
\begin{definition}[\DescribeMacro{\DTLpar}]%
\cs{DTLpar}
\end{definition}
For example:
\begin{verbatim}
\DTLnewdbentry{mydata}{Description}{First paragraph.\DTLpar
Second paragraph.}
\end{verbatim}

\begin{definition}[\DescribeMacro{\DTLaddentryforrow}]%
\cs{DTLaddentryforrow}\marg{db}\marg{assign list}%
\marg{condition}\marg{key}\allowbreak\marg{value}
\end{definition}
This adds the entry with the key given by \meta{key} and value
given by \meta{value} to the first row in the database 
\meta{db} which satisfies the condition given by
\meta{condition}. The \meta{assign list} argument is the same
as for \cs{DTLforeach} (described in \autoref{sec:dbforeach})
and may be used to set the values which are to be tested in 
\meta{condition} (where, again, \meta{condition} is the same
as for \cs{DTLforeach}). For example:
\begin{verbatim}
\DTLaddentryforrow{mydata}{\firstname=FirstName,\surname=Surname}%
{\DTLiseq{\firstname}{John}\and\DTLiseq{\surname}{Smith}}%
{Score}{75}
\end{verbatim}
Note that unlike \cs{DTLnewdbentry}, the value is always expanded
when adding an entry using \cs{DTLaddentryforrow}.

\begin{definition}[\DescribeMacro{\DTLsetheader}]%
\cs{DTLsetheader}\marg{db}\marg{key}\marg{header}
\end{definition}
This assigns a header for a given key in the database named 
\meta{db}. This is used by \cs{DTLdisplaydb} and 
\cs{DTLdisplaylongdb} in the header row (see 
\autoref{sec:displaydb}). If you don't assign
a header, the header will be given by the key.
For example:
\begin{verbatim}
\DTLsetheader{mydata}{Price}{Price (\$)}
\end{verbatim}

\begin{definition}[\DescribeMacro{\DTLaddcolumn}]
\cs{DTLaddcolumn}\marg{db}\marg{key}
\end{definition}
Adds a new column with the given key to the database \meta{db}. This
doesn't add any data to the column, just identifies it as an
available column. The starred version doesn't check if the database
exists.

\section{Loading a Database from an External ASCII File}
\label{sec:loaddb}

\begin{important}
\cs{DTLloaddb} and \cs{DTLloadrawdb}, described in this section, can't 
parse files that have newline characters within entries. The 
\app{datatooltk} application (see page~\pageref{datatooltk}) can 
parse them, so if you have multilined entries in a CSV file, you 
can convert it to \sty{datatool}'s internal database format using 
\app{datatooltk} and the input it using \ics{input} or \ics{DTLloaddbtex}. See the 
\app{datatooltk} documentation for further details.
\end{important}
Remember that row numbers in \sty{datatool} refer to the row index
of the internal data not to line numbers in the CSV file.

Instead of using the commands described in \autoref{sec:newdb}
to create a new database, you can load a database from an
external ASCII file using:
\begin{definition}[\DescribeMacro{\DTLloaddb}]%
\cs{DTLloaddb}\oarg{options}\marg{db name}\marg{filename}
\end{definition}

\begin{important}
Make sure your document uses the same encoding as \meta{filename}.
For example, if \meta{filename} is UTF-8, then include the following
in your document:
\begin{verbatim}
\usepackage[utf8]{inputenc}
\end{verbatim}
\end{important}
By default, \cs{DTLloaddb} creates a new database called \meta{db
name} before it loads the data given in the file \meta{filename}. 
If you want to append the data, use
\begin{definition}[\DTLnewdbonloadfalse]
\cs{DTLnewdbonloadfalse}
\end{definition}
before you use \cs{DTLloaddb}. You can reverse this using
\begin{definition}[\DTLnewdbonloadtrue]
\cs{DTLnewdbonloadtrue}
\end{definition}
The file (\meta{filename}) may have a header row at the start of the file,
which provides the \meta{key} when creating a new database entry
using \ics{DTLnewdbentry}.
The optional argument \meta{options}
is a \meta{key}=\meta{value} list of options.
\begin{important}
Some of the keys may take a comma-separated list as a value.
Note that spaces count in the default \TeX\ way within these lists.
Remember that \TeX\ considers an end-of-line character as a space.
If you have a long list, consider using the comment character (\%)
to suppress unwanted space caused by line breaks in the code.
\end{important}
Available options are:
\begin{description}
\item[\csopt{DTLloaddb}{noheader}] This is a boolean value and
indicates if the file does not contain a header. If no value is
supplied, "true" is assumed (i.e.\ the file doesn't contain a header
row).  If this option is omitted, it is assumed that the file
contains a header row.

\item[\csopt{DTLloaddb}{keys}] This is a comma-separated list of
keys to use, where the keys are listed in the same order as the
columns. If the file has a header, these keys will override the
values given in the header row. If the file has no header row and no
keys are supplied in \meta{options}, then the keys will be given by
\cs{dtldefaultkey}\meta{n}, where \meta{n} is the column number and
\DescribeMacro{\dtldefaultkey}\cs{dtldefaultkey} defaults to
``\dtldefaultkey''. Note that the list of keys must be delimited by
braces since they contain commas.
For example:
\begin{verbatim}
\DTLloaddb[noheader,keys={Temperature,Time,T2G}]{data}{data.csv}
\end{verbatim}

\item[\csopt{DTLloaddb}{autokeys}] This is a boolean option that
will automatically assign default keys (\cs{dtldefaultkey}\meta{n},
as above) for all columns, regardless of whether the file has a
header row. If true, this option overrides the
\csopt{DTLloaddb}{keys} option. You may want to use this if you have
a header row with many fields containing active characters but you
only want to use commands like \cs{DTLdisplaydb}. (In which case,
you may want to consider using \cs{DTLloadrawdb} to load the data.)

\item[\csopt{DTLloaddb}{headers}] This is a comma-separated list of 
headers. If not supplied, the header will be the same as that given
in the header row, or the key if there is no header row. Note that
the list of headers must be delimited by braces since they contain
commas. For example:
\begin{verbatim}
\DTLloaddb[noheader,keys={Temperature,Time,T2G},%
headers={\shortstack{Incubation\\Temperature},%
\shortstack{Incubation\\Time},%
\shortstack{Time to\\Growth}}]{data}{data.csv}
\end{verbatim}

\item[\csopt{DTLloaddb}{omitlines}]
This should be a non-negative integer that specifies how many rows
to skip at the start of the file.
\end{description}

By default, the entries in the database
must be separated by a comma, and optionally delimited by the
double quote character (\verb|"|). The separator can be changed
to a tab separator using the command:
\begin{definition}[\DescribeMacro{\DTLsettabseparator}]%
\cs{DTLsettabseparator}
\end{definition}
Note that this command changes the category code of the tab
character to~12 (other). If, after you have loaded your data, you want to
reset the tab category code to~10 (space), you can use:
\begin{definition}[\DescribeMacro{\DTLmaketabspace}]%
\cs{DTLmaketabspace}
\end{definition}
Don't use this command before you load any tab-separated data.

To set the separator to a character other than a tab, you need to use
\begin{definition}[\DescribeMacro{\DTLsetseparator}]%
\cs{DTLsetseparator}\marg{character}
\end{definition}
The delimiter can be changed using
\begin{definition}[\DescribeMacro{\DTLsetdelimiter}]%
\cs{DTLsetdelimiter}\marg{character}
\end{definition}
\begin{important}
Note that spaces count in the usual \TeX\ manner and won't be
trimmed from either side of the separators.
\end{important}

For example, suppose you have a file called \texttt{mydata.csv}
which contains the following:
\begin{verbatim}
FirstName,Surname,Score
John,"Smith, Jr",68
Jane,Brown,75
Andy,Brown,42
Z\"oe,Adams,52
\end{verbatim}
then
\begin{verbatim}
\DTLloaddb{mydata}{mydata.csv}
\end{verbatim}
is equivalent to:
\begin{verbatim}
\DTLnewdb{mydata}
\DTLnewrow{mydata}%
\DTLnewdbentry{mydata}{FirstName}{John}%
\DTLnewdbentry{mydata}{Surname}{Smith, Jr}%
\DTLnewdbentry{mydata}{Score}{68}%
\DTLnewrow{mydata}%
\DTLnewdbentry{mydata}{FirstName}{Jane}%
\DTLnewdbentry{mydata}{Surname}{Brown}%
\DTLnewdbentry{mydata}{Score}{75}%
\DTLnewrow{mydata}%
\DTLnewdbentry{mydata}{FirstName}{Andy}%
\DTLnewdbentry{mydata}{Surname}{Brown}%
\DTLnewdbentry{mydata}{Score}{42}%
\DTLnewrow{mydata}%
\DTLnewdbentry{mydata}{FirstName}{Z\"oe}%
\DTLnewdbentry{mydata}{Score}{52}%
\DTLnewdbentry{mydata}{Surname}{Adams}%
\end{verbatim}
Note that the entry \texttt{Smith, Jr} had to be delimited in
\texttt{mydata.csv} using the double quote character since it
contained a comma which is used as the separator.
The percent symbol \% can be used as a comment character within the
file.

The file used in the above example contained a \LaTeX\ command,
namely \verb|\"|. When using \ics{DTLloaddb} all the special 
characters that appear in the command retain their \LaTeX\ meaning
when the file is loaded. It is likely however that the data file
may have been created by another application that is not \TeX-aware,
such as a spreadsheet application. For example, suppose you
have a file called, say, \texttt{products.csv} which looks
like:
\begin{verbatim}
Product,Cost
Fruit & Veg,$1.25
Stationary,$0.80
\end{verbatim}
This file contains two of \TeX's special characters, namely
"&" and \verb|$|. In this case, if you try to load the file
using \ics{DTLloaddb}, you will encounter errors. Instead you
can use:
\begin{definition}[\DescribeMacro{\DTLloadrawdb}]%
\cs{DTLloadrawdb}\oarg{options}\marg{db name}\marg{filename}
\end{definition}
This is similar to \ics{DTLloaddb} except that it maps nine of
the ten special characters onto commands which produce that
symbol. The only character that retains its active state is the
backslash character, so you will still need to check the file
for backslash characters. The mappings used are listed in 
\autoref{tab:rawmappings}. So using the file \texttt{products.csv},
as described above, 
\begin{verbatim}
\DTLloadrawdb{mydata}{products.csv}
\end{verbatim}
is equivalent to:
\begin{verbatim}
\DTLnewdb{mydata}
\DTLnewrow{mydata}%
\DTLnewdbentry{mydata}{Product}{Fruit \& Veg}%
\DTLnewdbentry{mydata}{Cost}{\$1.25}%
\DTLnewrow{mydata}%
\DTLnewdbentry{mydata}{Product}{Stationary}%
\DTLnewdbentry{mydata}{Cost}{\$0.80}%
\end{verbatim}
As with \ics{DTLloaddb}, you can govern whether or not a new
database should be created with \ics{DTLnewdbonloadtrue} and
\ics{DTLnewdbonloadfalse}.

Note that \cs{DTLloadrawdb} is not recommended for CSV files that
contain commands. Any active characters occurring within the
file must be mapped. The mapping is done through
expansion. Version 2.28 now uses \cs{xdef} rather than
\cs{protected@xdef}, but \cs{DTLloadrawdb} isn't intended for files
containing \LaTeX\ code, but rather for files generated by a
non-\TeX\ aware method that doesn't guard against \LaTeX\ special
characters. 

If you have UTF-8 data, it's best to use an engine that natively
supports Unicode. Remember that there's no point using
\cs{DTLloadrawdb} (instead of \cs{DTLloaddb}) if your file doesn't
contain any special characters that require mapping. It's less
efficient to load if each row has to be converted, particularly for
large files.

\begin{table}[htbp]
\caption[Special character mappings used by 
\cs{DTLloadrawdb}]{Special character mappings used by 
\cs{DTLloadrawdb} (note that the backslash retains its active state)}
\label{tab:rawmappings}
\centering
\begin{tabular}{cl}
Character & Mapping\\
\verb|%| & \verb|\%|\\
\verb|$| & \verb|\$|\\
\verb|&| & \verb|\&|\\
\verb|#| & \verb|\#|\\
\verb|_| & \verb|\_|\\
\verb|{| & \verb|\{|\\
\verb|}| & \verb|\}|\\
\verb|~| & \cs{textasciitilde}\\
\verb|^| & \cs{textasciicircum}
\end{tabular}
\end{table}

It may be that there are other characters that require mapping.
For example, the file \texttt{products.csv} may instead look like:\par
\vskip\baselineskip
\begin{ttfamily}\obeylines\setlength{\parindent}{0pt}
Product,Cost
Fruit \& Veg,\pounds1.25
Stationary,\pounds0.80
\end{ttfamily}
\vskip\baselineskip
The pound character is not an internationally standard keyboard
character, and does not generally achieve the desired effect when used
in a \LaTeX\ document. It may therefore be necessary to convert
this symbol to an appropriate control sequence. This can be done 
using the command:
\begin{definition}[\DescribeMacro{\DTLrawmap}]%
\cs{DTLrawmap}\marg{string}\marg{replacement}
\end{definition}
For example:\par
\vskip\baselineskip
\begin{ttfamily}\obeylines\setlength{\parindent}{0pt}
\cs{DTLrawmap}\char`\{\pounds\char`\}\char`\{\cs{pounds}\char`\}
\end{ttfamily}
\vskip\baselineskip\noindent
will replace all occurrences\footnote{when it is loaded into the
\LaTeX\ database, it does not modify the data file!}\ of
\texttt{\pounds} with \cs{pounds}.  Naturally, the mappings must be
set \emph{prior} to loading the data with \cs{DTLloadrawdb}.

\begin{important}
Note that the warning in the previous section about no
paragraph breaks in an entry also applies to entries loaded from a
database.  If you do need a paragraph break, use \ics{DTLpar} instead
of \cs{par}, but remember that each row of data in an external data
file must not have a line break.
\end{important}

\section{Displaying the Contents of a Database}
\label{sec:displaydb}

Once you have created a database, either loading it from an
external file, as described in \autoref{sec:loaddb}, or using the
commands described in \autoref{sec:newdb}, you can display the
entire database in a \env{tabular} or \env{longtable}
environment.

\begin{definition}[\DescribeMacro{\DTLdisplaydb}]%
\cs{DTLdisplaydb}\oarg{omit list}\marg{db}
\end{definition}
This displays the database given by \meta{db} in a \env{tabular}
environment. The first row displays the headers for the database
in bold, the subsequent rows display the values for each key
in each row of the database. The optional argument \meta{omit list}
is a comma-separated list of column keys to omit. (All columns displayed by default.)

\begin{definition}[\DescribeMacro{\DTLdisplaylongdb}]%
\cs{DTLdisplaylongdb}\oarg{options}\marg{db}
\end{definition}
This is like \cs{DTLdisplaydb} except that it uses the 
\env{longtable} environment instead of the \env{tabular}
environment. Note that if you use this command, you must load the
\sty{longtable} package, as it is not loaded by \sty{datatool}.
The optional argument \meta{options} is a comma-separated list
of key=value pairs. The following keys are available:
\begin{description}
\item[caption] The caption for the \env{longtable}.

\item[contcaption] The continuation caption.

\item[shortcaption] The caption to be used in the list of tables.

\item[label] The label for this table.

\item[omit] Comma-separated list of column keys to omit.

\item[foot] The \env{longtable}'s foot.

\item[lastfoot] The foot for the last page of the \env{longtable}.
\end{description}
For example, suppose I have a database called "iris", then I can
display the contents in a \env{longtable} using:
\begin{verbatim}
\DTLdisplaylongdb[%
caption={Iris Data},%
label={tab:iris},%
contcaption={Iris Data (continued)},%
foot={\em Continued overleaf},%
lastfoot={}%
]{iris}
\end{verbatim}
I can then reference the table using \verb|\ref{tab:iris}|.

See the \sty{longtable} documentation for details on how to
change the \env{longtable} settings, such as how to change the
table so that it is left aligned instead of centred on the page.

Note that if you want more control over the way the data is 
displayed, for example, you want to filter rows or columns, you will
need to use \cs{DTLforeach}, described in \autoref{sec:dbforeach}.

\begin{example}{Displaying the Contents of a Database}{ex:displaydb}
Suppose I have a file called \texttt{t2g.csv} that contains the
following:
\begin{verbatim}
40,120,40
40,90,60
35,180,20
55,190,40
\end{verbatim}
This represents time to growth data, where the first column
is the incubation temperature, the second column is the incubation
time and the third column is the time to growth. This file has no
header row, so when it is loaded, the \csopt{DTLloaddb}{noheaders} 
option is required. Note that \cs{DTLdisplaydb} only puts the data 
in a \env{tabular} environment, so \cs{DTLdisplaydb} needs to be 
put in a \env{table} environment with a caption to make it a float.

First load the data base, setting the keys and headers:
\begin{verbatim}
\DTLloaddb[noheader,%
keys={Temperature,Time,T2G},%
headers={\shortstack{Incubation\\Temperature},%
\shortstack{Incubation\\Time},\shortstack{Time to\\Growth}}%
]{t2g}{t2g.csv}
\end{verbatim}
\DTLnewdb{t2g}\relax
\DTLnewrow*{t2g}\relax
\DTLnewdbentry*{t2g}{Temperature}{40}\relax
\DTLnewdbentry*{t2g}{Time}{120}\relax
\DTLnewdbentry*{t2g}{T2G}{40}\relax
\DTLnewrow*{t2g}\relax
\DTLnewdbentry*{t2g}{Temperature}{40}\relax
\DTLnewdbentry*{t2g}{Time}{90}\relax
\DTLnewdbentry*{t2g}{T2G}{60}\relax
\DTLnewrow*{t2g}\relax
\DTLnewdbentry*{t2g}{Temperature}{35}\relax
\DTLnewdbentry*{t2g}{Time}{180}\relax
\DTLnewdbentry*{t2g}{T2G}{20}\relax
\DTLnewrow*{t2g}\relax
\DTLnewdbentry*{t2g}{Temperature}{55}\relax
\DTLnewdbentry*{t2g}{Time}{190}\relax
\DTLnewdbentry*{t2g}{T2G}{40}\relax
\DTLsetheader*{t2g}{Temperature}{\shortstack{Incubation\\Temperature}}\relax
\DTLsetheader*{t2g}{Time}{\shortstack {Incubation\\Time}}\relax
\DTLsetheader*{t2g}{T2G}{\shortstack {Time to\\Growth}}\relax

Now display the data in a table:
\begin{verbatim}
\begin{table}[htbp]
\caption{Time to Growth Data}
\centering
\DTLdisplaydb{t2g}
\end{table} 
\end{verbatim}
The result is shown in \autoref{tab:t2g}.
\begin{table}[htbp]
\caption{Time to Growth Data}
\label{tab:t2g}
\centering
\DTLdisplaydb{t2g}
\end{table} 
\end{example}

Each column in the database has an associated data type which
indicates what type of data is in that column. This may be one
of: string, integer, real number or currency. If a column contains
more than one type, the data type is determined as follows:
\begin{itemize}
\item If the column contains at least one string, then the column
data type is string.

\item If the column doesn't contain a string, but contains at least
one currency, then the column data type is currency.

\item If the column contains only real numbers and integers, the
column data type is real number.

\item The column data type is integer if the column only
contains integers.
\end{itemize}
The column data type is updated whenever a new entry is added
to the database. Note that the column data type is not adjusted
when an entry is removed from the database.

The column alignments used by \cs{DTLdisplaydb} are given by:
\begin{definition}[\DescribeMacro{\dtlstringalign}]%
\cs{dtlstringalign}
\end{definition}\noindent
The string alignment defaults to \texttt{l} (left aligned).

\begin{definition}[\DescribeMacro{\dtlintalign}]%
\cs{dtlintalign}
\end{definition}\noindent
The integer alignment defaults to \texttt{r} (right aligned).

\begin{definition}[\DescribeMacro{\dtlrealalign}]%
\cs{dtlrealalign}
\end{definition}\noindent
The alignment for real numbers defaults to \texttt{r} (right
aligned).

\begin{definition}[\DescribeMacro{\dtlcurrencyalign}]%
\cs{dtlcurrencyalign}
\end{definition}\noindent
The currency alignment defaults to \texttt{r} (right aligned).

You can redefine these to change the column alignments. For
example, if you want columns containing strings to have the
alignment "p{2in}", then you can redefine \cs{dtlstringalign} as
follows:
\begin{verbatim}
\renewcommand{\dtlstringalign}{p{2in}}
\end{verbatim}

\begin{important}
You can't use \sty{siunitx}'s "S" column alignment
with either \cs{DTLdisplaydb} or \cs{DTLdisplaylongdb}. Instead, you
will need to use \cs{DTLforeach}. The \sty{siunitx} documentation
provides an example.
\end{important}

In addition to the \cs{dtl}\meta{type}"align" commands above, you
can also modify the \env{tabular} column styles by redefining the
following three commands:
\begin{definition}[\DescribeMacro{\dtlbeforecols}]
\cs{dtlbeforecols}
\end{definition}
(before the first column)
\begin{definition}[\DescribeMacro{\dtlbetweencols}]
\cs{dtlbetweencols}
\end{definition}
(between each column) and
\begin{definition}[\DescribeMacro{\dtlaftercols}]
\cs{dtlaftercols}
\end{definition}
(after the last column).

For example, to
place a vertical line before the start of the first column and
after the last column, do:
\begin{verbatim}
\renewcommand{\dtlbeforecols}{|}
\renewcommand{\dtlaftercols}{|}
\end{verbatim}
If you additionally want vertical lines between columns, do:
\begin{verbatim}
\renewcommand{\dtlbetweencols}{|}
\end{verbatim}

Limited modifications can be made to the way the data is displayed
with \cs{DTLdisplaydb} and \cs{DTLdisplaylongdb}.
The commands controlling the formatting are described below.
If a more complicated layout is required, you will need to use
\cs{DTLforeach} described in \autoref{sec:dbforeach}.

\begin{definition}[\DescribeMacro{\dtlheaderformat}]%
\cs{dtlheaderformat}\marg{header}
\end{definition}
This indicates how to format a column header, where the header is
given by \meta{header}. This defaults to 
\cs{null}\cs{hfil}\cs{textbf}\marg{header}\cs{hfil}\cs{null}.

\begin{definition}[\DescribeMacro{\dtlstringformat}]%
\cs{dtlstringformat}\marg{text}
\end{definition}
This specifies how to format each entry in the columns that contain
strings. This defaults to just displaying \meta{text}.

\begin{definition}[\DescribeMacro{\dtlintformat}]%
\cs{dtlintformat}\marg{text}
\end{definition}
This specifies how to format each entry in the columns that contain
only integers. This defaults to just displaying \meta{text}.

\begin{definition}[\DescribeMacro{\dtlrealformat}]%
\cs{dtlrealformat}\marg{text}
\end{definition}
This specifies how to format each entry in the columns that contain
only real numbers or a mixture of real numbers and integers. This 
defaults to just displaying \meta{text}.

\begin{definition}[\DescribeMacro{\dtlcurrencyformat}]%
\cs{dtlcurrencyformat}\marg{text}
\end{definition}
This specifies how to format each entry in the columns that contain
only currency or currency mixed with real numbers and/or integers. 
This defaults to just displaying \meta{text}.

\begin{definition}[\DescribeMacro{\dtldisplayvalign}]
\cs{dtldisplayvalign}
\end{definition}
Specifies the vertical alignment of the \env{tabular} environment
used by \cs{DTLdisplaydb}. Defaults to \texttt{c} (centred). May be redefined
to \texttt{t} (top) or \texttt{b} (bottom).

\begin{definition}[\DescribeMacro{\dtldisplaycr}]
\cs{dtldisplaycr}
\end{definition}
Specifies how to separate rows. Defaults to just
\ics{tabularnewline}.

\begin{definition}[\DescribeMacro{\dtldisplaystarttab}]%
\cs{dtldisplaystarttab}
\end{definition}
This is a hook to add something at the beginning of the
\env{tabular} environment. This defaults to nothing.
In the case of \cs{DTLdisplaylongdb}, this hook is done before the
header row on each page of the \env{longtable}.

\begin{definition}[\DescribeMacro{\dtldisplayendtab}]%
\cs{dtldisplayendtab}
\end{definition}
This is a hook to add something at the end of the
\env{tabular} environment. This defaults to nothing.
In the case of \cs{DTLdisplaylongdb}, this hook is only done on the
\emph{last} page of the \env{longtable}. You have to use the
\texttt{foot} option to specify some code to do at the end of each page.

\begin{definition}[\DescribeMacro{\dtldisplayafterhead}]%
\cs{dtldisplayafterhead}
\end{definition}
This is a hook to add something after the header row, before
the first row of data. This defaults to nothing.
In the case of \cs{DTLdisplaylongdb}, this hook is done after the
header row on each page of the \env{longtable}.

\begin{definition}[\DescribeMacro{\dtldisplaystartrow}]%
\cs{dtldisplaystartrow}
\end{definition}
This is a hook to add something at the start of each row, but
not including the header row or the first row of data. This 
defaults to nothing.

If you want to use the \sty{booktabs} package, you can redefine the
above three commands to use \ics{toprule}, \ics{midrule} and
\ics{bottomrule}:
\begin{verbatim}
\renewcommand{\dtldisplaystarttab}{\toprule}
\renewcommand{\dtldisplayafterhead}{\midrule}
\renewcommand{\dtldisplayendtab}{\\\bottomrule}
\end{verbatim}

\begin{example}{Balance Sheet}{ex:balance}
Suppose you have a file called "balance.csv" that contains
the following:
\begin{verbatim}
Description,In,Out,Balance
Travel expenses,,230,-230
Conference fees,,400,-630
Grant,700,,70
Train fare,,70,0
\end{verbatim}
\DTLnewdb{balance}\relax
\DTLnewrow{balance}\relax
\DTLnewdbentry{balance}{Description}{Travel expenses}\relax
\DTLnewdbentry{balance}{In}{}\relax
\DTLnewdbentry{balance}{Out}{230.00}\relax
\DTLnewdbentry{balance}{Balance}{-230.00}\relax
\DTLnewrow{balance}\relax
\DTLnewdbentry{balance}{Description}{Conference fees}\relax
\DTLnewdbentry{balance}{In}{}\relax
\DTLnewdbentry{balance}{Out}{400.00}\relax
\DTLnewdbentry{balance}{Balance}{-630.00}\relax
\DTLnewrow{balance}\relax
\DTLnewdbentry{balance}{Description}{Grant}\relax
\DTLnewdbentry{balance}{In}{700.00}\relax
\DTLnewdbentry{balance}{Out}{}\relax
\DTLnewdbentry{balance}{Balance}{70.00}\relax
\DTLnewrow{balance}\relax
\DTLnewdbentry{balance}{Description}{Train Fare}\relax
\DTLnewdbentry{balance}{In}{}\relax
\DTLnewdbentry{balance}{Out}{70.00}\relax
\DTLnewdbentry{balance}{Balance}{0.00}\relax
\DTLsetheader{balance}{In}{In (\pounds)}\relax
\DTLsetheader{balance}{Out}{Out (\pounds)}\relax
\DTLsetheader{balance}{Balance}{Balance (\pounds)}\relax
The data can be loaded using:
\begin{verbatim}
\DTLloaddb[headers={Description,In (\pounds),Out (\pounds),Balance
(\pounds)}]{balance}{balance.csv}
\end{verbatim}

Suppose I want negative numbers to be displayed in red. I can do
this by redefining \cs{dtlrealformat} to check if the entry
is negative. For example:
\begin{verbatim}
\begin{table}[htbp]
\caption{Balance Sheet}
\renewcommand*{\dtlrealformat}[1]{\DTLiflt{#1}{0}{\color{red}}{}#1}
\centering
\DTLdisplaydb{balance}
\end{table}
\end{verbatim}
This produces \autoref{tab:balance}.
\begin{table}[htbp]
\caption{Balance Sheet}
\label{tab:balance}
\renewcommand*{\dtlrealformat}[1]{\DTLiflt{#1}{0}{\color{red}}{}#1}
\centering
\DTLdisplaydb{balance}
\end{table}
\end{example}

\section{Iterating Through a Database}
\label{sec:dbforeach}

Once you have created a database, either loading it from an
external file, as described in \autoref{sec:loaddb}, or using the
commands described in \autoref{sec:newdb}, you can then iterate
through each row of the database and access elements in that row.

\begin{important}
The \cs{DTLforeach} command is provided as a convenient way of
iterating through databases with the option to filter rows, 
cross-reference rows or have nested loops, but for large databases it can be 
\emph{extremely slow}, especially if the unstarred version is used.
If you have a large database, consider pre-processing the data using
an external script. See also \sectionref{sec:advancediter}.
\end{important}

\begin{definition}[\DescribeMacro{\DTLforeach}]%
\cs{DTLforeach}\oarg{condition}\marg{db name}\marg{assign list}\marg{text}
\end{definition}
\begin{definition}[\DescribeMacro{\DTLforeach*}]%
\cs{DTLforeach*}\oarg{condition}\marg{db name}\marg{assign list}\marg{text}
\end{definition}
This will iterate through each row of the database called
\meta{db name}, applying \meta{text} to each row of the database
where \meta{condition} is met. The argument \meta{assign list} is a
comma separated list of \meta{cmd}"="\meta{key} pairs. At the
start of each row, each command \meta{cmd} in \meta{assign list}
will be set to the value of the entry given by \meta{key}.
These commands may then be used in \meta{text}.

\begin{important}
Note that this assignment is done globally to ensure
that \cs{DLTforeach} works correctly in a \env{tabular} environment.
Since you may want to use the same set of commands in a later
\cs{DTLforeach}, the commands are not checked to determine if they
already exist. It is therefore important that you check you are not
using an existing command whose value should not be changed.
\end{important}

The optional argument \meta{condition} is a condition in the
form allowed by \cs{ifthenelse}. This includes the commands
provided by the \sty{ifthen} package (such as \cs{not}, \cs{and},
\cs{or}), as well as the commands
described in \autoref{sec:ifthen}. The default value of 
\meta{condition} is "\boolean{true}".

The starred version \cs{DTLforeach*} is a read-only version.
If you want to modify the database using any of the commands
described in \autoref{sec:editdb}, you must use the unstarred
version. The starred version is faster.

\begin{important}
As is generally the case with command arguments, verbatim (for
example, using \cs{verb} or the \env{verbatim} environment) can't be used in any of
the arguments of \cs{DTLforeach}, specifically verbatim can't be
used in \meta{text}.
\end{important}

There are also environment alternatives:
\begin{definition}[\DescribeEnv{DTLenvforeach}]%
\cs{begin}\{DTLenvforeach\}\oarg{condition}\marg{db name}\marg{assign list}
\end{definition}
\begin{definition}[\DescribeEnv{DTLenvforeach*}]%
\cs{begin}\{DTLenvforeach*\}\oarg{condition}\marg{db name}\marg{assign list}
\end{definition}
However, note that since these environments gather the contents of
their body, they also suffer from the above limitation.
\begin{important}
Verbatim can't be used in the body of \env{DTLenvforeach} or
\env{DTLenvforeach*}.
\end{important}

\begin{example}{Student scores}{ex:scores}
Suppose you have a data file called \texttt{studentscores.csv} that 
contains the following:
\begin{verbatim}
FirstName,Surname,StudentNo,Score
John,"Smith, Jr",102689,68
Jane,Brown,102647,75
Andy,Brown,103569,42
Z\"oe,Adams,105987,52
Roger,Brady,106872,58
Clare,Verdon,104356,45
\end{verbatim}
\DTLnewdb{scores}\relax
\DTLnewrow{scores}\relax
\DTLnewdbentry{scores}{FirstName}{John}\relax
\DTLnewdbentry{scores}{Surname}{Smith, Jr}\relax
\DTLnewdbentry{scores}{StudentNo}{102689}\relax
\DTLnewdbentry{scores}{Score}{68}\relax
\DTLnewrow{scores}\relax
\DTLnewdbentry{scores}{FirstName}{Jane}\relax
\DTLnewdbentry{scores}{Surname}{Brown}\relax
\DTLnewdbentry{scores}{StudentNo}{102647}\relax
\DTLnewdbentry{scores}{Score}{75}\relax
\DTLnewrow{scores}\relax
\DTLnewdbentry{scores}{FirstName}{Andy}\relax
\DTLnewdbentry{scores}{Surname}{Brown}\relax
\DTLnewdbentry{scores}{StudentNo}{103569}\relax
\DTLnewdbentry{scores}{Score}{42}\relax
\DTLnewrow{scores}\relax
\DTLnewdbentry{scores}{FirstName}{Z\"oe}\relax
\DTLnewdbentry{scores}{Score}{52}\relax
\DTLnewdbentry{scores}{StudentNo}{105987}\relax
\DTLnewdbentry{scores}{Surname}{Adams}\relax
\DTLnewrow{scores}\relax
\DTLnewdbentry{scores}{FirstName}{Roger}\relax
\DTLnewdbentry{scores}{Score}{58}\relax
\DTLnewdbentry{scores}{StudentNo}{106872}\relax
\DTLnewdbentry{scores}{Surname}{Brady}\relax
\DTLnewrow{scores}\relax
\DTLnewdbentry{scores}{FirstName}{Clare}\relax
\DTLnewdbentry{scores}{Score}{45}\relax
\DTLnewdbentry{scores}{StudentNo}{104356}\relax
\DTLnewdbentry{scores}{Surname}{Verdon}\relax
and you load the data into a database called "scores" using:
\begin{verbatim}
\DTLloaddb{scores}{studentscores.csv}
\end{verbatim}
you can then display the database in a table as follows:
\begin{verbatim}
\begin{table}[htbp]
\caption{Student scores}
\centering
\begin{tabular}{llr}
\bfseries First Name &
\bfseries Surname &
\bfseries Score (\%)%
\DTLforeach{scores}{%
\firstname=FirstName,\surname=Surname,\score=Score}{%
\\% start new row
\firstname & \surname & \score}
\end{tabular}
\end{table}
\end{verbatim}
This produces \autoref{tab:scores}. (Note that since I didn't
need the student registration number, I didn't bother to 
assign a command to the key "StudentNo".)

\begin{important}
Note that the new row command \idbs\ has been placed at the
start of the final argument in the above example. This is necessary
as placing it at the end of the argument will cause an unwanted row
at the end of the table. This is a feature of the loop mechanism.
\end{important}

\begin{table}[htbp]
\caption[Student scores (displaying a database in a 
table)]{Student scores}\label{tab:scores}
\centering
\begin{tabular}{llr}
\bfseries First Name &
\bfseries Surname &
\bfseries Score (\%)\relax
\DTLforeach{scores}{\firstname=FirstName,\surname=Surname,\score=Score}{\relax
\\
\firstname & \surname & \score}
\end{tabular}
\end{table}
\end{example}

The macro \ics{DTLforeach} may be nested up to three times. Each
level uses the corresponding counters: 
\desctr{DTLrowi},
\desctr{DTLrowii} and
\desctr{DTLrowiii} which keep track of
the current row.

\begin{important}
Note that these counters are only incremented when
\meta{condition} is satisfied, therefore they will not have the
correct value in \meta{condition}. These counters are incremented
using \cs{refstepcounter} before the start of \meta{text}, so they
may be referenced using \cs{label}, however remember that \cs{label}
references the last counter to be incremented using
\cs{refstepcounter} \emph{in the current scope}. The \cs{label}
should therefore be the first command in \meta{text} to ensure that
it references the current row counter.
\end{important}

\begin{definition}[\DescribeMacro{\DTLcurrentindex}]%
\cs{DTLcurrentindex}
\end{definition}
At the start of each iteration in \cs{DTLforeach}, 
\cs{DTLcurrentindex} is set to the arabic value of the current row 
counter. Note that this is only set after the condition is tested,
so it should only be used in the body of \cs{DTLforeach} not in
the condition. It is also only set locally, so if you use it in
a tabular environment, it can only be used before the first instance
of \verb|\\| or \verb|&| in the current iteration.

Within the body of \ics{DTLforeach} (i.e.\ within \meta{text})
the following conditionals may be used:
\begin{definition}[\DescribeMacro{\DTLiffirstrow}]%
\cs{DTLiffirstrow}\marg{true part}\marg{false part}
\end{definition}
If the current row is the first row, then do \meta{true part},
otherwise do \meta{false part}.

\begin{definition}[\DescribeMacro{\DTLiflastrow}]%
\cs{DTLiflastrow}\marg{true part}\marg{false part}
\end{definition}
If the current row is the last row, then do \meta{true part},
otherwise do \meta{false part}.

\begin{definition}[\DescribeMacro{\DTLifoddrow}]%
\cs{DTLifoddrow}\marg{true part}\marg{false part}
\end{definition}
If the current row number is an odd number, then do \meta{true part},
otherwise do \meta{false part}.

\begin{definition}[\DescribeMacro{\DTLsavelastrowcount}]%
\cs{DTLsavelastrowcount}\marg{cmd}
\end{definition}
This command will store the value of the row counter used in
the last occurrence of \ics{DTLforeach} in the control sequence
\meta{cmd}.

\begin{definition}[\DescribeMacro{\DTLforeachkeyinrow}]%
\cs{DTLforeachkeyinrow}\marg{cmd}\marg{text}
\end{definition}
This iterates through each key in the current row, (globally) assigns
\meta{cmd} to the value of that key, and does \meta{text}
(\meta{cmd} must be a control sequence and may be used in
\meta{text}). This command may only be used in the body of
\cs{DTLforeach}. At each iteration, \cs{DTLforeachkeyinrow} sets
\cs{dtlkey} to the current key, \cs{dtlcol} to the current column
index, \cs{dtltype} to the data type for the current column,
and \cs{dtlheader} to the header for the current column. Note that
\cs{dtltype} corresponds to the column type but if the entries in
the column have mixed types, it may not correspond to the type
of the current entry.

\begin{definition}[\DescribeMacro{\dtlbreak}]%
\cs{dtlbreak}
\end{definition}
You can break out of most of the loops provided by \sty{datatool} 
using \cs{dtlbreak}. Note, however, that it doesn't break the loop
until the end of the current iteration. There is no provision for
a "next" or "continue" style command.

Additional loop commands provided by \sty{datatool} are
described in the documented code (datatool-code.pdf).

\begin{example}{Student Scores---Labelling}{ex:label}
In the previous example, the student scores, stored in the
database "scores" were placed in a table. In this example the
table will be modified slightly to number each student according
to the row. Suppose I also want to identify which row Jane Brown
is in, and reference it in the text. The easiest way to do this
is to construct a label on each row which uniquely identifies
that student. The label can't simply be constructed from the
surname, as there are two students with the same surname. In order
to create a unique label, I can either construct a label from
both the surname and the first name, or I can use the student's
registration number, or I can use the student's score, since all
the scores are unique. The former method will cause a problem since
one of the names (Z\"oe) contains an accent command. Although
the registration numbers are all unique, they are not particularly
memorable, so I shall instead use the scores.
\begin{verbatim}
\begin{table}[htbp]
\caption{Student scores}
\centering
\begin{tabular}{cllc}
\bfseries Row &
\bfseries First Name &
\bfseries Surname &
\bfseries Score (\%)%
\DTLforeach*{scores}{%
\firstname=FirstName,\surname=Surname,\score=Score}{%
\label{row:\score}\\\theDTLrowi &
\firstname & \surname & \score}%
\end{tabular}
\end{table}

Jane Brown scored the highest (75\%), her score can be seen on 
row~\ref{row:75}.
\end{verbatim}
This produces \autoref{tab:scoreslab} and the following text:
Jane Brown scored the highest (75\%), her score can be seen on 
row~\ref*{row:75}.

Notes:
\begin{itemize}
\item the \cs{label} command is placed before
"\\" to ensure that it is in the same scope as the command
"\refstepcounter{DTLrowi}".

\item To avoid unwanted spaces the end of line characters are
commented out with the percent (\texttt{\%}) symbol.
\end{itemize}

\begin{table}[htbp]
\caption[Student scores (labelling rows)]{Student 
scores}\label{tab:scoreslab}
\centering
\begin{tabular}{cllc}
\bfseries Row &
\bfseries First Name &
\bfseries Surname &
\bfseries Score (\%)\relax
\DTLforeach*{scores}{\firstname=FirstName,\surname=Surname,\score=Score}{\relax
\label{row:\score}\\\theDTLrowi &
\firstname & \surname & \score}\relax
\end{tabular}
\end{table}
\end{example}

\begin{example}{Filtering Rows}{ex:filter}
As mentioned earlier, the optional argument \meta{condition} of 
\ics{DTLforeach} provides a means to exclude certain rows.
This example uses the database defined in \autoref{ex:scores},
but only displays the information for students whose marks are
above 60. At the end of the table, \cs{DTLsavelastrowcount}
is used to store the number of rows in the table. (Note that
\cs{DTLsavelastrowcount} is outside of \ics{DTLforeach}.)
\begin{verbatim}
\begin{table}[htbp]
\caption{Top student scores}
\centering
\begin{tabular}{llr}
\bfseries First Name &
\bfseries Surname &
\bfseries Score (\%)%
\DTLforeach*[\DTLisgt{\score}{60}]{scores}{%
\firstname=FirstName,\surname=Surname,\score=Score}{%
\\
\firstname & \surname & \score}
\end{tabular}

\DTLsavelastrowcount{\n}%
\n\ students scored above 60\%.
\end{table}
\end{verbatim}
This produces \autoref{tab:topscores}. Note that in this example,
I could have specified the condition as "\score>60" since all
the scores are integers, however, as it's possible that an entry
may feasibly have a decimal score I have used \ics{DTLisgt} instead.

\begin{table}[htbp]
\caption[Top student scores (filtering rows using 
\cs{DTLisgt})]{Top student scores}\label{tab:topscores}
\centering
\begin{tabular}{llr}
\bfseries First Name &
\bfseries Surname &
\bfseries Score (\%)\relax
\DTLforeach*[\DTLisgt{\score}{60}]{scores}{\firstname=FirstName,\surname=Surname,\score=Score}{\relax
\\
\firstname & \surname & \score}
\end{tabular}

\DTLsavelastrowcount{\n}\relax
\n\ students scored above 60\%.
\end{table}

Suppose now, I only want to display the scores for students whose
surname begins with `B'. I can do this as follows:
\begin{verbatim}
\begin{table}[htbp]
\caption{Student scores (B)}
\centering
\begin{tabular}{llr}
\bfseries First Name &
\bfseries Surname &
\bfseries Score (\%)%
\DTLforeach*[\DTLisopenbetween{\surname}{B}{C}]{scores}{%
\firstname=FirstName,\surname=Surname,\score=Score}{%
\\
\firstname & \surname & \score}
\end{tabular}
\end{table}
\end{verbatim}
This produces \autoref{tab:Bscores}.

\begin{table}[htbp]
\caption[Student scores (B) --- filtering rows using \newline
\cs{DTLisopenbetween}]{Student scores (B)}\label{tab:Bscores}
\centering
\begin{tabular}{llr}
\bfseries First Name &
\bfseries Surname &
\bfseries Score (\%)\relax
\DTLforeach*[\DTLisopenbetween{\surname}{B}{C}]{scores}
{\firstname=FirstName,\surname=Surname,\score=Score}{\relax
\\
\firstname & \surname & \score}
\end{tabular}
\end{table}

\end{example}

\begin{example}{Checking for the First Row (booktabs)}{ex:iffirst}
Suppose I want to use the \sty{booktabs} package and I want to use
\ics{midrule} after the header row. I can use \ics{DTLiffirstrow} to
check if the loop is on the first row of the iteration. (Remember
that you need to load the \sty{booktabs} package in the preamble
with \cs{usepackage}.) Using the same database as before:
\begin{verbatim}
\begin{table}[htbp]
\caption{Student scores (booktabs)}
\centering
\begin{tabular}{llc}
\toprule
\bfseries First Name &
\bfseries Surname &
\bfseries Score (\%)%
\DTLforeach*{scores}%
 {\firstname=FirstName,\surname=Surname,\score=Score}%
 {%
   \DTLiffirstrow{\\\midrule}{\\}%
   \firstname & \surname & \score
 }% end of loop
 \\\bottomrule
\end{tabular}%
\end{table}
\end{verbatim}
(The commands \ics{toprule}, \ics{midrule} and \ics{bottomrule} are
all provided by \sty{booktabs}.) This produces
\autoref{tab:iffirst}.

\begin{table}[htbp]
\caption{Student scores (booktabs)}
\label{tab:iffirst}
\centering
\begin{tabular}{llc}
\toprule
\bfseries First Name &
\bfseries Surname &
\bfseries Score (\%)%
\DTLforeach*{scores}{%
\firstname=FirstName,\surname=Surname,\score=Score}{%
\DTLiffirstrow{\\\midrule}{\\}%
\firstname & \surname & \score}%
\\\bottomrule
\end{tabular}%
\end{table}

\end{example}

\begin{example}{Breaking Out of a Loop}{ex:dtlbreak}
Suppose I only want to display the first three rows of a database.
I could do:\footnote{Recall that \ctr{DTLrowi} is incremented
after the condition is tested, so it will be out by 1 when the
condition is tested which is why \texttt{<3} is used instead of
\texttt{<4}.}
\begin{verbatim}
\DTLforeach*[\value{DTLrowi}<3]{scores}%
{\firstname=FirstName,\surname=Surname,\score=Score}{%
\\\firstname & \surname & \score
}
\end{verbatim}
However, this isn't very efficient, as it still has to iterate
through the entire database, checking if the condition is met. If
the database has over 100 entries, this will slow the time taken
to create the table. It would therefore be much more efficient
to break out of the loop when row count exceeds 3:
\begin{verbatim}
\begin{table}[htbp]
\caption{First Three Rows}
\centering
\begin{tabular}{llr}
\bfseries First Name & \bfseries Surname & \bfseries Score (\%)%
\DTLforeach*{scores}%
{\firstname=FirstName,\surname=Surname,\score=Score}{%
\ifthenelse{\DTLcurrentindex=3}{\dtlbreak}{}%
\\\firstname & \surname & \score
}%
\end{tabular}
\end{table}
\end{verbatim}
This produces \autoref{tab:dtlbreak}. Note that the loop is not
broken until the end of the current iteration, so even though
\cs{dtlbreak} occurs at the start of the third row, the loop isn't
finished until the third row is completed. (Recall that 
\cs{DTLcurrentindex} must be used before the first instance of
\verb|\\| or \verb|&|.) Alternatively, you can use 
\ctr{DTLrowi} instead:
\begin{verbatim}
\DTLforeach{scores}%
{\firstname=FirstName,\surname=Surname,\score=Score}{%
\\\firstname & \surname & \score
\ifthenelse{\value{DTLrowi}=3}{\dtlbreak}{}%
}%
\end{verbatim}

\begin{table}[htbp]
\caption{First Three Rows}
\label{tab:dtlbreak}%
\centering
\begin{tabular}{llr}
\bfseries First Name & \bfseries Surname & \bfseries Score (\%)\relax
\DTLforeach*{scores}%
{\firstname=FirstName,\surname=Surname,\score=Score}{%
\ifthenelse{\DTLcurrentindex=3}{\dtlbreak}{}\relax
\\\firstname & \surname & \score
}%
\end{tabular}
\end{table}
\end{example}

\begin{example}{Stripy Tables}{ex:stripy}
This example uses the same database as in the previous examples.
It requires the \sty{colortbl} package, which provides the 
command \cs{rowcolor}. The command \cs{DTLifoddrow} is used
to produce a striped table.
\begin{verbatim}
\begin{table}[htbp]
\caption{A stripy table}\label{tab:stripy}
\centering
\begin{tabular}{llc}
\bfseries First Name &
\bfseries Surname &
\bfseries Score (\%)%
\DTLforeach*{scores}{%
\firstname=FirstName,\surname=Surname,\score=Score}{%
\\\DTLifoddrow{\rowcolor{blue}}{\rowcolor{green}}%
\firstname & \surname & \score}%
\end{tabular}
\end{table}
\end{verbatim}
This produces \autoref{tab:stripy}.

\begin{table}[htbp]
\caption[A stripy table (illustrating the use of 
\cs{DTLifoddrow})]{A stripy table}\label{tab:stripy}
\centering
\begin{tabular}{llc}
\bfseries First Name &
\bfseries Surname &
\bfseries Score (\%)\relax
\DTLforeach*{scores}{\firstname=FirstName,\surname=Surname,\score=Score}{\relax
\\\DTLifoddrow{\rowcolor{blue}}{\rowcolor{green}}\relax
\firstname & \surname & \score}\relax
\end{tabular}
\end{table}
\end{example}

\begin{example}{Two Database Rows per Tabular Row}{ex:2rows}
In order to save space, you may want two database rows per
tabular row, when displaying a database in a \env{tabular} 
environment. This can be accomplished using \ics{DTLifoddrow}.
For example
\begin{verbatim}
\begin{table}[htbp]
\caption{Two database rows per tabular row}
\centering
\begin{tabular}{llcllc}
\bfseries First Name &
\bfseries Surname &
\bfseries Score (\%) &
\bfseries First Name &
\bfseries Surname &
\bfseries Score (\%)%
\DTLforeach*{scores}{\firstname=FirstName,\surname=Surname,\score=Score}{%
\DTLifoddrow{\\}{&}%
\firstname & \surname & \score}%
\end{tabular}
\end{table}
\end{verbatim}
produces \autoref{tab:2rows}

\begin{table}[htbp]
\caption[Two database rows per tabular row (illustrating the
use of\newline \cs{DTLifoddrow})]{Two database rows per tabular 
row}\label{tab:2rows}
\centering
\begin{tabular}{llcllc}
\bfseries First Name &
\bfseries Surname &
\bfseries Score (\%) &
\bfseries First Name &
\bfseries Surname &
\bfseries Score (\%)\relax
\DTLforeach*{scores}{\firstname=FirstName,\surname=Surname,\score=Score}{\relax
\DTLifoddrow{\\}{&}\relax
\firstname & \surname & \score}\relax
\end{tabular}
\end{table}

(To order column-wise, instead of row-wise, see
\autoref{ex:twoblocks}.)
\end{example}

\begin{example}{Iterating Through Keys in a Row}{ex:foreachkey}
Suppose you have lots of columns in your database, and you
want to display them all without having to set a variable for
each column. You can leave the assignment list in \cs{DTLforeach} 
blank, and iterate through the keys using \cs{DTLforeachkeyinrow}.
For example:
\begin{verbatim}
\begin{table}[htbp]
\caption{Student Scores (Iterating Through Keys)}
\centering
\begin{tabular}{llll}
\bfseries First Name & \bfseries Surname &
\bfseries Registration No. &
\bfseries Score (\%)%
\DTLforeach*{scores}{}{%
\\\gdef\doamp{\gdef\doamp{&}}%
\DTLforeachkeyinrow{\thisValue}{\doamp\thisValue}}%
\end{tabular}
\end{table}
\end{verbatim}
This produces \autoref{tab:foreachkey}.
\begin{table}[htbp]
\caption{Student Scores (Iterating Through Keys)}
\label{tab:foreachkey}
\centering
\begin{tabular}{llll}
\bfseries First Name & \bfseries Surname &
\bfseries Registration No. &
\bfseries Score (\%)\relax
\DTLforeach*{scores}{}{%
\\\gdef\doamp{\gdef\doamp{&}}\relax
\DTLforeachkeyinrow{\thisValue}{\doamp\thisValue}}\relax
\end{tabular}
\end{table}

Note that the "&" must be between columns, so I have defined
a command called \cs{doamp} that on first use redefines 
itself to do "&". So, for each row, at the start of
the key iteration, \cs{doamp} does nothing, and on subsequent
iterations it does "&". This ensures that the correct number of
"&"s are used. Since each cell in the \env{tabular} environment
is scoped, \cs{gdef} is needed instead of \cs{def}.

In the above, I needed to know how many columns are in the
database, and the order that the headings should appear. If you
are unsure, you can use \cs{dtlforeachkey} to determine the
number of columns and to display the header row. For example:
\begin{verbatim}
 \begin{table}[htbp]
 \caption{Student Scores}
 \centering
 % Work out the column alignments.
 \def\colalign{}%
 \dtlforeachkey(\theKey,\theCol,\theType,\theHead)\in{scores}\do
 {\edef\colalign{\colalign l}}%
 % Begin the tabular environment.
 \edef\dobegintabular{\noexpand\begin{tabular}{\colalign}}%
 \dobegintabular
 % Do the header row.
 \gdef\doamp{\gdef\doamp{&}}%
 \dtlforeachkey(\theKey,\theCol,\theType,\theHead)\in{scores}\do
 {\doamp\bfseries \theHead}%
 % Iterate through the data.
 \DTLforeach*{scores}{}{%
 \\\gdef\doamp{\gdef\doamp{&}}%
 \DTLforeachkeyinrow{\thisValue}{\doamp\thisValue}}%
 \end{tabular}
 \end{table}
\end{verbatim}

\begin{table}[htbp]
\caption[Student Scores (Using \cs{dtlforeachkey} and \newline
\cs{DTLforeachkeyinrow})]{Student Scores (Using \cs{dtlforeachkey} and 
\cs{DTLforeachkeyinrow})}
\label{tab:foreachkey2}
\centering
\def\colalign{}%
\dtlforeachkey(\theKey,\theCol,\theType,\theHead)\in{scores}\do
{\edef\colalign{\colalign l}}%
\edef\dobegintabular{\noexpand\begin{tabular}{\colalign}}\relax
\dobegintabular
\gdef\doamp{\gdef\doamp{&}}\relax
\dtlforeachkey(\theKey,\theCol,\theType,\theHead)\in{scores}\do
{\doamp\bfseries \theHead}\relax
\DTLforeach*{scores}{}{%
\\\gdef\doamp{\gdef\doamp{&}}\relax
\DTLforeachkeyinrow{\thisValue}{\doamp\thisValue}}\relax
\end{tabular}
\end{table}

Notes:
\begin{itemize}
\item In order to determine the column alignment for the
\env{tabular} environment, I first define \cs{colalign} to
nothing, and then I iterate through the keys appending
\texttt{l} to \cs{colalign}. Since \cs{colalign} only contains
alphabetical characters, I can just use \cs{edef} for this. I
could modify this to check the data type and, say, use \texttt{l} 
(left alignment) for columns containing strings and \texttt{c}
(centred) for the other columns:
\begin{verbatim}
\dtlforeachkey(\theKey,\theCol,\theType,\theHead)\in{scores}\do
{\ifnum\theType=0\relax
   \edef\colalign{\colalign l}% column contains strings
 \else
   \edef\colalign{\colalign c}% column contains numerical values
 \fi
}%
\end{verbatim}

\item To ensure \cs{colalign} gets correct expanded when passed
to the \env{tabular} environment I temporarily define
\cs{dobegintabular} to the code required to start the 
\env{tabular} environment:
\begin{verbatim}
\edef\dobegintabular{\noexpand\begin{tabular}{\colalign}}%
\end{verbatim}
This sets \cs{dobegintabular} to \verb|\begin{tabular}{llll}|.
After defining \cs{dobegintabular}, I then need to use it.

\item As before, I use \cs{doamp} to put the ampersands between
columns.

\item Recall that I can set the headers using \cs{DTLsetheader}
or using the \csopt{DTLloaddb}{headers} key when loading the data 
from an external file. For example:
\begin{verbatim}
\DTLsetheaders{scores}{FirstName}{First Name}
\DTLsetheaders{scores}{Score}{Score (\%)}
\end{verbatim}
\end{itemize}

Recall that \cs{DTLforeachkeyinrow} sets \cs{dtlkey} to the
current key. This can be used to filter out columns. Alternatively,
if you know the column index, you can test \cs{dtlcol} instead.
The following code modifies the above example so that it filters
out the column whose key is \texttt{StudentNo}:
\begin{verbatim}
\begin{table}[htbp]
\caption{Student Scores (Filtering Out a Column)}
\centering
\def\colalign{}%
\dtlforeachkey(\theKey,\theCol,\theType,\theHead)\in{scores}\do
{\DTLifeq{\theKey}{StudentNo}{}{\edef\colalign{\colalign l}}}%
\edef\dobegintabular{\noexpand\begin{tabular}{\colalign}}%
\dobegintabular
\gdef\doamp{\gdef\doamp{&}}%
\dtlforeachkey(\theKey,\theCol,\theType,\theHead)\in{scores}\do
{\DTLifeq{\theKey}{StudentNo}{}{\doamp\bfseries \theHead}}%
\DTLforeach*{scores}{}{%
\\\gdef\doamp{\gdef\doamp{&}}%
\DTLforeachkeyinrow{\thisValue}{%
  \DTLifeq{\dtlkey}{StudentNo}{}{\doamp\thisValue}}}%
\end{tabular}
\end{table}
\end{verbatim}
The result is shown in \autoref{tab:foreachkey3}.

\begin{table}[htbp]
\caption{Student Scores (Filtering Out a Column)}
\label{tab:foreachkey3}
\centering
\def\colalign{}%
\dtlforeachkey(\theKey,\theCol,\theType,\theHead)\in{scores}\do
{\DTLifeq{\theKey}{StudentNo}{}{\edef\colalign{\colalign l}}}%
\edef\dobegintabular{\noexpand\begin{tabular}{\colalign}}\relax
\dobegintabular
\gdef\doamp{\gdef\doamp{&}}\relax
\dtlforeachkey(\theKey,\theCol,\theType,\theHead)\in{scores}\do
{\DTLifeq{\theKey}{StudentNo}{}{\doamp\bfseries \theHead}}\relax
\DTLforeach*{scores}{}{%
\\\gdef\doamp{\gdef\doamp{&}}\relax
\DTLforeachkeyinrow{\thisValue}{\relax
\DTLifeq{\dtlkey}{StudentNo}{}{\doamp\thisValue}}}\relax
\end{tabular}
\end{table}
\end{example}

\begin{example}{Nested \cs{DTLforeach}}{ex:nested}
In this example I have a CSV file called "index.csv" which
contains:
\begin{verbatim}
File,Temperature,NaCl,pH
exp25a.csv,25,4.7,0.5
exp25b.csv,25,4.8,1.5
exp30a.csv,30,5.12,4.5
\end{verbatim}
\DTLnewdb{index}\relax
\DTLnewrow{index}\relax
\DTLnewdbentry{index}{File}{exp25a.csv}\relax
\DTLnewdbentry{index}{Temperature}{25}\relax
\DTLnewdbentry{index}{NaCl}{4.7}\relax
\DTLnewdbentry{index}{pH}{0.5}\relax
\DTLnewrow{index}\relax
\DTLnewdbentry{index}{File}{exp25b.csv}\relax
\DTLnewdbentry{index}{Temperature}{25}\relax
\DTLnewdbentry{index}{NaCl}{4.8}\relax
\DTLnewdbentry{index}{pH}{1.5}\relax
\DTLnewrow{index}\relax
\DTLnewdbentry{index}{File}{exp30a.csv}\relax
\DTLnewdbentry{index}{Temperature}{30}\relax
\DTLnewdbentry{index}{NaCl}{5.12}\relax
\DTLnewdbentry{index}{pH}{4.5}\relax
The first column of this file contains the name of another
CSV file which has the results of a time to growth experiment 
performed at the given incubation temperature, salt concentration
and pH. The file "exp25a.csv" contains the following:
\begin{verbatim}
Time,Log Count
0,3.75
23,3.9
45,4.0
\end{verbatim}
\DTLnewdb{exp25a.csv}\relax
\DTLnewrow{exp25a.csv}\relax
\DTLnewdbentry{exp25a.csv}{Time}{0}\relax
\DTLnewdbentry{exp25a.csv}{Log Count}{3.75}\relax
\DTLnewrow{exp25a.csv}\relax
\DTLnewdbentry{exp25a.csv}{Time}{23}\relax
\DTLnewdbentry{exp25a.csv}{Log Count}{3.9}\relax
\DTLnewrow{exp25a.csv}\relax
\DTLnewdbentry{exp25a.csv}{Time}{45}\relax
\DTLnewdbentry{exp25a.csv}{Log Count}{4.0}\relax
The file "exp25b.csv" contains the following:
\begin{verbatim}
Time,Log Count
0,3.6
60,3.8
120,4.0
\end{verbatim}
\DTLnewdb{exp25b.csv}\relax
\DTLnewrow{exp25b.csv}\relax
\DTLnewdbentry{exp25b.csv}{Time}{0}\relax
\DTLnewdbentry{exp25b.csv}{Log Count}{3.6}\relax
\DTLnewrow{exp25b.csv}\relax
\DTLnewdbentry{exp25b.csv}{Time}{60}\relax
\DTLnewdbentry{exp25b.csv}{Log Count}{3.8}\relax
\DTLnewrow{exp25b.csv}\relax
\DTLnewdbentry{exp25b.csv}{Time}{120}\relax
\DTLnewdbentry{exp25b.csv}{Log Count}{4.0}\relax
The file "exp30a.csv" contains the following:
\begin{verbatim}
Time,Log Count
0,3.73
23,3.67
60,4.9
\end{verbatim}
\DTLnewdb{exp30a.csv}\relax
\DTLnewrow{exp30a.csv}\relax
\DTLnewdbentry{exp30a.csv}{Time}{0}\relax
\DTLnewdbentry{exp30a.csv}{Log Count}{3.73}\relax
\DTLnewrow{exp30a.csv}\relax
\DTLnewdbentry{exp30a.csv}{Time}{23}\relax
\DTLnewdbentry{exp30a.csv}{Log Count}{3.67}\relax
\DTLnewrow{exp30a.csv}\relax
\DTLnewdbentry{exp30a.csv}{Time}{60}\relax
\DTLnewdbentry{exp30a.csv}{Log Count}{4.9}\relax
Suppose I now want to iterate through "index.csv", load the given
file, and create a table for that data. I can do this using
nested \ics{DTLforeach} as follows:
\begin{verbatim}
 % load index data file
\DTLloaddb{index}{index.csv}

 % iterate through index database
\DTLforeach{index}{\theFile=File,\theTemp=Temperature,%
\theNaCl=NaCl,\thepH=pH}{%
 % load results file into database of the same name
\DTLloaddb{\theFile}{\theFile}%
 % Create a table
\begin{table}[htbp]
\caption{Temperature = \theTemp, NaCl = \theNaCl,
pH = \thepH}\label{tab:\theFile}
\centering
\begin{tabular}{rl}
\bfseries Time & \bfseries Log Count
\DTLforeach{\theFile}{\theTime=Time,\theLogCount=Log Count}{%
\\\theTime & \theLogCount}%
\end{tabular}
\end{table}
}
\end{verbatim}
This creates \autoref{tab:exp25a.csv} to \autoref{tab:exp30a.csv}.
(Note that each table is given a label that is based on the
database name, to ensure that it is unique.)

\DTLforeach{index}{\theFile=File,\theTemp=Temperature,\theNaCl
=NaCl,\thepH=pH}{
\begin{table}[htbp]
\caption[Temperature = \theTemp, NaCl = \theNaCl,
pH = \thepH\space (illustrating nested 
\cs{DTLforeach})]{Temperature = \theTemp, NaCl = \theNaCl,
pH = \thepH}\label{tab:\theFile}
\centering
\begin{tabular}{rl}
\bfseries Time & \bfseries Log Count
\DTLforeach{\theFile}{\theTime=Time,\theLogCount=Log Count}{
\\\theTime & \theLogCount}%
\end{tabular}
\end{table}
}
\end{example}

\begin{example}{Dynamically Allocating Field Name}{ex:dyn}
(This example was suggested by Bill~Hobbs.) Suppose you have a
directory containing members of multiple clubs. The CSV file
(say, \texttt{clubs.csv}) may look something like:
\begin{verbatim}
First Name,Surname,Rockin,Single
John,"Smith, Jr",member,
Jane,Brown,,friend
Andy,Brown,friend,member
Z\"oe,Adams,member,member
Roger,Brady,friend,friend
Clare,Verdon,member,
\end{verbatim}
\DTLnewdb{clubs}\relax
\DTLnewrow{clubs}\relax
\DTLnewdbentry{clubs}{First Name}{John}\relax
\DTLnewdbentry{clubs}{Surname}{Smith, Jr}\relax
\DTLnewdbentry{clubs}{Rockin}{member}\relax
\DTLnewdbentry{clubs}{Single}{}\relax
\DTLnewrow{clubs}\relax
\DTLnewdbentry{clubs}{First Name}{Jane}\relax
\DTLnewdbentry{clubs}{Surname}{Brown}\relax
\DTLnewdbentry{clubs}{Rockin}{}\relax
\DTLnewdbentry{clubs}{Single}{friend}\relax
\DTLnewrow{clubs}\relax
\DTLnewdbentry{clubs}{First Name}{Andy}\relax
\DTLnewdbentry{clubs}{Surname}{Brown}\relax
\DTLnewdbentry{clubs}{Rockin}{friend}\relax
\DTLnewdbentry{clubs}{Single}{member}\relax
\DTLnewrow{clubs}\relax
\DTLnewdbentry{clubs}{First Name}{Z\"oe}\relax
\DTLnewdbentry{clubs}{Surname}{Adams}\relax
\DTLnewdbentry{clubs}{Rockin}{member}\relax
\DTLnewdbentry{clubs}{Single}{member}\relax
\DTLnewrow{clubs}\relax
\DTLnewdbentry{clubs}{First Name}{Roger}\relax
\DTLnewdbentry{clubs}{Surname}{Brady}\relax
\DTLnewdbentry{clubs}{Rockin}{friend}\relax
\DTLnewdbentry{clubs}{Single}{friend}\relax
\DTLnewrow{clubs}\relax
\DTLnewdbentry{clubs}{First Name}{Clare}\relax
\DTLnewdbentry{clubs}{Surname}{Verdon}\relax
\DTLnewdbentry{clubs}{Rockin}{member}\relax
\DTLnewdbentry{clubs}{Single}{}\relax
(Blank entries indicate that the person is not a member of that
club.) The data can be loaded as follows:
\begin{verbatim}
\DTLloaddb{clubs}{clubs.csv}
\end{verbatim}
Suppose at the beginning of your document you have specified
which club you are interested in ("Rockin" or "Single") and
store it in \cs{DIdent}:
\begin{verbatim}
\newcommand{\DIdent}{Rockin}
\end{verbatim}
\newcommand{\DIdent}{Rockin}\relax
You can now display the members for this particular club as
follows:
\begin{verbatim}
\begin{table}[htbp]
\caption{Club Membership}
\centering
\begin{tabular}{lll}
\bfseries First Name & \bfseries Surname & \bfseries Status
\DTLforeach*[\not\DTLiseq{\status}{}]{clubs}
{\firstname=First Name,\surname=Surname,\status=\DIdent}{%
\\\firstname & \surname & \status
}%
\end{tabular}
\end{table}
\end{verbatim}
The result is shown in \autoref{tab:dyn}.

\begin{table}[htbp]
\caption{Club Membership}
\label{tab:dyn}
\centering
\begin{tabular}{lll}
\bfseries First Name & \bfseries Surname & \bfseries Status
\DTLforeach*[\not\DTLiseq{\status}{}]{clubs}
{\firstname=First Name,\surname=Surname,\status=\DIdent}{\relax
\\\firstname & \surname & \status
}\relax
\end{tabular}
\end{table}
\end{example}

\section{Null Values}
\begin{important}
Note that a null value is not the same as an empty value. Empty
values can be tested using \sty{etoolbox}'s \ics{ifdefempty} or
similar.
\end{important}

If a database is created using \cs{DTLnewdb}, \cs{DTLnewrow}
and \cs{DTLnewdbentry} (rather than loading it from an ASCII
file), it is possible for some of the entries to have null values
when a value is not assigned to a given key for a given row.
It's also possible for data fetched from a~SQL database using
\app{datatooltk} to contain null values, and you can use
the \app{datatooltk} GUI to assign null values, but data loaded using
\cs{DTLloaddb} (or \cs{DTLloadrawdb}) will have empty not null
values for any \emph{blank} cells. However, you will get null values when
loading a CSV file if cells are missing (rather than empty). For
example:
\begin{verbatim}
Column 1,Column 2,Column 3
Foo,Bar
Foo,,Baz
Foo,Bar,Baz
\end{verbatim}
This has an empty cell (the second column in \texttt{Foo,,Baz}) and
a null cell (the third column in \texttt{Foo,Bar}). If the first
row of data was instead
\begin{verbatim}
Foo,Bar,
\end{verbatim}
then the third column is now empty, not null.

When you iterate through the database using \cs{DTLforeach}
(described in \autoref{sec:dbforeach}), 
if an entry is missing for a given row, the associated command given 
in the \meta{values} argument will be set to a null value. This
value depends on the data type associated with the given key.

\begin{definition}[\DescribeMacro{\DTLstringnull}]%
\cs{DTLstringnull}
\end{definition}
This is the null value for a string.

\begin{definition}[\DescribeMacro{\DTLnumbernull}]%
\cs{DTLnumbernull}
\end{definition}
This is the null value for a number.

\begin{definition}[\DescribeMacro{\DTLifnull}]%
\cs{DTLifnull}\marg{cmd}\marg{true part}\marg{false part}
\end{definition}
This checks if \meta{cmd} is null where \meta{cmd} is a control
sequence, if it is, then \meta{true part}
is done, otherwise \meta{false part} is done. This macro is
illustrated in \autoref{ex:null} below.

\begin{definition}[\DescribeMacro{\DTLifnullorempty}]
\cs{DTLifnullorempty}\marg{cmd}\marg{true part}\marg{false part}
\end{definition}
This checks if \meta{cmd} is null or empty, where \meta{cmd} is a
control sequence. If it is it does \meta{true part}, otherwise
\meta{false part}.

\begin{example}{Null vs Empty Values}{ex:null}
Consider the following (which creates a database called "emailDB"):
\begin{verbatim}
\DTLnewdb{emailDB}
\DTLnewrow{emailDB}
\DTLnewdbentry{emailDB}{Surname}{Jones}
\DTLnewdbentry{emailDB}{FirstName}{Mary}
\DTLnewdbentry{emailDB}{Email1}{mj@my.uni.ac.uk}
\DTLnewdbentry{emailDB}{Email2}{mj@somewhere.com}
\DTLnewrow{emailDB}
\DTLnewdbentry{emailDB}{Surname}{Smith}
\DTLnewdbentry{emailDB}{FirstName}{Adam}
\DTLnewdbentry{emailDB}{Email1}{as@my.uni.ac.uk}
\DTLnewdbentry{emailDB}{RegNum}{12345}
\end{verbatim}
\DTLnewdb{emailDB}\relax
\DTLnewrow{emailDB}\relax
\DTLnewdbentry{emailDB}{Surname}{Jones}\relax
\DTLnewdbentry{emailDB}{FirstName}{Mary}\relax
\DTLnewdbentry{emailDB}{Email1}{mj@my.uni.ac.uk}\relax
\DTLnewdbentry{emailDB}{Email2}{mj@somewhere.com}\relax
\DTLnewrow{emailDB}\relax
\DTLnewdbentry{emailDB}{Surname}{Smith}\relax
\DTLnewdbentry{emailDB}{FirstName}{Adam}\relax
\DTLnewdbentry{emailDB}{Email1}{as@my.uni.ac.uk}\relax
\DTLnewdbentry{emailDB}{RegNum}{12345}\relax
In the above example, the first row of the database contains
an entry with the key "Email2", but the second row doesn't.
Whereas the second row contains an entry with the key "RegNum",
but the first row doesn't. That is, this database has two null (not
empty) values.

The following code puts the information in a \env{tabular} 
environment:
\begin{verbatim}
\begin{tabular}{lllll}
\bfseries First Name &
\bfseries Surname &
\bfseries Email 1 &
\bfseries Email 2 &
\bfseries Reg Num%
\DTLforeach*{emailDB}{\firstname=FirstName,\surname=Surname,%
\emailI=Email1,\emailII=Email2,\regnum=RegNum}{%
\\\firstname & \surname & \emailI & \emailII & \regnum}%
\end{tabular}
\end{verbatim}
This produces the following:\par\vskip\baselineskip\noindent
\begin{tabular}{lllll}
\bfseries First Name &
\bfseries Surname &
\bfseries Email 1 &
\bfseries Email 2 &
\bfseries Reg Num\relax
\DTLforeach*{emailDB}{\firstname=FirstName,\surname
=Surname,\emailI=Email1,\emailII=Email2,\regnum=RegNum}{\relax
\\\firstname & \surname & \emailI & \emailII & \regnum}\relax
\end{tabular}
\par\vskip\baselineskip
Note that on the first row of data, the registration number appears as
0, while on the next row, the second email address appears as
NULL. The \sty{datatool} package has identified the key "RegNum"
for this database as a numerical key, since all elements in the
database with that key are numerical, whereas it has 
identified the key "Email2" as a string, since there is at least
one element in this database with that key that is a string.  Null
numerical values are set to \cs{DTLnumbernull} (\DTLnumbernull), and
null strings are set to \cs{DTLstringnull} (\DTLstringnull).

The following code checks each value to determine whether it
is null using \cs{DTLifnull}. If it is, the text \emph{Missing}
is inserted, otherwise the value itself is used:
\begin{verbatim}
\begin{tabular}{lllll}
\bfseries First Name &
\bfseries Surname &
\bfseries Email 1 &
\bfseries Email 2 &
\bfseries Reg Num%
\DTLforeach{emailDB}{\firstname=FirstName,\surname=Surname,%
\emailI=Email1,\emailII=Email2,\regnum=RegNum}{%
\\\DTLifnull{\firstname}{\emph{Missing}}{\firstname} & 
\DTLifnull{\surname}{\emph{Missing}}{\surname} & 
\DTLifnull{\emailI}{\emph{Missing}}{\emailI} & 
\DTLifnull{\emailII}{\emph{Missing}}{\emailII} & 
\DTLifnull{\regnum}{\emph{Missing}}{\regnum}}%
\end{tabular}
\end{verbatim}
This produces the following:\par\vskip\baselineskip\noindent
\begin{tabular}{lllll}
\bfseries First Name &
\bfseries Surname &
\bfseries Email 1 &
\bfseries Email 2 &
\bfseries Reg Num\relax
\DTLforeach{emailDB}{\firstname=FirstName,\surname
=Surname,\emailI=Email1,\emailII=Email2,\regnum=RegNum}{\relax
\\\DTLifnull{\firstname}{\emph{Missing}}{\firstname} & 
\DTLifnull{\surname}{\emph{Missing}}{\surname} & 
\DTLifnull{\emailI}{\emph{Missing}}{\emailI} & 
\DTLifnull{\emailII}{\emph{Missing}}{\emailII} & 
\DTLifnull{\regnum}{\emph{Missing}}{\regnum}}\relax
\end{tabular}
\par\vskip\baselineskip\noindent
If you want to do this, you may find it easier to define a 
convenience command that will display some appropriate text
if an entry is missing, for example:
\begin{verbatim}
\newcommand*{\checkmissing}[1]{\DTLifnull{#1}{---}{#1}}
\end{verbatim}
Then instead of typing, say, 
\begin{verbatim}
\DTLifnull{\regnum}{---}{\regnum}
\end{verbatim}
you can instead type:
\begin{verbatim}
\checkmissing{\regnum}
\end{verbatim}

Now suppose that instead of defining the database using \cs{DTLnewdb},
\cs{DTLnewrow} and \cs{DTLnewdbentry}, you have a file with the 
contents:
\begin{ttfamily}\setlength{\parindent}{0pt}\par\vskip\baselineskip
Surname,FirstName,RegNum,Email1,Email2

\DTLforeach{emailDB}{\surname=Surname,\firstname=FirstName,\regNo
=RegNum,\emailI=Email1,\emailII
=Email2}{\relax
\surname,\firstname,\DTLifnull{\regNo}{}{\regNo},\emailI,\DTLifnull{\emailII}{}{\emailII}\par
}
\end{ttfamily}\par\vskip\baselineskip\noindent
and you load the data from this file using \cs{DTLloaddb}
(defined in \autoref{sec:loaddb}). Now the
database has \emph{no null values}, but has an \emph{empty} value for the
key "RegNum" on the first row of the database, and an empty
value for the key "Email2" on the second row of the database.
Now, the following code
\begin{verbatim}
\begin{tabular}{lllll}
\bfseries First Name &
\bfseries Surname &
\bfseries Email 1 &
\bfseries Email 2 &
\bfseries Reg Number%
\DTLforeach{emailDB}{\firstname=FirstName,\surname=Surname,%
\emailI=Email1,\emailII=Email2,\regnum=RegNum}{%
\\\DTLifnull{\firstname}{\emph{Missing}}{\firstname} & 
\DTLifnull{\surname}{\emph{Missing}}{\surname} & 
\DTLifnull{\emailI}{\emph{Missing}}{\emailI} & 
\DTLifnull{\emailII}{\emph{Missing}}{\emailII} & 
\DTLifnull{\regnum}{\emph{Missing}}{\regnum}}%
\end{tabular}
\end{verbatim}
produces:\par\vskip\baselineskip\noindent
\begin{tabular}{lllll}
\bfseries First Name &
\bfseries Surname &
\bfseries Email 1 &
\bfseries Email 2 &
\bfseries Reg Number\relax
\DTLforeach{emailDB}{\firstname=FirstName,\surname
=Surname,\emailI=Email1,\emailII=Email2,\regnum=RegNum}{\relax
\\\firstname & \surname & \emailI & \DTLifnull{\emailII}{}{\emailII} & \DTLifnull{\regnum}{}{\regnum}}\relax
\end{tabular}

\medskip

Now the missing entries are simply blank instead of containing
\emph{Missing}. This is because they're empty not null. In this
case, you may prefer to use \cs{DTLifnullorempty} instead of
\cs{DTLifnull}.
\end{example}

\section{Editing Database Rows}
\label{sec:editdb}

A row can be removed from a data base using:
\begin{definition}[\DescribeMacro{\DTLremoverow}]%
\cs{DTLremoverow}\marg{db name}\marg{row index}
\end{definition}
where \meta{row index} is the index of the unwanted row. For
example:
\begin{verbatim}
\DTLremoverow{scores}{2}
\end{verbatim}
will delete the second row in the database labelled ``scores''.
There is also a starred version that doesn't check for the 
existence of the database. Remember that the row index refers to the
internal data not to a reference in the external source if the data
has been input.

The following commands may be used in the body of the
\ics{DTLforeach} loop,\footnote{Only the unstarred version of
\cs{DTLforeach}; the starred version is read-only.}\ to edit the
current row of the loop. (See also \autoref{sec:currentrow}.)

\begin{definition}[\DescribeMacro{\DTLappendtorow}]%
\cs{DTLappendtorow}\marg{key}\marg{value}
\end{definition}
This appends a new entry with the given \meta{key} and \meta{value}
to the current row.
(\meta{value} is expanded.)

\begin{definition}[\DescribeMacro{\DTLreplaceentryforrow}]%
\cs{DTLreplaceentryforrow}\marg{key}\marg{value}
\end{definition}
This replaces the entry for \meta{key} with \meta{value}. 
(\meta{value} is expanded.)

\begin{definition}[\DescribeMacro{\DTLremoveentryfromrow}]%
\cs{DTLremoveentryfromrow}\marg{key}
\end{definition}
This removes the entry for \meta{key} from the current row.

\begin{definition}[\DescribeMacro{\DTLremovecurrentrow}]%
\cs{DTLremovecurrentrow}
\end{definition}
This removes the current row from the database.

\begin{example}{Editing Database Rows}{ex:editdb}
In this example I have a CSV file called "marks.csv" that contains
student marks for three assignments:
\DTLnewdb{marks}\relax
\DTLnewrow{marks}\relax
\DTLnewdbentry{marks}{FirstName}{John}\relax
\DTLnewdbentry{marks}{Surname}{Smith, Jr}\relax
\DTLnewdbentry{marks}{StudentNo}{102689}\relax
\DTLnewdbentry{marks}{Assignment 1}{68}\relax
\DTLnewdbentry{marks}{Assignment 2}{57}\relax
\DTLnewdbentry{marks}{Assignment 3}{72}\relax
\DTLnewrow{marks}\relax
\DTLnewdbentry{marks}{FirstName}{Jane}\relax
\DTLnewdbentry{marks}{Surname}{Brown}\relax
\DTLnewdbentry{marks}{StudentNo}{102647}\relax
\DTLnewdbentry{marks}{Assignment 1}{75}\relax
\DTLnewdbentry{marks}{Assignment 2}{84}\relax
\DTLnewdbentry{marks}{Assignment 3}{80}\relax
\DTLnewrow{marks}\relax
\DTLnewdbentry{marks}{FirstName}{Andy}\relax
\DTLnewdbentry{marks}{Surname}{Brown}\relax
\DTLnewdbentry{marks}{StudentNo}{103569}\relax
\DTLnewdbentry{marks}{Assignment 1}{42}\relax
\DTLnewdbentry{marks}{Assignment 2}{52}\relax
\DTLnewdbentry{marks}{Assignment 3}{54}\relax
\DTLnewrow{marks}\relax
\DTLnewdbentry{marks}{FirstName}{Z\"oe}\relax
\DTLnewdbentry{marks}{Surname}{Adams}\relax
\DTLnewdbentry{marks}{StudentNo}{105987}\relax
\DTLnewdbentry{marks}{Assignment 1}{52}\relax
\DTLnewdbentry{marks}{Assignment 2}{48}\relax
\DTLnewdbentry{marks}{Assignment 3}{57}\relax
\DTLnewrow{marks}\relax
\DTLnewdbentry{marks}{FirstName}{Roger}\relax
\DTLnewdbentry{marks}{Surname}{Brady}\relax
\DTLnewdbentry{marks}{StudentNo}{106872}\relax
\DTLnewdbentry{marks}{Assignment 1}{58}\relax
\DTLnewdbentry{marks}{Assignment 2}{60}\relax
\DTLnewdbentry{marks}{Assignment 3}{62}\relax
\DTLnewrow{marks}\relax
\DTLnewdbentry{marks}{FirstName}{Clare}\relax
\DTLnewdbentry{marks}{Surname}{Verdon}\relax
\DTLnewdbentry{marks}{StudentNo}{104356}\relax
\DTLnewdbentry{marks}{Assignment 1}{45}\relax
\DTLnewdbentry{marks}{Assignment 2}{50}\relax
\DTLnewdbentry{marks}{Assignment 3}{48}\relax
\begin{ttfamily}\setlength{\parindent}{0pt}\par
Surname,FirstName,StudentNo,Assignment 1,Assignment 2,Assignment 3

\DTLforeach{marks}{\surname=Surname,\firstname=FirstName,\regNo
=StudentNo,\assignI=Assignment 1,\assignII
=Assignment 2,\assignIII=Assignment 3}{\relax
\char`\"\surname\char`\",\DTLifstringeq{\firstname}{Zoe}{Z\string\"oe}{\firstname},\regNo,\assignI,\assignII,\assignIII\par
}
\end{ttfamily}\par\noindent
First load this into a database called "marks":
\begin{verbatim}
\DTLloaddb{marks}{marks.csv}
\end{verbatim}
Suppose now I want to compute the average mark for each
student, and append this to the database. I can do this as 
follows:
\begin{verbatim}
\DTLforeach{marks}{%
\assignI=Assignment 1,%
\assignII=Assignment 2,%
\assignIII=Assignment 3}{%
\DTLmeanforall{\theMean}{\assignI,\assignII,\assignIII}%
\DTLappendtorow{Average}{\theMean}}
\end{verbatim}
\DTLforeach{marks}{\assignI=Assignment 1,\assignII
=Assignment 2,\assignIII=Assignment 3}{\relax
\DTLmeanforall{\theMean}{\assignI,\assignII,\assignIII}%
\DTLappendtorow{Average}{\theMean}}\relax
For each row in the "marks" database, I now have an extra key
called "Average" that contains the average mark over all three
assignments for a given student. I can now put this data into
a table:
\begin{verbatim}
\begin{table}[htbp]
\caption{Student marks}
\centering
\begin{tabular}{llcccc}
\bfseries Surname & \bfseries First Name &
\bfseries Assign 1 &
\bfseries Assign 2 &
\bfseries Assign 3 &
\bfseries Average Mark%
\DTLforeach{marks}%
{% assign variable
  \surname=Surname,\firstname=FirstName,%
  \average=Average,\assignI=Assignment 1,%
  \assignII=Assignment 2,\assignIII=Assignment 3}%
{% start new row
 \\\surname & \firstname & \assignI & \assignII & \assignIII &
 \DTLround{\average}{\average}{2}% round to 2 dp
 \DTLclip{\average}{average}%clip unnecessary 0s
 \average
}\relax
\end{tabular}
\end{table}
\end{verbatim}
This produces \autoref{tab:meanmarks}.

Note that if I only wanted the averages for the table and nothing
else, I could simply have computed the average in each row of the
table and displayed it without adding the information to the
database, however I am going to reuse this information in
\autoref{ex:multibar}, so adding it to the database means that
I don't need to recompute the mean.

\begin{table}[htbp]
\caption[Student marks (with averages)]{Student 
marks}\label{tab:meanmarks}
\centering
\begin{tabular}{llcccc}
\bfseries Surname & \bfseries First Name &
\bfseries Assign 1 &
\bfseries Assign 2 &
\bfseries Assign 3 &
\bfseries Average Mark\relax
\DTLforeach{marks}{\surname=Surname,\firstname=FirstName,\average
=Average,\assignI=Assignment 1,\assignII=Assignment 2,\assignIII
=Assignment 3}{\\\surname
& \firstname & \assignI & \assignII & \assignIII &
\DTLround{\average}{\average}{2}\DTLclip{\average}{\average}\average}\relax
\end{tabular}
\end{table}

\end{example}

\section{Arithmetical Computations on Database Entries}

The commands used in \autoref{sec:fp} can be used on database
entries. You can, of course, directly use the commands provided
by the \sty{fp} package if you know that the values are in the
correct format (i.e.\ no currency symbols, no number group 
separators and a full stop as the decimal point) but if this is
not the case, then you should use the commands described in
\autoref{sec:fp}. If you want to use a command provided by the
\sty{fp} package, that does not have a wrapper function in 
\sty{datatool}, then you will need to convert the value using
\ics{DTLconverttodecimal}, and convert it back using either
\ics{DTLdecimaltolocale} or \ics{DTLdecimaltocurrency}.

\begin{example}{Arithmetical Computations}{ex:mean}
In this example, I am going to produce a table similar to
\autoref{tab:scores}, except that I want to add an extra row at the
end which contains the average score.
\begin{verbatim}
\begin{table}[htbp]
\caption{Student scores}\label{tab:mean}
\centering
\def\total{0}%
\begin{tabular}{llr}
\bfseries First Name &
\bfseries Surname &
\bfseries Score (\%)%
\DTLforeach{scores}{%
\firstname=FirstName,\surname=Surname,\score=Score}{%
\\
\firstname & \surname &
\DTLgadd{\total}{\score}{\total}%
\score
}\\
\multicolumn{2}{l}{\bfseries Average Score} &
\DTLsavelastrowcount{\n}%
\DTLdiv{\average}{\total}{\n}%
\DTLround{\average}{\average}{2}%
\average
\end{tabular}
\end{table}
\end{verbatim}
This produces \autoref{tab:mean}.
\textbf{Notes:}
\begin{itemize}
\item I had to use \ics{DTLgadd} rather than \ics{DTLadd} since it
occurs within a \env{tabular} environment which puts each entry
in a local scope.

\item I used \ics{DTLsavelastrowcount} to store the number of
rows produced by \ics{DTLforeach} in the control sequence \cs{n}.

\item I used \ics{DTLround} to round the average score to 2 decimal
places.
\end{itemize}

\begin{table}[htbp]
\caption[Student scores (using arithmetic computations)]{Student 
scores}\label{tab:mean}
\centering
\def\total{0}\relax
\begin{tabular}{llr}
\bfseries First Name &
\bfseries Surname &
\bfseries Score (\%)\relax
\DTLforeach{scores}{\firstname=FirstName,\surname=Surname,\score=Score}{\relax
\\
\firstname & \surname &
\DTLgadd{\total}{\score}{\total}\relax
\score
}\\
\multicolumn{2}{l}{\bfseries Average Score} &
\DTLsavelastrowcount{\n}\relax
\DTLdiv{\average}{\total}{\n}\relax
\DTLround{\average}{\average}{2}\relax
\average
\end{tabular}
\end{table}
\end{example}

\begin{definition}[\DescribeMacro{\DTLsumforkeys}]%
\cs{DTLsumforkeys}\oarg{condition}\oarg{assign list}\marg{db list}\marg{key list}\marg{cmd}
\end{definition}
This command sums all the entries over all the databases listed in
the comma separated list of database names \meta{db list} for each
key in \meta{key list} where the condition given by \meta{condition}
is true. The second optional argument \meta{assign list} is the same
as the assignment list used by \ics{DTLforeach}, so that you can use
the information in \meta{condition}. The result is stored in
\meta{cmd} which must be a control sequence. For example:
\begin{verbatim}
\DTLsumforkeys{scores}{Score}{\total}
\end{verbatim}
sets \cs{total} to the sum of all the scores in the database
called "scores".

\begin{definition}[\DescribeMacro{\DTLsumcolumn}]%
\cs{DTLsumcolumn}\marg{db}\marg{key}\marg{cmd}
\end{definition}
This is a faster version of \cs{DTLsumforkeys} that only sums
the entries in a single column (specified by \meta{key}) for
a single database (specified by \meta{db}) and doesn't provide any 
filtering. The result is stored in \meta{cmd} which must be
a control sequence.

\begin{definition}[\DescribeMacro{\DTLmeanforkeys}]%
\cs{DTLmeanforkeys}\oarg{condition}\oarg{assign list}\marg{db list}\marg{key list}\marg{cmd}
\end{definition}
This command computes the arithmetic mean of all the entries over
all the databases listed in \meta{db list} for all keys in \meta{key
list} where the condition given by \meta{condition} is true. The
second optional argument \meta{assign list} is the same as the
assignment list used by \ics{DTLforeach}, so that you can use the
information in \meta{condition}. The result is stored in \meta{cmd}
which must be a control sequence. For example:
\begin{verbatim}
\DTLmeanforkeys{scores}{Score}{\average}
\end{verbatim}
sets \cs{average} to the mean of all the scores in the database
called "scores".

\begin{definition}[\DescribeMacro{\DTLmeanforcolumn}]%
\cs{DTLmeanforcolumn}\marg{db}\marg{key}\marg{cmd}
\end{definition}
This is a faster version of \cs{DTLmeanforkeys} that only computes
the mean for a single column (specified by \meta{key}) for
a single database (specified by \meta{db}) and doesn't provide any 
filtering. The result is stored in \meta{cmd} which must be
a control sequence.

\begin{definition}[\DescribeMacro{\DTLvarianceforkeys}]%
\cs{DTLvarianceforkeys}\oarg{condition}\oarg{assign list}\marg{db list}\marg{key list}\marg{cmd}
\end{definition}
This command computes the variance of all the entries over all the
databases listed in \meta{db list} for all keys in \meta{key list}
where the condition given by \meta{condition} is true. The second
optional argument \meta{assign list} is the same as the assignment
list used by \ics{DTLforeach}, so that you can use the information
in \meta{condition}. The result is stored in \meta{cmd} which must
be a control sequence.

\begin{definition}[\DescribeMacro{\DTLvarianceforcolumn}]%
\cs{DTLvarianceforcolumn}\marg{db}\marg{key}\marg{cmd}
\end{definition}
This is a faster version of \cs{DTLvarianceforkeys} that only
computes the variance for a single column (specified by \meta{key})
for a single database (specified by \meta{db}) and doesn't provide
any filtering. The result is stored in \meta{cmd} which must be a
control sequence.

\begin{definition}[\DescribeMacro{\DTLsdforkeys}]%
\cs{DTLsdforkeys}\oarg{condition}\oarg{assign list}\marg{db list}\marg{key list}\marg{cmd}
\end{definition}
This command computes the standard deviation of all the entries over
all the databases listed in \meta{db list} for all keys in \meta{key
list} where the condition given by \meta{condition} is true. The
second optional argument \meta{assign list} is the same as the
assignment list used by \ics{DTLforeach}, so that you can use the
information in \meta{condition}. The result is stored in \meta{cmd}
which must be a control sequence.

\begin{definition}[\DescribeMacro{\DTLsdforcolumn}]%
\cs{DTLsdforcolumn}\marg{db}\marg{key}\marg{cmd}
\end{definition}
This is a faster version of \cs{DTLsdforkeys} that only computes the
standard deviation for a single column (specified by \meta{key}) for
a single database (specified by \meta{db}) and doesn't provide any
filtering. The result is stored in \meta{cmd} which must be a
control sequence.

\begin{definition}[\DescribeMacro{\DTLminforkeys}]%
\cs{DTLminforkeys}\oarg{condition}\oarg{assign list}\marg{db list}\marg{key list}\marg{cmd}
\end{definition}
This command determines the minimum value over all entries for all
keys in \meta{key list} over all the databases listed in \meta{db
list} where \meta{condition} is true.  The second optional argument
\meta{assign list} is the same as the assignment list used by
\ics{DTLforeach}, so that you can use the information in
\meta{condition}. The result is stored in \meta{cmd}, which must be
a control sequence. For example
\begin{verbatim}
\DTLminforkeys{scores}{Score}{\theMin}
\end{verbatim}
sets \cs{theMin} to the minimum score in the database.

\begin{definition}[\DescribeMacro{\DTLminforcolumn}]%
\cs{DTLminforcolumn}\marg{db}\marg{key}\marg{cmd}
\end{definition}
This is a faster version of \cs{DTLminforkeys} that only computes
the minimum for a single column (specified by \meta{key}) for
a single database (specified by \meta{db}) and doesn't provide any 
filtering. The result is stored in \meta{cmd} which must be
a control sequence.

\begin{definition}[\DescribeMacro{\DTLmaxforkeys}]%
\cs{DTLmaxforkeys}\oarg{condition}\oarg{assign list}\marg{db list}\marg{key list}\marg{cmd}
\end{definition}
This command determines the maximum value over all entries for all
keys in \meta{key list} over all the databases listed in \meta{db
list} where \meta{condition} is true.  The second optional argument
\meta{assign list} is the same as the assignment list used by
\ics{DTLforeach}, so that you can use the information in
\meta{condition}. The result is stored in \meta{cmd}, which must be
a control sequence. For example
\begin{verbatim}
\DTLminforkeys{scores}{Score}{\theMax}
\end{verbatim}
sets \cs{theMax} to the minimum score in the database.

\begin{definition}[\DescribeMacro{\DTLmaxforcolumn}]%
\cs{DTLmaxforcolumn}\marg{db}\marg{key}\marg{cmd}
\end{definition}
This is a faster version of \cs{DTLmaxforkeys} that only computes
the maximum for a single column (specified by \meta{key}) for
a single database (specified by \meta{db}) and doesn't provide any 
filtering. The result is stored in \meta{cmd} which must be
a control sequence.

\begin{definition}[\DescribeMacro{\DTLcomputebounds}]%
\cs{DTLcomputebounds}\marg{db list}\marg{x key}\marg{y key}\marg{minX cmd}\marg{minY cmd}\marg{maxX cmd}\marg{maxY cmd}
\end{definition}
Computes the maximum and minimum $x$ and $y$ values over all
the databases listed in \meta{db list} where the $x$ value
is given by \meta{x key} and the $y$ value is given by
\meta{y key}. The results are stored in \meta{minX cmd},
\meta{minY cmd}, \meta{maxX cmd} and \meta{maxY cmd}.

\begin{example}{Mail Merging}{ex:mailmerging}
This example uses the database given in \autoref{ex:scores} and
uses \ics{DTLmeanforkeys} to determine the average score. A letter
is then created for each student to inform them of their score
and the class average.

\begin{verbatim}
\documentclass{letter}

\usepackage{datatool}

\begin{document}
 % load database
\DTLloaddb{scores}{studentscores.csv}
 % compute arithmetic mean for key `Score'
\DTLmeanforkeys{scores}{Score}{\average}
 % Round the average to 2 decimal places
\DTLround{\average}{\average}{2}
 % Save the highest score in \maxscore
\DTLmaxforkeys{scores}{Score}{\maxscore}

\DTLforeach{scores}{\firstname=FirstName,\surname=Surname,%
\score=Score}{%
\begin{letter}{}
\opening{Dear \firstname\ \surname}

\DTLifnumgt{\score}{60}{Congratulations you}{You} achieved a score
of \score\% which was \DTLifnumgt{\score}{\average}{above}{below}
the average of \average\%. \DTLifnumeq{\score}{\maxscore}{You
achieved the highest score}{The top score was \maxscore}.

\closing{Yours Sincerely}
\end{letter}
}
\end{document}
\end{verbatim}

To determine a person's gender when mail merging, see
\autoref{sec:person}.
\end{example}

\section{Sorting a Database}
\label{sec:sort}

The sort methods described here use \TeX\ to sort, which is very
inefficient. The comparison handlers use a character code comparison
rather than locale-sensitive alphabetic ordering. It's more
efficient to use \app{datatooltk} to import and sort at the same
time.

\begin{definition}[\DescribeMacro{\dtlsort}]%
\cs{dtlsort}\oarg{replacement key list}\marg{sort criteria}\marg{db
name}\marg{handler}
\end{definition}
This will sort the database called \meta{db name} according to
the criteria given by \meta{sort criteria}, which must be a 
comma separated list of keys and optionally "="\meta{order}, where
\meta{order} is either "ascending" or "descending". If the order
is omitted, "ascending" is assumed. The database keeps track of
the data type for a given key, and uses this to determine whether
an alphabetical or numerical sort is required.

The final argument \meta{handler} is the command used for the
comparisons and is the same as the \meta{criteria cs} command
used by \cs{dtlsortlist} and \cs{dtlinsertinto}, described
in \sectionref{sec:csvlists}. The predefined handlers are:
\begin{description}
\item[\ics{dtlcompare}] A case-sensitive comparison.
\item[\ics{dtlicompare}] A case-insensitive comparison.
\item[\ics{dtlwordindexcompare}] English word-ordering comparison
for indexes, as described by the Oxford Style Manual.
\item[\ics{dtlletterindexcompare}] English letter-ordering
comparison for indexes.
\end{description}
The last two handlers, \cs{dtlwordindexcompare} and
\cs{dtlletterindexcompare}, assume that inversion commas are
indicated using one of the following commands:
\begin{itemize}
\item To indicate name inversion:
\begin{definition}[\DescribeMacro\datatoolpersoncomma]
\cs{datatoolpersoncomma}
\end{definition}
Example: \verb|Knuth\datatoolpersoncomma Donald E.|

\item To indicate place inversion:
\begin{definition}[\DescribeMacro\datatoolplacecomma]
\cs{datatoolplacecomma}
\end{definition}
Example:
\verb|New York\datatoolplacecomma USA|

\item To indicate subject inversion:
\begin{definition}[\DescribeMacro\datatoolsubjectcomma]
\cs{datatoolsubjectcomma}
\end{definition}
Example: \verb|New York\datatoolsubjectcomma population|
\end{itemize}
In addition, the start of parenthetical material should be indicated
with
\begin{definition}[\DescribeMacro\datatoolparenstart]
\cs{datatoolparenstart}
\end{definition}
Example:  \verb|High Water\datatoolparenstart play|

Following the guidelines of the Oxford Style Manual, when sorting
terms that have identical pre-inversion parts, the following
ordering is applied: people, places, subjects, no inversions and
parenthetical.

\begin{example}{Sorting a Database---Dealing with
Inversions}{ex:sortinversion}
This uses the example given in Chapter~16 of the Oxford Style
Manual. Suppose I~define my database as follows:
\begin{verbatim}
\DTLnewdb{inversiondata}
\DTLnewrow{inversiondata}
\DTLnewdbentry{inversiondata}{Term}{New York, New York}
\DTLnewrow{inversiondata}
\DTLnewdbentry{inversiondata}{Term}{New York\datatoolsubjectcomma
population}
\DTLnewrow{inversiondata}
\DTLnewdbentry{inversiondata}{Term}{New York\datatoolplacecomma
USA}
\DTLnewrow{inversiondata}
\DTLnewdbentry{inversiondata}{Term}{New York\datatoolpersoncomma
Earl of}
\end{verbatim}
\DTLnewdb{inversiondata}
\DTLnewrow{inversiondata}
\DTLnewdbentry{inversiondata}{Term}{New York, New York}
\DTLnewrow{inversiondata}
\DTLnewdbentry{inversiondata}{Term}{New York\datatoolsubjectcomma
population}
\DTLnewrow{inversiondata}
\DTLnewdbentry{inversiondata}{Term}{New York\datatoolplacecomma
USA}
\DTLnewrow{inversiondata}
\DTLnewdbentry{inversiondata}{Term}{New York\datatoolpersoncomma
Earl of}

First of all, display the unsorted data:
\begin{verbatim}
\DTLdisplaydb{inversiondata}
\end{verbatim}
This produces:
\begin{display}
\DTLdisplaydb{inversiondata}
\end{display}

Now sort the data using the \ics{dtlwordindexcompare} handler:
\begin{verbatim}
\dtlsort{Term}{inversiondata}{\dtlwordindexcompare}
\end{verbatim}
\dtlsort{Term}{inversiondata}{\dtlwordindexcompare}%
and display again:
\begin{verbatim}
\DTLdisplaydb{inversiondata}
\end{verbatim}
which now produces:
\begin{display}
\DTLdisplaydb{inversiondata}
\end{display}
There are three entries here with pre-inversion text that's simply
\texttt{New York}. Since each of these three entries has the same
pre-inversion text, they need to be sorted according to the type of
inversion: person, place, subject. The fourth entry (New York, New
York) doesn't have an inversion since the comma is part of the title
of the named work. It's therefore sorted according to \texttt{New
York, New York} rather than just \texttt{New York} and so comes
after all the \texttt{New York} entries.
\end{example}

If you want to write your own comparison handler, see the documented
code for details on the syntax of the handler. (You may want to
consider uploading your handler as a separate package to CTAN if you
think it will be of general use.)

\label{utf8support}%
As from version 2.24, the predefined handlers now have limited support for UTF-8
characters. \emph{This is still experimental.} The support will
automatically be switched on if the \sty{inputenc} package is loaded
with the \pkgopt{utf8} option before loading \sty{datatool-base}
(which is automatically loaded by \sty{datatool}). If \sty{inputenc}
is loaded after \sty{datatool-base}, you can use the (boolean) \pkgopt{utf8}
option when you load \sty{datatool-base} to enable it. For example,
either:
\begin{verbatim}
\usepackage[utf8]{inputenc}
\usepackage{datatool-base}
\end{verbatim}
or
\begin{verbatim}
\usepackage[utf8]{datatool-base}
\usepackage[utf8]{inputenc}
\end{verbatim}
(As from version 2.28, you can now also pass the \pkgopt{utf8}
option to \sty{datatool} and \sty{datagidx}.)
You can also enable this option after \sty{datatool-base} has been loaded
using
\begin{definition}[\DescribeMacro\dtlenableUTFviii]
\cs{dtlenableUTFviii}
\end{definition}
or disable it with
\begin{definition}[\DescribeMacro\dtldisableUTFviii]
\cs{dtldisableUTFviii}
\end{definition}
For example:
\begin{verbatim}
\usepackage[utf8]{inputenc}
\usepackage{datatool}
\dtldisableUTFviii
\end{verbatim}
This option isn't required for \XeLaTeX, which treats each UTF-8
character as an individual token.

With regular (pdf)\LaTeX\ (as opposed to \XeLaTeX) each UTF-8 character is
actually treated as two tokens that represent the first and second
octet of the UTF-8 character. This means that it's not possible for
\TeX\ to obtain a character code using the usual backtick method, so
(with this option enabled) if a UTF-8 character is detected, the handlers 
described above will pass both octet tokens to one of the following commands:
\begin{definition}[\DescribeMacro\dtlsetUTFviiicharcode]
\cs{dtlsetUTFviiicharcode}\marg{octet tokens}\marg{count}
\end{definition}
(for the case-sensitive handler \ics{dltcompare}) and
\begin{definition}[\DescribeMacro\dtlsetUTFviiilccharcode]
\cs{dtlsetUTFviiilccharcode}\marg{octet tokens}\marg{count}
\end{definition}
(for the case-insensitive handlers).

By default the first simply does:
\begin{definition}[\DescribeMacro\dtlsetdefaultUTFviiicharcode]
\cs{dtlsetdefaultUTFviiicharcode}\marg{octet tokens}\marg{count}
\end{definition}
and the second does
\begin{definition}[\DescribeMacro\dtlsetdefaultUTFviiilccharcode]
\cs{dtlsetdefaultUTFviiilccharcode}\marg{octet tokens}\marg{count}
\end{definition}

In all the above \meta{octet tokens} are the two octets forming the
UTF-8 character and \meta{count} is a count register in which to
store the relevant character code.

The default commands simply set the character code for certain
common accented Latin characters (such as \'e) to the code for their 
unaccented version (such as e). You can redefine
\cs{dtlsetUTFviiicharcode} and \cs{dtlsetUTFviiilccharcode} to test
for additional characters. For example, to add the Norwegian
characters \ae, \o\ and \aa\ to the end of the Latin alphabet, you
can do:
\begin{verbatim}
\renewcommand*{\dtlsetUTFviiicharcode}[2]{%
  \ifstrequal{#1}{Æ}%
  {%
    #2=91\relax
  }%
  {%
    \ifstrequal{#1}{Ø}%
    {%
      #2=92\relax
    }%
    {%
      \ifstrequal{#1}{Å}%
      {%
        #2=93\relax
      }%
      {%
        \ifstrequal{#1}{æ}%
        {%
          #2=123\relax
        }%
        {%
          \ifstrequal{#1}{ø}%
          {%
            #2=124\relax
          }%
          {%
            \ifstrequal{#1}{å}%
            {%
              #2=125\relax
            }%
            {%
              \dtlsetdefaultUTFviiicharcode{#1}{#2}%
            }%
          }%
        }%
      }%
    }%
  }%
}

\renewcommand*{\dtlsetUTFviiilccharcode}[2]{%
  \ifstrequal{#1}{Æ}%
  {%
    #2=123\relax
  }%
  {%
    \ifstrequal{#1}{Ø}%
    {%
      #2=124\relax
    }%
    {%
      \ifstrequal{#1}{Å}%
      {%
        #2=125\relax
      }%
      {%
        \ifstrequal{#1}{æ}%
        {%
          #2=123\relax
        }%
        {%
          \ifstrequal{#1}{ø}%
          {%
            #2=124\relax
          }%
          {%
            \ifstrequal{#1}{å}%
            {%
              #2=125\relax
            }%
            {%
              \dtlsetdefaultUTFviiilccharcode{#1}{#2}%
            }%
          }%
        }%
      }%
    }%
  }%
}
\end{verbatim}

In the case where a~character is a single token (for example, with
Latin-1 encoding), the code associated with that character is set
using
\begin{definition}[\DescribeMacro\dtlsetcharcode]
\cs{dtlsetcharcode}\marg{c}\marg{count register}
\end{definition}
(for case-sensitive comparison) and
\begin{definition}[\DescribeMacro\dtlsetlccharcode]
\cs{dtlsetlccharcode}\marg{c}\marg{count register}
\end{definition}
(for case-insensitive comparison), where \meta{c} is the character
and \meta{count register} is a count register. The default
definitions are:
\begin{verbatim}
\newcommand*{\dtlsetcharcode}[2]{#2=`#1\relax}
\newcommand*{\dtlsetlccharcode}[2]{#2=\lccode`#1\relax}
\end{verbatim}
So, for example, the letter \qt{e} has the code \number`e\ whereas
(with the Latin-1 encoding) the letter \qt{\'e} has the code 233.
This means that \qt{\'e} will be sorted after \qt{e}. It may be that
you want \qt{\'e} to be treated the same as \qt{e} when making the
comparison. In this case, you need to redefine \cs{dtlsetcharcode}
and \cs{dtlsetlccharcode}. For example, for the case-sensitive
comparisons:
\begin{verbatim}
\renewcommand*{\dtlsetcharcode}[2]{%
  \ifstrequal{#1}{É}%
  {%
    #2=`E\relax
  }%
  {%
    \ifstrequal{#1}{é}%
    {%
      #2=`e\relax
    }%
    {%
      #2=`#1\relax
    }%
  }%
}
\end{verbatim}
and for the case-insensitive comparisons:
\begin{verbatim}
\renewcommand*{\dtlsetlccharcode}[2]{%
  \ifboolexpr
  {%
       test {\ifstrequal{#1}{é}}
    or test {\ifstrequal{#1}{É}}
  }%
  {%
    #2=`e\relax
  }%
  {%
    #2=\lccode`#1\relax
  }%
}
\end{verbatim}
(Extra conditionals will need to be added for other diacritics.
Alternative conditionals, such as \cs{ifx} or \cs{ifnum}, may be more 
efficient. In the above I've used the \sty{etoolbox} commands for
clarity.)

There are two shortcut commands for \cs{dtlsort}:
\begin{definition}[\DescribeMacro{\DTLsort}]%
\cs{DTLsort}\oarg{replacement key list}\marg{sort criteria}\marg{db name}
\end{definition}
\begin{definition}[\DescribeMacro{\DTLsort*}]%
\cs{DTLsort*}\oarg{replacement key list}\marg{sort criteria}\marg{db name}
\end{definition}
these use the \ics{dtlcompare} and \ics{dtlicompare} handlers,
respectively.

The optional argument \meta{replacement key list} is a list of
keys to use if the current key given in \meta{sort criteria}
is null for a given entry. Null keys are unlikely to occur if
you have loaded the database from an external ASCII file, but
may occur if the database is created using \cs{DTLnewdb},
\cs{DTLnewrow} and \cs{DTLnewdbentry}. For example:
\begin{verbatim}
\DTLsort[Editor,Organization]{Author}{mydata}
\end{verbatim}
will sort according to the "Author" key, but if that key is missing
for a given row of the database, the "Editor" key will be used,
and if the "Editor" key is missing, it will use the "Organization"
key. Note that this is not the same as:
\begin{verbatim}
\DTLsort{Author,Editor,Organization}{mydata}
\end{verbatim}
which will first compare the "Author" keys, but if the author names
are the same, it will then compare the "Editor" keys, and if the
editor names are also the same, it will then compare the
"Organization" keys.

The unstarred version uses a case sensitive comparison for strings,
whereas the starred version ignores the case when comparing strings.
Note that the case sensitive comparison orders uppercase characters
before lowercase characters, so the letter B is considered to be 
lower than the letter a.

\begin{example}{Sorting a Database}{ex:sort}
This example uses the database called "scores" defined in 
\autoref{ex:scores}. First, I am going to sort the database
according to the student scores in descending order (highest to
lowest) and display the database in a table
\begin{verbatim}
\begin{table}[htbp]
\caption{Student scores (sorted by score)}
\centering
\DTLsort{Score=descending}{scores}%
\begin{tabular}{llr}
\bfseries First Name &
\bfseries Surname &
\bfseries Score (\%)%
\DTLforeach{scores}{%
\firstname=FirstName,\surname=Surname,\score=Score}{%
\\
\firstname & \surname & \score}
\end{tabular}
\end{table}
\end{verbatim}
This produces \autoref{tab:sortscores}.

\begin{table}[htbp]
\caption{Student scores (sorted by score)}\label{tab:sortscores}
\centering
\DTLsort{Score=descending}{scores}\relax
\begin{tabular}{llr}
\bfseries First Name &
\bfseries Surname &
\bfseries Score (\%)\relax
\DTLforeach{scores}{\firstname=FirstName,\surname=Surname,\score=Score}{\relax
\\
\firstname & \surname & \score}
\end{tabular}
\end{table}

Now I am going to sort the database according to 
surname and then first name, and display it in a table. Note that
since I want to sort in ascending order, I can omit the
"=ascending" part of the sort criteria. I have also decided to
reverse the first and second columns, so that the surname is
in the first column.
\begin{verbatim}
\begin{table}[htbp]
\caption{Student scores (sorted by name)}
\centering
\DTLsort{Surname,FirstName}{scores}%
\begin{tabular}{llr}
\bfseries Surname &
\bfseries First Name &
\bfseries Score (\%)%
\DTLforeach{scores}{%
\firstname=FirstName,\surname=Surname,\score=Score}{%
\\
\surname & \firstname & \score}
\end{tabular}
\end{table}
\end{verbatim}
This produces \autoref{tab:sortname}.
\begin{table}[htbp]
\caption{Student scores (sorted by name)}\label{tab:sortname}
\centering
\DTLsort{Surname,FirstName}{scores}\relax
\begin{tabular}{llr}
\bfseries Surname &
\bfseries First Name &
\bfseries Score (\%)\relax
\DTLforeach{scores}{\firstname=FirstName,\surname=Surname,\score=Score}{\relax
\\
\surname & \firstname & \score}
\end{tabular}
\end{table}

Now suppose I add two new students to the database:
\begin{verbatim}
\DTLnewrow{scores}%
\DTLnewdbentry{scores}{Surname}{van der Mere}%
\DTLnewdbentry{scores}{FirstName}{Henk}%
\DTLnewdbentry{scores}{Score}{71}%
\DTLnewrow{scores}%
\DTLnewdbentry{scores}{Surname}{de la Mere}%
\DTLnewdbentry{scores}{FirstName}{Jos}%
\DTLnewdbentry{scores}{Score}{58}%
\end{verbatim}
\DTLnewrow{scores}
\DTLnewdbentry{scores}{Surname}{van der Mere}\relax
\DTLnewdbentry{scores}{FirstName}{Henk}\relax
\DTLnewdbentry{scores}{Score}{71}\relax
\DTLnewrow{scores}\relax
\DTLnewdbentry{scores}{Surname}{de la Mere}\relax
\DTLnewdbentry{scores}{FirstName}{Jos}\relax
\DTLnewdbentry{scores}{Score}{58}\relax
and again I try sorting the database, and displaying the contents
as a table:
\begin{verbatim}
\begin{table}[htbp]
\caption{Student scores (case sensitive sort)}
\centering
\DTLsort{Surname,FirstName}{scores}%
\begin{tabular}{llr}
\bfseries Surname &
\bfseries First Name &
\bfseries Score (\%)%
\DTLforeach{scores}{%
\firstname=FirstName,\surname=Surname,\score=Score}{%
\\
\surname & \firstname & \score}
\end{tabular}
\end{table}
\end{verbatim}
This produces \autoref{tab:sortname2}. Notice that the surnames
aren't correctly ordered. This is because a case-sensitive
sort was used. Changing \cs{DTLsort} to \cs{DTLsort*} in the
above code produces \autoref{tab:sortname3}.

\begin{table}[htbp]
\caption{Student scores (case sensitive sort)}\label{tab:sortname2}
\centering
\DTLsort{Surname,FirstName}{scores}\relax
\begin{tabular}{llr}
\bfseries Surname &
\bfseries First Name &
\bfseries Score (\%)\relax
\DTLforeach{scores}{\firstname=FirstName,\surname=Surname,\score=Score}{\relax
\\
\surname & \firstname & \score}
\end{tabular}
\end{table}

\begin{table}[htbp]
\caption{Student scores (case ignored when 
sorting)}\label{tab:sortname3}
\centering
\DTLsort*{Surname,FirstName}{scores}\relax
\begin{tabular}{llr}
\bfseries Surname &
\bfseries First Name &
\bfseries Score (\%)\relax
\DTLforeach{scores}{\firstname=FirstName,\surname=Surname,\score=Score}{\relax
\\
\surname & \firstname & \score}
\end{tabular}
\end{table}

\end{example}

\begin{example}{Influencing the sort order}{ex:sortswitchargs}
Consider the data displayed in \autoref{tab:sortname3}, suppose that
you want the names ``van der Mere'' and ``de la Mere'' sorted
according to the actual surname ``Mere'' rather than by the ``von
part''.  There are two ways you can do this: firstly, you could store
the von part in a separate field, and then sort by surname, then von
part, then first name, or you could define a command called, say,
\cs{switchargs}, as follows:
\begin{verbatim}
\newcommand*{\switchargs}[2]{#2#1}
\end{verbatim}
\newcommand*{\switchargs}[2]{#2#1}\relax
then store the data as:
\begin{verbatim}
FirstName,Surname,StudentNo,Score
John,"Smith, Jr",102689,68
Jane,Brown,102647,75
Andy,Brown,103569,42
Z\"oe,Adams,105987,52
Roger,Brady,106872,58
Clare,Verdon,104356,45
Henk,\switchargs{Mere}{van der },106789,71
Jos,\switchargs{Mere}{de la },104256,58
\end{verbatim}
\DTLnewdb{scores2}\relax
\DTLnewrow{scores2}\relax
\DTLnewdbentry{scores2}{FirstName}{John}\relax
\DTLnewdbentry{scores2}{Surname}{Smith, Jr}\relax
\DTLnewdbentry{scores2}{StudentNo}{102689}\relax
\DTLnewdbentry{scores2}{Score}{68}\relax
\DTLnewrow{scores2}\relax
\DTLnewdbentry{scores2}{FirstName}{Jane}\relax
\DTLnewdbentry{scores2}{Surname}{Brown}\relax
\DTLnewdbentry{scores2}{StudentNo}{102647}\relax
\DTLnewdbentry{scores2}{Score}{75}\relax
\DTLnewrow{scores2}\relax
\DTLnewdbentry{scores2}{FirstName}{Andy}\relax
\DTLnewdbentry{scores2}{Surname}{Brown}\relax
\DTLnewdbentry{scores2}{StudentNo}{103569}\relax
\DTLnewdbentry{scores2}{Score}{42}\relax
\DTLnewrow{scores2}\relax
\DTLnewdbentry{scores2}{FirstName}{Z\"oe}\relax
\DTLnewdbentry{scores2}{Score}{52}\relax
\DTLnewdbentry{scores2}{StudentNo}{105987}\relax
\DTLnewdbentry{scores2}{Surname}{Adams}\relax
\DTLnewrow{scores2}\relax
\DTLnewdbentry{scores2}{FirstName}{Roger}\relax
\DTLnewdbentry{scores2}{Score}{58}\relax
\DTLnewdbentry{scores2}{StudentNo}{106872}\relax
\DTLnewdbentry{scores2}{Surname}{Brady}\relax
\DTLnewrow{scores2}\relax
\DTLnewdbentry{scores2}{FirstName}{Clare}\relax
\DTLnewdbentry{scores2}{Score}{45}\relax
\DTLnewdbentry{scores2}{StudentNo}{104356}\relax
\DTLnewdbentry{scores2}{Surname}{Verdon}\relax
\DTLnewrow{scores2}
\DTLnewdbentry{scores2}{Surname}{\switchargs{Mere}{van der }}\relax
\DTLnewdbentry{scores2}{FirstName}{Henk}\relax
\DTLnewdbentry{scores2}{Score}{71}\relax
\DTLnewrow{scores2}\relax
\DTLnewdbentry{scores2}{Surname}{\switchargs{Mere}{de la }}\relax
\DTLnewdbentry{scores2}{FirstName}{Jos}\relax
\DTLnewdbentry{scores2}{Score}{58}\relax
Now sort the data, and put it in table (this is the same code
as in the previous example:
\begin{verbatim}
\begin{table}[htbp]
\caption{Student scores (influencing the sort order)}
\centering
\DTLsort*{Surname,FirstName}{scores}%
\begin{tabular}{llr}
\bfseries Surname &
\bfseries First Name &
\bfseries Score (\%)%
\DTLforeach{scores}{%
\firstname=FirstName,\surname=Surname,\score=Score}{%
\\
\surname & \firstname & \score}
\end{tabular}
\end{table}
\end{verbatim}
This produces \autoref{tab:influencesort}.

\begin{table}[htbp]
\caption{Student scores (influencing the sort order)}\label{tab:influencesort}
\centering
\DTLsort*{Surname,FirstName}{scores2}\relax
\begin{tabular}{llr}
\bfseries Surname &
\bfseries First Name &
\bfseries Score (\%)\relax
\DTLforeach{scores2}{\firstname=FirstName,\surname=Surname,\score=Score}{\relax
\\
\surname & \firstname & \score}
\end{tabular}
\end{table}

\end{example}

\section{Saving a Database to an External File}
\label{sec:savedb}

\begin{definition}[\DescribeMacro{\DTLsavedb}]%
\cs{DTLsavedb}\marg{db name}\marg{filename}
\end{definition}
This writes the database called \meta{db name} to a file called
\meta{filename}. The separator and delimiter characters used
are as given by \ics{DTLsetseparator} (or \ics{DTLsettabseparator})
and \ics{DTLsetdelimiter}. For example:
\begin{verbatim}
\DTLsettabdelimiter
\DTLsavedb{scores}{scores.txt}
\end{verbatim}
will create a file called "scores.txt" and will save the data in a 
tab separated format. (The delimiters will only be used if a
given entry contains the separator character.)

\begin{definition}[\DescribeMacro{\DTLsavetexdb}]%
\cs{DTLsavetexdb}\marg{db name}\marg{filename}
\end{definition}
This writes the database called \meta{db name} to a \LaTeX\ file 
called \meta{filename}, where the database is stored as 
a combination of \ics{DTLnewdb}, \ics{DTLnewrow} and 
\ics{DTLnewdbentry} commands. This means that the file is in a
user-friendly format, but may be so to load, particularly if the
database is large. If you are more concerned with speed rather than
readability you can use:
\begin{definition}[\DescribeMacro{\DTLsaverawdb}]%
\cs{DTLsaverawdb}\marg{db name}\marg{filename}
\end{definition}
This saves the database to \meta{filename} in its internal
representation, which makes it faster to load. Fragile commands 
cause a~problem for \cs{DTLsaverawdb} so if your database contains
any use:
\begin{definition}[\DescribeMacro{\DTLprotectedsaverawdb}]%
\cs{DTLprotectedsaverawdb}\marg{db name}\marg{filename}
\end{definition}
instead. The \app{datatooltk} application
can read and write this raw format. To load a file in this format
you can just use \cs{input} or you can use:
\begin{definition}
\cs{DTLloaddbtex}\marg{cs}\marg{file}
\end{definition}
This checks for the file's existence and assigns the database name to the
control sequence \meta{cs}.

Databases saved using \cs{DTLsavetexdb}, \cs{DTLsaverawdb} and
\cs{DTLprotectedsaverawdb} can be loaded using \LaTeX's standard
\ics{input} command. As from version 2.15, the last line of the
database file defines
\begin{definition}[\DescribeMacro{\dtllastloadeddb}]
\cs{dtllastloadeddb}
\end{definition}
to the name of the database, in case it's required.

Databases saved using \cs{DTLsaverawdb} and
\cs{DTLprotectedsaverawdb} can also be loaded and edited by 
\app{datatooltk} (see page~\pageref{datatooltk}).

\section{Deleting or Clearing a Database}\label{sec:deletedb}

A database can be cleared or deleted when its contents are no
longer required.
\begin{definition}[\DescribeMacro{\DTLcleardb}]
\cs{DTLcleardb}\marg{db name}
\end{definition}
\begin{definition}[\DescribeMacro{\DTLgcleardb}]
\cs{DTLgcleardb}\marg{db name}
\end{definition}
Clears the database given by \meta{db name}. The database is emptied
but remains defined. The second form is required if you want a
global effect.

\begin{definition}[\DescribeMacro{\DTLdeletedb}]
\cs{DTLdeletedb}\marg{db name}
\end{definition}
\begin{definition}[\DescribeMacro{\DTLgdeletedb}]
\cs{DTLgdeletedb}\marg{db name}
\end{definition}
Deletes (undefines) the database given by \meta{db name}. The second
form is required if you want a global effect.

\begin{important}
Although \cs{DTLdeletedb} and \cs{DTLgdeletedb} undefine the macros
associated with the database, they don't unassign the registers
used.  (\TeX\ doesn't provide a command that performs the reverse of
commands such as \cs{newcount}.) If you want to keep making
temporary databases, it's better to just define a single database
(called, say, \texttt{temp}) and then just clear it rather than
delete it and define a new database. For example, if you are
iterating through a loop and want to have a temporary database
on each iteration. In that case, define the database before the
start of the loop and clear it on each iteration. If you are loading
data from an external file, remember to use \ics{DTLnewdbonloadfalse}
before \ics{DTLloaddb} (or \ics{DTLloadrawdb}).
\end{important}

\section{Advanced Database Commands}
\label{sec:advanced}

This section describes more advanced commands. Further details
can be found in the documented code (datatool-code.pdf).

\begin{definition}[\DescribeMacro{\DTLgetdatatype}]
\cs{DTLgetdatatype}\marg{cs}\marg{db}\marg{key}
\end{definition}
Gets the data type for the given key \meta{key} for the database
given by \meta{db}. The data type is stored in \meta{cs} which
must be a command name. The type will be one of:
\begin{itemize}
\item\DescribeMacro{\DTLunsettype}\cs{DTLunsettype} (not set),
\item\DescribeMacro{\DTLstringtype}\cs{DTLstringtype} (string),
\item\DescribeMacro{\DTLinttype}\cs{DTLinttype} (integer),
\item\DescribeMacro{\DTLrealtype}\cs{DTLrealtype} (real number) or
\item\DescribeMacro{\DTLcurrencytype}\cs{DTLcurrenttype} (currency).
\end{itemize}

\begin{definition}[\DescribeMacro{\DTLifdbexists}]
\cs{DTLifdbexists}\marg{db name}\marg{true part}\marg{false part}
\end{definition}
Determines if the database given by \meta{db name} exists.

\begin{definition}[\DescribeMacro{\DTLifhaskey}]
\cs{DTLifhaskey}\marg{db name}\marg{key}\marg{true part}\marg{false part}
\end{definition}
This determines if the database given by \meta{db name} has
any entries with the key given by \meta{key}. If so, it does
\meta{true part} otherwise it does \meta{false part}.

Each key has an associated column index. This can be obtained
using:
\begin{definition}[\DescribeMacro{\DTLgetcolumnindex}]
\cs{DTLgetcolumnindex}\marg{cs}\marg{db}\marg{key}
\end{definition}
where \meta{cs} is a command name, \meta{db} is the database label
and \meta{key} is the key. The column index is stored in 
\meta{cs}.

You can also do the reverse and find the key associated with a
given column index:
\begin{definition}[\DescribeMacro{\DTLgetkeyforcolumn}]
\cs{DTLgetkeyforcolumn}\marg{key cs}\marg{db}\marg{column index}
\end{definition}
The key is stored in \meta{key cs} (which must be a command name).

There is also a full expandable way of obtaining the column
index, but note that no check is performed to determine
if the database exists, or if it contains the given key:
\begin{definition}[\DescribeMacro{\dtlcolumnindex}]
\cs{dtlcolumnindex}\marg{db name}\marg{key}
\end{definition}

\begin{definition}[\DescribeMacro{\DTLgetkeydata}]
\cs{DTLgetkeydata}\marg{key}\marg{db}\marg{col cs}\marg{type cs}\marg{header cs}
\end{definition}
Gets data for given key in database \meta{db}: the column index is
stored in \meta{col cs} (as \cs{DTLgetcolumnindex}), the type is 
stored in \meta{type cs} (as \cs{DTLgetdatatype}) and the header
is stored in \meta{header cs}.

\begin{definition}[\DescribeMacro{\DTLgetvalue}]
\cs{DTLgetvalue}\marg{cs}\marg{db}\marg{r}\marg{c}
\end{definition}
This gets the value for row given by index \meta{r} and column
given by \meta{c} for the database \meta{db} and stores it in
\meta{cs} which must be a command name. If you want to get the
value by key rather than column index you can use 
\cs{dtlcolumnindex}. For example, the following gets the value
for row~3 with key \texttt{Surname} from the database 
\texttt{data} and stores in \cs{myval}:
\begin{verbatim}
\DTLgetvalue{\myval}{data}{3}{\dtlcolumnindex{data}{Surname}}
\end{verbatim}

\begin{definition}[\DescribeMacro{\DTLgetlocation}]
\cs{DTLgetlocation}\marg{row cs}\marg{column cs}\marg{database}%
\marg{value}
\end{definition}
Assigns \meta{row cs} and \meta{column cs} to the indices of the
first entry in \meta{database} that matches \meta{value}.

\begin{definition}[\DescribeMacro{\DTLgetvalueforkey}]
\cs{DTLgetvalueforkey}\marg{cmd}\marg{key}\marg{db name}\marg{ref
 key}\marg{ref value}
\end{definition}
This (globally) sets \meta{cmd} (a control sequence) to the
value of the key specified by \meta{key} in the first row
of the database called \meta{db name} which contains the key
\meta{ref key} which has the value \meta{value}.

\begin{definition}[\DescribeMacro{\DTLfetch}]
\cs{DTLfetch}\marg{db name}\marg{column1 name}\marg{column1
value}\marg{column2 name}
\end{definition}
This fetches and displays the value for \meta{column2 name} in the
first row where the value of \meta{column1 name} is \meta{column1
value}. (Note that all arguments are expanded.) So, for example, if
you have a column labelled ``regnum'' and a column labelled
``tutor'', then to fetch and display the value of the tutor in the
row where ``regnum'' is ``12345'' from the database called
``students'' you can do:
\begin{verbatim}
\DTLfetch{students}{regnum}{12345}{tutor}
\end{verbatim}
See \autoref{ex:join} on page~\pageref{ex:join}.

\begin{definition}[\DescribeMacro{\DTLassign}]
\cs{DTLassign}\marg{db name}\marg{row idx}\marg{assign list}
\end{definition}
This (globally) assigns the list of commands in \meta{assign list}
for row \meta{row idx} in database \meta{db name}, where
\meta{assign list} has the same format as in \ics{DTLforeach}.

\begin{definition}
\cs{DTLassignfirstmatch}\marg{db name}\marg{col key}\marg{value}\marg{assign list}
\end{definition}
This is similar to \cs{DTLassign} except that it applies to the
first row in the given database where the column identified by the
label \meta{col key} has the given value. Note that no expansion is
performed in the match. The value must be an exact match.

\begin{definition}
\cs{xDTLassignfirstmatch}\marg{db name}\marg{col key}\marg{value}\marg{assign list}
\end{definition}
This is like \cs{DTLassignfirstmatch} but performs a~\emph{one-level}
expansion on \meta{value}.

Two rows can be swapped using:
\begin{definition}[\DescribeMacro{\DTLswaprows}]%
\cs{DTLswaprows}\marg{db name}\marg{row1 index}\marg{row2 index}
\end{definition}
where \meta{row1 index} and \meta{row2 index} are the indices
of the rows to be swapped. For example:
\begin{verbatim}
\DTLswaprows{scores}{3}{5}
\end{verbatim}
will swap the third and fifth rows.

\begin{example}{Two Database Rows Per Tabular Row
(Column-Wise)}{ex:twoblocks}
This example adapts \autoref{ex:2rows} so that the list is ordered
vertically rather than horizontally.
\begin{verbatim}
\begin{table}[htbp]
 \caption{Two database rows per tabular row (column-wise)}
 \centering
 % store half number of rows
 \edef\maxrows{\DTLrowcount{scores}}%
 \DTLdiv{\halfrowidx}{\maxrows}{2}
 \begin{tabular}{llcllc}
 \bfseries First Name &
 \bfseries Surname &
 \bfseries Score (\%) &
 \bfseries First Name &
 \bfseries Surname &
 \bfseries Score (\%)%
 \DTLforeach*[\value{DTLrowi}<\halfrowidx]{scores}%
 {\firstname=FirstName,\surname=Surname,\score=Score}%
 {%
   \\%
   \firstname & \surname & \score
   &
   \edef\currentrowidx{\arabic{DTLrowi}}%
   \DTLadd{\rowidxII}{\halfrowidx}{\currentrowidx}%
   \DTLassign{scores}{\rowidxII}%
     {\firstnameII=FirstName,\surnameII=Surname,\scoreII=Score}%
   \firstnameII & \surnameII & \scoreII
 }%
 \end{tabular}
\end{table}
\end{verbatim}

This produces \autoref{tab:twoblocks}.

\begin{table}[htbp]
 \caption{Two database rows per tabular row (column-wise)}
 \label{tab:twoblocks}
 \centering
 % store half number of rows
 \edef\maxrows{\DTLrowcount{scores}}%
 \DTLdiv{\halfrowidx}{\maxrows}{2}
 \begin{tabular}{llcllc}
 \bfseries First Name &
 \bfseries Surname &
 \bfseries Score (\%) &
 \bfseries First Name &
 \bfseries Surname &
 \bfseries Score (\%)%
 \DTLforeach*[\value{DTLrowi}<\halfrowidx]{scores}%
 {\firstname=FirstName,\surname=Surname,\score=Score}%
 {%
   \\%
   \firstname & \surname & \score
   &
   \edef\currentrowidx{\arabic{DTLrowi}}%
   \DTLadd{\rowidxII}{\halfrowidx}{\currentrowidx}%
   \DTLassign{scores}{\rowidxII}{\firstnameII=FirstName,\surnameII=Surname,\scoreII=Score}%
   \firstnameII & \surnameII & \scoreII
 }%
 \end{tabular}
\end{table}
\end{example}

\subsection{Operating on Current Row}
\label{sec:currentrow}

If you want to select from or edit a particular row in a database without having
to iterate through the database using \cs{DTLforeach}, you can use
the commands described in this section. Remember that the row index
is a reference to the internal data and is unrelated to references
in the original source (such as line numbers in a CSV file).

\begin{definition}[\DescribeMacro{\DTLgetrowindex}]
\cs{DTLgetrowindex}\marg{row cs}\marg{db name}\marg{col
idx}\marg{value}
\end{definition}
Gets the row index of the first row in database \meta{db name} where
the value for column \meta{col idx} matches \meta{value} and stores
the result in \meta{row cs}, which must be a control sequence.
An error message is given if not found.
\begin{definition}[\DescribeMacro{\dtlgetrowindex}]
\cs{dtlgetrowindex}\marg{row cs}\marg{db name}\marg{col
idx}\marg{value}
\end{definition}
Similar to \cs{DTLgetrowindex} but doesn't produce an error if no match is
found. You can test the result by using \cs{ifx}\meta{row
cs}\cs{dtlnovalue}. For example:
\begin{verbatim}
\dtlgetrowindex{\myrowidx}{data}{\dtlcolumnindex{data}{Surname}}{Smith}
\ifx\myrowidx\dtlnovalue
  Not Found
\else
  Found in row \myrowidx.
\fi
\end{verbatim}

If you want \meta{value} to be fully expanded before testing
you can use
\begin{definition}[\DescribeMacro{\xdtlgetrowindex}]
\cs{xdtlgetrowindex}\marg{row cs}\marg{db name}\marg{col
idx}\marg{value}
\end{definition}
(The \sty{etoolbox} package provides \cs{expandonce} if you
only want one level of expansion. See the \sty{etoolbox} manual for
further details.)

\begin{definition}[\DescribeMacro{\dtlgetrow}]
\cs{dtlgetrow}\marg{db name}\marg{row idx}
\end{definition}
Gets the row with index \meta{row idx} from the database \meta{db
name}. The required row is stored in the token register
\begin{definition}[\DescribeMacro{\dtlcurrentrow}]
\cs{dtlcurrentrow}
\end{definition}
the preceding rows are stored in the token register 
\begin{definition}[\DescribeMacro{\dtlbeforerow}]
\cs{dtlbeforerow} 
\end{definition}
the following rows are stored in the token register 
\begin{definition}[\DescribeMacro{\dtlafterrow}]
\cs{dtlafterrow}
\end{definition}
the row index, \meta{row idx}, is stored in the register
\begin{definition}[\DescribeMacro{\dtlrownum}]
\cs{dtlrownum}
\end{definition}
and the database name is stored in the control sequence
\begin{definition}[\DescribeMacro{\dtldbname}]
\cs{dtldbname}
\end{definition}
\begin{important}
No check is made in \cs{dtlgetrow} to see 
if the database exists or if the row index is valid. You will probably get a \qt{Missing \{
inserted} error if you misspell the database name and a \qt{Runaway
argument} error if you specify a row index that is out of range.
\end{important}

\begin{definition}[\DescribeMacro{\dtlgetrowforvalue}]
\cs{dtlgetrowforvalue}\marg{db name}\marg{column index}\marg{value}
\end{definition}
Like \cs{dtlgetrow}, but this gets the row where the entry in column
\meta{column index} matches \meta{value}. This command produces an
error if no match is found. \textbf{Note that no expansion is
performed when matching \meta{value}.} If you want \meta{value}
expanded before comparison, use:
\begin{definition}[\DescribeMacro{\edtlgetrowforvalue}]
\cs{edtlgetrowforvalue}\marg{db name}\marg{column index}\marg{value}
\end{definition}

You can use the commands below to access or edit \cs{dtlcurrentrow}, but
they won't change the database. Instead, once you've finished
editing \cs{dtlcurrentrow}, you need to reconstruct the database
token by recombining \cs{dtlbeforerow}, \cs{dtlcurrentrow} and
\cs{dtlafterrow} using:
\begin{definition}[\DescribeMacro{\dtlrecombine}]
\cs{dtlrecombine}
\end{definition}
Alternatively, to recombine omitting the current row:
\begin{definition}[\DescribeMacro{\dtlrecombineomitcurrent}]
\cs{dtlrecombineomitcurrent}
\end{definition}
(This removes the current row from the database, shifting the row
indices in \cs{dtlafterrow}.) Note that these recombining commands
assume that you haven't altered \cs{dtlrownum}, \cs{dtldbname},
\cs{dtlbeforerow} and \cs{dtlafterrow}.

\begin{important}
\cs{dtlcurrentrow} stores the row information using \sty{datatool}'s
internal row syntax, described in the documented code (datatool-code.pdf).
Don't explicitly modify \cs{dtlcurrentrow} unless you have a good
understanding of the syntax.
\end{important}

\begin{definition}[\DescribeMacro{\dtlgetentryfromcurrentrow}]
\cs{dtlgetentryfromcurrentrow}\marg{cs}\marg{col idx}
\end{definition}
Gets the value from \cs{dtlcurrentrow} for the column given by
\meta{col idx} (an integer) and stores in \meta{cs}, which must be a
control sequence.

\begin{definition}[\DescribeMacro{\dtlreplaceentryincurrentrow}]
\cs{dtlreplaceentryincurrentrow}\marg{new value}\marg{col idx}
\end{definition}
Replaces the value in \cs{dtlcurrentrow} for the column given by
\meta{col idx} (an integer) with \meta{new value}.
\begin{important}
The new value doesn't get expanded.
\end{important}

\begin{definition}[\DescribeMacro{\dtlremoveentryincurrentrow}]
\cs{dtlremoveentryincurrentrow}\marg{col idx}
\end{definition}
Removes the value in \cs{dtlcurrentrow} for the column given by
\meta{col idx}.

\begin{definition}[\DescribeMacro{\dtlswapentriesincurrentrow}]
\cs{dtlswapentriesincurrentrow}\marg{col1 idx}\marg{col2 idx}
\end{definition}
Swaps entries in columns \meta{col1 idx} and \meta{col2 idx} in
\cs{dtlcurrentrow} (where \meta{col1 idx} and \meta{col2 idx} are
the column indices).

\begin{definition}[\DescribeMacro{\dtlappendentrytocurrentrow}]
\cs{dtlappendentrytocurrentrow}\marg{key}\marg{value}
\end{definition}
Appends \meta{value} to the current row for column given by
\meta{key}. (Produces an error if there is already an entry for that
column in the current row.)

\begin{definition}[\DescribeMacro{\dtlupdateentryincurrentrow}]
\cs{dtlupdateentryincurrentrow}\marg{key}\marg{value}
\end{definition}
Behaves like \cs{dtlappendentrytocurrentrow} if the current row
doesn't contain an entry for the column given by \meta{key},
otherwise behaves like \cs{dtlreplaceentryincurrentrow}.

\begin{example}{Joining Two Databases in a Single Table}{ex:join}
\DTLnewdb{cmp101}\DTLnewrow{cmp101}\relax
\DTLnewdbentry{cmp101}{regnum}{12345}\relax
\DTLnewdbentry{cmp101}{Autumn Marks}{80}\relax
\DTLnewdbentry{cmp101}{Spring Marks}{85}\relax
\DTLnewrow{cmp101}\relax
\DTLnewdbentry{cmp101}{regnum}{12346}\relax
\DTLnewdbentry{cmp101}{Autumn Marks}{70}\relax
\DTLnewdbentry{cmp101}{Spring Marks}{90}\relax
\DTLnewrow{cmp101}\relax
\DTLnewdbentry{cmp101}{regnum}{12347}\relax
\DTLnewdbentry{cmp101}{Autumn Marks}{75}\relax
\DTLnewdbentry{cmp101}{Spring Marks}{60}\relax
\DTLnewdb{students}\relax
\DTLnewrow{students}\relax
\DTLnewdbentry{students}{regnum}{12344}\relax
\DTLnewdbentry{students}{name}{Mary Brown}\relax
\DTLnewrow{students}\relax
\DTLnewdbentry{students}{regnum}{12345}\relax
\DTLnewdbentry{students}{name}{Joe Bloggs}\relax
\DTLnewrow{students}\relax
\DTLnewdbentry{students}{regnum}{12346}\relax
\DTLnewdbentry{students}{name}{Jane Doe}\relax
\DTLnewrow{students}\relax
\DTLnewdbentry{students}{regnum}{12347}\relax
\DTLnewdbentry{students}{name}{John Smith}\relax
\DTLnewrow{students}\relax
\DTLnewdbentry{students}{regnum}{12348}\relax
\DTLnewdbentry{students}{name}{Alice Jones}\relax
Suppose a lecturer has a CSV file for a particular course that
contains student registration numbers and marks for the Autumn and
Spring semesters. The file is called, say, \texttt{cmp101.csv} and
contains the following:

\vskip\baselineskip
\begin{ttfamily}\par
regnum,Autumn Marks,Spring Marks\par
\DTLforeach*{cmp101}{\RegNum=regnum,\Autumn=Autumn Marks,\Spring=Spring Marks}
{\RegNum,\Autumn,\Spring\par}
\end{ttfamily}
\vskip\baselineskip\noindent

This only contains the student registration numbers, not their
names, but suppose there's another CSV file that contains the
registration numbers and names for all students at the department
(or university). This file called, say, \texttt{students.csv} may
look something like:

\vskip\baselineskip
\begin{ttfamily}\par
regnum,name\par
\DTLforeach*{students}{\RegNum=regnum,\Name=name}
{\RegNum,\Name\par}
\end{ttfamily}
\vskip\baselineskip\noindent

Now suppose the lecturer wants a table of all the students on course
CMP101 listing each student's name and marks. Here's the code:
\begin{verbatim}
\DTLloaddb{cmp101}{cmp101.csv}% load course data
\DTLloaddb{students}{students.csv}% load student data

\begin{table}[htbp]
 \caption{Student Marks (Joining Databases)}
 \centering
 \begin{tabular}{lrr}
 \bfseries Name & \bfseries Autumn Marks & \bfseries Spring Marks%
 \DTLforeach*{cmp101}%
 {\RegNum=regnum,\Autumn=Autumn Marks,\Spring=Spring Marks}%
 {\\\DTLfetch{students}{regnum}{\RegNum}{name} & \Autumn & \Spring}%
 \end{tabular}
\end{table}
\end{verbatim}

The result is shown in \autoref{tab:join}.

\begin{table}[htbp]
 \caption{Student Marks (Joining Databases)}\label{tab:join}
 \centering
 \begin{tabular}{lrr}
 \bfseries Name & \bfseries Autumn Marks & \bfseries Spring Marks%
 \DTLforeach*{cmp101}{\RegNum=regnum,\Autumn=Autumn Marks,\Spring=Spring Marks}
 {\\\DTLfetch{students}{regnum}{\RegNum}{name} & \Autumn & \Spring}%
 \end{tabular}
\end{table}

Let's suppose now that the \texttt{students.csv} file has the first
name and surname in separate columns rather than single columns. So
the CSV file looks like:
\DTLcleardb{students}\relax
\DTLnewrow{students}\relax
\DTLnewdbentry{students}{regnum}{12344}\relax
\DTLnewdbentry{students}{forename}{Mary}\relax
\DTLnewdbentry{students}{surname}{Brown}\relax
\DTLnewrow{students}\relax
\DTLnewdbentry{students}{regnum}{12345}\relax
\DTLnewdbentry{students}{forename}{Joe}\relax
\DTLnewdbentry{students}{surname}{Bloggs}\relax
\DTLnewrow{students}\relax
\DTLnewdbentry{students}{regnum}{12346}\relax
\DTLnewdbentry{students}{forename}{Jane}\relax
\DTLnewdbentry{students}{surname}{Doe}\relax
\DTLnewrow{students}\relax
\DTLnewdbentry{students}{regnum}{12347}\relax
\DTLnewdbentry{students}{forename}{John}\relax
\DTLnewdbentry{students}{surname}{Smith}\relax
\DTLnewrow{students}\relax
\DTLnewdbentry{students}{regnum}{12348}\relax
\DTLnewdbentry{students}{forename}{Alice}\relax
\DTLnewdbentry{students}{surname}{Jones}\relax

\vskip\baselineskip
\begin{ttfamily}\par
regnum,forename,surname\par
\DTLforeach*{students}{\RegNum=regnum,\Forename=forename,\Surname=surname}
{\RegNum,\Forename,\Surname\par}
\end{ttfamily}
\vskip\baselineskip\noindent

You may be tempted to replace
\begin{verbatim}
\DTLfetch{students}{regnum}{\RegNum}{name}
\end{verbatim}
with
\begin{verbatim}
\DTLfetch{students}{regnum}{\RegNum}{forename}\space
\DTLfetch{students}{regnum}{\RegNum}{surname}
\end{verbatim}
in the above code, but this is inefficient as it requires two
searches for the same row. Instead, you can do:
\begin{verbatim}
\DTLfetch{students}{regnum}{\RegNum}{forename}\space
\dtlgetentryfromcurrentrow{\Surname}{\dtlcolumnindex{students}{surname}}%
\Surname
\end{verbatim}
This can be done because
\begin{verbatim}
\DTLfetch{students}{regnum}{\RegNum}{forename}
\end{verbatim}
is equivalent to 
\begin{verbatim}
\edtlgetrowforvalue{students}{\dtlcolumnindex{students}{regnum}}{\RegNum}%
\dtlgetentryfromcurrentrow
  {\dtlcurrentvalue}{\dtlcolumnindex{students}{forename}}%
\dtlcurrentvalue
\end{verbatim}
This means that \cs{dtlcurrentrow} has already been set by
\cs{DTLfetch} so we can just do another
\cs{dtlgetentryfromcurrentrow} for the surname field.
The new code for the table is now:

\begin{verbatim}
\DTLloaddb{cmp101}{cmp101.csv}% load course data
\DTLloaddb{students}{students.csv}% load student data

\begin{table}[htbp]
 \caption{Student Marks (Joining Databases)}
 \centering
 \begin{tabular}{lrr}
 \bfseries Name & \bfseries Autumn Marks & \bfseries Spring Marks%
 \DTLforeach*{cmp101}%
 {\RegNum=regnum,\Autumn=Autumn Marks,\Spring=Spring Marks}%
 {\\%
  \DTLfetch{students}{regnum}{\RegNum}{forename}\space
  \dtlgetentryfromcurrentrow{\Surname}{\dtlcolumnindex{students}{surname}}%
  \Surname
  & \Autumn & \Spring}%
 \end{tabular}
\end{table}
\end{verbatim}

The result is shown in \autoref{tab:join2}.

\begin{table}[htbp]
 \caption{Student Marks (Joining Databases)}\label{tab:join2}
 \centering
 \begin{tabular}{lrr}
 \bfseries Name & \bfseries Autumn Marks & \bfseries Spring Marks%
 \DTLforeach*{cmp101}%
 {\RegNum=regnum,\Autumn=Autumn Marks,\Spring=Spring Marks}%
 {\\%
  \DTLfetch{students}{regnum}{\RegNum}{forename}\space
  \dtlgetentryfromcurrentrow{\Surname}{\dtlcolumnindex{students}{surname}}%
  \Surname
  & \Autumn & \Spring}%
 \end{tabular}
\end{table}

\textbf{Caveat:} be careful of scoping issues. Suppose you want the
first name and surname in separate columns, you may consider doing:
\begin{verbatim}
  \DTLfetch{students}{regnum}{\RegNum}{forename}%
  &
  \dtlgetentryfromcurrentrow{\Surname}{\dtlcolumnindex{students}{surname}}%
  \Surname
\end{verbatim}
(and adding an extra column to the \env{tabular} environment).
However this will result in undefined values for \cs{Surname} as
\cs{dtlcurrentrow} is only locally set. After the \verb|&| special
character \cs{dtlcurrentrow} has lost its value as it's no longer in the same
scope. You can fix this problem in a~number of ways. Firstly you can
make \cs{dtlcurrentrow} global after \cs{DTLfetch} via
\begin{verbatim}
\global\dtlcurrentrow=\dtlcurrentrow
\end{verbatim}
or you could move the column break to just before \cs{Surname} and
make \cs{Surname} global:
\begin{verbatim}
  \DTLfetch{students}{regnum}{\RegNum}{forename}%
  \dtlgetentryfromcurrentrow{\Surname}{\dtlcolumnindex{students}{surname}}%
  \global\let\Surname\Surname
  &
  \Surname
\end{verbatim}
There are other possibilities as well, but the first method is
probably the best, especially if you have multiple columns you want
to fetch.

Here's the updated code:
\begin{verbatim}
\begin{table}[htbp]
 \caption{Student Marks (Joining Databases)}
 \centering
 \begin{tabular}{llrr}
 \bfseries Forename & \bfseries Surname &
 \bfseries Autumn Marks & \bfseries Spring Marks%
 \DTLforeach*{cmp101}%
 {\RegNum=regnum,\Autumn=Autumn Marks,\Spring=Spring Marks}%
 {\\%
  \DTLfetch{students}{regnum}{\RegNum}{forename}%
  \global\dtlcurrentrow=\dtlcurrentrow
  &
  \dtlgetentryfromcurrentrow{\Surname}{\dtlcolumnindex{students}{surname}}%
  \Surname
  & \Autumn & \Spring}%
 \end{tabular}
\end{table}
\end{verbatim}
The result is shown in \autoref{tab:join3}.

\begin{table}[htbp]
 \caption{Student Marks (Joining Databases)}\label{tab:join3}
 \centering
 \begin{tabular}{llrr}
 \bfseries Forename & \bfseries Surname &
 \bfseries Autumn Marks & \bfseries Spring Marks%
 \DTLforeach*{cmp101}%
 {\RegNum=regnum,\Autumn=Autumn Marks,\Spring=Spring Marks}%
 {\\%
  \DTLfetch{students}{regnum}{\RegNum}{forename}%
  \global\dtlcurrentrow=\dtlcurrentrow
  &
  \dtlgetentryfromcurrentrow{\Surname}{\dtlcolumnindex{students}{surname}}%
  \Surname
  & \Autumn & \Spring}%
 \end{tabular}
\end{table}
\end{example}

\subsection{Advanced Iteration}
\label{sec:advancediter}

The \ics{DTLforeach} command described in \sectionref{sec:dbforeach}
has some limitations, especially when trying to iterate through
large databases. This section describes lower-level user commands
that may be used for iteration instead of \cs{DTLforeach}.

\begin{definition}[\DescribeMacro{\dtlforeachkey}]%
\cs{dtlforeachkey}(\meta{key cs},\meta{col cs},\meta{type 
cs},\meta{header cs})\cs{in}\marg{db}\cs{do}\marg{body}
\end{definition}
This iterates through all the keys in the database given by
\meta{db}. In each iteration, \meta{key cs} is set to the key,
\meta{col cs} is set to the column index, \meta{type cs} is set to
the data type (as for \cs{DTLgetdatatype}), \meta{header cs} is set
to the header for that column, and then \meta{body} is done. Note
that \meta{key cs}, \meta{col cs}, \meta{type cs} and \meta{header
cs} must all be control sequences. No check is performed to
determine if that control sequence already exists, and the control
sequences are defined globally (since it's likely that
\cs{dtlforeachkey} may be used within a \env{tabular} environment),
so you need to make sure you don't override an existing command of
the same name.

\begin{definition}[\DescribeMacro{\dtlforcolumn}]%
\cs{dtlforcolumn}\marg{cs}\marg{db}\marg{key}\marg{body}
\end{definition}
This iterates through the column given by \meta{key} in the
database given by \meta{db} and applies \meta{body}. In each
iteration, \meta{cs} (which must be a control sequence) is set to
the current element in the column and may be used in \meta{body}.
Alternatively, if you want to identify the column by its index
rather than its key, use:
\begin{definition}[\DescribeMacro{\dtlforcolumnidx}]%
\cs{dtlforcolumnidx}\marg{cs}\marg{db}\marg{col index}\marg{body}
\end{definition}
Both \cs{dtlforcolumn} and \cs{dtlforcolumnidx} have a starred
version that doesn't check for the existence of the given database.
You may use \ics{dtlbreak} within \meta{body} to break out of the loop
at the end of the current iteration.

An alternative to \cs{DTLforeach} is to use \TeX's primitive
\ics{loop} with \ics{DTLgetvalue}, but this may not be faster. 
For example, suppose I have a CSV
file with a single column (labelled \qt{Word}) with 1000 rows of
data (where the CSV file is called \texttt{test-data-1000.csv}). 
First, let's use the unstarred version of \cs{DTLforeach}:
\begin{verbatim}
\batchmode
\documentclass{article}
\usepackage{datatool}

\DTLloaddb{data}{test-data-1000.csv}

\begin{document}
\DTLforeach{data}{\Word=Word}{\Word.\par}
\end{document}
\end{verbatim}
On my 64bit Linux computer, this document took 4.942s to compile.
Result from \texttt{time pdflatex test}:
\begin{verbatim}
real	0m4.942s
user	0m4.934s
sys	0m0.015s
\end{verbatim}
Now using the starred version of \cs{DTLforeach}:
\begin{verbatim}
\batchmode
\documentclass{article}
\usepackage{datatool}

\DTLloaddb{data}{test-data-1000.csv}

\begin{document}
\DTLforeach*{data}{\Word=Word}{\Word.\par}
\end{document}
\end{verbatim}
This took 2.138s to compile:
\begin{verbatim}
real	0m2.138s
user	0m2.122s
sys	0m0.020s
\end{verbatim}
Now using \TeX's \cs{loop}:
\begin{verbatim}
\batchmode
\documentclass{article}
\usepackage{datatool}

\DTLloaddb{data}{test-data-1000.csv}

\begin{document}

\newcount\rowctr
\loop
 \advance\rowctr by 1\relax
 \DTLgetvalue{\Word}{data}{\rowctr}{1}%
 \Word.\par
\ifnum\rowctr<\DTLrowcount{data}
\repeat

\end{document}
\end{verbatim}
(This takes advantage of the fact that I know I only have one column
of data, so I only need to reference column~1 in \cs{DTLgetvalue}.)
This takes 2.638s to compile:
\begin{verbatim}
real	0m2.638s
user	0m2.622s
sys	0m0.020s
\end{verbatim}
which is slightly longer than using \cs{DTLforeach*} (but not nearly
as long as using the unstarred version). Another possible method is
to use \cs{dtlgetrow} and \cs{dtlgetentryfromcurrentrow}:
\begin{verbatim}
\batchmode
\documentclass{article}
\usepackage{datatool}

\DTLloaddb{data}{test-data-1000.csv}

\begin{document}

\newcount\rowctr
\loop
 \advance\rowctr by 1\relax
 \dtlgetrow{data}{\rowctr}%
 \dtlgetentryfromcurrentrow{\Word}{1}%
 \Word.\par
\ifnum\rowctr<\DTLrowcount{data}
\repeat

\end{document}
\end{verbatim}
This took 3.596s to compile:
\begin{verbatim}
real	0m3.596s
user	0m3.582s
sys	0m0.019s
\end{verbatim}
Another possibility is to use \cs{dtlforcolumnidx}, described above:
\begin{verbatim}
\batchmode
\documentclass{article}
\usepackage{datatool}

\DTLloaddb{data}{test-data-1000.csv}

\begin{document}

\dtlforcolumnidx{\Word}{data}{1}{\Word.\par}

\end{document}
\end{verbatim}
This took 2.093s to compile:
\begin{verbatim}
real	0m2.093s
user	0m2.083s
sys	0m0.013s
\end{verbatim}
So if you only want to iterate through one column, this is the
fastest method, but it's still more efficient to pre-process the
data using an external script that creates a \texttt{.tex} file that
can be \cs{input} into the document.

Note that the build time increases with extra columns, \emph{even if
they're not required in the document}. For example, I created a new
CSV file called \texttt{test-data-1000-5.csv} that had four extra
columns (which were actually duplicates of the first column with
different headers, for simplicity). Just iterating through the first
column to obtain the same PDF as previously significantly increases
the time taken. The only modification to the above examples was an
edit to the \cs{DTLloaddb} line:
\begin{verbatim}
\DTLloaddb{data}{test-data-1000-5.csv}
\end{verbatim}
The fastest method using \cs{dtlforcolumnidx} took 25.688s:
\begin{verbatim}
real	0m25.688s
user	0m25.712s
sys	0m0.013s
\end{verbatim}
Using \cs{DTLforeach*} took 25.725s:
\begin{verbatim}
real	0m25.725s
user	0m25.744s
sys	0m0.018s
\end{verbatim}
The unstarred version took 35.665s:
\begin{verbatim}
real	0m35.665s
user	0m35.692s
sys	0m0.022s
\end{verbatim}
Using \cs{loop} and \cs{DTLgetvalue} took 27.844s:
\begin{verbatim}
real	0m27.844s
user	0m27.866s
sys	0m0.019s
\end{verbatim}
Using \cs{loop} and \cs{dtlgetrow} took 31.770s:
\begin{verbatim}
real	0m31.770s
user	0m31.785s
sys	0m0.028s
\end{verbatim}

On the other hand, if I want to only iterate through, say, the last
100 rows of the data, it's simpler to use \cs{loop}. For example:
\begin{verbatim}
\batchmode
\documentclass{article}

\usepackage{datatool}

\DTLloaddb{data}{test-data-1000-5.csv}

\begin{document}
\newcount\rowctr
\rowctr=\numexpr\DTLrowcount{data}-100\relax
\loop
 \advance\rowctr by 1\relax
 \DTLgetvalue{\Word}{data}{\rowctr}{1}%
 \Word.\par
\ifnum\rowctr<\DTLrowcount{data}
\repeat
\end{document}
\end{verbatim}
This took 25.124s:
\begin{verbatim}
real	0m25.124s
user	0m25.138s
sys	0m0.023s
\end{verbatim}
The equivalent using \cs{DTLforeach} is:
\begin{verbatim}
\batchmode
\documentclass{article}

\usepackage{datatool}

\DTLloaddb{data}{test-data-1000-5.csv}

\begin{document}
\newcount\firstidx
\firstidx=\numexpr\DTLrowcount{data}-100\relax
\DTLforeach*{data}{\Word=Word}%
 {\ifnum\DTLcurrentindex>\firstidx\relax\Word.\par\fi}
\end{document}
\end{verbatim}
(Remember that the row count can't be used in the optional argument
of \cs{DTLforeach} as it's only incremented when the condition is
true.) This took 26.375s:
\begin{verbatim}
real	0m26.375s
user	0m26.381s
sys	0m0.029s
\end{verbatim}
This takes longer because it's still iterating over every row of the
database and is applying the condition to each row.

\minisec{Summary}
\begin{itemize}
\item If possible, use an external script to pre-process the data so
that you can simply \cs{input} valid \LaTeX\ code into the document.
\item If you only want to iterate through one column of the data,
use \cs{dtlforcolumnidx}.
\item Remove\footnote{Naturally, make a copy of the original data, if
necessary.}\ unwanted columns and\slash or rows from the CSV file (and sort, if
necessary) using the spreadsheet application (or whatever) that was
used to generate the original CSV file. 
\end{itemize}

\chapter{Creating an index, glossary or list of acronyms
(\texorpdfstring{\sty{datagidx}}{datagidx} package)}
\label{sec:datagidx}

The \sty{datagidx} package is provided as an alternative to the
\sty{glossaries} package. Rather than relying on an external
indexing application, such as \app{xindy} or
\app{makeindex}, it uses the database mechanism of the
\sty{datatool} package. \emph{\sty{datagidx} and \sty{glossaries}
are not compatible.} (Note: \sty{glossaries} version 4.04 now has an
option that uses \TeX\ to sort the glossaries instead of using
\app{makeindex} or \app{xindy}.)

First a repeat of the caveat at the start of this manual:
\begin{important}
Use the right tool for the right job.
\end{important}
Don't expect \sty{datagidx} to perform as efficiently as an
application that is designed specifically to sort and collate entries.

If, however, you are happy to exchange efficiency for the
convenience of not having to invoke an external application in
between \LaTeX\ runs, read on.

Sections~\ref{sec:newgidx} and~\ref{sec:newterm} describe how to
create and populate a database that's used to store terms or
acronyms. By default the database is sorted when it's displayed
using \ics{printterms} (see section~\ref{sec:printterms}). This is
where the main inefficiency lies in this package. A~faster
alternative is to use \app{datatooltk} (see page~\pageref{datatooltk})
and its \texttt{datagidx} plugin, which will allow you to enter
terms in a graphical environment and sort the terms. This way, you
only need to sort the database after you enter a new term and the
sorting is done by a more efficient language than \TeX. Note that
this means returning to using an external helper application, but it
only needs to be used when you add a new term rather than between
each pair of \LaTeX\ runs.

Once you've edited and sorted the database in \app{datatooltk}, 
you can then just load it using:
\begin{definition}[\DescribeMacro{\loadgidx}]
\cs{loadgidx}\oarg{options}\marg{filename}\marg{title}
\end{definition}
where \meta{filename} is the name of the file saved in
\app{datatooltk}. The remaining arguments \meta{options} and
\meta{title} are the same as for \cs{newgidx}, described in
section~\ref{sec:newgidx}. This command automatically sets the
default database to the loaded database. You can change the default
database using \cs{DTLgidxSetDefaultDB}, described in
section~\ref{sec:newterm}.

Since \cs{loadgidx} is intended for use with presorted databases,
the \csopt{loadgidx}{sort} key defaults to nothing.

\begin{important}
If you've opted to use \sty{datagidx} over \sty{glossaries} because
you don't want to install Perl, then don't bother with
\app{datatooltk} because, although it's a Java application, it requires 
Perl for the plugins.
\end{important}

\section{Defining Index/Glossary Databases}
\label{sec:newgidx}

\begin{important}
The databases and their associated entries described here can only
be defined in the preamble. This is because the database must be set
up before the auxiliary file is read. If you don't want to lose your
place by constantly returning to the preamble to add a new term
while you edit your document, consider putting all your definitions
in a separate file which can be \cs{input} in the preamble.
You can then switch between files without losing your place
(provided you are using a decent text editor). Alternatively, use 
\app{datatooltk}'s \texttt{datagidx} plugin as described above.
\end{important}

First you need to define a customised database that will be used to
store the entries in your index, glossary or list of acronyms:
\begin{definition}[\DescribeMacro\newgidx]
\cs{newgidx}\oarg{options}\marg{label}\marg{title}
\end{definition}
This defines a new database with a unique label and a title.
For example:
\begin{verbatim}
\newgidx{index}{Index}
\end{verbatim}
I can now identify this database using the label \texttt{index}. The
title ``Index'' is the default heading when the database is
displayed using \cs{printterms} (see \autoref{sec:printterms}).

The optional argument \meta{options} should be a key=value list.
Available options:
\begin{description}
\item[\csopt{newgidx}{showgroups}] Boolean option that indicates
whether or not to insert group headings (and a group separator)
between index groups, if headings are supported by the given style.
If no value is supplied, \textsf{true} is assumed.

\item[\csopt{newgidx}{style}] The style to use. The value should
be the name of the style. Available styles are listed in
\autoref{sec:indexstyles}.

\item[\csopt{newgidx}{sort}] How to sort the database. See
\autoref{sec:indexsort} for further details.

\item[\csopt{newgidx}{balance}] This is a boolean option that
is only applied if \csopt{printterms}{columns} is greater than~1.
If \textsf{true}, the columns are balanced. If \textsf{false}, the
columns aren't balanced. If no value is specified, \textsf{true} is
assumed. If \verb|balance=false| and \verb|columns=2| 
\ics{twocolumn} is used instead of \env{multicols*}.

\item[\csopt{newgidx}{heading}] The heading at the start of the
index/glossary.

\item[\csopt{newgidx}{postheading}] What to put immediately after
the heading.

\end{description}

\section{Locations}
\label{sec:locations}

Each term in an index or glossary database has an associated
location list. This is initially null. When you display the database
using \ics{printterms} (see \autoref{sec:printterms}) only those
entries with a non-null location list or with a ``see''
cross-reference are displayed. The location by default is the page
number on which the entry has been used. This may be changed to
another counter by redefining
\begin{definition}[\DescribeMacro\DTLgidxCounter]
\cs{DTLgidxCounter}
\end{definition}
to the name of the required counter. For example:
\begin{verbatim}
\renewcommand*{\DTLgidxCounter}{section}
\end{verbatim}
The \styfmt{datagidx} package knows about the following counter
styles: \texttt{arabic}, \texttt{roman}, \texttt{Roman},
\texttt{alph} and \texttt{Alph}. If your location counter uses a
different style, you will need to add a new location type. This will
only work if the counter uses a command that expands to another
command that takes a number as its argument. For example, suppose I
want to use small caps Roman numeral page numbering. I need to
define a command (say \cs{myscroman}) that takes a counter name as
its argument but expands to another command that takes a number as
its argument, like this:
\begin{verbatim}
\newcommand*{\myscroman}[1]{\myscrromannum{\value{#1}}}
\newcommand*{\myscromannum}[1]{\textsc{\romannumeral#1}}
\end{verbatim}
Note that the font changing command \cs{textsc} is in the definition
of \cs{myscromannum} not in the definition of \cs{myscroman}.
The page counter can now be changed so that it uses \cs{myscroman}:
\begin{verbatim}
\renewcommand*{\thepage}{\myscroman{page}}
\end{verbatim}
I now have to indicate that \cs{myscromannum} is a valid location
type using:
\begin{definition}[\DescribeMacro\DTLgidxAddLocationType]
\cs{DTLgidxAddLocationType}\marg{cs name}
\end{definition}
where \meta{cs name} is the name of the control sequence without the
initial backslash. Like this:
\begin{verbatim}
\DTLgidxAddLocationType{myscromannum}
\end{verbatim}
Note that this is the command that takes a number as its argument
(\cs{myscromannum}) not the command that takes a counter name as its
argument (\cs{myscroman}).

As with \app{makeindex} and \app{xindy}, locations may have
a compositor. The default compositor is a full stop but may be
changed by redefining
\begin{definition}[\DescribeMacro\DTLgidxSetCompositor]
\cs{DTLgidxSetCompositor}
\end{definition}
Alternatively, you can use the package option \pkgopt{compositor}.

\section{Defining Terms}
\label{sec:newterm}

Once you have defined the database, you can now define terms
associated with that database using
\begin{definition}[\DescribeMacro\newterm]
\cs{newterm}\oarg{options}\marg{name}
\end{definition}
where \meta{name} is the term and \meta{options} is a
comma-separated list of \meta{key}=\meta{value} options. The
following keys are available:
\begin{description}
\item[\csopt{newterm}{database}] Identifies the database in which to
store this term. For example:
\begin{verbatim}
\newterm[database=index]{reptile}
\end{verbatim}
It can be somewhat cumbersome having to keep typing the database for
each new term. Instead you can identify the default database using
\begin{definition}[\DescribeMacro\DTLgidxSetDefaultDB]
\cs{DTLgidxSetDefaultDB}\marg{label}
\end{definition}
\textbf{Note:} the argument \meta{label} is not expanded.

Example:
\begin{verbatim}
 % define two indexes:
 \newgidx{index}{Index}
 \newgidx{people}{People}
 % Set "index" as the default database:
 \DTLgidxSetDefaultDB{index}
 % This batch of terms will be added to database "index":
 \newterm{reptile}
 \newterm{mammal}
 \newterm{insect}
 % Set "people" as the default database:
 \DTLgidxSetDefaultDB{people}
 % This batch of terms will be added to database "people":
 \newterm{Bob}
 \newterm{Mary}
 \newterm{Jane}
\end{verbatim}

\item[\csopt{newterm}{label}] A unique identifying label. This
should not contain any active characters. If omitted, the label is
extracted from \meta{name} (see below).

\item[\csopt{newterm}{sort}] The sort key. If omitted, this is
extracted from \meta{name} (see below).

\item[\csopt{newterm}{parent}] The parent entry, if this is a
sub-term. An entry may only have one parent. If you want the same
term to appear under two different parents, you'll have to define
two separate terms with the same name but different parents (and
different labels). This is the only way to avoid ambiguity with the
hyperlinks (if enabled).

\item[\csopt{newterm}{text}] How the entry should appear in the
document text. This is \meta{name} by default. If this option is
used, \meta{name} indicates how the entry should appear in the
index, glossary or list of acronyms.

\item[\csopt{newterm}{description}] An optional description. This is
usually not required for an index but needed for a glossary.

\item[\csopt{newterm}{plural}] The plural form of the term. If
omitted this is formed by appending ``s'' to \meta{name} (or the
value of the \csopt{newterm}{text} key if supplied).

\item[\csopt{newterm}{symbol}] An associated symbol if required.

\item[\csopt{newterm}{short}] An associated short form if required.
(Default \meta{name}.)

\item[\csopt{newterm}{long}] An associated long form if required.
(Default \meta{name}.)

\item[\csopt{newterm}{shortplural}] An associated short plural if
required. (Default formed by appending ``s'' to the value of the
\csopt{newterm}{short} key.)

\item[\csopt{newterm}{longplural}] An associated long plural if
required. (Default formed by appending ``s'' to the value of the
\csopt{newterm}{long} key.)

\item[\csopt{newterm}{see}] A cross-reference to a synonym. The
value should be the label of another entry. This entry will not have
a location list, just the reference to the other term.

\item[\csopt{newterm}{seealso}] A cross-reference to a closely
related term. This entry should have both a location list and a
reference to the other term.

\end{description}

If the \csopt{label} or \csopt{sort} key are omitted, \sty{datagidx}
tries to form sensible defaults. At the moment, this involves
stripping certain commands (\ics{MakeUppercase},
\ics{MakeLowercase},
\ics{MakeTextUppercase}, \ics{MakeTextLowercase}, \ics{acronymfont},
\ics{textsc}, \ics{textbf}, \ics{textmd}, \ics{textit},
\ics{textsl}, \ics{textrm}, \ics{texttt}, \ics{textsf}, \ics{emph},
\ics{ensuremath} and \ics{textsuperscript}), stripping accents
and replacing certain control characters or control sequences
(\verb|~| is replace with a space and \verb|\&| is replaced with
\ics{andname} (if defined) or ``and'' (if \cs{andname} isn't
defined)). The Greek letter commands (\ics{alpha} etc) are converted
to their name.

Examples:
\begin{enumerate}
\item \ics{ensuremath} is stripped and \ics{alpha} is converted to
``alpha'' so the following:
\begin{verbatim}
\newterm{\ensuremath{\alpha}}
\end{verbatim}
sets both the label and sort to \texttt{alpha} but the name and text
fields are set to \verb|\ensuremath{\alpha}|.

\item Accent commands are stripped so the following:
\begin{verbatim}
\newterm{mac\'edoine}
\end{verbatim}
sets both the label and sort fields to \texttt{macedoine} but the
name and text fields are set to \verb|mac\'edoine|.

\begin{important}
The first letter must be grouped if it's an accent or ligature
command.
\end{important}

\item This example must have the sort and label fields set manually
because the first letter has an accent:
\begin{verbatim}
\newterm[label=elite,sort=elite]{{\'e}lite}
\end{verbatim}

\item This used to also apply when using the \sty{inputenc} package
with older versions of \styfmt{datatool-base} and \styfmt{mfirstuc}.
However now it's no longer necessary for accents:
\begin{verbatim}
\newterm{élite}
\end{verbatim}

\item Commands such as \ics{oe} aren't dealt with, so you must
manually set the label and sort key:
\begin{verbatim}
\newterm[label=manoeuvre,sort=manoeuvre]{man\oe uvre}
\end{verbatim}

\item The same applies to plural terms set explicitly:
\begin{verbatim}
\newterm
 [%
   plural={{\oe}sophagi},%
   label={oesophagus},%
   sort={oesophagus}%
 ]
 {{\oe}sophagus}
\end{verbatim}

\item The same applies if you are using the \sty{inputenc} package
to enter ligatures:
\begin{verbatim}
\newterm[label=manoeuvre,sort=manoeuvre]{manœuvre}
\end{verbatim}

\begin{important}
Take care if any of the values to fields contain a comma or equal
sign. The value must be grouped.
\end{important}

\item This term contains a comma in some of the fields:
\begin{verbatim}
\newterm
 [%
   label={comma},%
   sort={,},%
   text={comma (,)}%
   plural={commas (,)}%
 ]
 {, (comma)}
\end{verbatim}
In the text, the entry is \texttt{comma (,)} but in the index the
entry is sorted according to the comma symbol and is displayed as
\texttt{, (comma)}.

\end{enumerate}

\subsection{Commands to Assist Sorting}

There are some situations where you will have to specify the sort
key, for example:
\begin{verbatim}
\newterm
[
  sort={Ten Downing Street}
]
{10 Downing Street}
\end{verbatim}

However, there are some commands provided to help set the default
sort for entries that are sorted differently from the way they are
typeset in the index/glossary, which can help reduce the number of times you
need to explicitly set the sort field.

\begin{definition}[\DescribeMacro\DTLgidxParen]
\cs{DTLgidxParen}\marg{text}
\end{definition}
This command is provided for parenthetical material that should be
typeset in the index, but should not contribute to the sort unless
there is an identical entry without parenthetical material.

For example:
\begin{verbatim}
\newterm{0\DTLgidxParen{zero}}
\end{verbatim}
This term is typeset as \texttt{0 (zero)}, but has the sort and
label fields set to \texttt{0}.

The default sort used is word-order sorting. This has a special
number group for entries where the sort field consists solely of
digits and they are sorted numerically rather than by string
comparison. Using \cs{DTLgidxParen} in this manner, the following
terms will appear in numerical order in the index:
\begin{verbatim}
\newterm{0\DTLgidxParen{zero}}
\newterm{1\DTLgidxParen{one}}
\newterm{2\DTLgidxParen{two}}
\newterm{3\DTLgidxParen{three}}
\newterm{10\DTLgidxParen{ten}}
\newterm{100\DTLgidxParen{one hundred}}
\newterm{20\DTLgidxParen{twenty}}
\end{verbatim}
If \cs{DTLgidxParen} was not used and the parentheses were
explicitly included, e.g.\ \texttt{0 (zero)}, then the entries would
be placed in the symbol group instead and be sorted according to
string (so \texttt{10 (ten)} would come before \texttt{2 (two)}).

\begin{definition}[\DescribeMacro\DTLgidxPlace]
\cs{DTLgidxPlace}\marg{country/county}\marg{city/town}
\end{definition}
Use this command to indicate a place. For example:
\begin{verbatim}
\newterm{\DTLgidxPlace{USA}{New York}}
\end{verbatim}
This sets the label and name to \texttt{New York, USA}, the text
field is set to just \texttt{New York} and the sort field is set to 
\verb|New York\datatoolplacecomma USA| (see \autoref{sec:sort}).

\begin{definition}[\DescribeMacro\DTLgidxSubject]
\cs{DTLgidxSubject}\marg{subject}\marg{text}
\end{definition}
Use this to indicate a subject, concept or object. Example:
\begin{verbatim}
\newterm{\DTLgidxSubject{population}{New York}}
\end{verbatim}
Both the label and name fields default to \texttt{New York,
population}, the text field defaults to \texttt{population} and the
sort field is set to \verb|New York\datatoolsubjectcomma population|
(see \autoref{sec:sort}).

\begin{definition}[\DescribeMacro\DTLgidxName]
\cs{DTLgidxName}\marg{forename(s)}\marg{surname}
\end{definition}
Use this command to index a person. The entry will be sorted
according to the surname then the forenames. The entry will be
displayed as \meta{surname}, \meta{forename(s)} in the index but
will be displayed as \meta{forename(s)} \meta{surname} when
referenced in the document. The label, on the other hand, is set to
just the surname. Example:
\begin{verbatim}
\newterm{\DTLgidxName{Donald E.}{Knuth}}
\end{verbatim}
This sets the name field to \texttt{Knuth, Donald E.}, the text
field to \texttt{Donald E. Knuth}, the label to \texttt{Knuth} and the sort field to
\texttt{Knuth\ics{datatoolpersoncomma} Donald E.} (see
\autoref{sec:sort}).

A person's title (such as ``Dr'') should typically not affect the
sort, unless there is another person with the same surname and
forenames (or initials) without a title. To assist this, you can
identify a person's title using:
\begin{definition}[\DescribeMacro\DTLgidxRank]
\cs{DTLgidxRank}\marg{title}\marg{forename(s)/initial(s)}
\end{definition}
Using examples from the Oxford Style Manual:
\begin{verbatim}
\newterm[label=AliceMeynell]{\DTLgidxName{Meynell}{Alice}}
\newterm[label=DrMeynell]{\DTLgidxName{Meynell}{\DTLgidxRank{Dr}{A.}}}
\newterm[label=AMeynell]{\DTLgidxName{Meynell}{A.}}
\end{verbatim}
Here the labels must be set as the surnames are identical for each
entry, but the entries will be sorted in the order: ``Meynell, A.'',
``Meynell, Dr~A.'' and ``Meynell, Alice''.

You can use
\begin{definition}[\DescribeMacro\DTLgidxNameNum]
\cs{DTLgidxNameNum}\marg{number}
\end{definition}
to indicate a number associated with a name. The number is typeset
as an uppercase Roman numeral in the text, but is sorted
numerically.

For example:
\begin{verbatim}
\newterm{James~\DTLgidxNameNum{1}}
\end{verbatim}
This is typeset as \verb|James~I|, but gets the label 
\texttt{James I} (note no tilde) and the sort field is set to
\texttt{James 01}. This means that if I want to index all the Kings
whose name is James, they will appear in the correct order in the
index.

If a term contains a variant of ``Mac'' you can also use:
\begin{definition}[\DescribeMacro\DTLgidxMac]
\cs{DTLgidxMac}\marg{text}
\end{definition}
The entry will be typeset with \meta{text} but the sort key will
have \meta{text} replaced with \texttt{Mac}. Examples:
\begin{verbatim}
\newterm{\DTLgidxName{Joe}{\DTLgidxMac{Mc}Cullers}}
\newterm{\DTLgidxName{Bob}{\DTLgidxMac{M'}Fingal}}
\newterm{\DTLgidxMac{Mc}Carthyism}
\newterm{\DTLgidxMac{Mc}Guffin}
\end{verbatim}

Similarly saints can be identified using:
\begin{definition}[\DescribeMacro\DTLgidxSaint]
\cs{DTLgidxSaint}\marg{text}
\end{definition}
Examples:
\begin{verbatim}
\newterm{\DTLgidxSaint{St} Julian}
\newterm{\DTLgidxName{Q.}{\DTLgidxSaint{St}~John-Smythe}}
\newterm{\DTLgidxPlace{\DTLgidxSaint{St}~Andrews}{Fife}}
\end{verbatim}
These will be sorted according to \texttt{Saint Julian}, 
\verb|Saint John-Smythe\datatoolpersoncomma Q.| and
\verb|Saint Andrews\datatoolplacecomma Fife|.

Particles, such as ``de'', ``von'' or ``of'' are usually ignored
when sorting. These can be identified using:
\begin{definition}[\DescribeMacro\DTLgidxParticle]
\cs{DTLgidxParticle}\marg{text}
\end{definition}
Examples:
\begin{verbatim}
\newterm{\DTLgidxName{Fred}{\DTLgidxParticle{de}{Winter}}}
\newterm{\DTLgidxName{Gustav}{\DTLgidxParticle{von}{Aschenbach}}}
\end{verbatim}
Here the names are sorted according to
\verb|Winter\datatoolpersoncomma Fred| and
\verb|Aschenbach\datatoolpersoncomma Gustav| but the labels are set
to \texttt{deWinter} and \texttt{vonAschenbach}.

A person can also be indicated by their office, for example ``Henry,
scribe of Bury St~Edmunds''. For this, you can use:
\begin{definition}[\DescribeMacro\DTLgidxOffice]
\cs{DTLgidxOffice}\marg{office}\marg{name}
\end{definition}
Here the label defaults to just \meta{name}, so you may need to set
the label manually to ensure uniqueness. Examples:
\begin{verbatim}
\newterm
[
  label={HenrySonJohn}
]
{\DTLgidxOffice{son \DTLgidxParticle{of}{John}}{Henry}}

\newterm
[
  label={HenryBeaumont}
]
{\DTLgidxOffice{bishop \DTLgidxParticle{of}{Bayeux}}{Henry
\DTLgidxParticle{de}{Beaumont}}}

\newterm
[
  label={HenryScribe}
]
{\DTLgidxOffice{scribe \DTLgidxParticle{of}{Bury}
   \DTLgidxSaint{St}~Edmunds}{Henry}}


\end{verbatim}

You can hook into the mechanism that sets the default sort key by
adding to the definition of
\begin{definition}[\DescribeMacro\newtermlabelhook]
\cs{newtermlabelhook}
\end{definition}
You can use \sty{etoolbox}'s \ics{appto} command to append to this
hook. For example, suppose you want to index the terms \verb|\TeX|,
\verb|e\TeX| and \verb|pdf\TeX|, but you want the terms to have the
label and sort fields to be just \verb|TeX|, \verb|eTeX| and
\verb|pdfTeX|, then you can add to the hook so that it automatically
converts \ics{TeX} to just \texttt{TeX}:
\begin{verbatim}
\appto\newtermlabelhook{\def\TeX{TeX}}
\end{verbatim}
(Note that it's important to use the local \ics{def} rather than the
global \ics{gdef} to ensure the redefinition is localised.)

Now the terms can simply be defined using:
\begin{verbatim}
\newterm{\TeX}
\newterm{e\TeX}
\newterm{pdf\TeX}
\end{verbatim}

To assist in using this mechanism, the following commands are
available (these commands may also be used in the mandatory argument
of \cs{newterm}):
\begin{definition}[\DescribeMacro\DTLgidxNoFormat]
\cs{DTLgidxNoFormat}\marg{text}
\end{definition}
This commands simply does its argument, so any commands that should
be stripped from the label or sort field without the loss of their
argument can be \ics{let} to \cs{DTLgidxNoFormat}. For example,
suppose you want to define a command called, say, \cs{appname} that
you want to use to identify application names, like this:
\begin{verbatim}
\newcommand*{\app}[1]{\texttt{#1}}
\end{verbatim}
This command needs to be stripped from the label and sort, so it can
be added to the hook like this:
\begin{verbatim}
\appto\newtermlabelhook{\let\app\DTLgidxNoFormat}
\end{verbatim}
Now you can define terms like this:
\begin{verbatim}
\newterm{\app{makeindex}}
\newterm{\app{xindy}}
\end{verbatim}
The label and sort keys are then set to \texttt{makeindex} (for the
first term) and \texttt{xindy} (for the second term).

\begin{definition}[\DescribeMacro\DTLgidxGobble]
\cs{DTLgidxGobble}\marg{text}
\end{definition}
This command discards its argument, so it can be used if you not
only want to strip a command but also its argument from the label
and sort fields.

For example, suppose you want some terms to have a footnote (both in the
index/glossary and in the document text) but the footnote shouldn't
form part of the sort or label fields. You can add to the hook like
this:
\begin{verbatim}
\appto\newtermlabelhook{\let\footnote\DTLgidxGobble}
\end{verbatim}
Now you can define some terms with footnotes:
\begin{verbatim}
\newterm{foo\footnote{a note about foo}}
\newterm{bar\footnote{a note about bar}}
\end{verbatim}
The label and sort keys are then set to \texttt{foo} (for the first
term) and \texttt{bar} (for the second term).

\begin{definition}[\DescribeMacro\DTLgidxIgnore]
\cs{DTLgidxIgnore}
\end{definition}
This is similar to \cs{DTLgidxGobble} but only affects the sort key
not the label. Example:
\begin{verbatim}
\newterm{de\DTLgidxIgnore{-}escalate}
\end{verbatim}
This is displayed as \texttt{de-escalate} and gets the label
\texttt{de-escalate} but is sorted according to \texttt{deescalate}.

\begin{definition}[\DescribeMacro\DTLgidxStripBackslash]
\cs{DTLgidxStripBackslash}\marg{control sequence}
\end{definition}
This can be used to ``stringify'' a control sequence and remove the
leading backslash. For example, suppose you want to index the
ampersand symbol (\&) but you want to sort it according to the
actual symbol \verb|&|, you can do:
\begin{verbatim}
\newterm
 [%
   label={amp},
   sort={\DTLgidxStripBackslash{\&}},
   text={ampersand (\&)},
   plural={ampersands (\&)},
 ]
 {\& (ampersand)}
\end{verbatim}

\section{Referencing Terms}
\label{sec:useentry}

You can reference terms using
\begin{definition}[\DescribeMacro\useentry]
\cs{useentry}\marg{label}\marg{field}
\end{definition}
This fetches the given field for the term identified by
\meta{label}, displays it and marks the term as having been used.
Example, suppose I have previous (in the preamble) defined the term
``reptile'' using:
\begin{verbatim}
\newterm{reptile}
\end{verbatim}
I can now reference this term in the document:
\begin{verbatim}
\useentry{reptile}{Text}
\end{verbatim}
or if I want the plural, I can use:
\begin{verbatim}
\useentry{reptile}{Plural}
\end{verbatim}
There are also uppercase versions:
\begin{definition}[\DescribeMacro\Useentry]
\cs{Useentry}\marg{label}\marg{field}
\end{definition}
This makes the first letter uppercase (using the \sty{mfirstuc}
package) or to make the whole text uppercase use:
\begin{definition}[\DescribeMacro\USEentry]
\cs{USEentry}\marg{label}\marg{field}
\end{definition}

If you use the \sty{hyperref} package, the above commands will
automatically create hyperlinks to the relevant entry in the
index/glossary.  You can suppress this action by using one of the
following analogous commands instead:
\begin{definition}[\DescribeMacro\useentrynl]
\cs{useentrynl}\marg{label}\marg{field}
\end{definition}
\begin{definition}[\DescribeMacro\Useentrynl]
\cs{Useentrynl}\marg{label}\marg{field}
\end{definition}
\begin{definition}[\DescribeMacro\USEentrynl]
\cs{USEentrynl}\marg{label}\marg{field}
\end{definition}

You can also specify your own custom text:
\begin{definition}[\DescribeMacro\glslink]
\cs{glslink}\marg{label}\marg{text}
\end{definition}

In all the above commands, the \meta{label} argument may optionally
start with \oarg{format}, where format is the name of a control
name \emph{without} the preceding backslash. This command will be
applied to this location in the entry's location list when it's
displayed in the index/glossary.

For example:
\begin{verbatim}
\useentry{[textbf]reptile}{Text}
\end{verbatim}
Note that the command (\cs{textbf} in the above example) should take
one argument (the location). If you attempt to use, say, a
declaration (such as \cs{bfseries}) the effect won't be localised.

You can display the value of a field without indexing it using:
\begin{definition}[\DescribeMacro\glsdispentry]
\cs{glsdispentry}\marg{label}\marg{field}
\end{definition}
To make the first letter uppercase, use:
\begin{definition}[\DescribeMacro\Glsdispentry]
\cs{Glsdispentry}\marg{label}\marg{field}
\end{definition}
The above commands aren't expandable. If you want to fetch a value
without displaying or using it, you can use:
\begin{definition}[\DescribeMacro\DTLgidxFetchEntry]
\cs{DTLgidxFetchEntry}\marg{cs}\marg{label}\marg{field}
\end{definition}
where \meta{cs} is a control sequence, \meta{label} is the label
that uniquely identifies the entry and \meta{field} is the required
field. The value of that field is stored in \meta{cs}.

The predefined database fields are:
\begin{description}
\item[Name] How the term appears in the index/glossary (as specified
by the mandatory argument of \ics{newterm}).
\item[Text] The value of the \csopt{newterm}{text} field.
\item[Plural] The value of the \csopt{newterm}{plural} field.
\item[Description] The value of the \csopt{newterm}{description}
field.
\item[Symbol] The value of the \csopt{newterm}{symbol} field.
\item[Long] The value of the \csopt{newterm}{long} field.
\item[Short] The value of the \csopt{newterm}{short} field.
\item[LongPlural] The value of the \csopt{newterm}{longplural}
field.
\item[ShortPlural] The value of the \csopt{newterm}{shortplural}
field.
\item[See] The value of the \csopt{newterm}{see} field.
\item[SeeAlso] The value of the \csopt{newterm}{seealso} field.
\item[Sort] The value of the \csopt{newterm}{sort} field.
\item[Parent] The value of the \csopt{newterm}{parent} field.
\item[Label] The entry's unique identifying label.
\item[Used] Has the value 1 (entry has been used) or either 0 or
undefined (entry hasn't been used). 
\item[Location] The entry's location list (picked up from the last
\LaTeX\ run).
\end{description}

In addition, there are some fields designed for internal use:
\texttt{Child}, \texttt{FirstId} and \texttt{CurrentLocation}.

You can add an entry to the index/glossary without displaying any
text using:
\begin{definition}[\DescribeMacro\glsadd]
\cs{glsadd}\marg{label}
\end{definition}
As with \cs{useentry}, \meta{label} maybe in the form
\oarg{format}\marg{label} where \meta{format} is the name of a
control sequence \emph{without} the leading backslash.

You can also add all entries from a particular database using
\begin{definition}[\DescribeMacro\glsaddall]
\cs{glsaddall}\marg{db name}
\end{definition}
where \meta{db name} is the name of the database.

Unlike the commands of the same name provided by the
\sty{glossaries} package, here there is a difference between
\cs{glsaddall} and using \cs{glsadd} on all entries in the database.
In the case of \cs{glsadd} a location is added to the location list
for that entry. However in the case of \cs{glsaddall} no location is
added to each entry's location list, but the location list is set to
non-null so the entry will appear in the index/glossary.

\subsection{Shortcut Commands}

There are some shortcuts to common fields (if you are used to the
\sty{glossaries} package, note that these commands have different
formats to the commands provided by \sty{glossaries} with the same
name):
\begin{definition}[\DescribeMacro\gls]
\cs{gls}\marg{label}
\end{definition}
This is equivalent to \cs{useentry}\marg{label}\verb|{Text}|.
\begin{definition}[\DescribeMacro\glspl]
\cs{glspl}\marg{label}
\end{definition}
This is equivalent to \cs{useentry}\marg{label}\verb|{Plural}|.
\begin{definition}[\DescribeMacro\glsnl]
\cs{glsnl}\marg{label}
\end{definition}
This is equivalent to \cs{useentrynl}\marg{label}\verb|{Text}|.
\begin{definition}[\DescribeMacro\glsplnl]
\cs{glsplnl}\marg{label}
\end{definition}
This is equivalent to \cs{useentrynl}\marg{label}\verb|{Plural}|.

\begin{definition}[\DescribeMacro\Gls]
\cs{Gls}\marg{label}
\end{definition}
This is equivalent to \cs{Useentry}\marg{label}\verb|{Text}|.
\begin{definition}[\DescribeMacro\Glspl]
\cs{Glspl}\marg{label}
\end{definition}
This is equivalent to \cs{Useentry}\marg{label}\verb|{Plural}|.
\begin{definition}[\DescribeMacro\Glsnl]
\cs{Glsnl}\marg{label}
\end{definition}
This is equivalent to \cs{Useentrynl}\marg{label}\verb|{Text}|.
\begin{definition}[\DescribeMacro\Glsplnl]
\cs{Glsplnl}\marg{label}
\end{definition}
This is equivalent to \cs{Useentrynl}\marg{label}\verb|{Plural}|.

\begin{definition}[\DescribeMacro\glssym]
\cs{glssym}\marg{label}
\end{definition}
This is equivalent to \cs{useentry}\marg{label}\verb|{Symbol}|.
\begin{definition}[\DescribeMacro\Glssym]
\cs{Glssym}\marg{label}
\end{definition}
This is equivalent to \cs{Useentry}\marg{label}\verb|{Symbol}|.

\section{Adding Extra Fields}

You can add new fields to the index/glossary database using:
\begin{definition}[\DescribeMacro\newtermaddfield]
\cs{newtermaddfield}\oarg{db list}\marg{field name}\marg{key
name}\marg{default value}
\end{definition}
The optional argument \meta{db list} is a comma-separated list of
databases that should have this new field. If omitted, the field
will be added to all the defined databases. The argument \meta{field
name} is the label to give this new column in the database(s). The
argument \meta{key name} is the name of the new key to use in the
optional argument of \ics{newterm}. The final argument \meta{default
value} is the default value if the key isn't used. Within
\meta{default value}, you may use
\begin{definition}[\DescribeMacro\field]
\cs{field}\marg{key}
\end{definition}
to indicate the value of another key.


For example, suppose I want to be able to specify an alternative
plural. I can add a new field like this:
\begin{verbatim}
\newtermaddfield{AltPlural}{altplural}{}
\end{verbatim}
This adds a new column with the label \texttt{AltPlural} to each
defined index/glossary database and adds a new key called
\texttt{altplural} that I can now use in \ics{newterm}. The default
is set to empty. Now I can define terms with an alternative plural:
\begin{verbatim}
\newterm[altplural=kine]{cow}
\end{verbatim}
In the document, I can use \verb|\gls{cow}| to display ``cow'',
\verb|\glspl{cow}| to display ``cows'' and
\verb|\useentry{cow}{AltPlural}| to display ``kine''. To make life a
little easier, I can define a new command to save typing:
\begin{verbatim}
\newcommand*{\glsaltpl}[1]{\useentry{#1}{AltPlural}}
\end{verbatim}
Now I can just do \verb|\glsaltpl{cow}| to display ``kine''.

Here's another example. Suppose I want to add a field that produces
the past tense of a verb. In this case, the default should be formed
by appending ``ed'' to the \csopt{newterm}{text} field. The new
field can be defined as follows:
\begin{verbatim}
\newtermaddfield{Ed}{ed}{\field{text}ed}
\end{verbatim}
This adds a new column labelled ``Ed'' and defines a new key called
``ed'' that can be used with \ics{newterm}. Now I can defined some
verbs:
\begin{verbatim}
\newterm{jump}
\newterm[ed=went]{go}
\end{verbatim}

Let's define a convenience command to access this field:
\begin{verbatim}
\newcommand*{\glsed}[1]{\useentry{#1}{Ed}}
\end{verbatim}
This new field can now be referenced in the document:
\begin{verbatim}
He \glsed{jump} over the gate.
She \glsed{go} to the shop.
\end{verbatim}
The above will be displayed as: He jumped over the gate. She went to
the shop.

\section{Acronyms}

You may have noticed that you can specify \csopt{newterm}{short} and
\csopt{newterm}{long} fields when you define a new term. There is a
convenient shortcut command which uses \ics{newterm} to define an
acronym. The syntax is:
\begin{definition}[\DescribeMacro\newacro]
\cs{newacro}\oarg{options}\marg{short}\marg{long}
\end{definition}
This is a shortcut for
\begin{alltt}
\ics{newterm}
 [%
   description=\{\ics{capitalisewords}\marg{long}\},%
   short=\{\cs{acronymfont}\marg{short}\},%
   long=\marg{long},%
   text=\{\cs{DTLgidxAcrStyle}\marg{long}\{\cs{acronymfont}\marg{short}\}\},%
   plural=\{\cs{DTLgidxAcrStyle}\{\meta{long}s\}\{\cs{acronymfont}\{\meta{short}s\}\}\},%
   sort=\marg{short},%
   \meta{options}%
 ]%
 {\ics{MakeTextUppercase}\marg{short}}
\end{alltt}
where \ics{capitalisewords} is defined in \sty{mfirstuc}
(automatically loaded by \styfmt{datagidx}) and
\ics{MakeTextUppercase} is defined in \sty{textcase} (automatically
loaded by \styfmt{datagidx}). The other commands used are defined by
\styfmt{datagidx}:
\begin{definition}[\DescribeMacro\acronymfont]
\cs{acronymfont}
\end{definition}
By default this just typesets its argument but can be redefined if
the acronyms need to be typeset in a certain style (such as small
caps).
\begin{definition}[\DescribeMacro\DTLgidxAcrStyle]
\cs{DTLgidxAcrStyle}\marg{long}\marg{short}
\end{definition}
This governs how the acronym is typeset in the \csopt{newterm}{text}
field. This defaults to: \meta{long} (\meta{short}).

\subsection{Using Acronyms}

You can use terms that represent acronyms via commands such as
\ics{useentry}. For example, if you define the following in the
preamble:
\begin{verbatim}
\newacro{css}{cascading style sheet}
\end{verbatim}
then later in the text you can use:
\begin{verbatim}
\useentry{css}{Short}
\end{verbatim}
to access the short form and
\begin{verbatim}
\useentry{css}{Long}
\end{verbatim}
to access the long form. You can also use
\begin{verbatim}
\useentry{css}{Text}
\end{verbatim}
(or \verb|\gls{css}|) to access the full version. However with
acronyms you generally only want the full form on first use and just
the short form on subsequent use. The following commands are
provided to do that. The singular form is obtained using:
\begin{definition}[\DescribeMacro\acr]
\cs{acr}\marg{label}
\end{definition}
The plural form is obtained using:
\begin{definition}[\DescribeMacro\acrpl]
\cs{acrpl}\marg{label}
\end{definition}

Note that, unlike the \sty{glossaries} package, \cs{acr} isn't the
same as \cs{gls}. With \styfmt{datagidx}, \cs{gls} always references
the \csopt{newterm}{text} field. There is no ``first'' field.

\begin{important}
Take care when using acronyms with \cls{beamer}. Using overlays can
cause problems with first use expansions.
\end{important}

As a general rule, you're not supposed to capitalise the first
letter of an acronym (especially if it is displayed in small caps) but if you need to
you can use:
\begin{definition}[\DescribeMacro\Acr]
\cs{Acr}\marg{label}
\end{definition}
and
\begin{definition}[\DescribeMacro\Acrpl]
\cs{Acrpl}\marg{label}
\end{definition}

\subsection{Unsetting and Resetting Acronyms}

You can reset a term so it's marked as not used with:
\begin{definition}[\DescribeMacro\glsreset]
\cs{glsreset}\marg{label}
\end{definition}
or you can unset a term so it's marked as used with:
\begin{definition}[\DescribeMacro\glsunset]
\cs{glsunset}\marg{label}
\end{definition}

You can reset all the terms defined in a given database using:
\begin{definition}[\DescribeMacro\glsresetall]
\cs{glsresetall}\marg{db name}
\end{definition}
or unset all the terms defined in a given database using:
\begin{definition}[\DescribeMacro\glsunsetall]
\cs{glsunsetall}\marg{db name}
\end{definition}
where \meta{db name} is the name of the database as supplied when
the database was defined using \ics{newgidx}.

\section{Conditionals}

You can test if a term exists using
\begin{definition}[\DescribeMacro\iftermexists]
\cs{iftermexists}\marg{label}\marg{true part}\marg{false part}
\end{definition}
You can test if a term has been used using:
\begin{definition}[\DescribeMacro\ifentryused]
\cs{ifentryused}\marg{label}\marg{true part}\marg{false part}
\end{definition}

\section{Displaying the Index or Glossary}
\label{sec:printterms}

The index or glossary can be displayed using
\begin{definition}[\DescribeMacro\printterms]
\cs{printterms}\oarg{options}
\end{definition}
You will need to run \LaTeX\ at least twice to ensure your
index/glossary is up-to-date. The first run will only display any
entries that have a ``See'' field defined.

The optional argument \meta{options} is a comma-separated list of
\meta{key}=\meta{value} options. Available keys:
\begin{description}
\item[\csopt{printterms}{database}] The name of the database (as
given in \ics{newgidx}).

\item[\csopt{printterms}{postdesc}] This may have the value
\textsf{dot} (put a full stop after the description) or 
\textsf{none} (don't put a full stop after the description). 

\item[\csopt{printterms}{prelocation}] This indicates what to put
before the location list. Available values:
\begin{description}
\item[\textsf{none}] Nothing.
\item[\textsf{enspace}] An en-space.
\item[\textsf{space}] An ordinary space.
\item[\textsf{dotfill}] A dotted line (\ics{dotfill}).
\item[\textsf{hfill}] Expandable space (\ics{hfill}).
\end{description}

\item[\csopt{printterms}{location}] This indicates how to display
the location list. Available values:
\begin{description}
\item[\textsf{hide}] Don't display the location list.
\item[\textsf{list}] Display the location list.
\item[\textsf{first}] Only display the first location in the list.
\end{description}

\item[\csopt{printterms}{symboldesc}] How to format the symbol in
relation to the description. Available values:
\begin{description}
\item[\textsf{symbol}] Display the symbol but not the description.
\item[\textsf{desc}] Display the description but not the symbol
field.
\item[\textsf{(symbol) desc}] Display the symbol (if defined) in parentheses
followed by the description.
\item[\textsf{desc (symbol)}] Display the description followed by
the symbol (if defined) in parentheses.
\item[\textsf{symbol desc}] Display the symbol (if defined) followed
by the description.
\item[\textsf{desc symbol}] Display the description followed by the
symbol (if defined).
\end{description}

\item[\csopt{printterms}{columns}] This should be a positive number
that indicates the page column layout. If the value is greater
than~1, the \env{multicols} environment is used (defined in the
\sty{multicol} package, which is automatically loaded).

\item[\csopt{printterms}{namecase}] Indicates whether any case
change should be applied to the entry's name field. Available
values:
\begin{description}
\item[\textsf{nochange}] Don't apply a case change.
\item[\textsf{uc}] Convert the name to uppercase.
\item[\textsf{lc}] Convert the name to lowercase.
\item[\textsf{firstuc}] Convert the first letter to uppercase (using
\ics{makefirstuc} defined in \sty{mfirstuc}).
\item[\textsf{capitalise}] Capitalise initial letters of each word
in the name (using \ics{capitalisewords} defined in \sty{mfirstuc}).
\end{description}

\item[\csopt{printterms}{namefont}] The font changing command to
apply to the name. (Include the initial backslash.) Declarations may
be used.

\item[\csopt{printterms}{postname}] What to put after the name.

\item[\csopt{printterms}{see}] Indicates how the cross-reference
(given in the ``See'' field) should be displayed. Available values:
\begin{description}
\item[\textsf{comma}] Insert a comma followed by a space in front of
the cross-reference.
\item[\textsf{brackets}] Insert a space before the cross-reference
and put the cross-reference in parentheses.
\item[\textsf{dot}] Insert a full stop followed by a space in front
of the cross-reference.
\item[\textsf{space}] Insert a space before the cross-reference.
\item[\textsf{nosep}] Don't insert anything before the
cross-reference.
\item[\textsf{semicolon}] Insert a semi-colon followed by a space in
front of the cross-reference.
\item[\textsf{location}] Display the cross-reference in the same way
as a location.
\end{description}

\item[\csopt{printterms}{child}] Indicates whether child entries
should have their name displayed. Available values: \textsf{named}
(display the child's name) and \textsf{noname} (don't display the
child's name).

\item[\csopt{printterms}{showgroups}] Boolean option that indicates
whether or not to insert group headings (and a group separator)
between index groups, if headings are supported by the given style.
If no value is supplied, \textsf{true} is assumed.

\item[\csopt{printterms}{style}] The style to use. The value should
be the name of the style. Available styles are listed in
\autoref{sec:indexstyles}.

\item[\csopt{printterms}{symbolwidth}] Some of the styles allow you
to specify a width for the symbol field. This width can be specified
with this option. The value will be ignored by some of the styles.

\item[\csopt{printterms}{locationwidth}] Some of the styles allow you
to specify a width for the location field. This width can be specified
with this option. The value will be ignored by some of the styles.

\item[\csopt{printterms}{childsort}] A boolean option that indicates
whether or not the child entries should be sorted. If \textsf{true}, the
child entries are listed using the same sort order as the sort
applied to the database. If \textsf{false}, the child entries are
listed in the order they were defined. If the value is missing,
\textsf{true} is assumed.

\item[\csopt{printterms}{heading}] The heading at the start of the
index/glossary.

\item[\csopt{printterms}{postheading}] What to put immediately after
the heading.

\item[\csopt{printterms}{sort}] How to sort the database. See
\autoref{sec:indexsort} for further details.

\item[\csopt{printterms}{balance}] This is a boolean option that
is only applied if \csopt{printterms}{columns} is greater than~1.
If \textsf{true}, the columns are balanced. If \textsf{false}, the
columns aren't balanced. If no value is specified, \textsf{true} is
assumed.

\item[\csopt{printterms}{condition}] This specifies a boolean
condition (as used by \ics{DTLforeach}) so you can display only
those entries where the condition is met. For example, to only
display entries starting with ``H'' (not including any entry that is
just the letter ``H'') you can do:
\begin{verbatim}
\printterms[condition={\DTLisiopenbetween{\Name}{H}{I}}]
\end{verbatim}
\end{description}

\subsection{Index or Glossary Styles}
\label{sec:indexstyles}

The index or glossary style is given by the
\csopt{newgidx,printterms}{style} key in the optional argument of
\ics{newgidx} or \ics{printterms}. The following styles are
available:

\begin{description}
\item[index]
The ``index'' style is a basic style for an index. This style
accepts the \csopt{printterms}{locationwidth} and
\csopt{printterms}{symbolwidth} keys in \ics{printterms}. This is
the default style.

\item[indexalign]
The ``indexalign'' style is similar to the ``index'' style but
aligns the descriptions.

\item[align]
The ``align'' style aligns the fields. This style
accepts the \csopt{printterms}{locationwidth} and
\csopt{printterms}{symbolwidth} keys in \ics{printterms}.

\item[gloss]
The ``gloss'' style is a basic glossary style. This style uses
\begin{definition}[\DescribeMacro\DTLgidxChildSep]
\cs{DTLgidxChildSep}
\end{definition}
as the separator between child entries (defaults to a space) and
\begin{definition}[\DescribeMacro\DTLgidxPostChild]
\cs{DTLgidxPostChild}
\end{definition}
to indicate what to put after the list of child entries (defaults to
nothing).

\item[dict]
The ``dict'' style is designed for dictionary-like glossaries. This
assumes a hierarchical structure where the top level entries have a
name. The next level is used to indicate a category (such as
``adjective'' or ``noun''). If there is only one meaning for the
term, this level also has a description. If there is more than one
meaning, each meaning should be a child of the category entry. Only
third level entries are numbered. No further levels are expected.
The symbol field is ignored.

If \csopt{printterms}{showgroups} is set, the group headers will be
placed in a \ics{chapter} (if defined) or in a \ics{section} (if
\ics{chapter} isn't defined).

This style uses:
\begin{definition}[\DescribeMacro\DTLgidxCategoryNameFont]
\cs{DTLgidxCategoryNameFont}\marg{text}
\end{definition}
The font used to display the name of the category (first child
level).
\begin{definition}[\DescribeMacro\DTLgidxCategorySep]
\cs{DTLgidxCategorySep}
\end{definition}
The category separator. (Defaults to a space).
\begin{definition}[\DescribeMacro\DTLgidxSubCategorySep]
\cs{DTLgidxSubCategorySep}
\end{definition}
The category separator. (Defaults to a space).
\begin{definition}[\DescribeMacro\DTLgidxDictPostItem]
\cs{DTLgidxDictPostItem}
\end{definition}
Indicates what to do at the end of each top-level item. (Defaults to
\cs{par}).

The indentation is given by the length register
\begin{definition}[\DescribeMacro\datagidxdictindent]
\cs{datagidxdictindent}
\end{definition}
This value defaults to 1em.

\end{description}

For additional commands that affect the style of the indexes or
glossaries, see the documented code \texttt{datatool-code.pdf}.

\subsection{Sorting the Index or Glossary Database}
\label{sec:indexsort}

By default the index/glossary databases are sorted according to the
\texttt{Sort} field using the \ics{dtlwordindexcompare} handler (see
\autoref{sec:sort}). Note that the \emph{entire} database is sorted,
which is less efficient that using external indexing applications,
such as \app{makeindex} or \app{xindy}, which only sort the
terms that have been used in the document. In addition, the sorting
algorithm used by \styfmt{datatool} is less efficient than that used
by a custom-built sorting and collation application.

The database is sorted at the start of \ics{printterms} according to
the value of the \csopt{printterms}{sort} key supplied by
\ics{printterms}. To completely suppress the sorting, set this key
to empty. Example:
\begin{verbatim}
\printterms[database=index,sort={},showgroups=false]
\end{verbatim}
Note that in the above, I also switched off the group headers as
they don't make sense with an unsorted index or glossary.

If you want to use a different comparison handler, you can set the
\csopt{printterms}{sort} key to the required sort command, where you
can use
\begin{definition}[\DescribeMacro\DTLgidxCurrentdb]
\cs{DTLgidxCurrentdb}
\end{definition}
to indicate the current database.

For example, to sort using letter rather than word comparison:
\begin{verbatim}
\printterms[database=index, 
  sort={\dtlsort{Sort}{\DTLgidxCurrentdb}{\dtlletterindexcompare}}]
\end{verbatim}

You may recall from earlier that the index/glossary databases have a
column labelled ``FirstId''. This can be used if you want to sort
the database according to the order of usage. Example:
\begin{verbatim}
\printterms[database=index,
 sort={\dtlsort{FirstId}{\DTLgidxCurrentdb}{\dtlcompare}}]
\end{verbatim}
Note that here I've used the \ics{dtlcompare} handler (which is the
fastest handler) as I'm only concerned with a numerical rather than
a string comparison.

The default value of the \csopt{printterms}{sort} key is actually:
\begin{verbatim}
\dtlsort{Sort,FirstId}{\DTLgidxCurrentdb}{\dtlwordindexcompare}}
\end{verbatim}
This means that entries with duplicate ``Sort'' fields are then
sorted according to use.

\subsubsection{Optimization}
\label{sec:optimize}

If you have used \app{xindy} or \app{makeindex}, you'll be
familiar with the document creation process. The document is first
compiled, then the indexing application is run to sort and collate
the entries, then the document is compiled again (and possible once
more). This involves two (or three) \LaTeX\ runs and one sort and
collate run. With the \styfmt{datagidx} package, the sorting and
collation is done every \LaTeX\ run. For a large index, this can be
quite slow. If you're not editing the index or glossary, you might
prefer not to have to keep sorting the database whenever you update
the document. To assist this, \styfmt{datagidx} provides the
\pkgopt{optimize} package option. This may take the following
values:
\begin{description}
\item[\pkgoptval{off}{optimize}] Don't use the optimize facility.
(The index/glossary databases will be sorted every run, unless the
sort is switched off by setting the \csopt{printterms}{sort} key to
empty.)

\item[\pkgoptval{low}{optimize}] Use the ``low'' optimize setting.
This only sorts the index/glossary databases every other run.
(Assuming that the sorting is done via the \cs{printterms}{sort} key
rather than explicitly using \ics{dtlsort} or \ics{DTLsort}
somewhere else in the document.) Don't use this option if sorting
the databases makes the document out-of-date. (For example, the
group headers use sectioning commands.)

\item[\pkgoptval{high}{optimize}] Use the ``high'' optimize setting.
This sorts the index/glossary databases on the first run, then
writes the sorted databases to external files, which are read in on
subsequent runs. Again this assumes that the sorting is done via the
\cs{printterms}{sort} key. Don't use this option if you want to edit
the index/glossary database.
\end{description}

\section{Package Options}
\label{sec:datagidxoptions}

The following package options are available for \styfmt{datagidx}:
\begin{description}
\item[\pkgopt{utf8}] A boolean option (same as for
\sty{datatool-base}). If you both load \sty{inputenc} with UTF-8
support and you use accent commands like \cs{'} or \cs{c} then make
sure you have at least version 2.05 of \sty{mfirstuc} if you want to
use commands like \cs{Gls}.

\item[\pkgopt{optimize}] Sets the optimization. (See
\autoref{sec:optimize}.)

\item[\pkgopt{columns}] Sets the default number of columns to use
for the indexes or glossaries. (See \autoref{sec:printterms}.)

\item[\pkgopt{child}] Sets whether or not to show the name in child
entries, where the style supports this option. (See \autoref{sec:printterms}.)

\item[\pkgopt{namecase}] Sets the case change for the entry's name.
(See \autoref{sec:printterms}.)

\item[\pkgopt{namefont}] Sets the font for the entry's name.
(See \autoref{sec:printterms}.)

\item[\pkgopt{postname}] Indicates what to put after the entry's name.
(See \autoref{sec:printterms}.)

\item[\pkgopt{postdesc}] Indicates what to put after the entry's
description.  (See \autoref{sec:printterms}.)

\item[\pkgopt{prelocation}] Indicates what to put before the entry's
location.  (See \autoref{sec:printterms}.)

\item[\pkgopt{location}] Indicates how to display the entry's
location.  (See \autoref{sec:printterms}.)

\item[\pkgopt{see}] Indicates how to display the entry's
cross-reference list.  (See \autoref{sec:printterms}.)

\item[\pkgopt{symboldesc}] Indicates how to display the entry's
symbol in relation to the description.  (See \autoref{sec:printterms}.)

\item[\pkgopt{compositor}] Sets the location compositor. (See
\autoref{sec:locations}.)

\item[\pkgopt{draft}] Displays additional information, such as
target names.

\item[\pkgopt{final}] Hides the draft information.

\item[\pkgopt{verbose}] Use \sty{datatool}'s verbose mode.

\item[\pkgopt{nowarn}] A boolean option that suppresses
\sty{datagidx}'s rerun warnings.

\end{description}

\begin{example}{Creating an Index}{ex:index}
In this document, I have used the \sty{datagidx} package and the
\sty{hyperref} package. In the preamble, I have the following:
\begin{verbatim}
\usepackage{datagidx}
\usepackage[colorlinks]{hyperref}

\newgidx{index}{Index}% define a database for the index

\DTLgidxSetDefaultDB{index}% set this as the default 

\newterm{mac\'edoine}
\newterm{macram\'e}
\newterm[label=elite]{{\'e}lite}
\newterm{reptile}
\newterm[seealso={reptile}]{crocodylian}

\newterm
 [%
   parent=crocodylian
 ]
 {crocodile}

\newterm
 [%
   parent=crocodylian
 ]
 {alligator}

\newterm
 [%
   parent=crocodylian,
   description={(also cayman)}
 ]
 {caiman}

\newterm[see={caiman}]{cayman}
\end{verbatim}

Now here's some code to go in the document:
\begin{verbatim}
Here are some words containing accents: \gls{macedoine},
\gls{macrame} and \gls{elite}. \Gls{elite} requires extra care as it
starts with an accented letter. A \gls{crocodylian} is a family of
\glspl{reptile} consisting of \glspl{crocodile}, \glspl{alligator} and
\glspl{caiman}.
\end{verbatim}
This produces the following:

Here are some words containing accents: \gls{macedoine},
\gls{macrame} and \gls{elite}. \Gls{elite} requires extra care as it
starts with an accented letter. A \gls{crocodylian} is a family of
\glspl{reptile} consisting of \glspl{crocodile}, \glspl{alligator} and
\glspl{caiman}.

The index can then be displayed using:
\begin{verbatim}
\printterms[heading={\section*},database=index]
\end{verbatim}
This requires two runs to ensure the index is up-to-date. The
resulting index is as follows:

\printterms
 [
   heading={\section*},
   database={gidx-index}
 ]

\par\vskip\baselineskip\noindent
Here's the code if you want to add the letter groups (I've also
added a dotted line before the location):
\begin{verbatim}
\printterms
 [
   heading={\section*},
   database=index,
   prelocation=dotfill,
   showgroups
 ]
\end{verbatim}
which produces:
\printterms
 [
   heading={\section*},
   database=gidx-index,
   prelocation=dotfill,
   showgroups
 ]

\end{example}

\chapter{Pie Charts (\texorpdfstring{\sty{datapie}}{datapie} package)}
\label{sec:datapie}

The \sty{datapie} package is not loaded by the \sty{datatool} package,
so you need to explicitly load \sty{datapie} if you want to use any of the
commands defined in this section. You will also need to have the
\sty{pgf}/\sty{tikz} packages installed. The \sty{datapie} package 
may be given the following options:
\begin{description}
\item[{\pkgopt{color}}] Colour option (default).
\item[{\pkgopt{gray}}] Grey scale option.
\item[{\pkgopt{rotateinner}}] 
Rotate inner labels so that they are aligned
with the pie chart radial axis.
\item[{\pkgopt{norotateinner}}] 
Don't rotate inner labels (default).
\item[{\pkgopt{rotateouter}}] 
Rotate outer labels so that they are aligned
with the pie chart radial axis.
\item[{\pkgopt{norotateouter}}] 
Don't rotate outer labels (default).
\end{description}

Numerical information contained in a database created by the
\sty{datatool} package can be converted into a pie chart using
\begin{definition}[\DescribeMacro{\DTLpiechart}]%
\cs{DTLpiechart}\oarg{condition}\marg{settings list}\marg{db name}\marg{values}
\end{definition}\noindent
where \meta{db name} is the name of the database, and 
\meta{condition} has the same form as the optional argument 
to \ics{DTLforeach} described in \autoref{sec:dbforeach}. If 
\meta{condition} is false, that information is omitted from the
construction of the pie chart. The argument \meta{values} is a
comma separated list of \meta{cmd}"="\meta{key} pairs, the same
as that required by the penultimate argument of \ics{DTLforeach}.
The \meta{settings list} is a comma separated list of 
\meta{setting}=\meta{value} pairs, where \meta{setting} can be any of
the following:
\begin{description}
\item[\csopt{DTLpiechart}{variable}] 
This specifies the control sequence to use that 
contains the value used to construct the pie chart. The control
sequence must be one of the control sequences to appear in
the assignment list \meta{values}. This setting is required.

\item[\csopt{DTLpiechart}{start}] 
This is the starting angle of the first segment. The
value is 0 by default.

\item[\csopt{DTLpiechart}{radius}] 
This is the radius of the pie chart. The default value
is 2cm.

\item[\csopt{DTLpiechart}{innerratio}] 
The distance from the centre of the
pie chart to the point where the inner labels are placed is given
by this value multiplied by the radius. The default value is 0.5. 

\item[\csopt{DTLpiechart}{outerratio}]
The distance from the centre of the
pie chart to the point where the outer labels are placed is given
by this value multiplied by the radius. The default value is 1.25.

\item[\csopt{DTLpiechart}{cutawayratio}]
The distance from the centre of the pie chart
to the point of cutaway segments is given by this value multiplied
by the radius. The default value is 0.2.

\item[\csopt{DTLpiechart}{inneroffset}] 
This is the absolute distance from the centre
of the pie chart to the point where the inner labels are placed.
You should use only one or other of \csopt{DTLpiechart}{innerratio}
and \csopt{DTLpiechart}{inneroffset}, not both. If you also want to 
specify the radius, you must use \csopt{DTLpiechart}{radius}
before \csopt{DTLpiechart}{inneroffset}. If omitted, the inner 
offset is obtained from the radius multiplied by the 
\csopt{DTLpiechart}{innerratio} value.

\item[\csopt{DTLpiechart}{outeroffset}] 
This is the absolute distance from the centre
of the pie chart to the point where the outer labels are placed.
You should use only one or other of \csopt{DTLpiechart}{outerratio} 
and \csopt{DTLpiechart}{outeroffset}, not both. If you also want to 
specify the radius, you must use \csopt{DTLpiechart}{radius}
before \csopt{DTLpiechart}{outeroffset}. If omitted, the outer 
offset is obtained from the radius multiplied by the 
\csopt{DTLpiechart}{outerratio} value.

\item[\csopt{DTLpiechart}{cutawayoffset}] 
This is the absolute distance from the centre of
the pie chart to the point of the cutaway segments.  You should use
only one or other of \csopt{DTLpiechart}{cutawayratio} and 
\csopt{DTLpiechart}{cutawayoffset}, not both. If
you also want to specify the radius, you must use 
\csopt{DTLpiechart}{ratio} before 
\csopt{DTLpiechart}{cutawayoffset}. If omitted, the cutaway offset 
is obtained from the ratio multiplied by the 
\csopt{DTLpiechart}{cutawayratio} value.

\item[\csopt{DTLpiechart}{cutaway}] 
This is a list of cutaway segments. This should be
a comma separated list of individual numbers, or number ranges
(separated by a dash). For example "cutaway={1,3}" will separate
the first and third segments from the rest of the pie chart, offset
by the value of the \csopt{DTLpiechart}{cutawayoffset} setting, 
whereas "cutaway={1-3}" will separate the
first three segments from the rest of the pie chart. If omitted,
the pie chart will be whole.

\item[\csopt{DTLpiechart}{innerlabel}] 
The value of this is positioned in the middle of each segment at a 
distance of \csopt{DTLpiechart}{inneroffset} from the centre
of the pie chart. The default is the same as the value of 
\csopt{DTLpiechart}{variable}.

\item[\csopt{DTLpiechart}{outerlabel}] 
The value of this is positioned at a distance of 
\csopt{DTLpiechart}{outeroffset} from the centre of the pie chart. 
The default is empty.

\item[\csopt{DTLpiechart}{rotateinner}] This is a boolean setting, 
so it can only take the values "true" and "false". If the value is 
omitted "true" is assumed. If true, the inner labels are rotated 
along the spokes of the pie chart, otherwise the inner labels are not 
rotated.  There are analogous package options 
\pkgopt{rotateinner} and \pkgopt{norotateinner}.

\item[\csopt{DTLpiechart}{rotateouter}] 
This is a boolean setting, so it can only take
the values "true" and "false". If the value is omitted "true" is
assumed. If true, the outer labels are rotated along the spokes of
the pie chart, otherwise the outer labels are not rotated.
There are analogous package options 
\pkgopt{rotateouter} and \pkgopt{norotateouter}.

\end{description}

\begin{example}{A Pie Chart}{ex:piechart}
This example loads data from a file called "fruit.csv" which contains
the following:
\begin{verbatim}
Name,Quantity
"Apples",30
"Pears",25
"Lemons,Limes",40.5
"Peaches",34.5
"Cherries",20
\end{verbatim}
First load the data:
\begin{verbatim}
\DTLloaddb{fruit}{fruit.csv}
\end{verbatim}
\DTLnewdb{fruit}\relax
\DTLnewrow{fruit}\relax
\DTLnewdbentry{fruit}{Name}{Apples}\relax
\DTLnewdbentry{fruit}{Quantity}{30}\relax
\DTLnewrow{fruit}\relax
\DTLnewdbentry{fruit}{Name}{Pears}\relax
\DTLnewdbentry{fruit}{Quantity}{25}\relax
\DTLnewrow{fruit}\relax
\DTLnewdbentry{fruit}{Name}{Lemons,Limes}\relax
\DTLnewdbentry{fruit}{Quantity}{40.5}\relax
\DTLnewrow{fruit}\relax
\DTLnewdbentry{fruit}{Name}{Peaches}\relax
\DTLnewdbentry{fruit}{Quantity}{34.5}\relax
\DTLnewrow{fruit}\relax
\DTLnewdbentry{fruit}{Name}{Cherries}\relax
\DTLnewdbentry{fruit}{Quantity}{20}\relax
Now create a pie chart in a figure:
\begin{verbatim}
\begin{figure}[htbp]
\centering
\DTLpiechart{variable=\quantity}{fruit}{\name=Name,\quantity=Quantity}
\caption{A pie chart}
\end{figure}
\end{verbatim}
This creates \autoref{fig:piechart}. The colours used are the
defaults. See \autoref{ex:piecolours} for an example that changes
the default colours.

\begin{figure}[htbp]
\centering
\DTLpiechart{variable=\quantity}{fruit}{\name=Name,\quantity=Quantity}
\caption{A pie chart}
\label{fig:piechart}
\end{figure}

There are no outer labels by default, but they can be set
using the \csopt{DTLpiechart}{outerlabel} setting. 
The following sets the outer label to the value of the "Name" key:
\begin{verbatim}
\begin{figure}[htbp]
\centering
\DTLpiechart{variable=\quantity,outerlabel=\name}{fruit}{%
\name=Name,\quantity=Quantity}
\caption{A pie chart (outer labels set)}
\end{figure}
\end{verbatim}
This creates \autoref{fig:piechartouter}.

\begin{figure}[htbp]
\centering
\DTLpiechart{variable=\quantity,outerlabel=\name}{fruit}
{\name=Name,\quantity=Quantity}
\caption{A pie chart (outer labels set)}
\label{fig:piechartouter}
\end{figure}

You may prefer the labels to be rotated. The following
switches on the rotation for the inner and outer labels:
\begin{verbatim}
\begin{figure}[htbp]
\centering
\DTLpiechart{variable=\quantity,outerlabel=\name,%
rotateinner,rotateouter}{fruit}{%
\name=Name,\quantity=Quantity}
\caption{A pie chart (rotation enabled)}
\end{figure}
\end{verbatim}
This creates \autoref{fig:piechartrot}.

\begin{figure}[htbp]
\centering
\DTLpiechart{variable=\quantity,outerlabel=\name
,rotateinner,rotateouter}{fruit}
{\name=Name,\quantity=Quantity}
\caption{A pie chart (rotation enabled)}
\label{fig:piechartrot}
\end{figure}
\end{example}

\begin{example}{Separating Segments from the Pie Chart}{ex:cutaway}
You may want to separate one or more segments from the pie chart,
perhaps to emphasize them. You can do this using the 
\csopt{DTLpiechart}{cutaway}
setting. The following separates the first and third segments
from the pie chart:
\begin{verbatim}
\begin{figure}[htbp]
\centering
\DTLpiechart{variable=\quantity,outerlabel=\name,%
cutaway={1,3}}{fruit}{%
\name=Name,\quantity=Quantity}
\caption{A pie chart with cutaway segments}
\end{figure}
\end{verbatim}
This produces \autoref{fig:piecutaway}.

\begin{figure}[htbp]
\centering
\DTLpiechart{variable=\quantity,outerlabel=\name
,cutaway={1,3}}{fruit}{\name=Name,\quantity=Quantity}
\caption{A pie chart with cutaway segments}
\label{fig:piecutaway}
\end{figure}

Alternatively I can specify a range of segments. The following
separates the first two segments:
\begin{verbatim}
\begin{figure}[htbp]
\centering
\DTLpiechart{variable=\quantity,outerlabel=\name,%
cutaway={1-2}}{fruit}{%
\name=Name,\quantity=Quantity}
\caption{A pie chart with cutaway segments (\texttt{cutaway=\{1-2\}})}
\end{figure}
\end{verbatim}
This produces \autoref{fig:piecutaway2}.

\begin{figure}[htbp]
\centering
\DTLpiechart{variable=\quantity,outerlabel=\name
,cutaway={1-2}}{fruit}{\name=Name,\quantity=Quantity}
\caption{A pie chart with cutaway segments (\texttt{cutaway=\{1-2\}})}
\label{fig:piecutaway2}
\end{figure}

Notice the difference between \autoref{fig:piecutaway2} and
\autoref{fig:piecutaway3} which was produced using:
\begin{verbatim}
\begin{figure}[htbp]
\centering
\DTLpiechart{variable=\quantity,outerlabel=\name,%
cutaway={1,2}}{fruit}{%
\name=Name,\quantity=Quantity}
\caption{A pie chart with cutaway segments (\texttt{cutaway=\{1,2\}})}
\end{figure}
\end{verbatim}

\begin{figure}[htbp]
\centering
\DTLpiechart{variable=\quantity,outerlabel=\name
,cutaway={1,2}}{fruit}{\name=Name,\quantity=Quantity}
\caption{A pie chart with cutaway segments (\texttt{cutaway=\{1,2\}})}
\label{fig:piecutaway3}
\end{figure}

\end{example}

\section{Pie Chart Variables}

\begin{definition}[\DescribeMacro{\DTLpievariable}]%
\cs{DTLpievariable}
\end{definition}
This command is set to the variable given by the
\csopt{DTLpiechart}{variable} setting in the \meta{settings list}
argument of \cs{DTLpiechart}. The \csopt{DTLpiechart}{innerlabel}
is set to \cs{DTLpievariable} by default.

\begin{definition}[\DescribeMacro{\DTLpiepercent}]%
\cs{DTLpiepercent}
\end{definition}
This command is set to the percentage value of \cs{DTLpievariable}.
The percentage value is rounded to \meta{n} digits, where \meta{n}
is the value of the \LaTeX\ counter 
\desctr{DTLpieroundvar}.

\begin{example}{Changing the Inner and Outer Labels}{ex:pielabels}
This example uses the database defined in \autoref{ex:piechart}.
The inner label is now set to the percentage value, rather than
the actual value, and the outer label is set to the name with
the actual value in parentheses.
\begin{verbatim}
\begin{figure}[htbp]
\centering
\DTLpiechart{variable=\quantity,%
innerlabel={\DTLpiepercent\%},%
outerlabel={\name\ (\DTLpievariable)}}{fruit},outerlabel={\name\ (\DTLpievariable)}}{fruit}{\name=Name,\quantity=Quantity}
\caption{A pie chart (changing the labels)}
\label{fig:piechartlabels}
\end{figure}
\end{example}

\section{Pie Chart Label Formatting}

\begin{definition}[\DescribeMacro{\DTLdisplayinnerlabel}]%
\cs{DTLdisplayinnerlabel}\marg{text}
\end{definition}
This governs how the inner label is formatted, where \meta{text}
is the text of the inner label. The default is to just do \meta{text}.

\begin{definition}[\DescribeMacro{\DTLdisplayouterlabel}]%
\cs{DTLdisplayouterlabel}\marg{text}
\end{definition}
This governs how the outer label is formatted, where \meta{text}
is the text of the outer label. The default is to just do \meta{text}.

\begin{example}{Changing the Inner and Outer Label 
Format}{ex:pielabelformat}
This example extends \autoref{ex:pielabels}.
The inner and outer labels are now both typeset in a sans-serif
font:
\begin{verbatim}
\begin{figure}[htbp]
\centering
\renewcommand*{\DTLdisplayinnerlabel}[1]{\textsf{#1}}
\renewcommand*{\DTLdisplayouterlabel}[1]{\textsf{#1}}
\DTLpiechart{variable=\quantity,%
innerlabel={\DTLpiepercent\%},%
outerlabel={\name\ (\DTLpievariable)}}{fruit},outerlabel={\name\ (\DTLpievariable)}}{fruit}{\name=Name,\quantity=Quantity}
\caption{A pie chart (changing the label format)}
\label{fig:piechartlabelformat}
\end{figure}
\end{example}

\section{Pie Chart Colours}

The \sty{datapie} package predefines colours for the first 
eight segments of the pie chart.  If you require more than 
eight segments or if you want to change the default colours, you 
will need to use
\begin{definition}[\DescribeMacro{\DTLsetpiesegmentcolor}]%
\cs{DTLsetpiesegmentcolor}\marg{n}\marg{color}
\end{definition}\noindent
The first argument \meta{n} is the segment index (starting from 1),
and the second argument \meta{color} is a colour specifier as used
in commands such as \cs{color}.

It is a good idea to set the colours so that each segment colour
is somehow relevant to whatever the segment represents. For
example, in the previous examples of pie charts depicting fruit, 
some of default colours were inappropriate. Whilst red is
appropriate for apples and green is appropriate for pears, blue
doesn't really correspond to lemons or limes.

\begin{definition}[\DescribeMacro{\DTLdopiesegmentcolor}]%
\cs{DTLdopiesegmentcolor}\marg{n}
\end{definition}
This sets the current text colour to that of the \meta{n}th
segment.

\begin{definition}[\DescribeMacro{\DTLdocurrentpiesegmentcolor}]%
\cs{DTLdocurrentpiesegmentcolor}
\end{definition}
This sets the current text colour to that of the current pie
segment. This command may only be used within a pie chart, or
within the body of \ics{DTLforeach}.

\begin{definition}[\DescribeMacro{\DTLpieoutlinecolor}]%
\cs{DTLpieoutlinecolor}
\end{definition}
This sets the outline colour for the pie chart. The default is
black.

\begin{definition}[\DescribeMacro{\DTLpieoutlinewidth}]%
\cs{DTLpieoutlinewidth}
\end{definition}
This is a length that governs the line width of the outline. The
default value is 0pt, but can be changed using \cs{setlength}.
The outline is only drawn if \cs{DTLpieoutlinewidth} is greater
than 0pt.

\begin{example}{Pie Segment Colours}{ex:piecolours}
This example extends \autoref{ex:pielabelformat}.
It sets the outline thickness to 2pt, and
the outer label is now set in the same colour as the fill colour 
of the segment to which it belongs. The third segment (lemons and
limes) is set to yellow and the fourth segment (peaches) is set
to pink.  In addition, a legend is created using \ics{DTLforeach}.
\begin{verbatim}
\begin{figure}[htbp]
\centering
\setlength{\DTLpieoutlinewidth}{2pt}
\DTLsetpiesegmentcolor{3}{yellow}
\DTLsetpiesegmentcolor{4}{pink}
\renewcommand*{\DTLdisplayinnerlabel}[1]{\textsf{#1}}
\renewcommand*{\DTLdisplayouterlabel}[1]{%
\DTLdocurrentpiesegmentcolor
\textsf{\shortstack{#1}}}
\DTLpiechart{variable=\quantity,%
innerlabel={\DTLpiepercent\%},%
outerlabel={\name\\(\DTLpievariable)}}{fruit}{%
\name=Name,\quantity=Quantity}
\begin{tabular}[b]{ll}
\DTLforeach{fruit}{\name=Name}{\DTLiffirstrow{}{\\}%
\DTLdocurrentpiesegmentcolor\rule{10pt}{10pt} &
\name
}
\end{tabular}
\caption{A pie chart (using segment colours and outline)}
\end{figure}
\end{verbatim}
This produces \autoref{fig:piesegcolour}. (The format of the
outer label has been changed to use \cs{shortstack} to 
prevent the outer labels from taking up so much horizontal
space. The \csopt{DTLpiechart}{outerlabel} setting has also been 
modified to use "\\" after the name to move the percentage value onto
the next row.)

\begin{figure}[htbp]
\centering
\setlength{\DTLpieoutlinewidth}{2pt}
\DTLsetpiesegmentcolor{3}{yellow}
\DTLsetpiesegmentcolor{4}{pink}
\renewcommand*{\DTLdisplayinnerlabel}[1]{\textsf{#1}}
\renewcommand*{\DTLdisplayouterlabel}[1]{\relax
\DTLdocurrentpiesegmentcolor
\textsf{\shortstack{#1}}}
\DTLpiechart{variable=\quantity,innerlabel={\DTLpiepercent
\%},outerlabel={\name\\(\DTLpievariable)}}{fruit}{\name=Name,\quantity=Quantity}
\begin{tabular}[b]{ll}
\DTLforeach{fruit}{\name=Name}{\DTLiffirstrow{}{\\}\relax
\DTLdocurrentpiesegmentcolor\rule{10pt}{10pt} &
\name
}
\end{tabular}
\caption{A pie chart (using segment colours and outline)}
\label{fig:piesegcolour}
\end{figure}
\end{example}

\section{Adding Extra Commands Before and After the Pie Chart}

The pie charts created using \ics{DTLpiechart} are placed inside
a \env{tikzpicture} environment (defined by the \sty{tikz} package).

\begin{definition}[\DescribeMacro{\DTLpieatbegintikz}]%
\cs{DTLpieatbegintikz}
\end{definition}
The macro \cs{DTLpieatbegintikz} is called at the start of the
\env{tikzpicture} environment, allowing you to change the
\env{tikzpicture} settings. By default \cs{DTLpieatbegintikz}
does nothing, but you can redefine it to, say, scale the pie
chart (but be careful not to distort the chart).

\begin{definition}[\DescribeMacro{\DTLpieatendtikz}]%
\cs{DTLpieatendtikz}
\end{definition}
The macro \cs{DTLpieatendtikz} is called at the end of the
\env{tikzpicture} environment, allowing you add additional
graphics to the pie chart. This does nothing by default.

\begin{example}{Adding Information to the Pie Chart}{ex:piescale}
This example modifies \autoref{ex:piechart}. It redefines
\cs{DTLpieatendtikz} to add an annotated arrow.
\begin{verbatim}
\begin{figure}[htbp]
\centering
\renewcommand*{\DTLpieatendtikz}{%
\draw[<-] (45:1.5cm) -- (40:2.5cm)node[right]{Apples};}
\DTLpiechart{variable=\quantity}{fruit}{%
\name=Name,\quantity=Quantity}
\caption{An annotated pie chart}
\end{figure}
\end{verbatim}
This produces \autoref{fig:pieannote}. (Note that the centre
of the pie chart is the origin of the TikZ picture.)

\begin{figure}[htbp]
\centering
\renewcommand*{\DTLpieatendtikz}{%
\draw[<-] (45:1.5cm) -- (40:2.5cm)node[right]{Apples};}
\DTLpiechart{variable=\quantity}{fruit}{\name=Name,\quantity=Quantity}
\caption{An annotated pie chart}
\label{fig:pieannote}
\end{figure}
\end{example}

\chapter{Scatter and Line Plots (\texorpdfstring{\sty{dataplot}}{dataplot}
package)}
\label{sec:dataplot}

The \sty{dataplot} package provides commands for creating
scatter or line plots from databases. It uses the pgf/TikZ plot 
handler library to create the plots. See the \sty{pgf} manual for
more detail on pgf streams and plot handles. The \sty{dataplot}
package is not loaded by \sty{datatool} so if you want to use
it you need to load it explicitly using "\usepackage{dataplot}".


\begin{definition}[\DescribeMacro{\DTLplot}]%
\cs{DTLplot}\oarg{condition}\marg{db list}\marg{settings}
\end{definition}
This command creates a plot (inside a \env{tikzpicture} environment)
of all the data given in the databases listed in \meta{db list},
which should be a comma separated list of database names.
The optional argument \meta{condition} is the same as that for
\ics{DTLforeach}. The \meta{settings} argument is a comma separated
list of \meta{setting}"="\meta{value} pairs. There are two settings
that must be specified \csopt{DTLplot}{x} and \csopt{DTLplot}{y}.
The other settings are optional. Note that any value that contains
a comma, must be enclosed in braces. For example
"colors={red,cyan,blue}". Note where any setting requires
a number, or list of numbers (such as \csopt{DTLplot}{bounds})
the number must be supplied in standard decimal notation (i.e.\
no currency, no number groups, and a full stop as the decimal
point). Available settings are as follows:
\begin{description}
\item[\csopt{DTLplot}{x}] The database key
that specifies the $x$ co-ordinates. This setting is required.

\item[\csopt{DTLplot}{y}] The database key that specifies
the $y$ co-ordinates. This setting is required.

\item[\csopt{DTLplot}{markcolors}] A comma separated list of colour
names for the markers. An empty value will use the current colour.

\item[\csopt{DTLplot}{linecolors}] A comma separated list of colour
names for the plot lines. An empty value will use the current colour.

\item[\csopt{DTLplot}{colors}] A comma separated list of colour
names for the lines and markers.

\item[\csopt{DTLplot}{marks}] A comma separated list of 
code to generate plot marks. (This should typically be a list
of \cs{pgfuseplotmark} commands, see the \sty{pgf} manual for
further details.) You may use \cs{relax} as an element
of the list to suppress markers for the corresponding plot.
For example: "marks={\pgfuseplotmark{o},\relax}" will use an
open circle marker for the first database, and no markers for the
second database listed in \meta{db list}.

\item[\csopt{DTLplot}{lines}] A comma separated list of 
line style settings. (This should typically be a list of 
\cs{pgfsetdash} commands, see the \sty{pgf} manual for
further details on how to set the line style.) An empty value 
will use the current line style.
You may use \cs{relax} as an element
of the list to suppress line for the corresponding plot.
For example: "lines={\relax,\pgfsetdash{}{0pt}}"
will have no lines for the first database, and a solid line
for the second database listed in \meta{db list}.

\item[\csopt{DTLplot}{width}] The width of the plot. This must
be a length. The plot width does not include outer tick marks or
labels.

\item[\csopt{DTLplot}{height}] The height of the plot. This must
be a length. The plot height does not include outer tick marks
or labels.

\item[\csopt{DTLplot}{style}] This setting governs whether
to use lines or markers in the plot, and may take one of
the following values: "both" (lines and markers), 
"lines" (only lines) or "markers" (only markers). The default is
"markers".

\item[\csopt{DTLplot}{axes}] This setting governs whether
to display the axes, and may take one of
the following values: "both", "x", "y" or "none". If no value
is specified, "both" is assumed.

\item[\csopt{DTLplot}{box}] This setting governs whether
or not to surround the plot in a box. It is a boolean setting,
taking only the values "true" and "false". If no value is
specified, "true" is assumed.

\item[\csopt{DTLplot}{xtics}] This setting governs whether
or not to display the $x$ tick marks. It is a boolean setting,
taking only the values "true" and "false". If no value is
specified "true" is assumed. If the \csopt{DTLplot}{axes}
setting is set to "both" or "x", this value will automatically
be set to "true", otherwise it will be set to "false".

\item[\csopt{DTLplot}{ytics}] This setting governs whether
or not to display the $y$ ticks. It is a boolean setting,
taking only the values "true" and "false". If no value is
specified "true" is assumed. If the \csopt{DTLplot}{axes}
setting is set to "both" or "y", this value will automatically
be set to "true", otherwise it will be set to "false".


\item[\csopt{DTLplot}{xminortics}] This setting governs whether
or not to display the $x$ minor tick marks. It is a boolean 
setting, taking only the values "true" and "false". If no value is
specified "true" is assumed. This setting also sets the
$x$ major tick marks on if the value is "true".

\item[\csopt{DTLplot}{yminortics}] This setting governs whether
or not to display the $y$ minor tick marks. It is a boolean 
setting, taking only the values "true" and "false". If no value is
specified "true" is assumed. This setting also sets the
$y$ major tick marks on if the value is "true".

\item[\csopt{DTLplot}{xticdir}] This sets the $x$ tick direction,
and may only take the values "in" or "out".

\item[\csopt{DTLplot}{yticdir}] This sets the $y$ tick direction,
and may only take the values "in" or "out".

\item[\csopt{DTLplot}{ticdir}] This sets the $x$ and $y$ tick 
direction, and may only take the values "in" or "out".

\item[\csopt{DTLplot}{bounds}] The value must be in the form
\meta{min x}","\meta{min y}","\meta{max x}","\meta{max y}. This
sets the graph bounds to the given values. If omitted the
bounds are computed from the maximum and minimum values of the
data. For example
\begin{verbatim}
\DTLplot{data1,data2}{x=Height,y=Weight,bounds={0,0,10,20}}
\end{verbatim}
Note that the \csopt{DTLplot}{bounds} setting overrides
the \csopt{DTLplot}{minx}, \csopt{DTLplot}{maxx},
\csopt{DTLplot}{miny} and \csopt{DTLplot}{maxy} settings.

\item[\csopt{DTLplot}{minx}] The value is the minimum value
of the $x$ axis.

\item[\csopt{DTLplot}{miny}] The value is the minimum value
of the $y$ axis.

\item[\csopt{DTLplot}{maxx}] The value is the maximum value
of the $x$ axis.

\item[\csopt{DTLplot}{maxy}] The value is the maximum value
of the $y$ axis.

\item[\csopt{DTLplot}{xticpoints}] The value must be a comma
separated list of decimal numbers indicating where to put the
$x$ tick marks. If omitted, the $x$ tick marks are placed at 
equal intervals along the $x$ axis such that each interval is
not less than the length given by \ics{DTLmintickgap}.
This setting overrides \csopt{DTLplot}{xticgap}.

\item[\csopt{DTLplot}{xticgap}] This value specifies the
gap between the $x$ tick marks.

\item[\csopt{DTLplot}{yticpoints}] The value must be a comma
separated list of decimal numbers indicating where to put the
$y$ tick marks. If omitted, the $y$ tick marks are placed at 
equal intervals along the $y$ axis such that each interval is
not less than the length given by \ics{DTLmintickgap}.
This setting overrides \csopt{DTLplot}{yticgap}.

\item[\csopt{DTLplot}{yticgap}] This value specifies the
gap between the $y$ tick marks.

\item[\csopt{DTLplot}{grid}] This is a boolean value that
specifies whether or not to display the grid. If no value
is given, "true" is assumed. The minor grid lines are only
displayed if the minor tick marks are set.

\item[\csopt{DTLplot}{xticlabels}] The value must be a comma
separated list of labels for each $x$ tick mark. If omitted,
the labels are the value of the $x$ tick position, rounded 
\meta{n} digits after the decimal point, where \meta{n} is 
given by the value of the counter \ctr{DTLplotroundXvar}.

\item[\csopt{DTLplot}{yticlabels}] The value must be a comma
separated list of labels for each $y$ tick mark. If omitted,
the labels are the value of the $y$ tick position, rounded 
\meta{n} digits after the decimal point, where \meta{n} is 
given by the value of the counter \ctr{DTLplotroundYvar}.

\item[\csopt{DTLplot}{xlabel}] The value is the label for
the $x$ axis. If omitted, the axis has no label.

\item[\csopt{DTLplot}{ylabel}] The value is the label for
the $y$ axis. If omitted, the axis has no label.

\item[\csopt{DTLplot}{legend}] This setting governs whether
or not to display the legend, and where it should be displayed.
It may take one of the following values "none" (don't display
the legend), "north", "northeast", "east", "southeast", "south",
"southwest", "west" or "northwest". If the value is omitted,
"northeast" is assumed.

\item[\csopt{DTLplot}{legendlabels}] The value must be a comma
separated list of labels for the legend. If omitted, the database
names are used.

\end{description}

\begin{example}{A Basic Graph}{ex:basicplot}
Suppose you have a file called "groupa.csv" that contains the 
following:
\DTLnewdb{groupa}\relax
\DTLnewrow{groupa}\relax
\DTLnewdbentry{groupa}{Height}{1.54}\relax
\DTLnewdbentry{groupa}{Weight}{48.0}\relax
\DTLnewrow{groupa}\relax
\DTLnewdbentry{groupa}{Height}{1.55}\relax
\DTLnewdbentry{groupa}{Weight}{45.4}\relax
\DTLnewrow{groupa}\relax
\DTLnewdbentry{groupa}{Height}{1.56}\relax
\DTLnewdbentry{groupa}{Weight}{58.0}\relax
\DTLnewrow{groupa}\relax
\DTLnewdbentry{groupa}{Height}{1.56}\relax
\DTLnewdbentry{groupa}{Weight}{50.2}\relax
\DTLnewrow{groupa}\relax
\DTLnewdbentry{groupa}{Height}{1.57}\relax
\DTLnewdbentry{groupa}{Weight}{46.0}\relax
\DTLnewrow{groupa}\relax
\DTLnewdbentry{groupa}{Height}{1.58}\relax
\DTLnewdbentry{groupa}{Weight}{48.3}\relax
\DTLnewrow{groupa}\relax
\DTLnewdbentry{groupa}{Height}{1.59}\relax
\DTLnewdbentry{groupa}{Weight}{56.5}\relax
\DTLnewrow{groupa}\relax
\DTLnewdbentry{groupa}{Height}{1.59}\relax
\DTLnewdbentry{groupa}{Weight}{58.1}\relax
\DTLnewrow{groupa}\relax
\DTLnewdbentry{groupa}{Height}{1.60}\relax
\DTLnewdbentry{groupa}{Weight}{60.9}\relax
\DTLnewrow{groupa}\relax
\DTLnewdbentry{groupa}{Height}{1.62}\relax
\DTLnewdbentry{groupa}{Weight}{56.3}\relax
\begin{ttfamily}\obeylines\setlength{\parindent}{0pt}
Height,Weight
\DTLforeach{groupa}{\x=Height,\y=Weight}{\x,\y
}\end{ttfamily}\par\noindent
First load this into a database called "groupa":
\begin{verbatim}
\DTLloaddb{groupa}{groupa.csv}
\end{verbatim}
The data can now be converted into a scatter plot as follows:
\begin{verbatim}
\begin{figure}[htbp]
\centering
\DTLplot{groupa}{x=Height,y=Weight}
\caption{A scatter plot}
\end{figure}
\end{verbatim}
This produces \autoref{fig:basicplot}.

\begin{figure}[htbp]
\centering
\DTLplot{groupa}{x=Height,y=Weight}
\caption{A scatter plot}
\label{fig:basicplot}
\end{figure}

Alternatively, you can use the \csopt{DTLplot}{style} setting
to change it into a line plot:
\begin{verbatim}
\begin{figure}[htbp]
\centering
\DTLplot{groupa}{x=Height,y=Weight,style=lines}
\caption{A line plot}
\end{figure}
\end{verbatim}
This produces \autoref{fig:basiclineplot}.

\begin{figure}[htbp]
\centering
\DTLplot{groupa}{x=Height,y=Weight,style=lines}
\caption{A line plot}
\label{fig:basiclineplot}
\end{figure}
\end{example}

\begin{example}{Plotting Multiple Data Sets}{ex:2db}
In this example, I shall use the database called "groupa" defined
in \autoref{ex:basicplot}, and another database called "groupb"
which is loaded from the file "groupb.csv" which contains the
following:
\DTLnewdb{groupb}
\DTLnewrow{groupb}\relax
\DTLnewdbentry{groupb}{Height}{1.54}\relax
\DTLnewdbentry{groupb}{Weight}{48.4}\relax
\DTLnewrow{groupb}\relax
\DTLnewdbentry{groupb}{Height}{1.54}\relax
\DTLnewdbentry{groupb}{Weight}{42.0}\relax
\DTLnewrow{groupb}\relax
\DTLnewdbentry{groupb}{Height}{1.55}\relax
\DTLnewdbentry{groupb}{Weight}{64.0}\relax
\DTLnewrow{groupb}\relax
\DTLnewdbentry{groupb}{Height}{1.56}\relax
\DTLnewdbentry{groupb}{Weight}{58.2}\relax
\DTLnewrow{groupb}\relax
\DTLnewdbentry{groupb}{Height}{1.56}\relax
\DTLnewdbentry{groupb}{Weight}{49.0}\relax
\DTLnewrow{groupb}\relax
\DTLnewdbentry{groupb}{Height}{1.57}\relax
\DTLnewdbentry{groupb}{Weight}{40.3}\relax
\DTLnewrow{groupb}\relax
\DTLnewdbentry{groupb}{Height}{1.58}\relax
\DTLnewdbentry{groupb}{Weight}{51.5}\relax
\DTLnewrow{groupb}\relax
\DTLnewdbentry{groupb}{Height}{1.58}\relax
\DTLnewdbentry{groupb}{Weight}{63.1}\relax
\DTLnewrow{groupb}\relax
\DTLnewdbentry{groupb}{Height}{1.59}\relax
\DTLnewdbentry{groupb}{Weight}{74.9}\relax
\DTLnewrow{groupb}\relax
\DTLnewdbentry{groupb}{Height}{1.59}\relax
\DTLnewdbentry{groupb}{Weight}{59.3}\relax
\begin{ttfamily}\obeylines\setlength{\parindent}{0pt}
Height,Weight
\DTLforeach{groupb}{\x=Height,\y=Weight}{\x,\y
}\end{ttfamily}\par\noindent
First load this into a database called "groupb":
\begin{verbatim}
\DTLloaddb{groupb}{groupb.csv}
\end{verbatim}
I can now plot both groups in the same graph, but I want a smaller
graph than \autoref{fig:basicplot} and \autoref{fig:basiclineplot}, 
so I am going to set the plot width and height to 3in:
\begin{verbatim}
\begin{figure}[htbp]
\centering
\DTLplot{groupa,groupb}{x=Height,y=Weight,width=3in,height=3in}
\caption{A scatter plot}
\end{figure}
\end{verbatim}
This produces \autoref{fig:2db}.

\begin{figure}[htbp]
\centering
\DTLplot{groupa,groupb}{x=Height,y=Weight,width=3in,height=3in}
\caption[A scatter plot (multiple datasets)]{A scatter plot}
\label{fig:2db}
\end{figure}

Now let's add a legend using the \csopt{DTLplot}{legend} setting, 
with the legend labels "Group A" and "Group B",
and set the $x$ tick intervals using \csopt{DTLplot}{xticpoints}
setting. I am also going to set the $x$ axis label to
"Height (m)" and the $y$ axis label to "Weight (kg)", and place
a box around the plot.
\begin{verbatim}
\begin{figure}[htbp]
\centering
\DTLplot{groupa,groupb}{x=Height,y=Weight,
width=3in,height=3in,legend,legendlabels={Group A,Group B},
xlabel={Height (m)},ylabel={Weight (kg)},box,
xticpoints={1.54,1.55,1.56,1.57,1.58,1.59,1.60,1.61,1.62}}
\caption{A scatter plot}
\end{figure}
\end{verbatim}
This produces \autoref{fig:legend}.

\begin{figure}[htbp]
\centering
\DTLplot{groupa,groupb}{x=Height,y=Weight,legend,
width=3in,height=3in,legendlabels={Group A,Group B},
xlabel={Height (m)},ylabel={Weight (kg)},box,
xticpoints={1.54,1.55,1.56,1.57,1.58,1.59,1.60,1.61,1.62}}
\caption[A scatter plot (with a legend)]{A scatter plot}
\label{fig:legend}
\end{figure}
\end{example}

\section{Adding Information to the Plot}

The \sty{datatool} package provides two hooks used at the beginning
and end of the \env{tikzpicture} environment:
\begin{definition}[\DescribeMacro{\DTLplotatbegintikz}]%
\cs{DTLplotatbegintikz}
\end{definition}\noindent
and
\begin{definition}[\DescribeMacro{\DTLplotatendtikz}]%
\cs{DTLplotatendtikz}
\end{definition}
They are both defined to do nothing by default, but can be redefined
to add commands to the image. The unit vectors are set prior to
using these hooks, so you can use the same co-ordinates as those
in the data sets.  However, to reduce the problem of exceeding \TeX's 
maximum dimension, \cs{DTLplot} scales the plot which may distort
plot marks. To get around this use
\begin{definition}[\DescribeMacro{\dtlplothandlermark}]
\cs{dtlplothandlermark}\marg{pgf code}
\end{definition}
instead of \ics{pgfplothandlermark}\marg{pgf code}. (See 
\autoref{ex:multikey}.) Note that \cs{dtlplothandlermark} is only
intended for use within the definition of \cs{DTLplotatbegintikz} or
\cs{DTLplotatendtikz}. If used elsewhere it will produce a warning
and act as though you'd just used \cs{pgfplothandlermark}.

\begin{definition}[\DescribeMacro{\DTLaddtoplotlegend}]%
\cs{DTLaddtoplotlegend}\marg{marker}\marg{line style}\marg{text}
\end{definition}
This adds a new row to the plot legend where \meta{marker} is
code to produce the marker, \meta{line style} is code to set
the line style and \meta{text} is a textual label. You can
use \cs{relax} to suppress the marker or line. For example:
\begin{verbatim}
\DTLaddtoplotlegend{\pgfuseplotmark{x}}{\relax}{Some Data}
\end{verbatim}
Note that the legend is plotted before \cs{DTLplotatendtikz},
so if you want to add information to the legend you will need
to do the in \cs{DTLplotatstarttikz}.

\begin{example}{Adding Information to a Plot}{ex:addtoplot}
Returning to the plots created in \autoref{ex:2db}, suppose
I now want to annotate the plot, say I want to draw your notice
to a particular point, say the point (1.58,48.3), then I can
redefine \cs{DTLplotatendtikz} to draw an annotated arrow to
that point:
\begin{verbatim}
\renewcommand*{\DTLplotatendtikz}{%
\draw[<-,line width=1pt] (1.58,48.3) -- (1.6,43)
 node[below]{interesting point};
}
\end{verbatim}
So \autoref{fig:legend} now looks like \autoref{fig:annote}.
(Obviously, \cs{DTLplotatendtikz} needs to be redefined before
using \cs{DTLplot}.)

\begin{figure}[htbp]
\renewcommand*{\DTLplotatendtikz}{%
\draw[<-,line width=1pt] (1.58,48.3) -- (1.6,43)
 node[below]{interesting point};
}
\centering
\DTLplot{groupa,groupb}{x=Height,y=Weight,legend,
width=3in,height=3in,legendlabels={Group A,Group B},box,
xlabel={Height (m)},ylabel={Weight (kg)},
xticpoints={1.54,1.55,1.56,1.57,1.58,1.59,1.60,1.61,1.62}}
\caption[A scatter plot (using the end plot hook to annotate the
plot)]{A scatter plot}

\label{fig:annote}
\end{figure}
\end{example}

\section{Global Plot Settings}

\subsection{Lengths}
This section describes the lengths that govern the appearance of
the plot created using \ics{DTLplot}. These lengths can be
changed using \cs{setlength}.

\begin{definition}[\DescribeMacro{\DTLplotwidth}]%
\cs{DTLplotwidth}
\end{definition}
This length governs the length of the $x$ axis. Note that the plot
width does not include any outer tick marks or labels. The default
value is 4in.

\begin{definition}[\DescribeMacro{\DTLplotheight}]%
\cs{DTLplotheight}
\end{definition}
This length governs the length of the $y$ axis. Note that the plot
height does not include any outer tick marks or labels. The default
value is 4in

\begin{definition}[\DescribeMacro{\DTLticklength}]%
\cs{DTLticklength}
\end{definition}
This governs the length of the tick marks. The default value is
5pt.

\begin{definition}[\DescribeMacro{\DTLminorticklength}]%
\cs{DTLminorticklength}
\end{definition}
This governs the length of the minor tick marks. The default value is
2pt.

\begin{definition}[\DescribeMacro{\DTLticklabeloffset}]%
\cs{DTLticklabeloffset}
\end{definition}
This governs the distance from the axis to the tick labels. The 
default value is 8pt.

\begin{definition}[\DescribeMacro{\DTLmintickgap}]%
\cs{DTLmintickgap}
\end{definition}
This is the minimum distance allowed between tick marks. If the
plot width or height is less than this distance there will only
be tick marks at either end of the axis. The default value is
20pt.

\begin{definition}[\DescribeMacro{\DTLlegendxoffset}]%
\cs{DTLlegendxoffset}
\end{definition}
This is the horizontal distance from the border of the plot to the 
outer border of the legend. The default value is 10pt.

\begin{definition}[\DescribeMacro{\DTLlegendyoffset}]%
\cs{DTLlegendyoffset}
\end{definition}
This is the vertical distance from the border of the plot to the 
outer border of the legend. The default value is 10pt.

\subsection{Counters}
These counters govern the appearance of plots 
created using \ics{DTLplot}. The value of the counters can be
changed using \cs{setcounter}.

\begin{definition}[\DescribeCounter{DTLplotroundXvar}]
\ctrfmt{DTLplotroundXvar}
\end{definition}
Unless you specify your own tick labels, the $x$ tick labels will 
be given by the tick points rounded to \meta{n} digits after the
decimal point, where \meta{n} is the value of the counter
\ctrfmt{DTLplotroundXvar}.

\begin{definition}[\DescribeCounter{DTLplotroundYvar}]
\ctrfmt{DTLplotroundYvar}
\end{definition}
Unless you specify your own tick labels, the $y$ tick labels will 
be given by the tick points rounded to \meta{n} digits after the
decimal point, where \meta{n} is the value of the counter
\ctrfmt{DTLplotroundYvar}.

\subsection{Macros}
These macros govern the appearance of plots created using
\ics{DTLplot}. They can be changed using \cs{renewcommand}.

\begin{definition}[\DescribeMacro{\DTLplotmarks}]%
\cs{DTLplotmarks}
\end{definition}
This must be a comma separated list of \sty{pgf} code to create the 
plot marks.  \ics{DTLplot} cycles through this list for each
database listed. The \sty{pgf} package provides convenient commands
for generating plots using \cs{pgfuseplotmark}. See the \sty{pgf}
manual for more details.

\begin{definition}[\DescribeMacro{\DTLplotmarkcolors}]%
\cs{DTLplotmarkcolors}
\end{definition}
This must be a comma separated list of defined colours to apply to the
plot marks. \ics{DTLplot} cycles through this list for each database
listed. If this macro is set to empty, the current colour will
be used instead.

\begin{definition}[\DescribeMacro{\DTLplotlines}]%
\cs{DTLplotlines}
\end{definition}
This must be a comma separated list of \sty{pgf} code to set the
style of the plot lines.  \ics{DTLplot} cycles through this list for 
each database listed. Dash patterns can be set using \cs{pgfsetdash},
see the \sty{pgf} manual for more details. If \cs{DTLplotlines} is
set to empty the current line style will be used instead.

\begin{definition}[\DescribeMacro{\DTLplotlinecolors}]%
\cs{DTLplotlinecolors}
\end{definition}
This must be a comma separated list of defined colours to apply to the
plot lines. \cs{DTLplot} cycles through this list for each database
listed. If this macro is set to empty, the current colour will
be used instead. The default is the same as \cs{DTLplotmarkcolors}.

\begin{definition}[\DescribeMacro{\DTLXAxisStyle}]%
\cs{DTLXAxisStyle}
\end{definition}
This governs the style of the $x$ axis. It is passed as the optional
argument to the TikZ \cs{draw} command. By default it is just "-"
which is a solid line style with no start or end arrows. The $x$
axis line starts from the bottom left corner of the plot and extends
to the bottom right corner of the plot. So if you want the $x$ axis
to have an arrow head at the right end, you can do:
\begin{verbatim}
\renewcommand*{\DTLXAxisStyle}{->}
\end{verbatim}

\begin{definition}[\DescribeMacro{\DTLYAxisStyle}]%
\cs{DTLYAxisStyle}
\end{definition}
This governs the style of the $y$ axis. It is analogous to
\cs{DTLXAxisStyle} described above.

\begin{definition}[\DescribeMacro{\DTLmajorgridstyle}]%
\cs{DTLmajorgridstyle}
\end{definition}
This specifies the format of the major grid lines.
It may be set to any TikZ setting that you can pass to the
optional argument of \cs{draw}. The default value is
"color=gray,-" which indicates a grey solid line.

\begin{definition}[\DescribeMacro{\DTLminorgridstyle}]%
\cs{DTLminorgridstyle}
\end{definition}
This specifies the format of the minor grid lines.
It may be set to any TikZ setting that you can pass to the
optional argument of \cs{draw}. The default value is
"color=gray,loosely dotted" which indicates a grey dotted line.

\begin{definition}[\DescribeMacro{\DTLformatlegend}]%
\cs{DTLformatlegend}\marg{legend}
\end{definition}
This formats the entire legend, which is passed as the argument.
The default is to set the legend with
a white background, a black frame.

\section{Adding to a Plot Stream}

\begin{definition}[\DescribeMacro{\DTLplotstream}]%
\cs{DTLplotstream}\oarg{condition}\marg{db name}\marg{x key}\marg{y key}
\end{definition}
This adds points to a stream from the database called \meta{db name}
where the $x$ co-ordinates are given by the key \meta{x key}
and the $y$ co-ordinates are given by the key \meta{y key}.
(\ics{DTLconverttodecimal} is used to convert locale dependent
values to a standard decimal that is recognised by the
\sty{pgf} package.)
The optional argument \meta{condition} is the same as that
for \ics{DTLforeach}.

\begin{example}{Adding to a Plot Stream}{ex:plotstream}
Suppose you have a CSV file called "data.csv" containing the 
following:
\begin{verbatim}
x,y
0,0
1,1
2,0.5
1.5,0.3
\end{verbatim}
\DTLnewdb{data}\relax
\DTLnewrow{data}\relax
\DTLnewdbentry{data}{x}{0}\relax
\DTLnewdbentry{data}{y}{0}\relax
\DTLnewrow{data}\relax
\DTLnewdbentry{data}{x}{1}\relax
\DTLnewdbentry{data}{y}{1}\relax
\DTLnewrow{data}\relax
\DTLnewdbentry{data}{x}{2}\relax
\DTLnewdbentry{data}{y}{0.5}\relax
\DTLnewrow{data}\relax
\DTLnewdbentry{data}{x}{1.5}\relax
\DTLnewdbentry{data}{y}{0.3}\relax
First load the file into a database called "data":
\begin{verbatim}
\DTLloaddb{data}{data.csv}
\end{verbatim}
Now create a figure containing this data:
\begin{verbatim}
\begin{figure}[tbhp]
\centering
\begin{tikzpicture}
\pgfplothandlermark{\pgfuseplotmark{o}}
\pgfplotstreamstart
\DTLplotstream{data}{x}{y}%
\pgfplotstreamend
\pgfusepath{stroke}
\end{tikzpicture}
\caption{Adding to a plot stream}
\end{figure}
\end{verbatim}
This produces \autoref{fig:plotstream}.

\begin{figure}[tbhp]
\centering
\begin{tikzpicture}
\pgfplothandlermark{\pgfuseplotmark{o}}
\pgfplotstreamstart
\DTLplotstream{data}{x}{y}%
\pgfplotstreamend
\pgfusepath{stroke}
\end{tikzpicture}
\caption{Adding to a plot stream}
\label{fig:plotstream}
\end{figure}
\end{example}

\begin{example}{Plotting Multiple Keys in the Same 
Database}{ex:multikey}
Suppose I have conducted two time to growth experiments. For each
experiment, I have recorded the log count at set times, and I have
recorded this information in the same data file called, say,
"growth.csv" which contains the following:
\begin{verbatim}
Time,Experiment 1,Experiment 2
0,3.73,3.6
23,3.67,3.7
60,4.9,3.8
\end{verbatim}
\DTLnewdb{growth}\DTLnewrow{growth}\relax
\DTLnewdbentry{growth}{Time}{0}\relax
\DTLnewdbentry{growth}{Experiment 1}{3.73}\relax
\DTLnewdbentry{growth}{Experiment 2}{3.6}\relax
\DTLnewrow{growth}\relax
\DTLnewdbentry{growth}{Time}{23}\relax
\DTLnewdbentry{growth}{Experiment 1}{3.67}\relax
\DTLnewdbentry{growth}{Experiment 2}{3.7}\relax
\DTLnewrow{growth}\relax
\DTLnewdbentry{growth}{Time}{60}\relax
\DTLnewdbentry{growth}{Experiment 1}{4.9}\relax
\DTLnewdbentry{growth}{Experiment 2}{3.8}\relax
I can load the data into a database using:
\begin{verbatim}
\DTLloaddb{growth}{growth.csv}
\end{verbatim}
However, I'd like to plot both results on the same graph. Since they
are contained in the same database, I can't use the method I used
in \autoref{ex:2db}. Instead I can use a combination of 
\ics{DTLplot} and \ics{DTLplotstream}:
\begin{verbatim}
\begin{figure}[tbhp]
\centering
 % compute bounds
\DTLminforkeys{growth}{Time}{\minX}
\DTLminforkeys{growth}{Experiment 1,Experiment 2}{\minY}
\DTLmaxforkeys{growth}{Time}{\maxX}
\DTLmaxforkeys{growth}{Experiment 1,Experiment 2}{\maxY}
 % round x tick labels to 1 d.p.
\setcounter{DTLplotroundXvar}{1}
 % redefine \DTLplotatbegintikz to plot the data for Experiment 1
\renewcommand*{\DTLplotatbegintikz}{%
  % set plot mark
  \dtlplothandlermark{\color{green}\pgfuseplotmark{x}}
  % start plot stream
  \pgfplotstreamstart
  % add data from Experiment 1 to plot stream
  \DTLplotstream{growth}{Time}{Experiment 1}%
  % end plot stream
  \pgfplotstreamend
  % stroke path
  \pgfusepath{stroke}
  % add information to legend (no line is require so use \relax)
  \DTLaddtoplotlegend{\color{green}%
  \pgfuseplotmark{x}}{\relax}{Experiment 1}
}
 % now plot the data for Experiment 2
\DTLplot{growth}{x=Time,y=Experiment 2,legend,
width=3in,height=3in,bounds={\minX,\minY,\maxX,\maxY},
xlabel={Time},ylabel={Log Count},
legendlabels={Experiment 2}}
\caption{Time to growth data}
\end{figure}
\end{verbatim}
This produces \autoref{fig:multikey}. 
Notes:
\begin{itemize}
\item I redefined \ics{DTLplotatbegintikz} in order to add
the new plot to the legend, since \ics{DTLplotatendtikz} is 
used after the legend is plotted. The $x$ and $y$ unit vectors
are set before \ics{DTLplotatbegintikz} so I don't need to 
worry about the co-ordinates, however I've had to use
\ics{dtlplothandlermark} instead of \ics{pgfplothandlermark}
to prevent the plot marks from being distorted.
\item I have used \ics{DTLminforkeys} and \ics{DTLmaxforkeys} to
determine the bounds since \ics{DTLplot} won't take the data
for Experiment~1 into account when computing the bounds.
\end{itemize}

\begin{figure}[htbp]
\DTLminforkeys{growth}{Time}{\minX}
\DTLminforkeys{growth}{Experiment 1,Experiment 2}{\minY}
\DTLmaxforkeys{growth}{Time}{\maxX}
\DTLmaxforkeys{growth}{Experiment 1,Experiment 2}{\maxY}
\setcounter{DTLplotroundXvar}{1}
\renewcommand*{\DTLplotatbegintikz}{%
\dtlplothandlermark{\color{green}\pgfuseplotmark{x}}
\pgfplotstreamstart
\DTLplotstream{growth}{Time}{Experiment 1}%
\pgfplotstreamend
\pgfusepath{stroke}
\DTLaddtoplotlegend{\color{green}\pgfuseplotmark{x}}{\relax}{Experiment 1}
}
\centering
\DTLplot{growth}{x=Time,y=Experiment 2,legend,
width=3in,height=3in,bounds={\minX,\minY,\maxX,\maxY},
xlabel={Time},ylabel={Log Count},
legendlabels={Experiment 2}}
\caption[Time to growth data (plotting from the same database using
different keys)]{Time to growth data}
\label{fig:multikey}
\end{figure}
\end{example}

\chapter{Bar Charts (\texorpdfstring{\sty{databar}}{databar} 
package)}
\label{sec:databar}

The \sty{databar} package provides commands for creating bar charts.
It is not loaded by the \sty{datatool} package, so if you want to
use it you will need to load it explicitly using 
"\usepackage{databar}". You must also have the \sty{pgf} package
installed.

Bar charts can either be vertical or horizontal, the default is
vertical. In this section the $x$ axis refers to the horizontal
axis when plotting a vertical bar chart and to the vertical axis
when plotting a horizontal bar chart. The $x$ axis units are in 
increments of one bar. The $y$ axis refers to the vertical axis
when plotting a vertical bar chart and to the horizontal axis when
plotting a horizontal bar chart. The $y$ axis uses the same 
co-ordinates as the data. The bars may have an upper and lower
label. In a vertical bar chart, the lower label is placed below
the $x$ axis and the upper label is placed above the top of the bar.
In a horizontal bar chart, the lower label is placed to the left of
the $x$ axis and the upper label is placed to the right of the 
end of the bar. (This is actually a misnomer as it is possible
for the ``upper'' label to be below the ``lower'' label if a
bar has a negative value, however the bars are considered to
be anchored on the $x$ axis, and the other end of the bar is
considered to be the ``upper'' end, regardless of its
direction.)

The \sty{databar} package options are as follows:
\begin{description}
\item[{\pkgopt{color}}] Created coloured bar charts 
(default).

\item[{\pkgopt{gray}}] Created grey scale bar charts.

\item[{\pkgopt{vertical}}] Created vertical bar charts
(default).

\item[{\pkgopt{horizontal}}] Created horizontal bar charts.
\end{description}



\begin{definition}[\DescribeMacro{\DTLbarchart}]%
\cs{DTLbarchart}\oarg{condition}\marg{settings}\marg{db name}\marg{values}
\end{definition}
\begin{definition}[\DescribeMacro{\DTLmultibarchart}]%
\cs{DTLmultibarchart}\oarg{condition}\marg{settings}\marg{db name}\marg{values}
\end{definition}
These commands both create a bar chart from the information in
the database \meta{db name}, where \meta{condition} is the same
as the optional argument for \ics{DTLforeach} described in
\autoref{sec:dbforeach}, and \meta{values} is the same as the
penultimate argument of \ics{DTLforeach}. The \meta{settings}
argument is a \meta{setting}"="\meta{value} list of settings. The
first command, \cs{DLTbarchart}, will draw a bar chart for a
given column of data in the database, whereas the second command,
\cs{DTLmultibarchart}, will draw a bar chart that is divided into
groups of bars where each bar within a group represents data from
several columns of a given row in the database.

The \csopt{DTLbarchart}{variable} setting is
required for \cs{DTLbarchart} and the 
\csopt{DTLmultibarchart}{variables}, the other settings are 
optional (though some may only be used for one of
\cs{DTLbarchart} and \cs{DLTmultibarchart}), and are as follows:

\begin{description}
\item[\csopt{DTLbarchart}{variable}] 
This specifies the control sequence to use that 
contains the value used to construct the bar chart. The control
sequence must be one of the control sequences to appear in
the assignment list \meta{values}. This setting is required
for \cs{DTLbarchart}, and is unavailable for \cs{DTLmultibarchart}.

\item[\csopt{DTLmultibarchart}{variables}] 
This specifies a list of control sequences to use which 
contain the values used to construct the bar chart. Each control
sequence must be one of the control sequences to appear in
the assignment list \meta{values}. This setting is required
for \cs{DTLmultibarchart}, and is unavailable for
\cs{DTLbarchart}.

\item[\csopt{DTLbarchart,DTLmultibarchart}{max}]
This specifies the maximum value on the $y$ axis. (This should
be a standard decimal value.)

\item[\csopt{DTLbarchart,DTLmultibarchart}{length}]
This specifies the overall length of the $y$ axis, and must
be a dimension.

\item[\csopt{DTLbarchart,DTLmultibarchart}{maxdepth}]
This must be a zero or negative number. It specifies the maximum
depth of the $y$ axis. (This should be a standard decimal value.)

\item[\csopt{DTLbarchart,DTLmultibarchart}{axes}]
This setting specifies which axes to display. This may take one
of the following values: "both", "x", "y" or "none".

\item[\csopt{DTLbarchart,DTLmultibarchart}{barlabel}]
This setting specifies the lower bar label. When used with
\cs{DTLmultibarchart} it indicates the group label.

\item[\csopt{DTLmultibarchart}{multibarlabels}]
This setting should contain a comma separated list of labels
for each bar within a group for \cs{DTLmultibarchart}. This
setting is not available for \cs{DTLbarchart}.

\item[\csopt{DTLbarchart}{upperbarlabel}]
This setting specifies the upper bar label. This setting
is not available for \cs{DTLmultibarchart}.

\item[\csopt{DTLmultibarchart}{uppermultibarlabels}]
This setting must be a comma separated list of upper bar
labels for each bar within a group. This setting is not 
available for \cs{DTLbarchart}.

\item[\csopt{DTLbarchart,DTLmultibarchart}{yticpoints}]
This must be a comma separated list of tick locations for the
$y$ axis. (These should be standard decimal values.)
This setting overrides \csopt{DTLbarchart}{yticgap}.

\item[\csopt{DTLbarchart,DTLmultibarchart}{yticgap}]
This specifies the gap between the $y$ tick marks. (This should
be a standard decimal value.)

\item[\csopt{DTLbarchart,DTLmultibarchart}{yticlabels}]
This must be a comma separated list of tick labels for the
$y$ axis.

\item[\csopt{DTLbarchart,DTLmultibarchart}{ylabel}]
This specifies the label for the $y$ axis.

\item[\csopt{DTLmultibarchart}{groupgap}] This specifies
the gap between groups when using \cs{DTLmultibarchart}.
This value is given as a multiple of the bar width. The
default value is 1, which indicates a gap of one bar width.
This setting is not available for \cs{DTLbarchart}.

\item[\csopt{DTLbarchart,DTLmultibarchart}{verticalbars}]
This is a boolean setting, so it can only take the values
"true" (do a vertical bar chart) or "false" (do a horizontal
bar chart). If the value is omitted, "true" is assumed.
\end{description}

\begin{example}{A Basic Bar Chart}{ex:barchart}
Recall \autoref{ex:piechart} defined a database called "fruit".
This example will be using that database to plot a bar chart.
The following plots a basic vertical bar chart:
\begin{verbatim}
\begin{figure}[htbp]
\centering
\DTLbarchart{variable=\theQuantity}{fruit}{\theQuantity=Quantity}
\caption{A basic bar chart}
\end{figure}
\end{verbatim}
This produces \autoref{fig:barchart}.

\begin{figure}[htbp]
\centering
\DTLbarchart{variable=\theQuantity}{fruit}{\theQuantity=Quantity}
\caption{A basic bar chart}
\label{fig:barchart}
\end{figure}
\end{example}

\section{Changing the Appearance of a Bar Chart}


\begin{definition}[\DescribeMacro{\DTLbarchartlength}]%
\cs{DTLbarchartlength}
\end{definition}
This specifies the total length of the $y$ axis. You must use
\cs{setlength} to change this value. The default value is 3in.

\begin{definition}[\DescribeMacro{\DTLbarwidth}]%
\cs{DTLbarwidth}
\end{definition}
This specifies the width of each bar. You must use \cs{setlength}
to change this value. The default value is 1cm.

\begin{definition}[\DescribeMacro{\DTLbarlabeloffset}]%
\cs{DTLbarlabeloffset}
\end{definition}
This specifies the distance from the $x$ axis to the lower bar label.
You must use \cs{setlength} to change this value. The default value
is 10pt.

\begin{definition}[\DescribeCounter{DTLbarroundvar}]
\ctrfmt{DTLbarroundvar}
\end{definition}
The $y$ tick labels are rounded to \meta{n} digits after the
decimal point, where \meta{n} is given by the value of the
counter \ctrfmt{DTLbarroundvar}. You must use \cs{setcounter}
to change this value.

\begin{definition}[\DescribeMacro{\DTLsetbarcolor}]%
\cs{DTLsetbarcolor}\marg{n}\marg{color}
\end{definition}
This sets the \meta{n}th bar colour to \meta{color}.
Only the first eight bars have a colour defined by default. If you
need more than eight bars, you will need to define more bar colours.
It is recommended that you set the colour of each bar to 
correspond with whatever the bar represents.

\begin{definition}[\DescribeMacro{\DTLdobarcolor}]%
\cs{DTLdobarcolor}\marg{n}
\end{definition}
This sets the current colour to the colour of the \meta{n}th bar.

\begin{definition}[\DescribeMacro{\DTLbaroutlinecolor}]%
\cs{DTLbaroutlinecolor}
\end{definition}
This macro contains the colour of the bar outlines. This defaults
to "black".

\begin{definition}[\DescribeMacro{\DTLbaroutlinewidth}]%
\cs{DTLbaroutlinewidth}
\end{definition}
This length specifies the line width for the bar outlines. If it
is 0pt, the outline is not drawn. The default value is 0pt.


\begin{definition}[\DescribeMacro{\DTLbaratbegintikz}]%
\cs{DTLbaratbegintikz}
\end{definition}
This specifies any additional commands to add to the start of
the plot. It defaults to nothing, and is called after the
unit vectors are set.

\begin{definition}[\DescribeMacro{\DTLbaratendtikz}]%
\cs{DTLbaratendtikz}
\end{definition}
This specifies any additional commands to add to the end of
the plot. It defaults to nothing.

\begin{definition}[\DescribeMacro{\DTLeverybarhook}]%
\cs{DTLeverybarhook}
\end{definition}
The specifies code to apply at every bar. Within the definition
of \cs{DTLeverybarhook} you can use the commands 
\DescribeMacro{\DTLstartpt}\cs{DTLstartpt} (the start
of the bar), \DescribeMacro{\DTLmidpt}\cs{DTLmidpt}
(the mid point of the bar) and 
\DescribeMacro{\DTLendpt}\cs{DTLendpt} (the end of the bar). For
example (using the earlier "fruit" database):
\begin{verbatim}
\renewcommand*{\DTLeverybarhook}{%
\pgftext[at=\DTLmidpt]{\insertName\space(\insertValue)}%
}
\DTLbarchart{variable=\insertValue,axes=both,
ylabel=Quantity,max=50,verticalbars=false
}%
{fruit}{\insertValue=Value,\insertName=Name}
\end{verbatim}
This puts the name followed by the quantity in brackets in the
middle of the bar.

\begin{definition}[\DescribeMacro{\ifDTLverticalbars}]%
\cs{ifDTLverticalbars}
\end{definition}
This conditional governs whether the chart uses vertical or
horizontal bars.

\begin{definition}[\DescribeMacro{\DTLbarXlabelalign}]%
\cs{DTLbarXlabelalign}
\end{definition}
This specifies the text alignment of the lower bar labels. This
defaults to "left,rotate=-90" if you use the \pkgopt{vertical}
package option or the \csopt{DTLbarchart}{verticalbars} setting,
and defaults to "right" if you use the \pkgopt{horizontal}
package option or the \csopt{DTLbarchart}{verticalbars}"=false"
setting.

\begin{definition}[\DescribeMacro{\DTLbarYticklabelalign}]%
\cs{DTLbarYlabelalign}
\end{definition}
This specifies the text alignment of the $y$ axis labels. This
defaults to "right" for vertical bar charts and "center" for
horizontal bar charts.

\begin{definition}[\DescribeMacro{\DTLbardisplayYticklabel}]%
\cs{DTLbardisplayYticklabel}\marg{text}
\end{definition}
This specifies how to display the $y$ tick label. The argument
is the tick label.

\begin{definition}[\DescribeMacro{\DTLdisplaylowerbarlabel}]%
\cs{DTLdisplaylowerbarlabel}\marg{text}
\end{definition}
This specifies how to display the lower bar label for
\cs{DTLbarchart} and the lower bar group label for
\cs{DTLmultibarchart}. The argument is the label.

\begin{definition}[\DescribeMacro{\DTLdisplaylowermultibarlabel}]%
\cs{DTLdisplaylowermultibarlabel}\marg{text}
\end{definition}
This specifies how to display the lower bar label for
\cs{DTLmultibarchart}. The argument is the label. This 
command is ignored by \cs{DTLbarchart}.

\begin{definition}[\DescribeMacro{\DTLdisplayupperbarlabel}]%
\cs{DTLdisplayupperbarlabel}\marg{text}
\end{definition}
This specifies how to display the upper bar label for
\cs{DTLbarchart} and the upper bar group label for
\cs{DTLmultibarchart}. The argument is the label.

\begin{definition}[\DescribeMacro{\DTLdisplayuppermultibarlabel}]%
\cs{DTLdisplayuppermultibarlabel}\marg{text}
\end{definition}
This specifies how to display the upper bar label for
\cs{DTLmultibarchart}. The argument is the label. This 
command is ignored by \cs{DTLbarchart}.

\begin{example}{A Labelled Bar Chart}{ex:annotebarchart}
This example extends \autoref{ex:barchart} so that the chart is
a bit more informative (which is after all the whole point of
a chart). This chart now has a label below each bar, as well 
as a label above the bar. The lower label uses the value of the
"Name" key, and the upper label uses the quantity. I have also
set the outline width so each bar has a border.
\begin{verbatim}
\begin{figure}[htbp]
\setlength{\DTLbaroutlinewidth}{1pt}
\centering
\DTLbarchart{variable=\theQuantity,barlabel=\theName,%
upperbarlabel=\theQuantity}{fruit}{%
\theQuantity=Quantity,\theName=Name}
\caption{A bar chart}
\end{figure}
\end{verbatim}
This produces \autoref{fig:annotebarchart}.

\begin{figure}[htbp]
\centering
\setlength{\DTLbaroutlinewidth}{1pt}
\DTLbarchart{variable=\theQuantity,barlabel=\theName,
upperbarlabel=\theQuantity}{fruit}{\theQuantity=Quantity,\theName=Name}
\caption[A bar chart (labelled)]{A bar chart}
\label{fig:annotebarchart}
\end{figure}
\end{example}

\begin{example}{Profit/Loss Bar Chart}{ex:profit}
Suppose I have a file called "profits.csv" that looks like:
\DTLnewdb{profits}
\DTLnewrow{profits}
\DTLnewdbentry{profits}{Year}{2000}\relax
\DTLnewdbentry{profits}{Profit}{\protect\pounds2,525}\relax
\DTLnewrow{profits}
\DTLnewdbentry{profits}{Year}{2001}\relax
\DTLnewdbentry{profits}{Profit}{\protect\pounds3,752}\relax
\DTLnewrow{profits}
\DTLnewdbentry{profits}{Year}{2002}\relax
\DTLnewdbentry{profits}{Profit}{-\protect\pounds1,520}\relax
\DTLnewrow{profits}
\DTLnewdbentry{profits}{Year}{2003}\relax
\DTLnewdbentry{profits}{Profit}{\protect\pounds1,270}\relax
\begin{verbatim}
Year,Profit
2000,\pounds2,535
2001,\pounds3,752
2002,-\pounds1,520
2003,\pounds1,270
\end{verbatim}
First I can load this file into a database called "profits":
\begin{verbatim}
\DTLloaddb{profits}{profits.csv}
\end{verbatim}
Now I can plot the data as a bar chart:
\begin{verbatim}
\begin{figure}[htbp]
\centering
 % Set the width of each bar to 10pt
\setlength{\DTLbarwidth}{10pt}
 % Set the outline width to 1pt
\setlength{\DTLbaroutlinewidth}{1pt}
 % Round the $y$ tick labels to integers
\setcounter{DTLbarroundvar}{0}
 % Adjust the tick label offset
\setlength{\DTLticklabeloffset}{20pt}
 % Change the y tick label alignment
\renewcommand*{\DTLbarYticklabelalign}{left}
 % Rotate the y tick labels
\renewcommand*{\DTLbardisplayYticklabel}[1]{\rotatebox{-45}{#1}}
 % Set the bar colours depending on the value of \theProfit
\DTLforeach{profits}{\theProfit=Profit}{%
\ifthenelse{\DTLislt{\theProfit}{0}}
{\DTLsetbarcolor{\DTLcurrentindex}{red}}
{\DTLsetbarcolor{\DTLcurrentindex}{blue}}}
 % Do the bar chart
\DTLbarchart{variable=\theProfit,upperbarlabel=\theYear,
ylabel={Profit/Loss (\pounds)},verticalbars=false,
maxdepth=-2000,max=4000}{profits}
{\theProfit=Profit,\theYear=Year}
\caption{Profits for 2000--2003}
\end{figure}
\end{verbatim}
This produces \autoref{fig:profits}. Notes:
\begin{enumerate}
\item This example uses \cs{rotatebox}, so the \sty{graphics}
or \sty{graphicx} package is required.
\item The $y$ tick labels are too wide to fit horizontally
so they have been rotated to avoid overlapping with their
neighbour.
\item Rotating the $y$ tick labels puts them too close to
the $y$ axis, so \ics{DTLticklabeloffset} is made larger to
compensate.
\item Remember not to use \cs{year} as an assignment command
as this command already exists!
\item Before the bar chart is created I have iterated through
the database, setting the bar colour to red or blue
depending on the value of \cs{theProfit}.
\end{enumerate}

Both \cs{DTLbarchart} and \cs{DTLmultibarchart} set the following
macros, which may be used in \cs{DTLbaratbegintikz} and
\cs{DTLbaratendtikz}:
\begin{definition}[\DescribeMacro{\DTLbarchartwidth}]%
\cs{DTLbarchartwidth}
\end{definition}
This is the overall width of the bar chart. In the case of
\cs{DTLbarchart} this is just the number of bars. In the case
of \cs{DTLmultibarchart} it is computed as:
\begin{displaymath}
m \times n + (m-1)\times g
\end{displaymath}
where $m$ is the number of bar groups (i.e.\ the number of rows
of data), $n$ is the number of bars within a group (i.e.\ 
the number of commands listed in the 
\csopt{DTLmultibarchart}{variables}) setting and $g$ is the
group gap (as specified by the \csopt{DTLmultibarchart}{groupgap} setting).

\begin{definition}[\DescribeMacro{\DTLnegextent}]%
\cs{DTLnegextent}
\end{definition}
This is set to the negative extent of the bar chart. (This value
may either be zero or negative, and corresponds to the
\csopt{DTLbarchart,DTLmultibarchart}{maxdepth} setting.)

\begin{definition}[\DescribeMacro{\DTLbarmax}]%
\cs{DTLbarmax}
\end{definition}
This is set to the maximum extent of the bar chart. (This value
corresponds to the
\csopt{DTLbarchart,DTLmultibarchart}{max} setting.)

\begin{figure}[htbp]
\centering
\setlength{\DTLbarwidth}{10pt}
\setlength{\DTLbaroutlinewidth}{1pt}
\setlength{\DTLticklabeloffset}{20pt}
\setcounter{DTLbarroundvar}{0}
\renewcommand*{\DTLbarYticklabelalign}{left}
\renewcommand*{\DTLbardisplayYticklabel}[1]{\rotatebox{-45}{#1}}
\DTLforeach{profits}{\theProfit=Profit}{\relax
\ifthenelse{\DTLislt{\theProfit}{0}}
{\DTLsetbarcolor{\DTLcurrentindex}{red}}
{\DTLsetbarcolor{\DTLcurrentindex}{blue}}}
\DTLbarchart{variable=\theProfit,upperbarlabel=\theYear,
ylabel={Profit/Loss (\pounds)},verticalbars=false,
maxdepth=-2000,max=4000
}{profits}
{\theProfit=Profit,\theYear=Year}
\caption[Profits for 2000--2003 (a horizontal bar chart)]{Profits for 
2000--2003}
\label{fig:profits}
\end{figure}

\end{example}

\begin{example}{A Multi-Bar Chart}{ex:multibar}
This example uses the "marks" database described in 
\autoref{ex:editdb}.
Recall that this database stores student marks for three 
assignments. The keys for the assignment marks are 
\texttt{Assignment 1}, \texttt{Assignment 2} and 
\texttt{Assignment 3}, respectively. I can convert this
data into a bar chart using the following:
\begin{verbatim}
\begin{figure}[htbp]
\centering
\DTLmultibarchart{variables={\assignI,\assignII,\assignIII},
barwidth=10pt,uppermultibarlabels={\assignI,\assignII,\assignIII},
barlabel={\firstname\ \surname}}{marks}{%
\surname=Surname,\firstname=FirstName,\assignI=Assignment 1,%
\assignII=Assignment 2,\assignIII=Assignment 3}
\caption{Student marks}
\end{figure}
\end{verbatim}
This produces \autoref{fig:multibar}. Notes:
\begin{enumerate}
\item I used "variables={\assignI,\assignII,\assignIII}" to
set the variable to use for each bar within a group. This means
that there will be three bars in each group.
\item I have set the bar width to 10pt, otherwise the chart
will be too wide.
\item I used "uppermultibarlabels={\assignI,\assignII,\assignIII}"
to set the upper labels for each bar within a group. This
will print the assignment mark above the relevant bar.
\item I used "barlabel={\firstname\ \surname}" to place the
student's name below the group corresponding to that student.
\end{enumerate}

\begin{figure}[htbp]
\centering
\DTLmultibarchart{variables={\assignI,\assignII,\assignIII},
barwidth=10pt,uppermultibarlabels={\assignI,\assignII,\assignIII},
barlabel={\firstname\ \surname}}{marks}
{\surname=Surname,\firstname=FirstName,\assignI=Assignment 1,\assignII=Assignment 2,\assignIII=Assignment 3}
\caption[Student marks (a multi-bar chart)]{Student 
marks}\label{fig:multibar}
\end{figure}

Recall that \autoref{ex:editdb} computed the average score over
for each student, and saved it with the key "Average". This
information can be added to the bar chart. It might also be
useful to compute the average over all students and add this
information to the chart. This is done as follows:
\begin{verbatim}
 \begin{figure}[htbp]
 \centering
 % compute the overall mean
 \DTLmeanforkeys{marks}{Average}{\overallmean}
 % round it to 2 decimal places
 \DTLround{\overallmean}{\overallmean}{2}
 % draw a grey dotted line indicating the overall mean
 % covering the entire width of the bar chart
 \renewcommand*{\DTLbaratendtikz}{%
   \draw[lightgray,loosely dotted] (0,\overallmean) --
     (\DTLbarchartwidth,\overallmean)
     node[right,black]{Average (\overallmean)};}
 % Set the lower bar labels to draw a brace across the current
 % group, along with the student's name and average score
 \renewcommand*{\DTLdisplaylowerbarlabel}[1]{%
 \tikz[baseline=(current bounding box.center)]{
 \draw[snake=brace,rotate=-90](0,0) -- (\DTLbargroupwidth,0);}
 \DTLround{\theMean}{\theMean}{2}%
 \shortstack{#1\\(Average: \theMean)}}
 % draw the bar chart
 \DTLmultibarchart{variables={\assignI,\assignII,\assignIII},
 barwidth=10pt,uppermultibarlabels={\assignI,\assignII,\assignIII},
 barlabel={\firstname\ \surname}}{marks}
 {\surname=Surname,\firstname=FirstName,\assignI=Assignment 1,%
 \assignII=Assignment 2,\assignIII=Assignment 3,\theMean=Average}
 \caption{Student marks}
 \end{figure}
\end{verbatim}
which produces \autoref{fig:multibarmean}. Notes:
\begin{enumerate}
\item I've used the TikZ snake library to create a brace, so
I need to put
\begin{verbatim}
\usetikzlibrary{snakes}
\end{verbatim}
in the preamble. See the \sty{pgf} manual for more details on
how to use this library.

\item I used \ics{DTLbargroupwidth} to indicate the width of
each bar group.

\item I used \ics{DTLbarchartwidth} to indicate the width of the
entire bar chart
\end{enumerate}

\begin{figure}[htbp]
\centering
\DTLmeanforkeys{marks}{Average}{\overallmean}
\DTLround{\overallmean}{\overallmean}{2}
\renewcommand*{\DTLbaratendtikz}{\draw[lightgray,loosely dotted] (0,\overallmean) --
(\DTLbarchartwidth,\overallmean) node[right,black]{Average (\overallmean)};}
\renewcommand*{\DTLdisplaylowerbarlabel}[1]{\relax
\tikz[baseline=(current bounding box.center)]{
\draw[snake=brace,rotate=-90](0,0) -- (\DTLbargroupwidth,0);}
\DTLround{\theMean}{\theMean}{2}\relax
\shortstack{#1\\(Average: \theMean)}}
\DTLmultibarchart{variables={\assignI,\assignII,\assignIII},
barwidth=10pt,uppermultibarlabels={\assignI,\assignII,\assignIII},
barlabel={\firstname\ \surname}}{marks}
{\surname=Surname,\firstname=FirstName,\assignI=Assignment 1,\assignII=Assignment 2,\assignIII=Assignment 3,\theMean=Average}
\caption[Student marks (annotating a bar chart)]{Student marks}
\label{fig:multibarmean}
\end{figure}

\end{example}

\chapter{Converting a \texorpdfstring{\BibTeX}{BibTeX} database 
into a \texorpdfstring{\sty{datatool}}{datatool} database
(\texorpdfstring{\sty{databib}}{databib} package)}
\label{sec:databib}

The \sty{databib} package provides the means of converting a \BibTeX\
database into a \sty{datatool} database. The database can then be
sorted using \cs{DTLsort}, described in \autoref{sec:sort}.
For example, you may want to sort the bibliography in 
reverse chronological order. Once you have sorted the bibliography,
you can display it using \cs{DTLbibliography}, described in
\autoref{sec:thebib}, or you can iterate through the database
using \cs{DTLforeachbibentry}, described in \autoref{sec:foreachbib}.

Note that the \sty{databib}
package is not automatically loaded by \sty{datatool}, so if
you want to use it, you must load it using
"\usepackage{databib}".

\begin{important}
The purpose of this package is to provide a means for
authors to format their own bibliography style where there is no
bibliography style file available that produces the desired results.
The \cs{DTLsort} macro uses a much less efficient sorting algorithm
than \BibTeX, and loading the bibliography as a \sty{datatool}
database is much slower than loading a standard \filetype{bbl} file. If
you have a large database, and you are worried that \LaTeX\ may have
become stuck, try using the \pkgopt{verbose} option to \sty{datatool}
or use the command \cs{dtlverbosetrue}.  This will print informative
messages to the console and transcript file, to let you know what's
going on.
\end{important}

\section{\texorpdfstring{\BibTeX}{BibTeX}: An Overview}
\label{sec:bibtex}
This document assumes that you have at least some passing
familiarity with \BibTeX, but here follows a brief refresher.

\BibTeX\ is an external application used in conjunction with \LaTeX.
When you run \BibTeX, you need to specify the name of the document's
auxiliary file (without the \filetype{aux} extension). \BibTeX\ then reads
this file and looks for the commands \cs{bibstyle} (which indicates
which bibliography style (\filetype{bst}) file to load), \cs{bibdata}
(which indicates which bibliography database (\filetype{bib}) files to
load) and \cs{citation} (produced by \cs{cite} and \cs{nocite}, which
indicates which entries should be included in the bibliography).
\BibTeX\ then creates a file with the extension \filetype{bbl} which
contains the bibliography, formatted according to the layout defined
in the bibliography style file.

In general, given a document called, say, \texttt{mydoc.tex}, you
will have to perform the following steps to ensure that the
bibliography and all citations are up-to-date:
\begin{enumerate}
\item
\begin{verbatim}
latex mydoc
\end{verbatim}
This writes the citation information to the auxiliary file.
The bibliography currently doesn't exists, so it isn't displayed.
Citations will appear in the document as ?? since the internal
cross-references don't exist yet.

\item
\begin{verbatim}
bibtex mydoc
\end{verbatim}
This reads the auxiliary file, and creates a file with the extension
\filetype{bbl} which typically contains the typeset bibliography.

\item
\begin{verbatim}
latex mydoc
\end{verbatim}
Now that the \filetype{bbl} file exists, the bibliography can be input
into the document. The internal cross-referencing information for the 
bibliography can now be written to the auxiliary file.

\item
\begin{verbatim}
latex mydoc
\end{verbatim}
The cross-referencing information can be read from the auxiliary
file.
\end{enumerate}

\subsection{\texorpdfstring{\BibTeX}{BibTeX} database}

The bibliographic data required by \BibTeX\ must be stored in 
a file with the extension \filetype{bib}, where each entry is stored
in the form:\par\vskip\baselineskip\noindent
{\obeylines
\noindent\texttt{@}\meta{entry\_type}\verb|{|\meta{cite\_key}\texttt,
  \meta{field\_name} \texttt{=} \char`\"\meta{value}\char`\"\texttt,
  \mbox{}\vdots
  \meta{field\_name} \texttt{=} \char`\"\meta{value}\char`\"
\noindent\verb|}|
}
\par\vskip\baselineskip\noindent
Note that curly braces "{" and "}" may be used instead of \verb|"|
and \verb|"|.

The entry type, given by \meta{entry\_type} above, indicates
the type of document. This may be one of: "article", "book", 
"booklet", "inbook", "incollection", "inproceedings"\footnote
{Note that \texttt{conference} is a synonym for \texttt{inproceedings}.},
"manual",
"mastersthesis", "misc", "phdthesis", "proceedings", \linebreak
"techreport" or "unpublished".

The \meta{cite\_key} above is a unique label identifying this
entry, and is the label used in the argument of \cs{cite} or
\cs{nocite}. The available fields depends on the entry type, for
example, the field "journal" is required for the "article" entry
type, but is ignored for the "inproceedings" entry type. The standard
fields are: "address", "author", "booktitle", "chapter", "edition",
"editor", "howpublished", "institution", "journal", "key", "month",
"note", "number", "organization", "pages", "publisher", "school",
"series", "title", "type", "volume" and "year".

Author and editor names must be entered in one of the following
ways:
\begin{enumerate}
\item \meta{First names} \meta{von part} \meta{Surname}, \meta{Jr part}

The \meta{von part} is optional and is identified by the name(s)
starting with lowercase letters. The final comma followed by 
\meta{Jr part} is also optional. Examples:
\begin{verbatim}
author = "Henry James de Vere"
\end{verbatim}
In the above, the first names are Henry James, the ``von part'' is
de and the surname is Vere. There is no ``junior part''.
\begin{verbatim}
author = "Mary-Jane Brown, Jr"
\end{verbatim}
In the above, the first name is Mary-Jane, there is no von part,
the surname is Brown and the junior part is Jr.
\begin{verbatim}
author = "Peter {Murphy Allen}"
\end{verbatim}
In the above, the first name is Peter, and the surname is Murphy 
Allen.  Note that in this case, the surname must be grouped, otherwise
Murphy would be considered part of the forename.
\begin{verbatim}
author = "Maria Eliza {\uppercase{d}e La} Cruz"
\end{verbatim}
In the above, the first name is Maria Eliza, the von part is
De La, and the surname is Cruz. In this case, the von part starts
with an uppercase letter, but specifying
\begin{verbatim}
author = "Maria Eliza De La Cruz"
\end{verbatim}
would make \BibTeX\ incorrectly classify ``Maria Eliza De La'' as
the first names, and the von part would be empty. Since \BibTeX\
doesn't understand \LaTeX\ commands, using "{\uppercase{d}e La}"
will trick \BibTeX\ into thinking that it starts with a lower
case letter.

\item \meta{von part} \meta{Surname}, \meta{Forenames}

Again the \meta{von part} is optional, and is determined by the
case of the first letter. For example:
\begin{verbatim}
author = "de Vere, Henry James"
\end{verbatim}
\end{enumerate}

Multiple authors or editors should be separated by the key word
"and", for example:
\begin{verbatim}
author = "Michel Goossens and Frank Mittlebach and Alexander Samarin"
\end{verbatim}

Below is an example of a book entry:
\begin{verbatim}
@book{latexcomp,
  title     = "The \LaTeX\ Companion",
  author    = "Michel Goossens and Frank Mittlebach and 
               Alexander Samarin",
  publisher = "Addison-Wesley",
  year      = 1994
}
\end{verbatim}
Note that numbers may be entered without delimiters, as in "year = 1994".
There are also some predefined strings, including those for the month
names. You should always use these strings instead of the actual
month name, as the way the month name is displayed depends on the
bibliography style. For example:
\begin{verbatim}
@article{Cawley2007b,
author = "Gavin C. Cawley and Nicola L. C. Talbot",
title  = "Preventing over-fitting in model selection via {B}ayesian
          regularisation of the hyper-parameters",
journal = "Journal of Machine Learning Research",
volume  = 8,
pages   = "841--861",
month   = APR,
year    = 2007
}
\end{verbatim}

You can concatenate strings using the "#" character, for example:
\begin{verbatim}
month  = JUL # "~31~--~" # AUG # "~4",
\end{verbatim}
Depending on the bibliography style, this may be displayed as:
July~31~--~August~4, or it may be displayed as:
Jul~31~--~Aug~4. For further information, see~\cite{Goossens}.

\section{Loading a \texorpdfstring{\sty{databib}}{databib} 
database}
\label{sec:loadbbl}

The \sty{databib} package always requires the \texttt{databib.bst}
bibliography style file (which is supplied with this bundle).
You need to use \cs{cite} or \cs{nocite} as usual. If you want to
add all entries in the \filetype{bib} file to the \sty{datatool} database,
you can use "\nocite{*}".

\begin{definition}[\DescribeMacro{\DTLloadbbl}]%
\cs{DTLloadbbl}\oarg{bbl name}\marg{db name}\marg{bib list}
\end{definition}
This command performs several functions:

\begin{enumerate}
\item it writes the following line in the auxiliary file:
\begin{verbatim}
\bibstyle{databib}
\end{verbatim}
which tells \BibTeX\ to use the \texttt{databib.bst}
\BibTeX\ style file, 

\item it writes \cs{bibdata}\marg{bib list} to the
auxiliary file, which tells \BibTeX\ which \filetype{bib} files to use,

\item it creates a \sty{datatool} database called \meta{db name},

\item it loads the file \meta{bbl name} if it exists. (The value
defaults to \cs{jobname}".bbl", which is the usual name for
a \filetype{bbl} file.) If the \filetype{bbl} file doesn't exist, the
database \meta{db name} will remain empty.
\end{enumerate}

You then need to run your document through \LaTeX\ (or PDF\LaTeX)
and then run \BibTeX\ on the auxiliary file, as described in
\autoref{sec:bibtex}. This will create a \filetype{bbl} file which
contains all the commands required to add the bibliography information
to the \sty{datatool} database called \meta{db name}. The next
time you \LaTeX\ your document, this file will be read, and the
information will be added to \meta{db name}. 

\begin{important}
Note that \cs{DTLloadbbl} doesn't generate any text. Once you have
loaded the data, you can display the bibliography uses
\cs{DTLbibliography} (described below) or you can iterate through it
using \cs{DTLforeachbibentry} described in \autoref{sec:foreachbib}.
\end{important}

Note that the \texttt{databib.bst} \BibTeX\ style file provides
the following additional fields: "isbn", "doi", "pubmed", "url"
and "abstract".
However these fields are ignored by the three predefined
\sty{databib} styles ("plain", "abbrv" and "alpha"). If you
want these fields to be displayed in the bibliography you will
need to modify the bibliography style (see \autoref{sec:modbibstyle}).

\section{Displaying a \texorpdfstring{\sty{databib}}{databib} 
database}
\label{sec:thebib}

A \sty{databib} database which has been loaded using 
\cs{DTLloadbbl} (described in \autoref{sec:loadbbl}) can be 
displayed using:
\begin{definition}[\DescribeMacro{\DTLbibliography}]
\cs{DTLbibliography}\oarg{conditions}\marg{db name}
\end{definition}\noindent
where \meta{db name} is the name of the database.

Within the optional argument \meta{condition}, you may use any of the 
commands that may be used within the optional argument of
\cs{DTLforeach} \emph{In addition}, you may use the following
commands:

\begin{definition}[\DescribeMacro{\DTLbibfieldexists}]%
\cs{DTLbibfieldexists}\marg{field label}
\end{definition}
This tests whether the field with the given label exists for the
current entry.
The field label may be one of: "Address", "Author", 
"BookTitle", "Chapter", "Edition", "Editor", "HowPublished",
"Institution", "Journal", "Key", "Month", "Note", "Number", 
"Organization", "Pages", "Publisher", "School", "Series",
"Title", "Type", "Volume", "Year", "ISBN", "DOI", "PubMed",
"Abstract", "Url" or "Eprints".

For example, suppose you have loaded a \sty{databib} database
called "mybib" using \cs{DTLloadbbl} (described in 
\autoref{sec:loadbbl}) then the following bibliography will only
include those entries which have a "Year" field:
\begin{verbatim}
\DTLbibliography[\DTLbibfieldexists{Year}]{mybib}
\end{verbatim}

\begin{definition}[\DescribeMacro{\DTLbibfieldiseq}]%
\cs{DTLbibfieldiseq}\marg{field label}\marg{value}
\end{definition}
This tests whether the value of the field given by 
\meta{field label} equals \meta{value}. If the field doesn't
exist for the current entry, this evaluates to false.
For example, the following will produce a bibliography which
only contains entries which have the "Year" field set to 2004:
\begin{verbatim}
\DTLbibliography[\DTLbibfieldiseq{Year}{2004}]{mybib}
\end{verbatim}

\begin{definition}[\DescribeMacro{\DTLbibfieldcontains}]%
\cs{DTLbibfieldcontains}\marg{field label}\marg{sub string}
\end{definition}
This tests whether the value of the field given by \meta{field label}
contains \meta{sub string}. For example, the following will produce
a bibliography which only contains entries where the author field
contains the name "Knuth":
\begin{verbatim}
\DTLbibliography[\DTLbibfieldcontains{Author}{Knuth}]{mybib}
\end{verbatim}

\begin{definition}[\DescribeMacro{\DTLbibfieldislt}]%
\cs{DTLbibfieldislt}\marg{field label}\marg{value}
\end{definition}
This tests whether the value of the field given by \meta{field label}
is less than \meta{value}. If the field doesn't exist for the
current entry, this evaluates to false.
For example, the following will produce a bibliography which only
contains entries whose "Year" field is less than 1983:
\begin{verbatim}
\DTLbibliography[\DTLbibfieldislt{Year}{1983}]{mybib}
\end{verbatim}

\begin{definition}[\DescribeMacro{\DTLbibfieldisle}]%
\cs{DTLbibfieldisle}\marg{field label}\marg{value}
\end{definition}
This tests whether the value of the field given by \meta{field label}
is less than or equal to \meta{value}. If the field doesn't exist 
for the current entry, this evaluates to false.
For example, the following will produce a bibliography which only
contains entries whose "Year" field is less than or equal to 1983:
\begin{verbatim}
\DTLbibliography[\DTLbibfieldisle{Year}{1983}]{mybib}
\end{verbatim}

\begin{definition}[\DescribeMacro{\DTLbibfieldisgt}]%
\cs{DTLbibfieldisgt}\marg{field label}\marg{value}
\end{definition}
This tests whether the value of the field given by \meta{field label}
is greater than \meta{value}. If the field doesn't exist for the
current entry, this evaluates to false.
For example, the following will produce a bibliography which only
contains entries whose "Year" field is greater than 1983:
\begin{verbatim}
\DTLbibliography[\DTLbibfieldisgt{Year}{1983}]{mybib}
\end{verbatim}

\begin{definition}[\DescribeMacro{\DTLbibfieldisge}]%
\cs{DTLbibfieldisge}\marg{field label}\marg{value}
\end{definition}
This tests whether the value of the field given by \meta{field label}
is greater than or equal to \meta{value}. If the field doesn't exist 
for the current entry, this evaluates to false.
For example, the following will produce a bibliography which only
contains entries whose "Year" field is greater than or equal to 1983:
\begin{verbatim}
\DTLbibliography[\DTLbibfieldisge{Year}{1983}]{mybib}
\end{verbatim}

Note that \cs{DTLbibliography} uses \cs{DTLforeachbibentry}
(described in \autoref{sec:foreachbib}) so you may also use
test the value of the counter \ctr{DTLbibrow} within
\meta{conditions}. You may also use the boolean commands defined
by the \sty{ifthen} package, such as \ics{not}.

\begin{example}{Creating a list of publications since a given year}{ex:bibsince2000}
Suppose my boss has asked me to produce a list of my 
publications in reverse chronological order, but doesn't want any 
publications published prior 
to the year 2000. I have a file called "nlct.bib" which contains
all my publications which I keep in the directory 
\verb!$HOME/texmf/bibtex/bib/!. I could look through this file,
work out the labels for all the publications whose year field
is greater or equal to 2000, and
create a file with a \cs{nocite} command containing all those labels
in a comma separated list in reverse chronological order, 
but I really can't be bothered to do that.
Instead, I can create the following document:
\begin{verbatim}
\documentclass{article}
\usepackage{databib}
\begin{document}
\nocite{*}
\DTLloadbbl{mybib}{nlct}
\DTLsort{Year=descending,Month=descending}{mybib}
\DTLbibliography[\DTLbibfieldisge{Year}{2000}]{mybib}
\end{document}
\end{verbatim}
Suppose I save this file as "mypubs.tex", then I need to do:
\begin{verbatim}
latex mypubs
bibtex mypubs
latex mypubs
\end{verbatim}
Notes:
\begin{enumerate}
\item "\nocite{*}" is used to add all the citations in the 
bibliography file ("nlct.bib" in this case) to the \sty{databib}
database.

\item "\DTLloadbbl{mybib}{nlct}" does the following:
\begin{enumerate}
\item writes the line
\begin{verbatim}
\bibstyle{databib}
\end{verbatim}
to the auxiliary file. This tells \BibTeX\ to use "databib.bst"
(which is supplied with this package). You therefore shouldn't
use \cs{bibliographystyle}.

\item writes the line
\begin{verbatim}
\bibdata{nlct}
\end{verbatim}
to the auxiliary file. This tells \BibTeX\ that the bibliography
data is stored in the file "nlct.bib". Since I have placed this
file in \TeX's search path, \BibTeX\ will be able to find it.

\item creates a \sty{datatool} database called "mybib".

\item if the \filetype{bbl} file ("mypubs.bbl" in this example) exists,
it loads this file (which adds the bibliography data to the
database), otherwise it does nothing further.

\end{enumerate}

\item In my \BibTeX\ database ("nlct.bib" in this example), I 
have remembered to use the \BibTeX\ month macros: "jan", "feb"
etc. This means that the months are stored in the database in
the form \cs{DTLmonthname}\marg{nn}, where \meta{nn} is a two
digit number from 01 to 12. \cs{DTLsort} ignores command names
when it compares strings, which means I can not only sort by
year, but also by month\footnote{as long as I haven't put anything
before the month name in the bibliography file, e.g.\ 
\mbox{\texttt{month = "2~" \# apr}} will sort by 2~03, instead of
03}.

\item Once I have loaded and sorted my database, I can then
display it using \cs{DTLbibliography}. This uses the style
given by the \sty{databib} \pkgopt{style} package option,
or the \cs{DTLbibliographystyle} command, both of which are
described in \autoref{sec:bibstyle}.

\item I have filtered the bibliography using the optional
argument\linebreak "[\DTLbibfieldisge{Year}{2000}]", which checks if the
year field of the current entry is greater than or equal to
2000. (Note that if an entry has no year field, the condition
evaluates to false, and the entry will be omitted from the
bibliography.)

\item If the bibliography database is large, sorting and creating
the bibliography may take a while. Using \sty{databib} is much
slower than using a standard \BibTeX\ style file.
\end{enumerate}
\end{example}

\begin{example}{Creating a list of my 10 most recent 
publications}{ex:top10bib}
Suppose now my boss has asked me to produce a list of my ten most 
recent publications (in reverse chronological order). 
As in the previous example, I have a file called "nlct.bib"
which contains all my publications. I can create the required
document as follows:
\begin{verbatim}
\documentclass{article}
\usepackage{databib}
\begin{document}
\nocite{*}
\DTLloadbbl{mybib}{nlct}
\DTLsort{Year=descending,Month=descending}{mybib}
\DTLbibliography[\value{DTLbibrow}<10]{mybib}
\end{document}
\end{verbatim}
\end{example}

\section{Changing the bibliography style}
\label{sec:bibstyle}
The style of the bibliography produced using \cs{DTLbibliography} 
depends on the \pkgopt{style}
package option, or can be set using
\begin{definition}[\DescribeMacro{\DTLbibliographystyle}]%
\cs{DTLbibliographystyle}\marg{style}
\end{definition}
Note that this is \emph{not} the same as \cs{bibliographystyle},
as the \sty{databib} package uses its custom \texttt{databib.bst}
bibliography style file.

Example:
\begin{verbatim}
\usepackage[style=plain]{databib}
\end{verbatim}
This sets the plain bibliography style. This is, in fact, the
default style, so it need not be specified.

Available styles are: "plain", "abbrv" and "alpha". These are similar to the
standard \BibTeX\ styles of the same name, but are by no means 
identical. The most notable difference is that these styles
do not sort the bibliography. It is up to you to sort the
bibliography using \cs{DTLsort} (described in \autoref{sec:sort}).

\subsection{Modifying an existing style}
\label{sec:modbibstyle}

This section describes some of the commands which are used to 
format the bibliography. You can choose whichever predefined 
style best fits your required style, and then modify the commands 
described in this section. A description of the remaining
commands not listed in this section can be found in 
\autoref{sec:src:bibnames}, \autoref{sec:src:displaybib}
and \autoref{sec:src:bibstyle}.

\begin{definition}[\DescribeMacro{\DTLformatauthor}]%
\cs{DTLformatauthor}\marg{von part}\marg{surname}\marg{jr part}\marg{forenames}
\end{definition}
\begin{definition}[\DescribeMacro{\DTLformateditor}]%
\cs{DTLformateditor}\marg{von part}\marg{surname}\marg{jr part}\marg{forenames}
\end{definition}
These commands are used to format an author/editor's name,
respectively. The list of authors and editors are stored in
the \sty{databib} database as a comma separated list of
\marg{von part}\marg{surname}\marg{jr part}\marg{forenames}
data. This ensures that when you sort on the Author or Editor
field, the names will be sorted by the first author or editor's
surname.

Within \cs{DTLformatauthor} and \cs{DTLformateditor}, you may
use the following commands:
\begin{definition}[\DescribeMacro{\DTLformatforenames}]%
\cs{DTLformatforenames}\marg{forenames}
\end{definition}
This is used by the "plain" style to display the author's
forenames\footnote{It also checks 
whether \meta{forenames} ends with a full stop using 
\cs{DTLcheckendsperiod} to prevent a sentence ending full stop
from following an abbreviation full stop}.

\begin{definition}[\DescribeMacro{\DTLformatabbrvforenames}]%
\cs{DTLformatabbrvforenames}\marg{forenames}
\end{definition}
This is used by the "abbrv" style to display the author's
initials (which are determined from \meta{forenames}).
Note that if any of the authors has a name starting with an
accent, the accented letter must be grouped in order for this
command to work. For example:
\begin{verbatim}
author = "{\'E}lise {\"E}awyn Edwards",
\end{verbatim}
The initials are formed using \cs{DTLstoreinitials} described
in \autoref{sec:strings}, so if you want to change the way the
initials are displayed (e.g.\ put a space between them) you will
need to redefine the commands used by \cs{DTLstoreinitials} (such 
as \cs{DTLbetweeninitials}).

\begin{definition}[\DescribeMacro{\DTLformatsurname}]%
\cs{DTLformatsurname}\marg{surname}
\end{definition}
This displays its argument by default\footnote{It also checks 
whether the surname ends with a full stop using 
\cs{DTLcheckendsperiod}}.

\begin{definition}[\DescribeMacro{\DTLformatvon}]%
\cs{DTLformatvon}\marg{von part}
\end{definition}
If the \meta{von part} is empty, this command does nothing,
otherwise it displays its argument followed by a
non-breakable space.
\begin{definition}[\DescribeMacro{\DTLformatjr}]%
\cs{DTLformatjr}\marg{jr part}
\end{definition}
If the \meta{jr part} is empty, this command displays nothing, 
otherwise it displays a comma followed by its argument\footnote
{again, it also checks \meta{jr part} to determine if it ends
with a full stop}.

For example, suppose you want the author's surname to appear
first in small capitals, followed by a comma and the forenames. This
can be achieved by redefining \cs{DTLformatauthor} as follows:
\begin{verbatim}
\renewcommand*{\DTLformatauthor}[4]{%
\textsc{\DTLformatvon{#1}%
\DTLformatsurname{#2}\DTLformatjr{#3}},
\DTLformatforenames{#4}%
}
\end{verbatim}

\begin{definition}[\DescribeCounter{DTLmaxauthors}]
\ctrfmt{DTLmaxauthors}
\end{definition}
The counter \ctrfmt{DTLmaxauthors} is used to determine the
maximum number of authors to display for a given entry. If the
entry's author list contains more than that number of authors,
\cs{etalname} is used, the definition of which is given in
\autoref{sec:src:bibnames}. The default value of 
\ctrfmt{DTLmaxauthors} is \theDTLmaxauthors.

\begin{definition}[\DescribeCounter{DTLmaxeditors}]
\ctrfmt{DTLmaxeditors}
\end{definition}
The \ctrfmt{DTLmaxeditors} counter is analogous to the
\ctrfmt{DTLmaxauthors} counter. It is used to determine the
maximum number of editor names to display. The default value
of \ctrfmt{DTLmaxeditors} is \theDTLmaxeditors.

\DescribeMacro{\DTLandlast}
Within a list of author or editor names, \cs{DTLandlast} is
used between the last two names, otherwise 
\DescribeMacro{\DTLandnotlast}\cs{DTLandnotlast} is used between
names.
However, if there are only two author or editor names, 
\DescribeMacro{\DTLtwoand}\cs{DTLtwoand}
is used instead of \cs{DTLandlast}.

\DescribeMacro\DTLbibitem
The command \cs{DTLbibitem} is used at the start of each
bibliography item. It uses \cs{bibitem} to provide a marker, 
such as [1], and writes the citation information to the
\texttt{.aux} file.

\DescribeMacro\DTLmbibitem
The command \cs{DTLmbibitem} is analogous to \cs{DTLbibitem} but is
for use with \cs{DTLmbibliography}.

\DescribeMacro{\DTLendbibitem}
The command \cs{DTLendbibitem} is a hook provided to add 
additional information at the end of each bibliography item.
This does nothing by default, but if you want to display the
additional fields provided by the "databib.bst" style file,
you can redefine \cs{DTLendbibitem} so that it displays a
particular field, if it is defined. Within this command, you
may use the commands \cs{DTLbibfield}, \cs{DTLifbibfieldexist}
and \cs{DTLifanybibfieldexist}, which are described in 
\autoref{sec:foreachbib}. For example, if you have used the
"abstract" field in any of your entries, you can display the
abstract as follows:
\begin{verbatim}
\renewcommand{\DTLendbibitem}{%
\DTLifbibfieldexists{Abstract}{\DTLpar\textbf{Abstract}
\begin{quote}\DTLbibfield{Abstract}\end{quote}}{}}
\end{verbatim}
(Note that \cs{DTLpar} needs to be used instead of
\cs{par}.)

\begin{example}{Compact bibliography}{ex:compactbib}
Suppose I don't have much space in my document, and I need to
produce a compact bibliography. Firstly, I can use the
bibliography style "abbrv", either through the package option:
\begin{verbatim}
\usepackage[style=abbrv]{databib}
\end{verbatim}
or using:
\begin{verbatim}
\DTLbibliographystyle{abbrv}
\end{verbatim}
Once I have set the style, I can further modify it thus:
\begin{verbatim}
\renewcommand*{\andname}{\&}
\renewcommand*{\editorname}{ed.}
\renewcommand*{\editorsname}{eds.}
\renewcommand*{\pagesname}{pp.}
\renewcommand*{\pagename}{p.}
\renewcommand*{\volumename}{vol.}
\renewcommand*{\numbername}{no.}
\renewcommand*{\editionname}{ed.}
\renewcommand*{\techreportname}{T.R.}
\renewcommand*{\mscthesisname}{MSc thesis}
\end{verbatim}
Now I can load\footnote{I can load the bibliography earlier, but
obviously the bibliography should only be displayed after the
bibliography styles have been set, otherwise they will have no
effect} and display the bibliography:
\begin{verbatim}
 % create a database called mybib from the information given
 % in mybib1.bib and mybib2.bib
 \DTLloadbbl{mybib}{mybib1,mybib2}
 % display the bibliography
 \DTLbibliography{mybib}
\end{verbatim}
\end{example}

\begin{example}{Highlighting a given author}{ex:highlightauthor}
Suppose my boss wants me to produce a list of all my publications
(which I have stored in the file "nlct.bib", as in 
\autoref{ex:bibsince2000}). Most of my 
publications have multiple co-authors, but suppose my boss would
like me to highlight my name so that when he skims through
the document, he can easily see my name in the list of 
co-authors. I can do this by redefining \cs{DTLformatauthor}
so that it checks if the given surname matches mine. (This 
assumes that none of the other co-author's share my surname.)
\begin{verbatim}
\renewcommand*{\DTLformatauthor}[4]{%
{\DTLifstringeq{#2}{Talbot}{\bfseries }{}%
\DTLformatforenames{#4}
\DTLformatvon{#1}%
\DTLformatsurname{#2}%
\DTLformatjr{#3}}}
\end{verbatim}
Notes:
\begin{enumerate}
\item I have used \cs{DTLifstringeq} (described in 
\autoref{sec:ifconditions}) to perform the string comparison.
\item If one or more of my co-authors shared the same surname as
me, I would also have had to check the first name, however there
is regrettably a lack of consistency in my \filetype{bib} file when
it comes to my forenames. Sometimes my name is given as
\texttt{Nicola L. C. Talbot}, sometimes the middle initials
are omitted, \texttt{Nicola Talbot}, or sometimes, just initials
are used, \texttt{N. L. C. Talbot}. This can cause problems
when checking the forenames, but as long as the other authors
who share the same surname as me, don't also share the same
first initial, I can use \cs{DTLifStartsWith} or \cs{DTLisPrefix},
which are described in \autoref{sec:ifconditions} and
\autoref{sec:ifthen}, respectively. Using the first approach
I can do:
\begin{verbatim}
\renewcommand*{\DTLformatauthor}[4]{%
{\DTLifstringeq{#2}{Talbot}{\DTLifStartsWith{#4}{N}{\bfseries }{}}{}%
\DTLformatforenames{#4}
\DTLformatvon{#1}%
\DTLformatsurname{#2}%
\DTLformatjr{#3}}}
\end{verbatim}
Using the second approach I can do:
\begin{verbatim}
\renewcommand*{\DTLformatauthor}[4]{%
{\ifthenelse{\DTLiseq{#2}{Talbot}\and
\DTLisPrefix{#4}{N}}{\bfseries }{}%
\DTLformatforenames{#4}
\DTLformatvon{#1}%
\DTLformatsurname{#2}%
\DTLformatjr{#3}}}
\end{verbatim}

\item I have used a group to localise the effect of \cs{bfseries}.
\end{enumerate}
\end{example}
 
\section{Iterating through a 
\texorpdfstring{\sty{databib}}{databib} database}
\label{sec:foreachbib}

\cs{DTLbibliography} (described in \autoref{sec:thebib}) may still
not meet your needs. For example, you may be required
to list journal papers and conference proceedings in separate
sections. In which case, you may find it easier to iterate through
the bibliography using:
\begin{definition}[\DescribeMacro{\DTLforeachbibentry}]%
\cs{DTLforeachbibentry}\oarg{condition}\marg{db name}\marg{text}
\end{definition}
\begin{definition}[\DescribeMacro{\DTLforeachbibentry*}]%
\cs{DTLforeachbibentry*}\oarg{condition}\marg{db name}\marg{text}
\end{definition}
This iterates through the \sty{databib} database called
\meta{db name} and does \meta{text} if \meta{condition} is met.
As with \cs{DTLforeach}, the starred version is read-only.

\cs{DTLforeachbibentry} only makes local assignments, which means that it's unsuitable to display the references
in a~\env{tabular}-like environment (for example, the
\env{europecv} environment provided by the \cls{europecv} class). It's also a~short command, so
\meta{text} can't contain any paragraph breaks. Instead you can use
the analogous commands:
\begin{definition}[\DescribeMacro{\gDTLforeachbibentry}]%
\cs{gDTLforeachbibentry}\oarg{condition}\marg{db name}\marg{text}
\end{definition}
\begin{definition}[\DescribeMacro{\gDTLforeachbibentry*}]%
\cs{gDTLforeachbibentry*}\oarg{condition}\marg{db name}\marg{text}
\end{definition}

For each row of the database, the following commands are set:
\begin{itemize}
\item \cs{DBIBcitekey} \DescribeMacro{\DBIBcitekey}This is the
unique label which identifies the current entry (as used in the
argument of \cs{cite} and \cs{nocite}).

\item \cs{DBIBentrytype} \DescribeMacro{\DBIBentrytype}This
is the current entry type, and will be one of: "article", "book", 
"booklet", "inbook", "incollection", "inproceedings",
"manual", "mastersthesis", "misc", "phdthesis", "proceedings", 
"techreport" or "unpublished". (Note that even if you used the
entry type "conference" in your \filetype{bib} file, its entry type
will be set to "inproceedings").
\end{itemize}

The remaining fields may be accessed using:
\begin{definition}[\DescribeMacro{\DTLbibfield}]%
\cs{DTLbibfield}\marg{field label}
\end{definition}\noindent
where \meta{field label} may be one of: "Address", "Author", 
"BookTitle", "Chapter", "Edition", "Editor", "HowPublished",
"Institution", "Journal", "Key", "Month", "Note", "Number", 
"Organization", "Pages", "Publisher", "School", "Series",
"Title", "Type", "Volume", "Year", "ISBN", "DOI", "PubMed",
"Abstract" or "Url".

Alternatively, you can assign the value of a field to a control
sequence \meta{cs} using:
\begin{definition}[\DescribeMacro\DTLbibfieldlet]
\cs{DTLbibfieldlet}\marg{cs}\marg{field label}
\end{definition}

You can determine if a field exists for a given entry using
\begin{definition}[\DescribeMacro{\DTLifbibfieldexists}]%
\cs{DTLifbibfieldexists}\marg{field label}\marg{true part}\marg{false
part}
\end{definition}
If the field given by \meta{field label} exists for the current
bibliography entry, it does \meta{true part}, otherwise it
does \meta{false part}.

\begin{definition}[\DescribeMacro{\DTLifbibanyfieldexists}]%
\cs{DTLifanybibfieldexists}\marg{field label list}\marg{true 
part}\marg{false part}
\end{definition}
This is similar to \cs{DTLifbibfieldexists} except that the
first argument is a list of field names. If one or more of
the fields given in \meta{field label list} exists for the
current bibliography item, this does \meta{true part}, otherwise
it does \meta{false part}.

\begin{definition}[\DescribeMacro{\DTLformatbibentry}]%
\cs{DTLformatbibentry}
\end{definition}
This formats the bibliography entry for the current row. It
checks for the existence of the command 
\cs{DTLformat}\meta{entry type}, where \meta{entry type}
is given by \cs{DBIBentrytype}. These commands are defined
by the bibliography style. There is also a~version for use with
\cs{gDTLforeachbibentry}:
\begin{definition}[\DescribeMacro{\gDTLformatbibentry}]%
\cs{gDTLformatbibentry}
\end{definition}
It's also possible to use \cs{DTLformatbibentry} for a specific key,
rather than using it within \ics{DTLforeachbibentry} using:
\begin{definition}
\cs{DTLformatthisbibentry}\marg{db}\marg{cite key}
\end{definition}
where \meta{db} is the database name and \meta{cite key} is the
citation label. Note that none of these three commands use
\cs{bibitem}. You can manually insert \cs{bibitem}\marg{cite key}
in front of the command, or you can use:
\begin{definition}
\cs{DTLcustombibitem}\marg{marker code}\marg{ref text}\marg{cite key}
\end{definition} 
This is like \cs{bibitem}\oarg{text}\marg{cite key} except that it uses
\meta{marker code} instead of \cs{item}\oarg{text} and it uses
\meta{ref text} instead of \verb|\the\value{\@listctr}|.

\begin{definition}[\DescribeMacro{\DTLcomputewidestbibentry}]%
\cs{DTLcomputewidestbibentry}\marg{conditions}\marg{db 
name}\marg{bib label}\marg{cmd}
\end{definition}
This computes the widest bibliography entry over all entries
satisfying \meta{conditions} in the database \meta{db name},
where the label is given by \meta{bib label}, and the result
is stored in \meta{cmd}, which may then be used in the
argument of the \env{thebibliography} environment.

The counter \desctr{DTLbibrow} keeps track of the current
bibliography entry. This is reset at the start of each
\cs{DTLforeachbibentry} and is incremented if \meta{conditions}
is met.

Within the optional argument \meta{condition}, you may use any of the 
commands that may be used within the optional argument of
\cs{DTLbibliography}, described in \autoref{sec:thebib}.

\begin{example}{Separate List of Journals and Conference Papers}{ex:jcbib}
Suppose now my boss has decided that I need to produce a list
of all my publications, but they need to be separated so that
all the journal papers appear in one section, and all the
conference papers appear in another section. The journal papers
need to be labelled [J1], [J2] and so on, while the conference
papers need to be labelled [C1], [C2] and so on. (My boss isn't
interested in any of my other publications!) Again, all my
publications are stored in the \BibTeX\ database "nlct.bib". The
following creates the required document:
\begin{verbatim}
\documentclass{article}
\usepackage{databib}
\begin{document}
\nocite{*}
\DTLloadbbl{mybib}{nlct}

\renewcommand*{\refname}{Journal Papers}
\DTLcomputewidestbibentry{\equal{\DBIBentrytype}{article}}
{mybib}{J\theDTLbibrow}{\widest}

\begin{thebibliography}{\widest}
\DTLforeachbibentry[\equal{\DBIBentrytype}{article}]{mybib}{%
\bibitem[J\theDTLbibrow]{\DBIBcitekey} \DTLformatbibentry}
\end{thebibliography}

\renewcommand*{\refname}{Conference Papers}
\DTLcomputewidestbibentry{\equal{\DBIBentrytype}{inproceedings}}
{mybib}{C\theDTLbibrow}{\widest}

\begin{thebibliography}{\widest}
\DTLforeachbibentry[\equal{\DBIBentrytype}{inproceedings}]{mybib}{%
\bibitem[C\theDTLbibrow]{\DBIBcitekey} \DTLformatbibentry}
\end{thebibliography}

\end{document}
\end{verbatim}
\end{example}

\section{Multiple Bibliographies}
\label{sec:multibib}

It is possible to have more than one bibliography in a document,
but it then becomes necessary to have a separate auxiliary file
for each bibliography, and each auxiliary file must then be
passed to \BibTeX. In order to do this, you need to use
\begin{definition}[\DescribeMacro{\DTLmultibibs}]%
\cs{DTLmultibibs}\marg{name list}
\end{definition}
where \meta{name list} is a comma separated list of names, 
\meta{name}. For each \meta{name}, this command creates an
auxiliary file called \meta{name}".aux" (note that this
command may only be used in the preamble).

When you want to cite an entry for a given bibliography named
in \cs{DTLmultibibs}, you must use:
\begin{definition}[\DescribeMacro{\DTLcite}]%
\cs{DTLcite}\oarg{text}\marg{mbib}\marg{cite key list}
\end{definition}\noindent
This is analogous to \cs{cite}\oarg{text}\marg{cite key list}, 
but writes the \cs{citation} command to \meta{mbib}".aux" instead
of to the document's main auxiliary file. It also ensures that
the cross-referencing labels are based on \meta{mbib}, to allow
you to have the same reference in more than one bibliography
without incurring a ``multiply defined'' warning message. Note
that you can still use \cs{cite} to add citation information to
the main auxiliary file.

If you want to add an entry to the bibliography without producing
any text, you can use
\begin{definition}[\DescribeMacro{\DTLnocite}]
\cs{DTLnocite}\marg{mbib}\marg{cite key list}
\end{definition}\noindent
which is analogous to \cs{nocite}\marg{cite key list}, where
again the citation information is written to \meta{mbib}".aux"
instead of the document's main auxiliary file.

Note that for both \cs{DTLcite} and \cs{DTLnocite} the
\meta{mbib} part must be one of the names listed in 
\cs{DTLmultibibs}.

\begin{definition}[\DescribeMacro{\DTLloadmbbl}]%
\cs{DTLloadmbbl}\marg{mbib}\marg{db name}\marg{bib list}
\end{definition}
This is analogous to \cs{DTLloadbbl}\marg{db name}\marg{bib list}
described in \autoref{sec:loadbbl}. (Again \meta{mbib} must be
one of the names listed in \cs{DTLmultibibs}.) This creates
a new \sty{datatool} database called \meta{db name} and loads the
bibliography information from \meta{mbib}".bbl" (if it exists).

\begin{definition}[\DescribeMacro{\DTLmbibliography}]%
\cs{DTLmbibliography}\oarg{condition}\marg{mbib}\marg{db name}
\end{definition}
This is analogous to \cs{DTLbibliography}\oarg{condition}\marg{db name},
but is required when displaying a bibliography in which elements have
been cited using \cs{DTLcite} and \cs{DTLnocite}. 

\begin{example}{Multiple Bibliographies}{ex:multibib}
Suppose I need to create a document which contains a section listing
all my publications, but I also need to have separate sections
covering each of my research topics, with a mini-bibliography
at the end of each section. As in the earlier examples, all my
publications are stored in the file "nlct.bib" which is somewhere
on \TeX's path. Note that there will be some duplication as the
references in the mini-bibliographies will also appear in the
main bibliography at the end of the document, but using
\cs{DTLcite} and \cs{DTLmbibliography} ensures that all the
cross-referencing labels (and hyperlinks if they are enabled)
are unique. 
\begin{verbatim}
\documentclass{article}
\usepackage{databib}
\DTLmultibibs{kernel,food}
\begin{document}
\section{Kernel methods}
In this section I'm going to describe some research work into
kernel methods, and in the process I'm going to cite some related
papers \DTLcite{kernel}{Cawley2007a,Cawley2006a}.

\DTLloadmbbl{kernel}{kernelDB}{nlct}
\DTLmbibliography{kernel}{kernelDB}

\section{Food research}

In this section I'm going to describe some research work
in the area of food safety, and in the process, I'm going
to cite some related papers \DTLcite{food}{Peck1999,Barker1999a}

\DTLloadmbbl{food}{foodDB}{nlct}
\DTLmbibliography{food}{foodDB}

\cite{*}
\renewcommand{\refname}{Complete List of Publications}
\DTLloadbbl{fullDB}{nlct}
\DTLbibliography{fullDB}
\end{document}
\end{verbatim}
Notes:
\begin{enumerate}
\item This will create the files "kernel.aux" and "food.aux".
These will have to be passed to \BibTeX, in addition to the
documents main auxiliary file. So, if my document is called
"researchwork.tex", then I need to do:
\begin{verbatim}
latex researchwork
bibtex researchwork
bibtex kernel
bibtex food
latex researchwork
latex researchwork
\end{verbatim}

\item "\cite{*}" is used to add all the entries in the bib file
to the main bibliography database. As before, \cs{DTLloadbbl}
and \cs{DTLbibliography} are used to load and display the main
bibliography.
\end{enumerate}
\end{example}

\begin{important}
Don't try to directly input the ".bbl" file using \cs{input} (or
\cs{include}) instead of using \cs{DTLloadbbl} or \cs{DTLloadmbbl}
as these commands store the name of the required database and
initialise the database before loading the \texttt{.bbl} file.
Similarly, don't just copy the contents of the ".bbl" file into your
document without first defining the database using \ics{DTLnewdb}
and setting \ics{DTLBIBdbname} to the name of the database.
\end{important}

\chapter{Referencing People (\sty{person} package)}
\label{sec:person}

Sometimes when mail-merging, it may be necessary to reference a
person by their pronoun which can lead to the cumbersome and
impersonal ``he/she'' construct. The \sty{person} package 
allows you to define a person by their full name, familiar name and
gender. You can then use the commands described in
\autoref{sec:refperson} to produce the appropriate pronoun.

This can also be useful for other types of documents, such as an
order of service for a baptism or funeral. Since the
document is much the same from one person to the next, documents of
this nature are frequently simply copied and a search and replace
edit is used to change the relevant text.  However this can lead to
errors (especially if the previous person's name was Mary!) With
the \sty{person} package, you need only change the definition of
the person by modifying the arguments of \cs{newperson}.

\section{Defining and Undefining People}

A person is defined (globally) using the command:
\begin{definition}[\DescribeMacro{\newperson}]
\cs{newperson}\oarg{label}\marg{full name}\marg{familiar name}\marg{gender}
\end{definition}
The optional argument is a unique label identifying this person,
in the event that there is more than one person. If \meta{label}
is omitted \texttt{anon} is used. (This is also the case for 
subsequent commands that take an optional label.) 
The gender may be any of those given by
\begin{definition}[\DescribeMacro{\malelabels}]
\cs{malelabels}
\end{definition}
or 
\begin{definition}[\DescribeMacro{\femalelabels}]
\cs{femalelabels}
\end{definition}
The default definition of \cs{malelabels} is \texttt{\malelabels}
and the default definition of \cs{femalelabels} is
\texttt{\femalelabels}. You can add extra identifiers using
\begin{definition}[\DescribeMacro{\addmalelabel}]
\cs{addmalelabel}\marg{identifier}
\end{definition}
or
\begin{definition}[\DescribeMacro{\addfemalelabel}]
\cs{addfemalelabel}\marg{identifier}
\end{definition}
For example:
\begin{verbatim}
\addmalelabel{boy}
\addfemalelabel{girl}
\end{verbatim}

The total number of defined people is given by:
\begin{definition}[\DescribeMacro{\thepeople}]
\cs{thepeople}
\end{definition}

A person can be undefined using:
\begin{definition}[\DescribeMacro{\removeperson}]
\cs{removeperson}\oarg{label}
\end{definition}
where the person is given by \meta{label}.

If more than one person has been defined, they can all be 
removed using:
\begin{definition}[\DescribeMacro{\removeallpeople}]
\cs{removeallpeople}
\end{definition}
or you can remove a subset using:
\begin{definition}[\DescribeMacro{\removepeople}]
\cs{removepeople}\marg{list}
\end{definition}
where \meta{list} is a comma-separated list of labels.

\section{Displaying Information}
\label{sec:refperson}

Once a person has been defined, you can display their name using:
\begin{definition}[\DescribeMacro{\personfullname}]
\cs{personfullname}\oarg{label}
\end{definition}
where \meta{label} is the unique label used in the optional
argument to \cs{newperson}. The person's familiar name is displayed
using:
\begin{definition}[\DescribeMacro{\personname}]
\cs{personname}\oarg{label}
\end{definition}
The person's pronoun (``he'' or ``she'') is displayed using:
\begin{definition}[\DescribeMacro{\personpronoun}]
\cs{personpronoun}\oarg{label}
\end{definition}
The objective pronoun (``him'' or ``her'') is displayed using:
\begin{definition}[\DescribeMacro{\personobjpronoun}]
\cs{personobjpronoun}\oarg{label}
\end{definition}
The possessive adjective (``his'' or ``her'') is displayed using:
\begin{definition}[\DescribeMacro{\personpossadj}]
\cs{personpossadj}\oarg{label}
\end{definition}
The possessive pronoun ``his'' or ``hers'' is displayed using:
\begin{definition}[\DescribeMacro{\personposspronoun}]
\cs{personposspronoun}\oarg{label}
\end{definition}
The person's relationship to their parent (``son'' or ``daughter'')
is displayed using:
\begin{definition}[\DescribeMacro{\personchild}]
\cs{personchild}\oarg{label}
\end{definition}
The person's relationship to their child (``mother'' or ``father'')
is displayed using:
\begin{definition}[\DescribeMacro{\personparent}]
\cs{personparent}\oarg{label}
\end{definition}
The person's relationship to their sibling (``brother'' or 
``sister'') is displayed using:
\begin{definition}[\DescribeMacro{\personsibling}]
\cs{personsibling}\oarg{label}
\end{definition}

If the word occurs at the start of a sentence, you will need one
of the following commands, which are as the above, except the
first letter is converted to upper case:
\begin{definition}[\DescribeMacro{\Personpronoun}]
\cs{Personpronoun}\oarg{label}
\end{definition}
\begin{definition}[\DescribeMacro{\Personobjpronoun}]
\cs{Personobjpronoun}\oarg{label}
\end{definition}
\begin{definition}[\DescribeMacro{\Personpossadj}]
\cs{Personpossadj}\oarg{label}
\end{definition}
\begin{definition}[\DescribeMacro{\Personposspronoun}]
\cs{Personposspronoun}\oarg{label}
\end{definition}
\begin{definition}[\DescribeMacro{\Personchild}]
\cs{Personchild}\oarg{label}
\end{definition}
\begin{definition}[\DescribeMacro{\Personparent}]
\cs{Personparent}\oarg{label}
\end{definition}
\begin{definition}[\DescribeMacro{\Personsibling}]
\cs{Personsibling}\oarg{label}
\end{definition}

\begin{example}{Order of Service (Memorial)}{ex:memorial}
This example is for a memorial order of service.

\begin{verbatim}
\documentclass{article}

\usepackage{person}

\newperson{Jane Doe}{Jane}{female}

\begin{document}
\begin{center}
\Large
In Memory of \personfullname
\end{center}

We are gathered here to remember our \personsibling\ \personname.
\Personpronoun\ will be much missed, and \personpossadj\
family are in our prayers.
\end{document}
\end{verbatim}

\newperson{Jane Doe}{Jane}{female}
\begin{center}
\Large
In Memory of \personfullname
\end{center}

We are gathered here to remember our \personsibling\ \personname.
\Personpronoun\ will be much missed, and \personpossadj\
family are in our prayers.
\removeperson

\end{example}

If there is more than one person, you will need to use the 
optional argument \meta{label} to \cs{newperson} to uniquely
identify each person. You can then list all of the people's full
or familiar names using:
\begin{definition}[\DescribeMacro{\peoplefullname}]
\cs{peoplefullname}
\end{definition}
\begin{definition}[\DescribeMacro{\peoplename}]
\cs{peoplename}
\end{definition}
Note that if there is only one person defined, these commands behave
the same as \cs{personfullname}\oarg{label} and
\cs{personname}[\meta{label}].

Similarly for the pronouns:
\begin{definition}[\DescribeMacro{\peoplepronoun}]
\cs{peoplepronoun}
\end{definition}
\begin{definition}[\DescribeMacro{\Peoplepronoun}]
\cs{Peoplepronoun}
\end{definition}
\begin{definition}[\DescribeMacro{\peopleobjpronoun}]
\cs{peopleobjpronoun}
\end{definition}
\begin{definition}[\DescribeMacro{\Peopleobjpronoun}]
\cs{Peopleobjpronoun}
\end{definition}
\begin{definition}[\DescribeMacro{\peoplepossadj}]
\cs{peoplepossadj}
\end{definition}
\begin{definition}[\DescribeMacro{\Peoplepossadj}]
\cs{Peoplepossadj}
\end{definition}
\begin{definition}[\DescribeMacro{\peopleposspronoun}]
\cs{peopleposspronoun}
\end{definition}
\begin{definition}[\DescribeMacro{\Peopleposspronoun}]
\cs{Peopleposspronoun}
\end{definition}
where, again, if only one person has been defined, each of these
commands is equivalent to 
\cs{person}\ldots\oarg{label} or
\cs{Person}\ldots\oarg{label}. If more than 
one person has been defined, these commands will display
they/them/their/theirs or They/Them/Their/Theirs, as appropriate.

Likewise for relationship commands:
\begin{definition}[\DescribeMacro{\peoplechild}]
\cs{peoplechild}
\end{definition}
\begin{definition}[\DescribeMacro{\Peoplechild}]
\cs{Peoplechild}
\end{definition}
\begin{definition}[\DescribeMacro{\peopleparent}]
\cs{peopleparent}
\end{definition}
\begin{definition}[\DescribeMacro{\Peopleparent}]
\cs{Peopleparent}
\end{definition}
\begin{definition}[\DescribeMacro{\peoplesibling}]
\cs{peoplesibling}
\end{definition}
\begin{definition}[\DescribeMacro{\Peoplesibling}]
\cs{Peoplesibling}
\end{definition}

\begin{example}{Order of Service (Baptism)}{ex:baptism}
In this example two people are defined.
\begin{verbatim}
\documentclass{article}

\usepackage{person}

\newperson[john]{John Joseph}{John}{male}
\newperson[jane]{Jane Mary}{Jane}{female}

\begin{document}
\begin{center}
\Large
Baptism of \peoplefullname.
\end{center}

Today we welcome \peoplename\ into God's family, may He guide
and protect \peopleobjpronoun.
\end{document}
\end{verbatim}

This is produces the following text:

\newperson[john]{John Joseph}{John}{male}
\newperson[jane]{Jane Mary}{Jane}{female}

\begin{center}
\Large
Baptism of \peoplefullname.
\end{center}

Today we welcome \peoplename\ into God's family, may He guide
and protect \peopleobjpronoun.
\removeallpeople
\end{example}

\begin{example}{Mail Merging Using Appropriate Gender}{ex:personmerge}
In this example I have a CSV file called \texttt{students.csv} 
containing the following:
\begin{verbatim}
FirstName,Surname,Gender,Parent,Address
John,"Smith, Jr",M,Mr and Mrs Smith,1 The Street\\Newtown
Jane,Brown,F,Ms Brown,2 The Avenue\\Oldtown
Andy,Brown,male,Mr Brown and Miss Sepia,3 The Road\\Newtown
Z\"oe,Adams,f,Mr and Mrs Adams,5 The Street\\Newtown
Roger,Brady,m,Mrs Brady,6 The Avenue\\Oldtowm
Clare,Vernon,female,Mr Vernon,7 The Close\\Anytown
\end{verbatim}
Suppose I have to write to each student's parents regarding their
child. I can load the information using \ics{DTLloaddb} (described
in \autoref{sec:loaddb}). I can then iterate through the database
and define the student as a person and use the commands defined in
the \sty{person} package to display the correct gender related text.
I could give each person a unique label based on the row count
(\ics{DTLcurrentindex}), but since I don't need to reuse the 
information, I can use the default "anon" label and undefine the
person when no longer required.

Note that in the CSV file, the gender label isn't consistent.
For some students the gender is identified by a single letter 
(``m'' or ``f'') and for others the gender is identified by a
complete word (``male'' or ``female''). There's also no regard
for case. This doesn't matter to \ics{newperson} as all the
identifiers used are listed in \ics{malelabels} and 
\ics{femalelabels}.

The following is an example letter sent to all parents:
\begin{verbatim}
\documentclass{letter}
\usepackage{person}

% load student information from file "students.csv"
\DTLloaddb{students}{students.csv}
\begin{document}
% Iterate through the student database:
\DTLforeach{students}{\FirstName=FirstName,\Surname=Surname,%
\Gender=Gender,\Parent=Parent,\Address=Address}{%
% Define "anon":
  \newperson{\FirstName\space\Surname}{\FirstName}{\Gender}%
% Do the letter:
  \begin{letter}{\Parent\\\Address}
  \opening{Dear \Parent}
    Your \personchild\ \personname\ has been awarded a
    place. We look forward to seeing \personobjpronoun\
    on \personpossadj\ arrival.
  \closing{Yours Sincerely}
  \end{letter}
% Undefine "anon":
  \removeperson
}
\end{document}
\end{verbatim}

The body of the first letter appears as follows:
\newperson{John Smith Jr}{John}{m}
\par\vskip\baselineskip

    Your \personchild\ \personname\ has been awarded a
    place. We look forward to seeing \personobjpronoun\
    on \personpossadj\ arrival.

\removeperson
\par\vskip\baselineskip
Whereas the body of the second letter appears as follows:
\newperson{Jane Brown}{Jane}{f}
\par\vskip\baselineskip

    Your \personchild\ \personname\ has been awarded a
    place. We look forward to seeing \personobjpronoun\
    on \personpossadj\ arrival.

\removeperson
\end{example}

\section{Advanced Commands}

This section describes additional commands provided by the
\sty{person} package. More detail can be found in 
the documented code (datatool-code.pdf).

\subsection{Conditionals}

\begin{definition}[\DescribeMacro{\ifpersonexists}]
\cs{ifpersonexists}\marg{label}\marg{true part}\marg{false part}
\end{definition}
Tests if the person identified by \meta{label} has been defined.
If true, do \meta{true part} otherwise do \meta{false part}.

\begin{definition}[\DescribeMacro{\ifmale}]
\cs{ifmale}\marg{label}\marg{true part}\marg{false part}
\end{definition}
Test if the person identified by \meta{label} is male.
If true, do \meta{true part} otherwise do \meta{false part}.

\begin{definition}[\DescribeMacro{\iffemale}]
\cs{iffemale}\marg{label}\marg{true part}\marg{false part}
\end{definition}
Test if the person identified by \meta{label} is female.
If true, do \meta{true part} otherwise do \meta{false part}.

\begin{definition}[\DescribeMacro{\ifallmale}]
\cs{ifallmale}\oarg{label list}\marg{true part}\marg{false part}
\end{definition}
Tests if all the people listed in \meta{label list} are male.
If true, do \meta{true part} otherwise do \meta{false part}.
If \meta{label list} is omitted, applied to all defined people.
\begin{definition}[\DescribeMacro{\ifallfemale}]
\cs{ifallfemale}\oarg{label list}\marg{true part}\marg{false part}
\end{definition}
Likewise to test if all the people tested are female.

To determine if a string is an allowed male label:
\begin{definition}[\DescribeMacro{\ifmalelabel}]
\cs{ifmalelabel}\marg{identifier}\marg{true part}\marg{false part}
\end{definition}
where \meta{identifier} is the string to be tested. If true, do
\meta{true part} otherwise do \meta{false part}.
For example:
\begin{verbatim}
\def\gender{M}
\ifmalelabel{\gender}{male}{not male}
\end{verbatim}
Similarly to for an allowed female label:
\begin{definition}[\DescribeMacro{\iffemalelabel}]
\cs{iffemalelabel}\marg{identifier}\marg{true part}\marg{false part}
\end{definition}
For example:
\begin{verbatim}
\ifmalelabel{\gender}{Male}{%
  \iffemalelabel{\gender}{Female}%
   {Undefined Gender}%
}
\end{verbatim}

\subsection{Iterating Through Defined People}

You can iterate through all defined people using:
\begin{definition}[\DescribeMacro{\foreachperson}]
\cs{foreachperson}(\meta{name cs},\meta{full name cs},\meta{gender cs},\meta{label cs})\cs{do}\marg{body}
\end{definition}
At each iteration, \meta{name cs}, \meta{full name cs}, 
\meta{gender cs} and \meta{label cs} are set to the current person's
name, full name, gender and label, respectively. (These arguments
must all be command names.) Note that the gender is set to 
the definition of \DescribeMacro{\malename}\cs{malename} or
\DescribeMacro{\femalename}\cs{femalename}, as appropriate.\footnote{Predefined names provided by the \sty{person} package are described
in the documented code (datatool-code.pdf).} Once these
commands are set, \meta{body} is applied.

If you only want to iterate through a subset of defined people,
you can use:
\begin{definition}
\cs{foreachperson}(\meta{name cs},\meta{full name cs},\meta{gender cs},\meta{label cs})\cs{in}\marg{list}\cs{do}\marg{body}
\end{definition}
where \meta{list} is a comma-separated list of labels.

\subsection{Accessing Individual Information}

\begin{definition}[\DescribeMacro{\getpersongender}]
\cs{getpersongender}\marg{cs}\marg{label}
\end{definition}
Gets the gender of the person identified by \meta{label} and
stores in \meta{cs} (which must be a command name). This
sets \meta{cs} to the definition of \ics{malename} or 
\ics{femalename} as appropriate.

\begin{definition}[\DescribeMacro{\getpersonname}]
\cs{getpersonname}\marg{cs}\marg{label}
\end{definition}
Gets the name of the person identified by \meta{label} and
stores in \meta{cs} (which must be a command name).

\begin{definition}[\DescribeMacro{\getpersonfullname}]
\cs{getpersonfullname}\marg{cs}\marg{label}
\end{definition}
Gets the full name of the person identified by \meta{label} and
stores in \meta{cs} (which must be a command name).

\clearpage
\phantomsection
\addcontentsline{toc}{chapter}{\bibname}
\DTLbibliography{docbib}

\clearpage
\phantomsection
\addcontentsline{toc}{chapter}{Acknowledgements}
\chapter*{Acknowledgements}

Many thanks to Morten~H\o gholm for providing a much more
efficient way of storing the information in databases which
has significantly improved the time it takes to \LaTeX\ documents
containing large databases.

\clearpage
\phantomsection
\addcontentsline{toc}{chapter}{\indexname}
\PrintIndex

\end{document}
