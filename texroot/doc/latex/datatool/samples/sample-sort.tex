% This file is public domain

\documentclass{article}

\usepackage[margin=1in]{geometry}

\usepackage[T1]{fontenc}
\usepackage[utf8]{inputenc}
\usepackage{datatool}

% Sample database

\DTLnewdb{data}

% Add a word to the database. (The Definition column is provided to
% distinguish the entries when the database is displayed. The
% Definition column is not used by the sort code.)

% Syntax: \newword{string}

\makeatletter
\newcommand*{\newword}[1]{%
  \DTLnewrow{data}%
  \DTLnewdbentry{data}{Word}{#1}%
% Convert word to its definition and add
  {%
    \def\thisword{#1}%
    \@onelevel@sanitize\thisword
    \dtlexpandnewvalue % ensure \thisword gets expanded
    \DTLnewdbentry{data}{Definition}{\thisword}%
  }%
}
\makeatother

\newword{High Water}
\newword{!}
\newword{[}
\newword{~}
\newword{\#define}
\newword{High water}
\newword{highwater}
\newword{1st Street}
\newword{high water}
\newword{1st~Street}
\newword{10~Downing Street}
\newword{10~Downing Avenue}
\newword{10 Downing Road}
\newword{\#include}
\newword{1 (one)}
\newword{1\datatoolparenstart(one)}
\newword{10\datatoolparenstart(ten)}
\newword{10 (ten)}
\newword{42\datatoolparenstart(forty-two)}
\newword{42 (forty-two)}
\newword{100\datatoolparenstart(one hundred)}
\newword{100 (one hundred)}
\newword{4\datatoolparenstart(four)}
\newword{4 (four)}
\newword{The Ten Samurai}
\newword{London, Jack}
\newword{London}
\newword{10 Samurai, The}
\newword{10 Samurai}
\newword{Ten Samurai, The}
\newword{Ten Samurai}
\newword{10 Samurai\datatoolsubjectcomma The}
\newword{Ten Samurai\datatoolsubjectcomma The}
\newword{London\datatoolpersoncomma Jack}
\newword{London\datatoolplacecomma UK}
\newword{London\datatoolsubjectcomma History of}
\newword{Zebra}
\newword{zebra}
\newword{alpha}
\newword{Alpha}
\newword{seal}
\newword{sea lion}
\newword{elephant}
\newword{Élite}

\begin{document}
This is a sample document testing datatool's comparison
handlers. Note that UTF-8 characters are treated as two tokens
(the first and second octets) with regular (pdf)\LaTeX, so the
definition column on the following pages shows the two octets
for the UTF-8 character É as ÃĽ, which is how \LaTeX\ sees them. XeLaTeX will treat it as a single
token.

\newpage

\section{Unsorted data}

\begin{tabular}{ll}
\bfseries Word & \bfseries Definition
\DTLforeach*{data}{\Word=Word,\Definition=Definition}{%
\\\Word & \ttfamily \Definition
}
\end{tabular}

\section{Case-Insensitive Sort}

\DTLsort*{Word}{data}

\begin{tabular}{ll}
\bfseries Word & \bfseries Definition
\DTLforeach*{data}{\Word=Word,\Definition=Definition}{%
\\\Word & \ttfamily \Definition
}
\end{tabular}


\section{Case-Sensitive Sort}

\DTLsort{Word}{data}

\begin{tabular}{ll}
\bfseries Word & \bfseries Definition
\DTLforeach*{data}{\Word=Word,\Definition=Definition}{%
\\\Word & \ttfamily \Definition
}
\end{tabular}

\section{Word-Order Sort}

\dtlsort{Word}{data}{\dtlwordindexcompare}

\begin{tabular}{ll}
\bfseries Word & \bfseries Definition
\DTLforeach*{data}{\Word=Word,\Definition=Definition}{%
\\\Word & \ttfamily \Definition
}
\end{tabular}

\end{document}
