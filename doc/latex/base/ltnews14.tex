% \iffalse meta-comment
%
% Copyright (C) 1993-2020
% The LaTeX3 Project and any individual authors listed elsewhere
% in this file.
%
% This file is part of the LaTeX base system.
% -------------------------------------------
%
% It may be distributed and/or modified under the
% conditions of the LaTeX Project Public License, either version 1.3c
% of this license or (at your option) any later version.
% The latest version of this license is in
%    http://www.latex-project.org/lppl.txt
% and version 1.3c or later is part of all distributions of LaTeX
% version 2008 or later.
%
% This file has the LPPL maintenance status "maintained".
%
% The list of all files belonging to the LaTeX base distribution is
% given in the file `manifest.txt'. See also `legal.txt' for additional
% information.
%
% The list of derived (unpacked) files belonging to the distribution
% and covered by LPPL is defined by the unpacking scripts (with
% extension .ins) which are part of the distribution.
%
% \fi
% Filename: ltnews14.tex
%
% This is issue 14 of LaTeX News.

\documentclass
%    [lw35fonts]     % uncomment this line to get Palatino
   {ltnews}[2001/07/12]

% \usepackage[T1]{fontenc}

\publicationmonth{June}
\publicationyear{2001}
\publicationissue{14}


\begin{document}

\maketitle

\raisefirstsection
\section{Future releases}

We are currently exploring how to best support the very large
community of individuals, organisations and enterprises that depend on
the robustness and availability of the current standard \LaTeX{}
distribution.  The results of this may lead to some changes in the
regular release schedule and the handling of bug reports during the
next year.

\section{New release of \textsf{Babel} (required)}

Earlier this year a new release of \textsf{Babel} (3.7) became
available. You can read about its new features in
\begin{latexonly}
  \file{http://www.ctan.org/tex-archive/macros/}\\
  \hspace*{4em}\file{latex/required/babel/announce.txt}
\end{latexonly}
\begin{htmlonly}
  \url{http://www.ctan.org/tex-archive/macros/latex/required/babel/announce.txt}
\end{htmlonly}

One of the bugs that got fixed in this release deals with how labels
are handled by \LaTeX{}.  Because this part of the kernel is modified
by \textsf{babel}, the relevant changes need to be coordinated.
Therefore to use \textsf{Babel} with this release of \LaTeX{} you will
need to update your version of \textsf{babel} to at least 3.7.

\section{New input encoding \package{latin9}}

The package \package{inputenc} has, thanks to Karsten Tinnefeld, been
extended to cover the \package{latin9} input encoding.  The
ISO-Latin~9 encoding is a useful modern replacement for ISO-Latin~1
that contains a few characters needed for French and Finnish. Of wider
interest, it also contains the euro currency sign; this could be the
killer argument for many 8-bit texts to use Latin-9 in the future.

According to a Linux manpage, ISO~Latin-9 supports Albanian, Basque,
Breton, Catalan, Danish, Dutch, English, Estonian, Faroese, Finnish,
French, Frisian, Galician, German, Greenlandic, Icelandic, Irish
Gaelic, Italian, Latin, Luxembourgish, Norwegian, Portuguese,
Rhaeto-Romanic, Scottish Gaelic, Spanish and Swedish.\\
The characters added in \package{latin9} are (in \LaTeX{} notation):\\
\begin{small}
\verb| \texteuro  \v S  \v s  \v Z  \v z  \OE  \oe  \" Y |
\end{small}\\
They displace the following characters from \package{latin1}:\\
\begin{small}
\verb| \textcurrency  \textbrokenbar  \"{}   \'{}   \c{} |\\
\verb| \textonequarter  \textonehalf  \textthreequarters |
\end{small}


\section{New tools}

The new package \package{trace} provides many commands to control
\LaTeX{}'s tracing and debugging output, including the excellent new
information available with \eTeX{} such as the extremely useful
tracing of local assignments.  You will find it in the tools
distribution.

It offers the command \verb|\traceon|, which is similar to
\verb|\tracingall| but suppresses uninteresting stuff such as font
loading by NFSS (which can go on for pages if you are unlucky).  It
also offers \verb|\traceoff| to \ldots\ guess what!  Full details are
in the documented source file, \file{trace.dtx}.

In the base \package{ifthen} package we have added
the uppercase synonyms \verb|\NOT| \verb|\AND| and \verb|\OR|.

\section{New experimental code}

In \textit{\LaTeX{} News~12} we announced some ongoing work towards a
`Designer Interface for \LaTeX' and we presented some early results
thereof.  Since then, at Gutenberg\,2000 in Toulouse and TUG\,2000 in
Oxford, we described a new output routine and an improved method of
handling vertical mode material between paragraphs.  In combination
these support higher quality \emph{automated}\footnote
                             {The stress here is on automated!}
page-breaking and page make-up\latex{\\}
for complex pages---the best yet achieved with \TeX{}!

More recently we have added material to handle the complex front
matter requirements of journal articles; this was presented at
Gutenberg\,2001 in Metz.

A paper describing the new output routine is
\begin{latexonly}
  at\\
\begin{small}
  \file{http://www.latex-project.org/papers/xo-pfloat.pdf}\\
\end{small}
\end{latexonly}
\begin{htmlonly}
  at \url{http://www.latex-project.org/papers/xo-pfloat.pdf}.
\end{htmlonly}
All code examples and documentation are available
\begin{latexonly}
  at\\
\begin{small}
   \file{http://www.latex-project.org/code/experimental}
\end{small}
\end{latexonly}
\begin{htmlonly}
  at \url{http://www.latex-project.org/code/experimental/}.
\end{htmlonly}

This directory has been extended to contain the following.
\begin{description}
 \item[galley] Prototype implementation of the interface\latex{\\}
   for manipulating vertical material in galleys.
 \item[xinitials] Prototype implementation of the interface\latex{\\}
   for paragraph initials (needs the \texttt{galley} package).
 \item[xtheorem] Contributed example using the \texttt{template}
   package to provide a designer interface for theorem environments.
 \item[xor] A prototype implementation of the new output routine
   as described in the \texttt{xo-pfloat.pdf} paper.
 \item[xfrontm] A prototype version of
   the new font matter interface.
\end{description}

\end{document}
