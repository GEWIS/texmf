% Copyright 2006 by Till Tantau
%
% This file may be distributed and/or modified
%
% 1. under the LaTeX Project Public License and/or
% 2. under the GNU Free Documentation License.
%
% See the file doc/generic/pgf/licenses/LICENSE for more details.


\section{Paper-Folding Diagrams Library}
\label{section-calender-folding}

\begin{tikzlibrary}{folding}
    This library defines pic types for creating paper-folding diagrams. Many
    thanks to Nico van Cleemput for providing most of the code.
\end{tikzlibrary}

Here is a big example that produces a diagram for a calendar:
%
\begin{codeexample}[
    leave comments,
    preamble={\usetikzlibrary{calendar,folding}},
]
\sffamily\scriptsize
\tikz \pic [
  transform shape,
  every calendar/.style={
    at={(-8ex,4ex)},
    week list,
    month label above centered,
    month text=\bfseries\textcolor{red}{\%mt} \%y0,
    if={(Sunday) [black!50]}
  },
  folding line length=2.5cm,
  face 1={ \calendar [dates=\the\year-01-01 to \the\year-01-last];},
  face 2={ \calendar [dates=\the\year-02-01 to \the\year-02-last];},
  face 3={ \calendar [dates=\the\year-03-01 to \the\year-03-last];},
  face 4={ \calendar [dates=\the\year-04-01 to \the\year-04-last];},
  face 5={ \calendar [dates=\the\year-05-01 to \the\year-05-last];},
  face 6={ \calendar [dates=\the\year-06-01 to \the\year-06-last];},
  face 7={ \calendar [dates=\the\year-07-01 to \the\year-07-last];},
  face 8={ \calendar [dates=\the\year-08-01 to \the\year-08-last];},
  face 9={ \calendar [dates=\the\year-09-01 to \the\year-09-last];},
  face 10={\calendar [dates=\the\year-10-01 to \the\year-10-last];},
  face 11={\calendar [dates=\the\year-11-01 to \the\year-11-last];},
  face 12={\calendar [dates=\the\year-12-01 to \the\year-12-last];}
] {dodecahedron folding};
\end{codeexample}

The foldings are sorted by number of faces.

\begin{pictype}{tetrahedron folding}{}
    This pic type draws a folding diagram for a tetrahedron. The following keys
    influence the pic:
    %
    \begin{key}{/tikz/folding line length=\meta{dimension}}
        Sets the length of the base line for folding. For the dodecahedron this
        is the length of all the sides of the pentagons.
    \end{key}
    %
    \begin{key}{/tikz/face 1=\meta{code}}
        The \meta{code} is executed for the first face of the dodecahedron.
        When it is executed, the coordinate system will have been shifted and
        rotated such that it lies at the middle of the first face of the
        dodecahedron.
    \end{key}
    %
    \begin{key}{/tikz/face 2=\meta{code}}
        Same as |face 1|, but for the second face.
    \end{key}
    %
    \begin{key}{/tikz/face 3=\meta{code}}
    \end{key}
    %
    \begin{key}{/tikz/face 4=\meta{code}}
    \end{key}
    %
    There are further similar options for more faces (for commands shown
    later).

    Here is a simple example:
    %
\begin{codeexample}[preamble={\usetikzlibrary{folding}}]
\tikz \pic [
  transform shape,
  folding line length=6mm,
  face 1={ \node[red] {1};},
  face 2={ \node      {2};},
  face 3={ \node      {3};},
  face 4={ \node      {4};}
] {tetrahedron folding};
\end{codeexample}

    The appearance of the cut and folding lines can be influenced using the
    following styles:
    %
    \begin{stylekey}{/tikz/every cut (initially \normalfont empty)}
        Executed for every line that should be cut using scissors.
    \end{stylekey}
    %
    \begin{stylekey}{/tikz/every fold (initially help lines)}
        Executed for every line that should be folded.
        %
\begin{codeexample}[preamble={\usetikzlibrary{folding}}]
\tikz \pic[
  every cut/.style=red,
  every fold/.style=dotted,
  folding line length=6mm
] { tetrahedron folding };
\end{codeexample}
    \end{stylekey}

    There is one style that is mainly useful for the present documentation:
    %
    \begin{stylekey}{/tikz/numbered faces}
        Sets |face |\meta{i} to |\node {|\meta{i}|};| for all~$i$.
    \end{stylekey}
\end{pictype}

\begin{pictype}{tetrahedron truncated folding}{}
    A folding of a truncated tetrahedron.
    %
\begin{codeexample}[width=5cm,preamble={\usetikzlibrary{folding}}]
\tikz \pic [folding line length=6mm, numbered faces, transform shape]
  { tetrahedron truncated folding };
\end{codeexample}
    %
\end{pictype}

\begin{pictype}{cube folding}{}
    A folding of a cube.
    %
\begin{codeexample}[preamble={\usetikzlibrary{folding}}]
\tikz \pic [folding line length=6mm, numbered faces, transform shape]
  { cube folding };
\end{codeexample}
    %
\end{pictype}

\begin{pictype}{cube truncated folding}{}
    A folding of a truncated cube.
    %
\begin{codeexample}[width=5cm,preamble={\usetikzlibrary{folding}}]
\tikz \pic [folding line length=6mm, numbered faces, transform shape]
  { cube truncated folding };
\end{codeexample}
    %
\end{pictype}

\begin{pictype}{octahedron folding}{}
    A folding of an octahedron.
    %
\begin{codeexample}[preamble={\usetikzlibrary{folding}}]
\tikz \pic [folding line length=6mm, numbered faces, transform shape]
  { octahedron folding };
\end{codeexample}
    %
\end{pictype}

\begin{pictype}{octahedron folding}{}
    A folding of a truncated octahedron.
    %
\begin{codeexample}[preamble={\usetikzlibrary{folding}}]
\tikz \pic [folding line length=6mm, numbered faces, transform shape]
  { octahedron truncated folding };
\end{codeexample}
    %
\end{pictype}

\begin{pictype}{dodecahedron folding}{}
    A folding of a dodecahedron.
    %
\begin{codeexample}[preamble={\usetikzlibrary{folding}}]
\tikz \pic [folding line length=6mm, numbered faces, transform shape]
  { dodecahedron folding };
\end{codeexample}
    %
\end{pictype}

\begin{pictype}{dodecahedron' folding}{}
    This is an alternative folding of a dodecahedron.
    %
\begin{codeexample}[preamble={\usetikzlibrary{folding}}]
\tikz \pic [folding line length=6mm, numbered faces, transform shape]
  { dodecahedron' folding };
\end{codeexample}
    %
\end{pictype}

\begin{pictype}{cuboctahedron folding}{}
    A folding of a cuboctahedron.
    %
\begin{codeexample}[preamble={\usetikzlibrary{folding}}]
\tikz \pic [folding line length=6mm, numbered faces, transform shape]
  { cuboctahedron folding };
\end{codeexample}
    %
\end{pictype}

\begin{pictype}{cuboctahedron truncated folding}{}
    A folding of a truncated cuboctahedron.
    %
\begin{codeexample}[preamble={\usetikzlibrary{folding}}]
\tikz \pic [folding line length=6mm, numbered faces, transform shape]
  { cuboctahedron truncated folding };
\end{codeexample}
    %
\end{pictype}

\begin{pictype}{icosahedron folding}{}
    A folding of an icosahedron.
    %
\begin{codeexample}[preamble={\usetikzlibrary{folding}}]
\tikz \pic [folding line length=6mm, numbered faces, transform shape]
  { icosahedron folding };
\end{codeexample}
    %
\end{pictype}

\begin{pictype}{rhombicuboctahedron folding}{}
    A folding of an rhombicuboctahedron.
    %
\begin{codeexample}[preamble={\usetikzlibrary{folding}}]
\tikz \pic [folding line length=6mm, numbered faces, transform shape]
  { rhombicuboctahedron folding };
\end{codeexample}
    %
\end{pictype}

\begin{pictype}{snub cube folding}{}
    A folding of a snub cube.
    %
\begin{codeexample}[width=5cm,preamble={\usetikzlibrary{folding}}]
\tikz \pic [folding line length=6mm, numbered faces, transform shape]
  { snub cube folding };
\end{codeexample}
    %
\end{pictype}

\begin{pictype}{icosidodecahedron folding}{}
    A folding of an icosidodecahedron.
    %
\begin{codeexample}[preamble={\usetikzlibrary{folding}}]
\tikz \pic [folding line length=6mm, numbered faces, transform shape]
  { icosidodecahedron folding };
\end{codeexample}
    %
\end{pictype}


%%% Local Variables:
%%% mode: latex
%%% TeX-master: "pgfmanual-pdftex-version"
%%% End:
