% Copyright 2015 by Till Tantau
%
% This file may be distributed and/or modified
%
% 1. under the LaTeX Project Public License and/or
% 2. under the GNU Free Documentation License.
%
% See the file doc/generic/pgf/licenses/LICENSE for more details.


\section{Animations}
\label{section-base-animations}

\begin{pgfmodule}{animations}
    This module contains the basic layer support of animations, which is
    documented in the following.
\end{pgfmodule}

This section described the basic layer support of animations, the \tikzname\
support is described in Section~\ref{section-tikz-animations}. As always,
\tikzname\ mainly converts syntactic constructs (like the special colon or
quote syntax) to appropriate basic layer commands, which are documented here.
Note, however, that while many attributes and options are the same on both
layers, some things are handled differently on the basic layer.


\subsection{Overview}

An \emph{animation} changes the way some part of a graphic looks like over
time. The archetypical animation is, of course, a \emph{movement} of node, but
a change of, say, the opacity of a path is also an animation. \pgfname\ allows
you to specify such animations using a set of commands and keys that are
documented in the following.
%
\begin{codeexample}[
    preamble={\usepgfmodule{animations}},
    animation list={0.5,1,1.5,2},
]
\tikz {
  \pgfanimateattribute{opacity}{
    whom = node, begin on = {click}, entry = {0s}{1}, entry = {2s}{0} }
  \node (node) [fill = blue!20, draw = blue, very thick, circle] {Click me!};
}
\end{codeexample}

Differently from other packages, the animations created by \pgfname\ are not
precomputed sequences of pictures that are displayed in rapid succession.
Rather, an animation created by \pgfname\ consists mainly of an annotation in
the output that a certain attribute of a certain object should change over time
in some specific way when the object is displayed. It is the job of the
document viewer application to actually compute and display the animation.
Interestingly, this means that animations neither increase the size of the
output files noticeably nor does it put a special burden on \TeX. The hard and
complicated calculations are done by the viewer application, not by \TeX\ and
\pgfname.

Only few viewer applications and formats are currently ``up to the job'' of
displaying animations. In particular, the popular \textsc{pdf} format does
\emph{not} allow one to specify animations in this way (one can partly ``fake''
animations at the high price of including a great number of precomputed
pictures and using JavaScript in special viewers, but this is really not the
same thing as what \pgfname\ does). Indeed, currently only the \textsc{svg}
format allows one to specify animations in a sensible way. Thus, \pgfname's
animations will only be displayed when \textsc{svg} is used as output format.

Because of the shortcomings of the other formats and, also, for purposes of
printing and depicting animations in a sequential manner, \pgfname\ also allows
you to create ``snapshots'' of animations. As an example, the following code
shows how the same drawing is shown at different ``time snapshots'':
%
\begin{codeexample}[
    width=3.9cm,
    preamble={\usetikzlibrary{animations}
\def\pgfname{\textsc{pgf}}
}]
\tikz [make snapshot of=0.5s] \scoped :rotate = { 0s = "0", 2s = "90" }
  \node [draw=blue, very thick] {\pgfname};
\tikz [make snapshot of=1s]   \scoped :rotate = { 0s = "0", 2s = "90" }
  \node [draw=blue, very thick] {\pgfname};
\tikz [make snapshot of=1.5s] \scoped :rotate = { 0s = "0", 2s = "90" }
  \node [draw=blue, very thick] {\pgfname};
\tikz [make snapshot of=2s]   \scoped :rotate = { 0s = "0", 2s = "90" }
  \node [draw=blue, very thick] {\pgfname};
\end{codeexample}


\subsection{Animating an Attribute}

\subsubsection{The Main Command}

Creating an animation is done using the command |\pgfanimateattribute|, which
takes a to-be-animated attribute and options specifying the timeline:

\begin{command}{\pgfanimateattribute\marg{attribute}\marg{options}}
    Adds an animation of the \meta{attribute} of a future \emph{object} to the
    current graphic. Attributes are things like the ``fill opacity'' or the
    transformation matrix or the line width.

    The \meta{options} are keys that configure how the attribute changes over
    time. Using the |entry| key multiple times, you specify which value the
    chosen attribute should have at different points in time. Unless special
    keys are used, ``outside'' the specified timeline the animation has no
    effect:
    %
\begin{codeexample}[
    preamble={\usepgfmodule{animations}},
    animation list={0.5,1,1.5,2,2.5},
]
\tikz {
  \pgfanimateattribute{opacity}{
    whom = node, begin on = {click}, entry = {0s}{1}, entry = {2s}{0} }
  \node (node) [fill = blue!20, draw = blue, very thick, circle] {Click me!};
}
\end{codeexample}

    Other keys, like |repeats|, allow you to specify how the animation behaves
    ``as a whole''. These keys are documented later in this section.


    \medskip
    \textbf{The Attributes}

    In detail, the |\pgfanimateattribute| command opens a \TeX-scope, looks up
    the \emph{type} values of the specified \meta{attribute} have (if you wish
    to animate the |opacity| of an object, the type is ``scalar'' meaning that
    entries must be scalar numbers; when you animate the |fill| attribute, the
    type is ``color'' and values must be colors, and so on), and then executes
    the \meta{options} with the path prefix |/pgf/animation|. Finally, an
    appropriate system layer command |\pgfsysanimate...| is called to create
    the actual animation and the scope is closed.

    The following \meta{attributes} are permissible:

    \begin{tabular}{ll}
        \emph{Attribute}  & \emph{Type} \\
        |draw|, |fill|    & color \\
        |line width|      & dimension \\
        |motion|          & scalar \\
        |opacity|, |fill opacity|, |draw opacity| & scalar \\
        |path|            & path \\
        |rotate|          & scalar \\
        |scale|           & scaling \\
        |softpath|        & softpath \\
        |translate|       & point \\
        |view|            & viewbox \\
        |visible|         & boolean \\
        |stage|           & boolean \\
        |xskew|, |yskew|  & scalar \\
    \end{tabular}

    These attributes are detailed in Sections
    \ref{section-base-animation-painting}
    to~\ref{section-base-animation-views}, but here is a quick overview:
    %
    \begin{itemize}
        \item |draw| and |fill| refer to the color used to draw (stroke) and
            fill paths in an object, respectively. Typical values for this
            attribute are |red| or |black!10|.
        \item |line width| is, of course, the line width used in an object.
            Typical values are |0.4pt| or |1mm|. Note that you (currently)
            cannot use keys like |thin| or |thick| here, but this may change in
            the future.
        \item |motion| is a slightly special attribute: It allows you to
            specify a path along which the object should be moved (using the
            |along| key). The values given to the |entry| key for this
            attribute refer to a \emph{fraction of the distance along the
            path}. See the |along| key for details.
        \item |opacity| and the variants |fill opacity| and |draw opacity|
            animate the opacity of an object. Allowed values range between 0
            and 1.
        \item |path| allows you to animate a path (it will morph). The
            ``values'' are now paths themselves. See
            Section~\ref{section-base-animation-paths} for details.
        \item |rotate| refers to a rotation of the object. Values for the
            |entry| key are the rotation angles like |0| or |90|.
        \item |scale| refers to the scaling of the object. Values are either
            single scalars values (like |1| or  |1.5|) or two numbers separated
            by a comma (like |1,1.5| or |0.5,2|), referring to the $x$-scaling
            and $y$-scaling.
        \item |softpath| is a special case of the |path| attribute, see
            Section~\ref{section-base-animation-paths} once more.
        \item |translate| shifts the object by a certain vector. Values are
            points like |\pgfpoint{1cm}{2cm}|.
        \item |view| allows you to animate the view box of a view, see
            Section~\ref{section-base-animation-views} for details.
        \item |visible| refers to the visibility of an object. Allowed values
            are |true| and |false|.
        \item |stage| is identical to |visible|, but when the object is not
            animated, it will be hidden by default.
        \item |xskew| and |yskew| skew the object. Attributes are angles like
            |0| or |45| or even |90|.
    \end{itemize}


    \medskip
    \textbf{The Target Object}

    As stated earlier, the \meta{options} are used to specify the object whose
    attribute for which an animation should be added to the picture. Indeed,
    you \emph{must} specify the object explicitly using the |whom| key and you
    must do so \emph{before} the object is created. Note that, in contrast, in
    \textsc{svg} you can specify an animation more or less anywhere and then
    use hyper-references to link the animation to the to-be-animated object;
    \pgfname\ insists that you specify the animation before the object. This is
    a bit of a bother in some situations, but it is the only way to ensure that
    \pgfname\ has a fighting chance to attach some additional code to the
    object (which is necessary for almost all animations of the transformation
    matrix).

    \begin{key}{/pgf/animation/whom=\meta{id}\opt{|.|\meta{type}}}
        You \emph{must} use this key once which each call of the
        |\pgfanimateattribute| command. The \meta{id} and the optional
        \meta{type} (which is whatever follows the first dot) will be passed to
        |\pgfidrefnextuse|, see that command for details.
    \end{key}

    As explained in the introduction of this chapter, an ``animation'' is just
    a bit of special text in the output document asking a viewer application to
    animate the object at some later time. The |\pgfanimateattribute| command
    inserts this special text immediately, even though it refers to an object
    created only later on. Normally, this is not a problem, but the special
    text should be on the same page as the to-be-animated object. To ensure
    this, it suffices to call |\pgfanimateattribute| no earlier than the
    beginning of the |pgfpicture| containing the object.


    \medskip
    \textbf{Naming the Animation}

    You can assign a name to an animation for later (or early) reference. In
    particular, it is possible to begin \emph{another} animation relative to
    the beginning or end of this animation and for referencing this animation
    must be assigned a name. See the |of| and |of next| keys for details.

    \begin{key}{/pgf/animation/name=\meta{name}}
        Assigns a name to the animation by which it can be referenced using the
        |of| and |of next| keys in another animation.
        %
\begin{codeexample}[
    preamble={\usepgfmodule{animations}},
    animation list={.5,1,1.5,2,2.5,3,3.5,4,4.5,5,5.5,6},
]
\tikz [very thick] {
  \pgfanimateattribute{rotate}{
    whom = node, begin on = {end, of next = my move animation, delay = 1s},
    entry = {0s}{0}, entry = {2s}{90}, begin snapshot = 3s, }
  \pgfanimateattribute{translate}{
    name = my move animation, whom = node, begin on = {click},
    entry = {0s}{\pgfpointorigin}, entry = {2s}{\pgfpoint{0cm}{-5mm}} }
  \node (node) [fill = blue!20, draw = blue, circle] {Here!};
}
\end{codeexample}
    \end{key}
\end{command}

\begin{command}{\pgfanimateattributecode\marg{attribute}\marg{code}}
    The command works like |\pgfanimateattribute|, only instead of
    \meta{options} you specify some \meta{code} whose job is to setup the
    options.
\end{command}


\subsubsection{Specifying the Timeline}

The core key for specifying how an attribute varies over time is the |entry|
key:
%
\begin{key}{/pgf/animation/entry=\marg{time}\marg{value}}
    You use this key repeatedly to specify the different values that the
    \meta{attribute} should have over time. At the \meta{time} specified, the
    \meta{attribute} will have the value specified as \meta{value}:
    %
\begin{codeexample}[
    preamble={\usepgfmodule{animations}},
    animation list={.5,1,1.5,2},
]
\tikz {
  \pgfanimateattribute{rotate}{
    whom = node, begin on = {click},
    entry = {0s}{0}, entry = {1s}{90}, entry = {1.1s}{45}, entry = {2s}{90}
  }
  \node (node) [fill = blue!20, draw = blue, very thick, circle] {Click me!};
}
\end{codeexample}

    You need to call |entry| once for each time in the timeline for which you
    want to specify a \meta{value} explicitly. Between these times, the values
    get interpolated (see below for details). You need to specify the
    \meta{time}s in non-decreasing order (it is permissible and sometimes also
    necessary to specify the same time twice, namely to create a ``jump'' of
    the value of some attribute).

    The \meta{time} is parsed using the command |\pgfparsetime| described
    later.


    \medskip
    \textbf{Start and end of the timeline.}
    The first and last times of the timeline are a bit special: The timeline
    starts on the first time and the duration of the timeline is the difference
    between the first and last time. ``Starting'' on the start time actually
    means that any beginnings (see the |begin| and |end| keys) get offset by
    the start time; similarly end times are offset by this value.


    \medskip
    \textbf{Syntax of the values.}
    The syntax of the \meta{value} varies according to the type of the
    \meta{attribute}. In detail, these are:

    \begin{tabular}{lp{12cm}}
        \emph{Type} & \emph{Syntax} \\
        color       & Standard color syntax like |red| or |black!10| \\
        scalar      & A value parsed using |\pgfmathparse| \\
        dimension   & A dimension parsed using |\pgfmathparse| \\
        path        & A sequence of path construction commands \\
        softpath    & A sequence of soft path construction commands \\
        scaling     & A scalar value or a pair of scalar values separated by a comma \\
        point       & A \pgfname-point like |\pgfpoint{1cm}{5mm}| \\
        viewbox     & Two \pgfname-points \\
        boolean     & |true| or |false| \\
    \end{tabular}


    \medskip
    \textbf{Interpolation between key times.}
    You use the |entry| key repeatedly, namely once for each ``key time'',
    which is a time point for which you specify the value of the attribute
    explicitly. Between these key times, the attribute's value is interpolated.
    Normally, this is just a linear interpolation, but you can influence this
    using the following keys, see Section~\ref{section-anim-smooth} for
    details.

    \begin{key}{/pgf/animations/exit control=\marg{time fraction}\marg{value fraction}}
        Same as |/tikz/animate/options/exit control|.
        %
\begin{codeexample}[
    preamble={\usepgfmodule{animations}},
    animation list={0.333/\frac{1}{3},0.666/\frac{2}{3},1,1.333/1\frac{1}{3},1.666/1\frac{2}{3}},
]
\tikz {
  \foreach \i in {0,0.1,...,1} \draw (-0.9,.9-\i) -- ++(1.8,0);
  \pgfanimateattribute{translate}{
    whom = node, begin on = {click},
    exit control={1}{0},
    entry = {0s}{\pgfpointorigin},
    linear, % revert to default
    entry = {1s}{\pgfpoint{0cm}{-5mm}},
    entry control={0}{1},
    entry = {2s}{\pgfpoint{0cm}{-10mm}} }
  \node (node) [fill = blue!20, draw = blue, very thick, circle] {Click me!};
}
\end{codeexample}
    \end{key}

    \begin{key}{/pgf/animations/entry control=\marg{time fraction}\marg{value fraction}}
        Works like |exit control|.
    \end{key}

    \begin{key}{/pgf/animations/linear}
        A shorthand for |exit control={0}{0}, entry control={1}{1}|. This will
        (re)install a linear curve.
    \end{key}

    \begin{key}{/pgf/animations/stay}
        Same as |/tikz/animate/options/stay|.
        %
\begin{codeexample}[
    preamble={\usepgfmodule{animations}},
    animation list={0.5,1,1.5,2,2.5},
]
\tikz {
  \foreach \i in {0,0.1,...,1} \draw (-0.9,.9-\i) -- ++(1.8,0);
  \pgfanimateattribute{translate}{
    whom = node, begin on = {click},
    entry = {0s}{\pgfpointorigin},
    stay,
    entry = {1s}{\pgfpoint{0cm}{-5mm}},
    linear,
    entry = {2s}{\pgfpoint{0cm}{-10mm}},
    entry = {3s}{\pgfpoint{0cm}{-15mm}} }
  \node (node) [fill = blue!20, draw = blue, very thick, circle] {Click me!};
}
\end{codeexample}
    \end{key}

    \begin{key}{/pgf/animations/jump}
        Same as |/tikz/animate/options/jump|.
        %
\begin{codeexample}[
    preamble={\usepgfmodule{animations}},
    animation list={0.5,1,1.5,2},
]
\tikz {
  \foreach \i in {0,0.1,...,1} \draw (-0.9,.9-\i) -- ++(1.8,0);
  \pgfanimateattribute{translate}{
    whom = node, begin on = {click},
    entry = {0s}{\pgfpointorigin},
    jump,
    entry = {1s}{\pgfpoint{0cm}{-1cm}},
    linear,
    entry = {2s}{\pgfpoint{0cm}{-2cm}} }
  \node (node) [fill = blue!20, draw = blue, very thick, circle] {Click me!};
}
\end{codeexample}
    \end{key}
\end{key}

When the time of an animation lies outside the timeline specified by the
|entry| keys, no animation is present. This means that the value of the
attribute is the object's scope is used instead. Using the following key, you
can set this value directly:

\begin{key}{/tikz/animations/base=\meta{value}}
    The syntax of the \meta{value} is the same as for the |entry| key. The
    \meta{value} is installed as the value of the object's attribute whenever
    the timeline is not active. This makes it easy to specify the value of an
    attribute when the animation is ``not running''.
    %
\begin{codeexample}[
    preamble={\usepgfmodule{animations}},
    animation list={-1,0,1,2,3},
]
\tikz {
  \pgfanimateattribute{rotate}{
    whom = node, begin on = {click},
    entry = {0s}{90}, entry = {2s}{180},
    base  = 45
  }
  \node (node) [fill = blue!20, draw = blue, very thick, circle] {Click me!};
}
\end{codeexample}
    %
\end{key}

It may happen that there is more than one timeline active that is ``trying to
modify'' a given attribute. In this case, the following rules are used to
determine, which timeline ``wins'':
%
\begin{enumerate}
    \item If no animation is active at the current time (all animation either
        have not yet started or they have already ended), then the base value
        given in the animation encountered last in the code is used. (If there
        are no base values, the attribute is taken from the surrounding scope.)
    \item If there are several active animations, the one that has started last
        is used and the its value is used.
    \item If there are several active animations that have started at the same
        time, the one that comes last in the code is used.
\end{enumerate}

Note that these rules do not apply to transformations of the canvas since these
are always additive (or, phrased differently, they are always all active and
the effects accumulate).

\begin{command}{\pgfparsetime\marg{time}}
    This command works like |\pgfmathparse| (indeed, it calls is internally),
    but returns the result in the macro |\pgftimeresult| rather than
    |\pgfmathresult|. Furthermore, the following changes are installed:
    %
    \begin{itemize}
        \item The postfix operator |s| is added, which has no effect.
        \item The postfix operator |ms| is added, which divides a number by
            1000, so |2ms| equals 0.002s.
        \item The postfix operator |min| is added, which multiplies a number by
            60.
        \item The postfix operator |h| is added, which multiplies a number by
            3600.
        \item The infix operator |:| is redefined, so that it multiplies its
            first argument by 60 and adds the second. This implies that |1:20|
            equals 80s and |01:00:00| equals 3600s.
        \item The parsing of octal numbers is switched off to allow things like
            |01:08| for 68s.
    \end{itemize}
\end{command}


\subsubsection{``Anti-Animations'': Snapshots}

There are a number of situations in which you want the ``opposite'' of an
animation to happen: You want to create a ``still image''. For instance, when
you want to print an animation you will typically wish to show one or more
``temporal snapshots'' of the animation. Also, you may wish to specify a value
for an object when it is \emph{not} being animated.

Let us start with creating a snapshot:

\begin{command}{\pgfsnapshot\marg{time}}
    When this command is used inside a \TeX\ scope, the behaviour of
    |\pgfanimateattribute| changes: Instead of adding an animation to the
    object and the attribute, the object's attribute is set to value it would
    have during the animation at time \meta{time}. Note that when this command
    is used in a \TeX\ scope, no animation is created and no support by the
    driver is needed (so, it works with \textsc{pdf}).
    %
\begin{codeexample}[preamble={\usetikzlibrary{animations}}]
\tikz [make snapshot of=1s,
       animate = { myself: = {
         :rotate = { 0s = "0", 2s = "90" },
         :color  = { 0s = "red", 2s = "green" },
         :line width = { 0s = "0mm", 4s = "4mm" }
       }}]
  \node [fill=black!20, draw] { Node };
\end{codeexample}


    \medskip\textbf{Timing and Events.}
    The timeline of an animation normally starts at a ```moment |0s|'' and the
    \meta{time} is considered relative to this time. For instance, if a
    timeline contains, say, the settings |entry={2s}{0}| and |entry={3s}{10}|
    and \marg{time} is set to |2.5s|, then the value the attribute will get is
    5.

    It is, however, also possible to specify that animations begin and end at
    certain times relative to events like a |click| event. \emph{These events
    are not relevant with respect to snapshots.} However, there is one key that
    allows you to specify the beginning of the snapshot timeline:
    %
    \begin{key}{/tikz/animations/begin snapshot=\meta{begin time}}
        When this key is used inside the options of |\pgfanimateattribute|,
        with respect to snapshots, the timeline begins at \meta{begin time}.
        This means that, if the snapshot time is set to \meta{time} and the
        beginning of the snapshot's timeline is set to \meta{begin time}, the
        attribute is set to the value of the timeline at time $\meta{time} -
        \meta{begin time}$.

        The idea is that when you make a snapshot of several animations and all
        of them have started at different times because of different events,
        you use |begin snapshot| with each object and attribute to directly
        specify when these different events have happened.
    \end{key}

    Note that the |end| keys have no effect with snapshots, that is, with a
    snapshot all animations always run till the end of the timeline (which may
    or may not be ``forever'').


    \medskip\textbf{Limitations.}
    For snapshots, the value an animation has at time \meta{time} must be
    computed by \TeX. While in many cases this is easy to achieve, in some
    cases this is not trivial such as a timeline for a path with repeats plus
    smoothing via splines. An additional complication is the fact that an
    animation may be specified at a place far removed from the actual
    to-be-animated object. For these reasons, certain limitations apply to
    snapshots:
    %
    \begin{itemize}
        \item The |begin| and |begin on| keys have no effect (but 
            |begin snapshot| has one.
        \item The |end| and |end on| keys have no effect.
        \item The |current value| may not be used in a timeline (since
            \pgfname\ cannot really determine this value).
        \item The |accumulating| specification may not be used with paths,
            views, or motions.
        \item Since the timing computations are done using \TeX\ code, they are
            not necessarily stable. For instance, when a time interval is very
            small and there are many repeats or when a spline is very
            complicated, the calculated values may not be fully accurate.
    \end{itemize}
\end{command}

\begin{command}{\pgfsnapshotafter\marg{time}}
    This command works exactly like |\pgfsnapshot| only the ``moment'' that
    \meta{time} refers to is conceptually $\meta{time} + \epsilon$: When
    timeline specifies several values for \meta{time}, this command will select
    the last value at \meta{time}, while |\pgfsnapshot| will select the first
    value at \meta{time}. Similarly, when a timeline ends at \meta{time},
    |\pgfsnapshot| will select the last value of the timeline while
    |\pgfsnapshotafter| will not apply the animation any more:
    %
\begin{codeexample}[preamble={\usepgfmodule{animations}}]
\foreach \t in {0,1,2,3,4} {
  \pgfsnapshot{\t}
  \tikz :rotate = { 0s = "0", 2s = "90", 2s = "180", 4s = "270" }
    \node [draw=blue, very thick] {f}; }
\end{codeexample}
    %
\begin{codeexample}[preamble={\usepgfmodule{animations}}]
\foreach \t in {0,1,2,3,4} {
  \pgfsnapshotafter{\t}
  \tikz :rotate = { 0s = "0", 2s = "90", 2s = "180", 4s = "270" }
    \node [draw=blue, very thick] {f}; }
\end{codeexample}
    %
\end{command}


\subsection{Animating Color, Opacity, Visibility, and Staging}
\label{section-base-animation-painting}

\begin{animateattribute}{fill}
    You can animate the color of the target object of an animation using the
    attributes |fill| or |draw|, which animate the fill color and the drawing
    (stroking) color, respectively. To animate both the fill and draw color,
    you need to create two animations, one for each.
    %
\begin{codeexample}[
    preamble={\usepgfmodule{animations}},
    animation list={0.5,1,1.5,2},
]
\tikz {
  \pgfanimateattribute{fill}{
    whom = node.background, begin on = {click},
    entry = {0s}{white}, entry = {2s}{red} }
  \node (node) [fill = blue!20, draw = blue, very thick, circle] {Click me!};
}
\end{codeexample}
    %
\end{animateattribute}

\begin{animateattribute}{draw}
\begin{codeexample}[
    preamble={\usepgfmodule{animations}},
    animation list={0.5,1,1.5,2},
]
\tikz {
  \pgfanimateattribute{draw}{
    whom = node.background, begin on = {click},
    entry = {0s}{white}, entry = {2s}{red} }
  \node (node) [fill = blue!20, draw = blue, very thick, circle] {Click me!};
}
\end{codeexample}
\end{animateattribute}

When the target of a color animation is a scope, you animate the color ``used
in this scope'' for filling or stroking. However, when an object inside the
scope has its color set explicitly, this color overrules the color of the
scope:
%
\begin{codeexample}[
    preamble={\usepgfmodule{animations}},
    animation list={0.5,1,1.5,2},
    animation bb={(1.5,-0.75) rectangle (3,0.75)},
]
\tikz {
  \pgfanimateattribute{fill}{
    whom = example, begin on = {click, of next=node},
    entry = {0s}{white}, entry = {2s}{red} }
  \node (node) [fill = blue!20, draw = blue, very thick, circle] {Click me!};
  \begin{scope}[name = example]
    \fill (1.5,-0.75) rectangle ++ (1,1);
    \fill [blue] (2,-0.25) rectangle ++ (1,1);
  \end{scope}
}
\end{codeexample}

Note that in certain cases, a graphic scope may contain graphic objects with
their colors set explicitly ``in places where you do not expect it'': In
particular, a node normally consists at least of a background path and a text.
For both the text and for the background path, colors will be set for the text
and also for the path explicitly. This means that when you pick the fill
attribute of a node as the target of an animation, you will \emph{not} animate
the color of the background path in case this color has been set explicitly.
Instead, you must choose the background path of the node as the target of the
animation. Fortunately, this is easy to achieve since when the background path
of a node is created, the identifier type is set to |background|, which in turn
allows you to access it as \meta{node}|.background| through the |whom| key.

The text of a node also gets it color set explicitly, which means that a change
of the node's scope's color has no effect on the text color. Instead, you must
choose \meta{name}|.text| as the target (or, if the node has more parts, use
the name of the part as the identifier type instead of |text|).
%
\begin{codeexample}[
    preamble={\usepgfmodule{animations}},
    animation list={0.5,1,1.5,2},
    animation bb={(1.1,-0.9) rectangle (2.9,0.9)},
]
\tikz {
  \pgfanimateattribute{fill}{
    whom = example, begin on = {click, of next=node},
    entry = {0s}{white}, entry = {2s}{red} }
  \node (node) [fill = blue!20, draw = blue, very thick, circle] {Click me!};
  \node at (2,0) (example) [fill = blue!20, circle] {No effect}; }
\end{codeexample}

\begin{codeexample}[
    preamble={\usepgfmodule{animations}},
    animation list={0.5,1,1.5,2},
    animation bb={(1.1,-0.9) rectangle (2.9,0.9)},
]
\tikz {
  \pgfanimateattribute{fill}{
    whom = example.background, begin on = {click, of next=node},
    entry = {0s}{white}, entry = {2s}{red} }
  \node (node) [fill = blue!20, draw = blue, very thick, circle] {Click me!};
  \node at (2,0) (example) [fill = blue!20, circle] {Effect}; }
\end{codeexample}

\begin{codeexample}[
    preamble={\usepgfmodule{animations}},
    animation list={0.5,1,1.5,2},
    animation bb={(1.1,-0.9) rectangle (2.9,0.9)},
]
\tikz {
  \pgfanimateattribute{fill}{
    whom = example.text, begin on = {click, of next=node},
    entry = {0s}{white}, entry = {2s}{red} }
  \node (node) [fill = blue!20, draw = blue, very thick, circle] {Click me!};
  \node at (2,0) (example) [fill = blue!20, circle, font=\huge] {Text}; }
\end{codeexample}

Similarly to the color, you can also set the opacity used for filling and for
drawing. You specify the opacity using a number between 0 (transparent) and 1
(opaque).

\begin{animateattribute}{fill opacity}
\begin{codeexample}[
    preamble={\usepgfmodule{animations}},
    animation list={0.5,1,1.5,2},
]
\tikz {
  \pgfanimateattribute{fill opacity}{
    whom = node, begin on = {click}, entry = {0s}{1}, entry = {2s}{0} }
  \node (node) [fill = blue!20, draw = blue, very thick, circle] {Click me!};
}
\end{codeexample}
\end{animateattribute}

\begin{animateattribute}{draw opacity}
\begin{codeexample}[
    preamble={\usepgfmodule{animations}},
    animation list={0.5,1,1.5,2},
]
\tikz {
  \pgfanimateattribute{draw opacity}{
    whom = node, begin on = {click}, entry = {0s}{1}, entry = {2s}{0} }
  \node (node) [fill = blue!20, draw = blue, very thick, circle] {Click me!};
}
\end{codeexample}
\end{animateattribute}

\begin{animateattribute}{opacity}
    Unlike colors, where there is no joint attribute for filling and stroking,
    there is a single |opacity| attribute in addition to the above two
    attributes. If supported by the driver, it treats the graphic object to
    which it is applied as a transparency group. In essence, ``this attribute
    does what you want'' at least in most situations.
    %
\begin{codeexample}[
    preamble={\usepgfmodule{animations}},
    animation list={0.5,1,1.5,2},
]
\tikz {
  \pgfanimateattribute{opacity}{
    whom = node, begin on = {click}, entry = {0s}{1}, entry = {2s}{0} }
  \node (node) [fill = blue!20, draw = blue, very thick, circle] {Click me!};
}
\end{codeexample}
    %
\end{animateattribute}

\begin{animateattribute}{visible}
    The difference between the |visible| attribute and an opacity of |0| is
    that an invisible object cannot be clicked and does not need to be
    rendered. The (only) two possible values for this attribute are |false| and
    |true|.
    %
\begin{codeexample}[
    preamble={\usepgfmodule{animations}},
    animation list={1,2,3,4},
]
\tikz {
  \pgfanimateattribute{visible}{
    whom = node, begin on = {click}, entry = {0s}{false}, entry = {2s}{false} }
  \node (node) [fill = blue!20, draw = blue, very thick, circle] {Click me!};
}
\end{codeexample}
    %
\end{animateattribute}

\begin{animateattribute}{stage}
    This attribute is the same as the |visible| attribute, only |base=false| is
    set by default. This means that the object is \emph{only} visible when you
    explicitly during the time the entries are set to |true|. The idea behind
    the name ``stage'' is that the object is normally ``off stage'' and when
    you explicitly set the ``stage attribute'' to |true| the object ``enters''
    the stage and ``leaves'' once more when it is no longer ``on stage''.
    %
\begin{codeexample}[
    preamble={\usepgfmodule{animations}},
    animation list={-1,0,1,2,3},
    animation bb={(1.3,-0.7) rectangle (2.7,0.7)},
]
\tikz {
  \pgfanimateattribute{stage}{
    whom = example, begin on = {click, of next=node},
    entry = {0s}{true}, entry = {2s}{true} }
  \node (node) [fill = blue!20, draw = blue, very thick, circle] {Click me!};
  \node at (2,0) (example) [fill = blue!20, circle] {Effect}; }
\end{codeexample}
    %
\end{animateattribute}


\subsection{Animating Paths and their Rendering}
\label{section-base-animation-paths}

You can animate the appearance of a path in the following ways:

\begin{animateattribute}{line width}
\begin{codeexample}[
    preamble={\usepgfmodule{animations}},
    animation list={0.5,1,1.5,2},
]
\tikz [very thick] {
  \pgfanimateattribute{line width}{
    whom = node, begin on = {click}, entry = {0s}{1pt}, entry = {2s}{5mm} }
  \node (node) [fill = blue!20, draw = blue, circle] {Click me!};
}
\end{codeexample}
    %
    The possible values passed to the |entry| key are, of course, dimensions.
\end{animateattribute}

\begin{animateattribute}{dash}
\begin{codeexample}[
    preamble={\usepgfmodule{animations}},
    animation list={0.5,1,1.5,2},
]
\tikz {
  \pgfanimateattribute{dash}{
    whom = node, begin on = {click}, entry = {0s}{{{10pt}{1pt}}{0pt}},
                                     entry = {2s}{{{1pt}{10pt}}{0pt}} }
  \node (node) [fill = blue!20, draw = blue, very thick, circle] {Click me!};
}
\end{codeexample}
    %
\begin{codeexample}[
    preamble={\usepgfmodule{animations}},
    animation list={0.5,1,1.5,2},
]
\tikz {
  \pgfanimateattribute{dash}{
    whom = node, begin on = {click}, entry = {0s}{{{1cm}{1pt}}{0pt}},
                                     entry = {2s}{{{1cm}{1pt}}{1cm}} }
  \node (node) [fill = blue!20, draw = blue, very thick, circle] {Click me!};
}
\end{codeexample}
    %
    To specify the dash pattern, you specify a sequence of ``on and off''
    dimensions; see |\pgfsetdash| for details. Note that you \emph{must}
    specify the same number of elements in all patterns of a timeline: You
    cannot specify that the dash pattern for |1s| is |{1pt}{2pt}| and for |2s|
    is |{1pt}{3pt}{2pt}| since the number of elements would differ. In
    particular, you cannot (sensibly) use |current value| for the first entry
    since this corresponds to an empty dash pattern (even when you have
    specified a dash pattern for the target object: this pattern will not be
    attached to the to-be-animated scope or object but to a surrounding scope
    and, thus, the to-be-animated scope will not have any dash pattern
    installed).
\end{animateattribute}

\begin{animateattribute}{path}
    You can animate the path itself:
    %
\begin{codeexample}[
    preamble={\usepgfmodule{animations}},
    animation list={0.5,1,1.5,2},
]
\tikz {
  \pgfanimateattribute{path}{
    whom = node.background.path, begin on = {click, of next=node},
    entry = {0s}{\pgfpathellipse{\pgfpointorigin}
                  {\pgfpointxy{1}{0}}{\pgfpointxy{0}{1.5}}},
    entry = {2s}{\pgfpathellipse{\pgfpointxy{.5}{0}}
                  {\pgfpointxy{.5}{.5}}{\pgfpointxy{0.25}{.25}}}}
  \node (node) [fill = blue!20, draw = blue, very thick, circle] {Click me!};
}
\end{codeexample}
    %
    The path is specified by giving path construction commands as in the above
    example. They will be executed in a special protected scope to ensure that
    they have only little side effects.

    As for the dash pattern, you must ensure that all paths in the timeline
    have the same structure (same sequence of path construction commands); only
    the coordinates may differ. In particular, you cannot say that the path at
    |1s| is a rectangle using |\pgfpathrectangle| and at |2s| is a circle using
    |\pgfpathcircle|. Instead, you would have to ensure that at both times that
    path consists of appropriate Bézier curves.

    Unlike the dash pattern, the to-be-animated object is, indeed, the path
    itself and not some special scope. This means that you can use the 
    |current value| for the start path. However, this also means that you really must
    pick \emph{the path object} as the target of the animation. In conjunction
    with \tikzname, this will be an object of type |path| as in the above
    example.

    When a path is animated, it cannot have ``normal'' arrows attached to it
    since due to the way \pgfname\ adds arrow tips to paths, these would not
    ``move along'' with the path (you get an error message if you try).
    However, it still \emph{is} possible to add arrow tips to an animated path,
    only you need to use the |arrows| key described next.

    Concerning the bounding box computation of the path, a bounding box for all
    paths mentioned for any time point is used as the overall bounding box.
\end{animateattribute}

\begin{key}{/pgf/animation/arrows=\meta{start tip spec}|-|\meta{end tip spec}}
    This key specifies arrow tips during the animation of the path. The syntax
    for the arrow tips is the same syntax as the |\pgfsetarrow| command or
    \tikzname's |arrows| key.  The specified start and end arrow tips are
    rendered as ``markers'', which are added to the path \emph{only} during the
    animation. The markers are rotated along with the path in exactly the  same
    way as normal arrow tips would be. To be precise, the rules used for the
    computation of where arrow tips go and in which direction they head is not
    always the same for ``static'' arrow tips (arrow tips added to a normal
    path) and the ``dynamic'' arrow tips based on markers; namely when the
    paths are very short or closed. For this reason, you should add arrow tips
    to animated paths only when the paths are ``nice and simple'' in the sense
    that they consist of a single segment whose ends are reasonably long.

    In addition to adding the arrow tips to the path during the animation, the
    path gets shortened as necessary to compensate for the extend of the arrow
    tips. However, for this to work, the arrow tips have to be specified before
    path values are specified (since the shortening is done immediately when a
    path value is parsed).
    %
\begin{codeexample}[
    preamble={\usepgfmodule{animations}},
    animation list={0,1,2,3,4},
    animation bb={(0.9,-0.1)rectangle(2.1,1.1)},
]
\tikz {
  \pgfanimateattribute{path}{
    whom = p.path, begin on = {click, of next=node}, arrows = ->,
    entry = {1s}{\pgfpathmoveto{\pgfpoint{1cm}{0cm}}
                 \pgfpathlineto{\pgfpoint{2cm}{1cm}}},
    entry = {3s}{\pgfpathmoveto{\pgfpoint{1cm}{1cm}}
                 \pgfpathlineto{\pgfpoint{2cm}{5mm}}}}
  \node (node) [fill = blue!20, draw = blue, very thick, circle] {Click me!};
  \draw [very thick, blue, name=p] (1,0.5) -- (2,0.5);
}
\end{codeexample}

    Note that the markers that visualize the arrow tips are rendered only once
    per animation. In consequence, ``bending'' arrow tips cannot be rendered
    correctly: As a path ``morphs'' a bend arrow tip needs not only to rotate
    along, but must actually ``bend along'', which is not supported (neither by
    \pgfname\ nor by \textsc{svg}).

    As pointed out earlier, an animated path cannot have ``static'' arrow tips.
    However, when you specify a |base| value, which is the path used whenever
    there is no active animation, \emph{will} use the arrow tips. As a result,
    you can use this to animate a path with an arrow tip:
    %
\begin{codeexample}[
    preamble={\usepgfmodule{animations}},
    animation list={0,1,2,3,4},
    animation bb={(0.9,-0.1)rectangle(2.1,1.1)},
]
\tikz {
  \pgfanimateattribute{path}{
    whom = p.path, begin on = {click, of next=node}, arrows = ->,
    base = {\pgfpathmoveto{\pgfpoint{1cm}{5mm}}
            \pgfpathlineto{\pgfpoint{2cm}{5mm}}},
    entry = {1s}{\pgfpathmoveto{\pgfpoint{1cm}{0cm}}
                 \pgfpathlineto{\pgfpoint{2cm}{1cm}}},
    entry = {3s}{\pgfpathmoveto{\pgfpoint{1cm}{1cm}}
                 \pgfpathlineto{\pgfpoint{2cm}{5mm}}}}
  \node (node) [fill = blue!20, draw = blue, very thick, circle] {Click me!};
  \draw [very thick, blue, name=p];
}
\end{codeexample}
    %
\end{key}

\begin{key}{/pgf/animation/shorten >=\meta{dimension}}
    Just like the normal \tikzname\ key |shorten >|, this key specifies an
    extra shortening of to-be-animated paths. Whenever a path is parsed as a
    value for a path animation, it gets shortened at the end by the
    \meta{dimension} (and, additionally, by the length of the attached arrow
    tip). Just like the |arrows| key, this key must be given before the path
    entries are specified.
    %
\begin{codeexample}[
    preamble={\usepgfmodule{animations}},
    animation list={0,1,2,3,4},
    animation bb={(0.9,-0.1)rectangle(2.1,1.1)},
]
\tikz {
  \pgfanimateattribute{path}{
    whom = p.path, begin on = {click, of next=node}, arrows = ->,
    shorten > = 2mm,
    base = {\pgfpathmoveto{\pgfpoint{1cm}{5mm}}
            \pgfpathlineto{\pgfpoint{2cm}{5mm}}},
    entry = {1s}{\pgfpathmoveto{\pgfpoint{1cm}{0cm}}
                 \pgfpathlineto{\pgfpoint{2cm}{1cm}}},
    entry = {3s}{\pgfpathmoveto{\pgfpoint{1cm}{1cm}}
                 \pgfpathlineto{\pgfpoint{2cm}{5mm}}}}
  \node (node) [fill = blue!20, draw = blue, very thick, circle] {Click me!};
  \draw              (0.9,-0.1) grid (2.1,1.1);
  \draw [help lines] (0.9,-0.1) grid[step=1mm] (2.1,1.1);
  \draw [very thick, blue, name=p];
}
\end{codeexample}
    %
\end{key}

\begin{key}{/pgf/animation/shorten <=\meta{dimension}}
    Works like |shorten >|.
\end{key}


\subsection{Animating Transformations  and Views}
\label{section-base-animation-views}

In order to animate the canvas transformation matrix, you do not animate an
attribute called ``|transform|'' (or something similar). Rather, there are
several keys that all manipulate the canvas transformation matrix in different
ways. These keys, taken in appropriate combination, allow you to achieve any
particular canvas transformation matrix. All keys that animate the
transformation matrix \emph{always} accumulate.

Some, but not all, of these keys also have an effect on the bounding box
computation: The |translate| and |motion| attribute change the computation of
the bounding box in such a way that it is chosen large enough as to include the
whole picture during all stages of the animation (however, if there are
multiple transformations active at the same time, the computation may not be
correct). In contrast, |scale|, |rotate| and |skew| animations change the
canvas transformation, but are ignored for the bounding box computation. When
in doubt, please set the bounding box explicitly.

Let us start with the basic keys that allow you to change the canvas
transformation matrix directly:

\begin{animateattribute}{scale}
    The |scale| attribute adds an animation of the scaling:
    %
\begin{codeexample}[
    preamble={\usepgfmodule{animations}},
    animation list={0.5,1,1.5,2},
]
\tikz [very thick] {
  \pgfanimateattribute{scale}{
    whom = node, begin on = {click},
    entry = {0s}{0.5}, entry = {2s}{0.75,1.5} }
  \node (node) [fill = blue!20, draw = blue, circle] {Click me!};
}
\end{codeexample}
    %
    The values passed to the |entry| key must either be single scalar values or
    a pair of such numbers separated by a comma (which then refer to the $x$-
    and $y$-scaling).
\end{animateattribute}

\begin{animateattribute}{rotate}
    The |rotate| key adds an animation of the rotation:
    %
\begin{codeexample}[
    preamble={\usepgfmodule{animations}},
    animation list={0.5,1,1.5,2},
]
\tikz [very thick] {
  \pgfanimateattribute{rotate}{
    whom = node, begin on = {click},
    entry = {0s}{45}, entry = {2s}{90} }
  \node (node) [fill = blue!20, draw = blue, circle] {Click me!};
}
\end{codeexample}
    %
    The values are scalar values representing a rotation in degrees.
\end{animateattribute}

\begin{animateattribute}{xskew}
    The |xskew| and |yskew| keys (and also |skew x| and |skew y|, which are
    aliases) add an animation of the skew (given in degrees, not as a slant):
    %
\begin{codeexample}[
    preamble={\usepgfmodule{animations}},
    animation list={0.5,1,1.5,2},
]
\tikz [very thick] {
  \pgfanimateattribute{xskew}{
    whom = node, begin on = {click}, entry = {0s}{0}, entry = {2s}{45} }
  \node (node) [fill = blue!20, draw = blue, circle] {Click me!};
}
\end{codeexample}
    %
    The values are scalar values.
\end{animateattribute}

\begin{animateattribute}{yskew}
    See |xskew|.
\end{animateattribute}

\begin{animateattribute}{skew x}
    An alias of |xskew|.
\end{animateattribute}

\begin{animateattribute}{skew y}
    An alias of |yskew|.
\end{animateattribute}

\begin{animateattribute}{translate}
    The |translate| key adds an animation of the translation (shift):
    %
\begin{codeexample}[
    preamble={\usepgfmodule{animations}},
    animation list={0.5,1,1.5,2},
]
\tikz [very thick] {
  \pgfanimateattribute{translate}{
    whom = node, begin on = {click},
    entry = {0s}{\pgfpointorigin}, entry = {2s}{\pgfpoint{5mm}{-5mm}} }
  \node (node) [fill = blue!20, draw = blue, circle] {Click me!};
}
\end{codeexample}
    %
    The values are \pgfname-points.

    Unlike for the previous canvas transformations, for this key the bounding
    box computation is changed: All points in the to-be-animated scope do not
    only contribute to the normal bounding box, but they also contribute
    shifted by all points in the entry list. The effect is that a bounding box
    is computed that encompasses the animated scope at all stages.
\end{animateattribute}

For all of these attributes, the following key is of importance:
%
\begin{key}{/pgf/animation/origin=\meta{pgf point}}
    An animation of the canvas transformation is added to all other
    transformations from surrounding or interior scopes. This means that, in
    particular, the origin of a canvas transformation is, by default, the
    origin of the canvas of the scope surrounding the transformation object.

    For some canvas animations, like a rotation or a scaling, you will
    typically wish to use a different origin (like the center of an object that
    is to be rotated or scaled). You can achieve this effect by surrounding the
    object by a scope that shifts the canvas to the desired origin, followed by
    a scope whose transformation matrix you animate, followed by a scope that
    shifts back the canvas.

    The |origin| key simplifies this process by allowing you to specify the
    origin of the transformation directly. Internally, however, all this key
    does is to create the above-mentioned scopes with the necessary shifts.
    %
\begin{codeexample}[
    preamble={\usepgfmodule{animations}},
    animation list={0.5,1,1.5,2},
]
\tikz [very thick] {
  \pgfanimateattribute{rotate}{
    whom = node, begin on = {click},
    origin = \pgfpoint{-5mm}{0mm}, entry = {0s}{0}, entry = {2s}{90} }
  \node (node) [fill = blue!20, draw = blue, circle] {Click me!};
}
\end{codeexample}
    %
\end{key}

\begin{animateattribute}{motion}
    A second way of changing the canvas transformation matrix is to use the
    |motion| attribute:
    %
\begin{codeexample}[
    preamble={\usepgfmodule{animations}},
    animation list={0.5,1,1.5,2},
]
\tikz [very thick] {
  \pgfanimateattribute{motion}{
    whom = node, begin on = {click},
    along = \pgfpathcircle{\pgfpointorigin}{5mm},
    entry = {0s}{.25}, entry = {2s}{.5} }
  \node (node) [fill = blue!20, draw = blue, circle] {Click me!}; }
\end{codeexample}

    Just like the |translate| attribute, this key also changes the bounding box
    computation.
    %
    \begin{key}{/pgf/animation/along=\meta{path}}
        This key must be used with |motion| attribute to specify a path along
        which the transformation matrix will be ``moved'' (that is, a shift
        transformation will be added to the different points on the path).

        The values passed to the |entry| key specify fractions of the distance
        along the \meta{path}. That means, when you provide a value of |0|, you
        reference the start point of the path, a value of |1| references the
        end of the path and |0.5| referenced the point halfway along the path.
        %
\begin{codeexample}[
    preamble={\usepgfmodule{animations}},
    animation list={0.25,0.5,0.75,1,1.25,1.5,1.75,2,2.25,2.5},
]
\tikz [very thick] {
  \pgfanimateattribute{motion}{
    whom = node, begin on = {click},
    along = \pgfpathmoveto{\pgfpointorigin}
            \pgfpathlineto{\pgfpoint{0mm}{5mm}},
    entry = {0s}{0}, entry = {1s}{0.5}, entry = {2s}{0.25}, entry={3s}{1} }
  \node (node) [fill = blue!20, draw = blue, circle] {Click me!}; }
\end{codeexample}
    \end{key}

    \begin{key}{/pgf/animation/rotate along=\meta{Boolean} (default true)}
        When set to |true|, the |along| key additionally adds a rotation that
        varies in such a way that a tangent to the path always points right.
        %
\begin{codeexample}[
    preamble={\usepgfmodule{animations}},
    animation list={0.5,1,1.5,2},
]
\tikz [very thick] {
  \pgfanimateattribute{motion}{
    whom = node, begin on = {click},
    rotate along = true,
    along = \pgfpathmoveto {\pgfpointorigin}
            \pgfpathcurveto{\pgfpoint{5mm}{0cm}}{\pgfpoint{5mm}{0cm}}
                           {\pgfpoint{5mm}{5mm}},
    entry = {0s}{0}, entry = {2s}{1} }
  \node (node) [fill = blue!20, draw = blue, circle] {Click me!}; }
\end{codeexample}
    \end{key}
\end{animateattribute}

The final method of changing the transformation matrix is to animate a
\emph{view.}

\begin{animateattribute}{view}
    A view is a canvas transformation that shifts and scales the canvas in such
    a way that a certain rectangle ``matches'' another rectangle: The idea is
    that you ``look through'' a ``window'' (the view) and ``see'' a certain
    area of the canvas. View animation do not allow you to do anything that
    cannot also be done using the |translate| and |scale| keys in combination,
    but it often much more natural to animate which area of a graphic you wish
    to see than to compute and animate a scaling and shift explicitly.

    In order to use a view, you first need to create a view, which is done
    using a |{pgfviewboxscope}|, see Section~\ref{section-base-view}, which is
    used by the |views| library internally. You can then animate the view using
    the |view| attribute. The values passed to the |entry| key must be two
    \pgfname-points, each surrounded by parentheses.
    %
\begin{codeexample}[
    preamble={\usepgfmodule{animations}
\usetikzlibrary{views}},
    animation list={0.5,1,1.5,2},
    animation bb={(1.1,-0.9) rectangle (2.9,0.9)},
]
\tikz [very thick] {
  \pgfanimateattribute{view}{
    whom = me.view, begin on = {click, of next=node}, freeze at end,
    entry = {0s}{{\pgfpoint{0mm}{0mm}}{\pgfpoint{20mm}{20mm}}},
    entry = {2s}{{\pgfpoint{10mm}{10mm}}{\pgfpoint{15mm}{15mm}}} }
  \node (node) [fill = blue!20, draw = blue, very thick, circle] {Click me!};

  \draw [green!50!black] (1.2,-0.8) rectangle (2.7,0.8);
  \begin{scope}[name = me, view = {(0,0) (2,2) at (1.2,-0.8) (2.7,0.8)}]
    \draw [red] (10mm,10mm) rectangle (15mm,15mm);
    \node at (10mm,10mm) [circle, fill=red, text=white, font=\tiny] {red};
  \end{scope}
}
\end{codeexample}

\begin{codeexample}[
    width=2cm,
    preamble={\usepgfmodule{animations}
\usetikzlibrary{views}},
]
\tikz [very thick] {
  \pgfanimateattribute{view}{
    whom = me.view, begin on = {click, of next=n1}, freeze at end,
    entry = {0s}{\pgfpoint{0mm}{0mm}}{\pgfpoint{2cm}{2cm}},
    entry = {2s}{{\pgfpoint{5mm}{5mm}}{\pgfpoint{15mm}{20mm}}} }
  \pgfanimateattribute{view}{
    whom = me.view, begin on = {click, of next=n2}, freeze at end,
    entry = {0s}{\pgfpoint{0mm}{0mm}}{\pgfpoint{2cm}{2cm}},
    entry = {2s}{{\pgfpoint{10mm}{10mm}}{\pgfpoint{15mm}{15mm}}} }
  \node (n1) at (0,0) [fill = blue!20, draw = blue, circle] {Zoom blue};
  \node (n2) at (2,0) [fill = blue!20, draw = blue, circle] {Zoom red};

  \draw [green!50!black] (4,0) rectangle (6,2);
  \begin{scope}[name = me, view = {(0,0) (2,2) at (4,0) (6,2)}]
    \draw [blue] (5mm,5mm) rectangle (15mm,20mm);
    \node at (5mm,5mm) [circle, fill=blue, text=white] {blue};

    \draw [red] (10mm,10mm) rectangle (15mm,15mm);
    \node at (10mm,10mm) [circle, fill=red, text=white, font=\tiny] {red};
  \end{scope}
}
\end{codeexample}
    %
\end{animateattribute}


\subsection{Commands for Specifying Timing: Beginnings and Endings}

Using the |entry| key repeatedly, you specify a timeline: The \meta{time} used
with the first use of the |entry| key in a timeline is the start time and the
\meta{time} in the last |entry| key is the stop time. However, this leaves open
then question of when the whole timeline is to be started: The moment the
document is opened? When the page is displayed? When the user scrolls to the
to-be-animated object? When some other object is clicked? The key |begin|, and
also the key |end|, allow you to specify answers to these questions.

\begin{key}{/pgf/animation/begin=\meta{time}}
    This key specifies when the ``moment |0s|'' should be relative to the
    moment when the current graphic is first displayed. You can use this key
    multiple times, in this case the timeline is restarted for each of the
    times specified (if it is already running, it will be reset). If no |begin|
    key is given at all, the effect is the same as if |begin=0s| had been
    specified.

    It is permissible to set \meta{time} to a negative value.
\end{key}

\begin{key}{/pgf/animation/end=\meta{time}}
    This key will truncate the timeline so that it ends \meta{time} after the
    display of the graphic, provided the timeline begins before the specified
    end time. For instance, if you specify a timeline starting at 2\,s and
    ending at 5\,s and you set |begin| to 1\,s and |end| to 4\,s, the timeline
    will run, relative to the moment when the graphic is displayed from 3\,s to
    4\,s.
    %
\begin{codeexample}[preamble={\usepgfmodule{animations}},width=3cm]
\tikz [very thick] {
  \pgfanimateattribute{rotate}{
    whom = node, begin = 2s, end = 4s,
    entry = {1s}{0}, entry = {2s}{90}, entry = {3s}{180}, entry = {4s}{270} }
  \node (node) [fill = blue!20, draw = blue, circle] {Turn after 3s!}; }
\end{codeexample}
    %
\end{key}

It is not immediately clear what should happen with the attribute of an object
when an animation ends: Should it revert to its original value ``as if there
had never been an animation'' or should it ``stay at the last value''? The
following key governs what should happen:

\begin{key}{/pgf/animation/freeze at end=\meta{true or false} (default true, initially false)}
    When set to |true|, whenever a timeline ends (either because the last time
    of timeline has been reached or because an |end| or |end of| key have ended
    it prematurely), the last value the attribute had because of the animation
    ``stays put''. When set to |false|, which is the initial value, once an
    animation ends, its effect will be removed ``as if it never happened''.
    %
\begin{codeexample}[
    preamble={\usepgfmodule{animations}},
    animation list={0,1,2,3,4},
]
\tikz [very thick] {
  \pgfanimateattribute{rotate}{
    whom = node, begin on = {click}, freeze at end = false,
    entry = {0s}{0}, entry = {2s}{90} }
  \node (node) [fill = blue!20, draw = blue, circle] {Here!}; }
\end{codeexample}
    %
\begin{codeexample}[
    preamble={\usepgfmodule{animations}},
    animation list={0,1,2,3,4},
]
\tikz [very thick] {
  \pgfanimateattribute{rotate}{
    whom = node, begin on = {click}, freeze at end,
    entry = {0s}{0}, entry = {2s}{90} }
  \node (node) [fill = blue!20, draw = blue, circle] {Here!}; }
\end{codeexample}
    %
\end{key}

Instead of specifying the beginning of the timeline relative to the moment to
to-be-animated graphic is displayed, you can also set the ``moment |0s|'' to
the moment a specific \emph{event} happens using the following key:

\begin{key}{/pgf/animation/begin on=\meta{options}}
    Has the same effect as |/tikz/animate/option/begin on|, see
    Section~\ref{section-anim-begin-end}.
\end{key}

When you use |begin on| to start an animation when a certain event is
triggered, it is not clear what should happen when the event is triggered
\emph{again}. Should this be ignored completely? Should it only be ignored
while the animation is running? The following key allows you to specify when
should happen:

\begin{key}{/pgf/animation/restart=\meta{choice} (default true)}
    Has the same effect as |/tikz/animate/option/restart|, see
    Section~\ref{section-anim-begin-end}.
\end{key}

Just like |begin on| specifies when a timeline begins relative to some event,
the |end on| allows you to stop is early when some event happens:

\begin{key}{/pgf/animation/end on=\meta{options}}
    Works exactly like |begin on|, one possible end of the timeline is
    specified using the \meta{options}.
\end{key}


\subsection{Commands for Specifying Timing: Repeats}

Normally, a timeline is displayed once and then ends. You can, however, request
that the timeline should be repeated a certain number of times or indefinitely.

\begin{key}{/pgf/animation/repeats=\meta{specification}}
    Use this key to specify that the timeline animation should repeat at the
    end. The \meta{specification} must consist of two parts, each of which may
    be empty. The first part is one of the following:
    %
    \begin{itemize}
        \item Empty, in which case the timeline repeats forever.
            %
\begin{codeexample}[
    preamble={\usepgfmodule{animations}},
    animation list={1,2,3,4,5},
]
\tikz [very thick] {
  \pgfanimateattribute{rotate}{
    whom = node, begin on = {click}, repeats,
    entry = {0s}{0}, entry = {2s}{90} }
  \node (node) [fill = blue!20, draw = blue, circle] {Click me!}; }
\end{codeexample}
            %
        \item A \meta{number} (like |2| or |3.25|), in which case the timeline
            repeats \meta{number} times.
            %
\begin{codeexample}[
    preamble={\usepgfmodule{animations}},
    animation list={1,2,3,4,5},
]
\tikz [very thick] {
  \pgfanimateattribute{rotate}{
    whom = node, begin on = {click}, repeats = 1.75,
    entry = {0s}{0}, entry = {2s}{90} }
  \node (node) [fill = blue!20, draw = blue, circle] {Click me!}; }
\end{codeexample}
            %
        \item The text ``|for| \meta{time}'' (like |for 2s| or |for 300ms|), in
            which case the timeline repeats however often necessary so that it
            stops exactly after \meta{time}.
            %
\begin{codeexample}[
    preamble={\usepgfmodule{animations}},
    animation list={1,2,3,4,5},
]
\tikz [very thick] {
  \pgfanimateattribute{rotate}{
    whom = node, begin on = {click}, repeats = for 3.5s,
    entry = {0s}{0}, entry = {2s}{90} }
  \node (node) [fill = blue!20, draw = blue, circle] {Click me!}; }
\end{codeexample}
    \end{itemize}
    %
    The second part of the specification must be one of the following:
    %
    \begin{itemize}
        \item Empty, in which case each time the timeline is restarted, the
            attribute's value undergoes the same series of values it did
            previously.
        \item The text |accumulating|. This has the effect that each time the
            timeline is restarted, the attribute values specified by the
            timeline are \emph{added} to the value from the previous
            iteration(s). A typical example is an animation that shifts a scope
            by, say, 1\,cm over a time of 1\,s. Now, if you repeat this five
            times, normally the scope will shift 1\,cm for 1\,s then ``jump
            back'', shift again, jump back, and so on for five times. In
            contrast, when the repeats are accumulating, the scope will move by
            5\,cm over 5\,s in total.
            %
\begin{codeexample}[
    preamble={\usepgfmodule{animations}},
    animation list={1,2,3,4,5},
]
\tikz [very thick] {
  \pgfanimateattribute{rotate}{
    whom = node, begin on = {click}, repeats = accumulating,
    entry = {0s}{0}, entry = {2s}{90} }
  \node (node) [fill = blue!20, draw = blue, circle] {Click me!}; }
\end{codeexample}
            %
\begin{codeexample}[
    preamble={\usepgfmodule{animations}},
    animation list={1,2,3,4,5},
]
\tikz [very thick] {
  \pgfanimateattribute{rotate}{
    whom = node, begin on = {click}, repeats = 2 accumulating,
    entry = {0s}{0}, entry = {2s}{90} }
  \node (node) [fill = blue!20, draw = blue, circle] {Click me!}; }
\end{codeexample}
    \end{itemize}
\end{key}

\begin{key}{/pgf/animation/repeat=\meta{specification}}
    This is an alias for |repeats|.
\end{key}


%%% Local Variables:
%%% mode: latex
%%% TeX-master: "pgfmanual"
%%% End:
