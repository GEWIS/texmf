%% GEWIS Letterhead Template
%% Stijl
%% All documentation can be found on https://gewis.nl/~stijl/

\documentclass[
 digital,         % Use toggle digital to include letterhad in exported pdf
 %printletterhead,% Use toggle printbackground to print letterhead in exported pdf - implies digital
 %english         % Use toggle English for having the template printed in English. You might need to clear all compiled files to prevent having errors after switching the language
]{GEWISLetter}

%% This example letter requires version 1.0 of the GEWISLetter package, which was released on April 0, 2020
\makeatletter
    \@ifclasslater{GEWISLetter}{2020/05/09}{}{
        \ClassError{GEWISLetter}{Dit voorbeeld vereist GEWISLetter v1.0 of hoger}%
    }
\makeatother

%% In this example we show some Lorem Ipsum; it is not neccessary to include this package if you are not going to use it
\usepackage{lipsum}

%% In case we want to use a different sender or contact details, we set it here
%\setSenderName{Ouderdagcommissie}
%\setSenderMail{odc@gewis.nl}
%\setSenderPhone{+31 123 456789}
%\setSenderWeb{odc.gewis.nl}

%% Set the recipient (required parameters have to be set, but can be set as empty string if needed)
\setType{Persoonlijk}           % (optional) Set the Type which is printed above the address - e.g. "Confidential"
\setRecipient{Het Sprookjesbos} % (required) Set the recipient for this letter
\setAttn{Roodkapje}             % (optional) Set the "Attn."/"T.a.v." line in the address

\setStreet{Europalaan 1}        % (required) Set the address of the recipient (for Dutch addresses usually street + number)
\setPostcode{5171 KW}           % (required) Set postal code
\setCity{Kaatsheuvel}           % (required) Set city of recipient
%\setPostcodecity{2020}         % (optional) For countries that don't use postal code + city addressing, you may want to set the line differently
%\setCountry{The Netherlands}   % (optional) Set the country of the recipient

%% Set letter properties
\setYourreference{KHM26}        % (optional) Set the reference ("Uw kenmerk") the recipient uses/used
\setMyreference{32768}          % (optional) Set your own reference ("Ons kenmerk")
\setSubject{De Grote Boze Wolf} % (highly recommended) Set the subject of the letter

\setDate{\today}                % (highly recommended) Set the date the letter is to be sent/was sent

%% Usually you dont wan't the names of peprsons to be split to multiple lines, so you can define them here
%% You can also define new commands to be able to reuse the names or other variables. Using tildes instead of space, allows you to prevent linebreaks within the command
\newcommand{\dewolf}{De~Grote~Boze~Wolf\xspace}
\hyphenation{De Grote Boze Wolf}
\newcommand{\rood}{Roodkapje\xspace}
\hyphenation{Roodkapjë}

%% Now it is time to start the letter
\begin{document}
\GEWISfirstpage                 % We want to print the address information of GEWIS on the first page
\printadresenkenmerk            % We want to print the information we just set on the first page

%% We now can write the letter - Note that we can use the variables we set for the names
Geachte heer/mevrouw \rood,\\

%% We can also use the variables we set for the address to repeat the recipients details e.g.
Namens Studievereniging GEWIS mag ik u meedelen dat we onlangs uw bestelling van rode kapjes met GEWIS-logo's verstuurd hebben naar uw adres (\GEWISstreet, \GEWISpostcodecity).\\

In de bijlage vindt u een overzicht van de kosten.\\

\lipsum[1]\\

\lipsum[2-5]

Met vriendelijke groet,\\[3\baselineskip] \\
\dewolf\\
Commissaris Sprookjesactige Betrekkingen GEWIS 2020-2021


%% Perhaps we want the attachment to start with page 1 again
\resetpagecount
%% Maybe we want to hide the GEWIS logo in the top right corner (can be undone by \GEWISlargeheading)
%% Note that this is not suitable for printed letters, since we only have preprinted full pages
\GEWISsmallheading
\newpage
\subsection{Financi\"en}
\begin{tabularx}{\textwidth}{X r}\toprule
	Beschrijving           & Bedrag(EUR)\\\midrule
	37 Rode kapjes         & 500,00\\
	37 Stenen voor de wolf & 800,00\\
	\cmidrule{2-2} \textbf{Totaal} & {\bfseries 1.300,00}\\\bottomrule
\end{tabularx}



%% Maybe we want some other attachment to start with a fresh page count, or we are generating letters in some kind of fancy for-loop
%% We reset the page count
\resetpagecount
%%We update the recipient and print all the information again
\setRecipient{De Efteling}
\GEWISfirstpage
\printadresenkenmerk

\lipsum[1]


\end{document}
